% Author: Alfredo Sánchez Alberca (asalber@ceu.es)

\chapter[Intervalos de Confianza para Medias y Proporciones]{Intervalos de Confianza\\ para Medias y Proporciones}

\section{Fundamentos teóricos}

\subsection{Inferencia Estadística y Estimación de Parámetros}
El objetivo de un estudio estadístico es doble: describir la muestra elegida de una población en la que se quiere
estudiar alguna característica, y realizar inferencias, es decir, sacar conclusiones y hacer predicciones sobre la
población de la que se ha extraído dicha muestra.

La metodología que conduce a obtener conclusiones sobre la población, basadas en la información contenida en la muestra,
constituye la \emph{Inferencia Estadística}.

Puesto que la muestra contiene menos información que la población, las predicciones serán aproximadas. Por eso, uno de
los objetivos de la inferencia estadística es determinar la probabilidad de que una conclusión obtenida a partir del
análisis de una muestra sea cierta, y para ello se apoya en la teoría de la probabilidad.

Cuando se desea conocer el valor de alguno de los parámetros de la población, el procedimiento a utilizar es la
\emph{Estimación de Parámetros, }que a su vez se divide en \emph{Estimación Puntual}, cuando se da un único valor como
estimación del parámetro poblacional considerado, y \emph{Estimación por Intervalos}, cuando interesa conocer no sólo un
valor aproximado del parámetro sino también la precisión de la estimación. En este último caso el resultado es un
intervalo, dentro del cual estará, con una cierta confianza, el verdadero valor del parámetro poblacional. A este
intervalo se le denomina \emph{intervalo de confianza}. A diferencia de la estimación puntual, en la que se utiliza un
único estimador, en la estimación por intervalo se emplean dos estimadores, uno para cada extremo del intervalo.

\subsection{Intervalos de Confianza}
Dados dos estadísticos muestrales $L_1$ y $L_2$, se dice que el intervalo $I=(L_1,\ L_2)$ es un \emph{Intervalo de
Confianza} para un parámetro poblacional $\theta$, con \emph{nivel de confianza} $1-\alpha$ (o \emph{nivel de
significación} $\alpha $), si la probabilidad de que los estadísticos que determinan los límites del intervalo tomen
valores tales que $\theta$ esté comprendido entre ellos, es igual a $1-\alpha$, es decir, \[ P\left( L_{1}<\theta
<L_{2}\right) =1-\alpha \]

Los extremos del intervalo son variables aleatorias cuyos valores dependen de la muestra considerada. Es decir, los
extremos inferior y superior del intervalo serían $L_{1}\left( X_{1},...,X_{n}\right) $ y $L_{2}\left(
X_{1},...,X_{n}\right)$ respectivamente, aunque habitualmente escribiremos $L_{1}$ y $L_{2}$ para simplificar la
notación. Designaremos mediante $l_{1}$ y $l_{2}$ los valores que toman dichas variables para una muestra determinada
$\left( x_{1},...,x_{n}\right).$

Cuando en la definición se dice que la probabilidad de que el parámetro $\theta $ esté en el intervalo $\left( L_{1},\
L_{2}\right) $ es $1-\alpha $, quiere decir que en el $100 \left( 1-\alpha \right) \ \% $ de las posibles muestras, el
valor de $\theta $ estaría en los correspondientes intervalos $\left( l_{1},\ l_{2}\right) .$

Una vez que se tiene una muestra, y a partir de ella se determina el intervalo correspondiente $\left( l_{1},\
l_{2}\right) $, no tendría sentido hablar de la probabilidad de que el parámetro $\theta $ esté en el intervalo $\left(
l_{1},\ l_{2}\right) $, pues al ser $l_{1}$ y $l_{2}$ números, el parámetro $\theta $, que también es un número, aunque
desconocido, estará o no estará en dicho intervalo, y por ello hablamos de confianza en lugar de probabilidad.

Así, cuando hablemos de un intervalo de confianza para el parámetro $\theta $ con nivel de confianza $1-\alpha $,
entenderemos que antes de tomar una muestra, hay una probabilidad $1-\alpha $ de que el intervalo que se construya a
partir de ella, contenga el valor del parámetro $\theta$. O, dicho de otro modo, si tomasemos 100 muestras del mismo
tamaño y calculásemos sus respectivos intervalos, el $1-\alpha\%$ de estos contendrían el verdadero valor del parámetro a
estimar (ver figura~\ref{g:100intervalos}).

\begin{figure}[h!]
\begin{center}
\scalebox{1}{%% Input file name: 100_intervalos_confianza_media.fig
%% FIG version: 3.2
%% Orientation: Landscape
%% Justification: Flush Left
%% Units: Inches
%% Paper size: A4
%% Magnification: 100.0
%% Resolution: 1200ppi

\begin{pspicture}(5.03cm,4.29cm)(16.66cm,13.45cm)
\psset{unit=0.8cm}
%%
%% Depth: 2147483647
%%
\newrgbcolor{mycolor0}{1.00 0.50 0.31}\definecolor{mycolor0}{rgb}{1.00,0.50,0.31}
\newgray{mycolor1}{0.74}\definecolor{mycolor1}{gray}{0.74}
%%
%% Depth: 100
%%
\psset{linestyle=solid,linewidth=0.03175,linecolor=mycolor0}
\qdisk(10.61,12.00){0.07}
\qdisk(10.70,12.70){0.07}
\qdisk(10.80,10.13){0.07}
\qdisk(10.89,12.78){0.07}
\qdisk(10.99,11.13){0.07}
\qdisk(11.08,11.80){0.07}
\qdisk(11.17,11.91){0.07}
\qdisk(11.27,11.74){0.07}
\qdisk(11.36,10.96){0.07}
\qdisk(11.46,11.37){0.07}
\qdisk(11.55,11.56){0.07}
\qdisk(11.64,11.75){0.07}
\qdisk(11.74,11.05){0.07}
\qdisk(11.83,11.21){0.07}
\qdisk(11.93,12.90){0.07}
\qdisk(12.02,11.82){0.07}
\qdisk(12.12,11.97){0.07}
\qdisk(12.21,12.10){0.07}
\qdisk(12.30,11.25){0.07}
\qdisk(12.40,13.05){0.07}
\qdisk(12.49,11.01){0.07}
\qdisk(12.59,12.06){0.07}
\qdisk(12.68,12.40){0.07}
\qdisk(12.78,12.46){0.07}
\qdisk(12.87,11.07){0.07}
\qdisk(12.96,12.05){0.07}
\qdisk(13.06,11.62){0.07}
\qdisk(13.15,11.18){0.07}
\qdisk(13.25,11.89){0.07}
\qdisk(13.34,11.36){0.07}
\qdisk(13.43,11.66){0.07}
\qdisk(13.53,12.03){0.07}
\qdisk(13.62,12.06){0.07}
\qdisk(13.72,11.74){0.07}
\qdisk(13.81,11.50){0.07}
\qdisk(13.91,10.91){0.07}
\qdisk(14.00,12.03){0.07}
\qdisk(14.09,11.86){0.07}
\qdisk(14.19,10.68){0.07}
\qdisk(14.28,13.03){0.07}
\qdisk(14.38,10.96){0.07}
\qdisk(14.47,11.92){0.07}
\qdisk(14.56,12.51){0.07}
\qdisk(14.66,10.45){0.07}
\qdisk(14.76,13.07){0.07}
\qdisk(14.85,11.95){0.07}
\qdisk(14.94,10.90){0.07}
\qdisk(15.04,12.40){0.07}
\qdisk(15.13,11.28){0.07}
\qdisk(15.23,12.70){0.07}
\qdisk(15.32,11.85){0.07}
\qdisk(15.41,11.55){0.07}
\qdisk(15.51,13.07){0.07}
\qdisk(15.60,12.53){0.07}
\qdisk(15.70,12.96){0.07}
\qdisk(15.79,11.48){0.07}
\qdisk(15.89,12.46){0.07}
\qdisk(15.98,10.94){0.07}
\qdisk(16.07,11.20){0.07}
\qdisk(16.17,10.95){0.07}
\qdisk(16.26,12.44){0.07}
\qdisk(16.36,11.36){0.07}
\qdisk(16.45,11.49){0.07}
\qdisk(16.54,13.00){0.07}
\qdisk(16.64,10.70){0.07}
\qdisk(16.73,10.59){0.07}
\qdisk(16.83,10.98){0.07}
\qdisk(16.92,12.82){0.07}
\qdisk(17.02,12.18){0.07}
\qdisk(17.11,11.02){0.07}
\qdisk(17.20,11.94){0.07}
\qdisk(17.30,10.64){0.07}
\qdisk(17.39,12.29){0.07}
\qdisk(17.49,11.39){0.07}
\qdisk(17.58,12.75){0.07}
\qdisk(17.68,11.13){0.07}
\qdisk(17.77,10.79){0.07}
\qdisk(17.86,12.43){0.07}
\qdisk(17.96,11.23){0.07}
\qdisk(18.05,11.17){0.07}
\qdisk(18.15,12.10){0.07}
\qdisk(18.24,10.95){0.07}
\qdisk(18.33,10.80){0.07}
\qdisk(18.43,9.57){0.07}
\qdisk(18.52,12.15){0.07}
\qdisk(18.62,11.75){0.07}
\qdisk(18.71,12.63){0.07}
\qdisk(18.81,10.15){0.07}
\qdisk(18.90,11.85){0.07}
\qdisk(18.99,11.53){0.07}
\qdisk(19.09,10.88){0.07}
\qdisk(19.18,11.60){0.07}
\qdisk(19.28,11.84){0.07}
\qdisk(19.37,10.95){0.07}
\qdisk(19.47,12.66){0.07}
\qdisk(19.56,11.45){0.07}
\qdisk(19.66,11.35){0.07}
\qdisk(19.75,11.34){0.07}
\qdisk(19.84,11.92){0.07}
\qdisk(19.94,11.09){0.07}
\psset{linecolor=black,fillstyle=none}
\psline(10.51,7.49)(19.94,7.49)
\psline(10.51,7.49)(10.51,7.28)
\psline(12.40,7.49)(12.40,7.28)
\psline(14.28,7.49)(14.28,7.28)
\psline(16.17,7.49)(16.17,7.28)
\psline(18.05,7.49)(18.05,7.28)
\psline(19.94,7.49)(19.94,7.28)
\rput(10.51,6.73){0}
\rput(12.40,6.73){20}
\rput(14.28,6.73){40}
\rput(16.17,6.73){60}
\rput(18.05,6.73){80}
\rput(19.94,6.73){100}
\psline(10.23,8.37)(10.23,14.79)
\psline(10.23,8.37)(10.02,8.37)
\psline(10.23,9.98)(10.02,9.98)
\psline(10.23,11.58)(10.02,11.58)
\psline(10.23,13.18)(10.02,13.18)
\psline(10.23,14.79)(10.02,14.79)
\rput{90}(9.73,8.37){-2}
\rput{90}(9.73,9.98){-1}
\rput{90}(9.73,11.58){0}
\rput{90}(9.73,13.18){1}
\rput{90}(9.73,14.79){2}
\psline(10.23,7.49)(20.31,7.49)(20.31,16.30)(10.23,16.30)(10.23,7.49)
\rput(15.27,5.88){Nº de muestra}
\rput{90}(8.88,11.90){Intervalo de confianza}
\psline(10.23,11.58)(20.31,11.58)
\psset{linewidth=0.0635,linecolor=mycolor1}
\psline(10.61,10.27)(10.61,13.73)
\psline(10.70,10.15)(10.70,15.26)
\psset{linecolor=red}
\psline(10.80,9.04)(10.80,11.22)
\psset{linecolor=mycolor1}
\psline(10.89,10.79)(10.89,14.76)
\psline(10.99,10.28)(10.99,11.98)
\psline(11.08,10.26)(11.08,13.33)
\psline(11.17,10.28)(11.17,13.55)
\psline(11.27,9.27)(11.27,14.22)
\psline(11.36,8.68)(11.36,13.25)
\psline(11.46,8.37)(11.46,14.37)
\psline(11.55,10.49)(11.55,12.64)
\psline(11.64,10.46)(11.64,13.03)
\psline(11.74,9.79)(11.74,12.30)
\psline(11.83,8.67)(11.83,13.74)
\psline(11.93,10.63)(11.93,15.17)
\psline(12.02,10.44)(12.02,13.20)
\psline(12.12,8.86)(12.12,15.07)
\psline(12.21,11.07)(12.21,13.13)
\psline(12.30,10.21)(12.30,12.29)
\psset{linecolor=red}
\psline(12.40,12.49)(12.40,13.60)
\psset{linecolor=mycolor1}
\psline(12.49,9.70)(12.49,12.31)
\psline(12.59,10.42)(12.59,13.71)
\psline(12.68,9.20)(12.68,15.60)
\psset{linecolor=red}
\psline(12.78,11.95)(12.78,12.97)
\psset{linecolor=mycolor1}
\psline(12.87,10.00)(12.87,12.14)
\psline(12.96,10.36)(12.96,13.74)
\psline(13.06,9.88)(13.06,13.37)
\psline(13.15,9.36)(13.15,13.00)
\psline(13.25,10.71)(13.25,13.08)
\psline(13.34,8.49)(13.34,14.22)
\psline(13.43,10.29)(13.43,13.03)
\psline(13.53,9.99)(13.53,14.07)
\psline(13.62,10.90)(13.62,13.23)
\psline(13.72,10.47)(13.72,13.00)
\psline(13.81,8.74)(13.81,14.26)
\psline(13.91,8.93)(13.91,12.90)
\psline(14.00,9.65)(14.00,14.41)
\psline(14.09,10.46)(14.09,13.25)
\psline(14.19,8.78)(14.19,12.58)
\psline(14.28,10.11)(14.28,15.95)
\psline(14.38,8.47)(14.38,13.45)
\psline(14.47,10.26)(14.47,13.59)
\psline(14.56,11.47)(14.56,13.56)
\psline(14.66,9.01)(14.66,11.89)
\psline(14.76,11.54)(14.76,14.60)
\psline(14.85,9.86)(14.85,14.04)
\psline(14.94,9.73)(14.94,12.08)
\psline(15.04,11.05)(15.04,13.74)
\psline(15.13,10.14)(15.13,12.41)
\psline(15.23,10.92)(15.23,14.48)
\psline(15.32,9.92)(15.32,13.78)
\psline(15.41,9.64)(15.41,13.46)
\psset{linecolor=red}
\psline(15.51,12.07)(15.51,14.06)
\psset{linecolor=mycolor1}
\psline(15.60,10.57)(15.60,14.49)
\psline(15.70,10.45)(15.70,15.46)
\psline(15.79,8.96)(15.79,13.99)
\psline(15.89,10.59)(15.89,14.32)
\psline(15.98,9.17)(15.98,12.71)
\psline(16.07,9.41)(16.07,12.99)
\psline(16.17,9.22)(16.17,12.68)
\psline(16.26,9.77)(16.26,15.10)
\psline(16.36,9.77)(16.36,12.96)
\psline(16.45,8.76)(16.45,14.22)
\psline(16.54,10.03)(16.54,15.97)
\psline(16.64,8.57)(16.64,12.83)
\psline(16.73,7.82)(16.73,13.36)
\psline(16.83,9.31)(16.83,12.64)
\psline(16.92,11.41)(16.92,14.23)
\psline(17.02,10.70)(17.02,13.65)
\psline(17.11,9.43)(17.11,12.60)
\psline(17.20,10.46)(17.20,13.43)
\psline(17.30,9.39)(17.30,11.89)
\psline(17.39,9.42)(17.39,15.15)
\psline(17.49,9.07)(17.49,13.72)
\psline(17.58,9.74)(17.58,15.76)
\psline(17.68,9.53)(17.68,12.74)
\psline(17.77,9.52)(17.77,12.06)
\psline(17.86,10.04)(17.86,14.82)
\psline(17.96,8.66)(17.96,13.80)
\psline(18.05,8.31)(18.05,14.04)
\psline(18.15,10.51)(18.15,13.69)
\psline(18.24,9.01)(18.24,12.88)
\psline(18.33,7.86)(18.33,13.75)
\psset{linecolor=red}
\psline(18.43,8.42)(18.43,10.72)
\psset{linecolor=mycolor1}
\psline(18.52,8.75)(18.52,15.55)
\psline(18.62,10.31)(18.62,13.19)
\psline(18.71,10.09)(18.71,15.17)
\psline(18.81,7.89)(18.81,12.42)
\psline(18.90,10.34)(18.90,13.37)
\psline(18.99,8.79)(18.99,14.28)
\psline(19.09,8.75)(19.09,13.01)
\psline(19.18,9.30)(19.18,13.90)
\psline(19.28,9.66)(19.28,14.03)
\psline(19.37,8.69)(19.37,13.21)
\psline(19.47,10.89)(19.47,14.43)
\psline(19.56,9.67)(19.56,13.22)
\psline(19.66,9.20)(19.66,13.49)
\psline(19.75,10.10)(19.75,12.59)
\psline(19.84,9.91)(19.84,13.93)
\psline(19.94,8.31)(19.94,13.88)
\end{pspicture}
%% End
}
\caption{Intervalos de confianza del 95\% para la media de 100 muestras
tomadas de una población normal $N(0,1)$. Como se puede apreciar, de los 100
intervalos, sólo 5 no contienen el valor de la media real $\mu=0$. }
\label{g:100intervalos}
\end{center}
\end{figure}

Cuando se realiza la estimación de un parámetro mediante un intervalo de confianza, el nivel de confianza se suele fijar
a niveles altos (los más habituales son $0.90$, $0.95$ ó $0.99$), para tener una alta confianza de que el parámetro está
dentro del intervalo. Por otro lado, también interesa que la amplitud del intervalo sea pequeña para delimitar con
precisión el valor del parámetro poblacional (esta amplitud del intervalo se conoce como \emph{imprecisión} de la
estimación). Pero a partir de una muestra, cuanto mayor sea el nivel de confianza deseado, mayor amplitud tendrá el
intervalo y mayor imprecisión la estimación, y si se impone que la estimación sea más precisa (menor imprecisión), el
nivel de confianza correspondiente será más pequeño. Por consiguiente, hay que llegar a una solución de compromiso entre
el nivel de confianza y la precisión de la estimación. No obstante, si con la muestra disponible no es posible obtener un
intervalo de amplitud suficientemente pequeña (imprecisión pequeña) con un nivel de confianza aceptable, hay que emplear
una muestra de mayor tamaño. Al aumentar el tamaño muestral se consiguen intervalos de menor amplitud sin disminuir el
nivel de confianza, o niveles de confianza más altos manteniendo la amplitud.


\subsubsection{Intervalos de confianza para la media}
Apoyándose en conclusiones extraídas del Teorema Central del Límite se obtiene que, siempre que las muestras sean grandes
(como criterio habitual se toma que el tamaño muestral, $n$, sea mayor o igual que 30), e independientemente de la
distribución original de la variable de partida $X$, de media $\mu$ y desviación típica $\sigma$, la variable \[
Z=\dfrac{\overline{X}-\mu }{\sigma/\sqrt{n}} \] sigue una distribución Normal tipificada, $N(0,1)$.

Si la desviación típica $\sigma$ de la variable de partida es desconocida, se utiliza como estimación la cuasidesviación
típica muestral: \[ \hat S=\sqrt{\dfrac{\sum \left( X_{i}-\overline{X}\right) ^{2}}{n-1}} \] y con ello, la nueva
variable \[
\dfrac{\overline{X}-\mu }{\hat S/\sqrt{n}}
\] sigue una distribución $t$ de Student con $n-1$ grados de libertad, $T(n-1)$.

Para muestras pequeñas ($n<30$) también pueden aplicarse los resultados anteriores, siempre y cuando la variable
aleatoria de partida $X$, siga una distribución Normal.

A partir de lo anterior y teniendo en cuenta los tres factores de clasificación expuestos: si la población de partida en
la que obtenemos la muestra sigue o no una distribución Normal, si la varianza de dicha población es conocida o
desconocida, y si la muestra es grande ($n\geq30$) o no, pueden deducirse las siguientes expresiones correspondientes a
los diferentes intervalos de confianza.

\subsubsection{Intervalo de confianza para la media de una población normal con varianza conocida en muestras de
cualquier tamaño}
\[
\left( \overline{x}-z_{\alpha /2}\cdot \dfrac{\sigma }{\sqrt{n}},\ \overline{%
x}+z_{\alpha /2}\cdot \dfrac{\sigma }{\sqrt{n}}\right)
\]
En la figura \ref{g:intervalomedia} aparece un esquema explicativo de la construcción de este intervalo.
\begin{figure}[h!]
\begin{center}
\scalebox{0.8}{%% Input file name: calculo_intervalo_confianza_media.fig
%% FIG version: 3.2
%% Orientation: Landscape
%% Justification: Flush Left
%% Units: Inches
%% Paper size: A4
%% Magnification: 100.0
%% Resolution: 1200ppi
%% Include the following in the preamble:
%% \usepackage{textcomp}
%% End

\begin{pspicture}(6.65cm,2cm)(17.03cm,13.56cm)
\psset{unit=0.8cm}
%%
%% Depth: 2147483647
%%
\newrgbcolor{mycolor0}{1.00 0.50 0.31}\definecolor{mycolor0}{rgb}{1.00,0.50,0.31}
\newgray{mycolor1}{0.74}\definecolor{mycolor1}{gray}{0.74}
%%
%% Depth: 100
%%
\psset{linestyle=solid,linewidth=0.03175,linecolor=black,fillstyle=solid,fillcolor=mycolor0}
\psline(10.59,6.47)(10.68,6.48)(10.78,6.50)(10.88,6.51)(10.98,6.53)(11.08,6.55)(11.18,6.57)(11.28,6.60)(11.38,6.64)(11.47,6.68)(11.57,6.72)(11.67,6.78)(11.77,6.84)(11.87,6.91)(11.97,6.99)(12.07,7.09)(12.17,7.19)(12.27,7.31)(12.37,7.44)(12.47,7.59)(12.56,7.76)(12.66,7.94)(12.76,8.14)(12.86,8.35)(12.96,8.59)(12.96,6.47)
\psline(18.01,6.47)(18.01,8.59)(18.11,8.35)(18.21,8.14)(18.30,7.94)(18.40,7.76)(18.50,7.59)(18.60,7.44)(18.70,7.31)(18.80,7.19)(18.90,7.09)(19.00,6.99)(19.10,6.91)(19.20,6.84)(19.30,6.78)(19.40,6.72)(19.49,6.68)(19.59,6.64)(19.69,6.60)(19.79,6.57)(19.89,6.55)(19.99,6.53)(20.09,6.51)(20.19,6.50)(20.28,6.48)(20.38,6.47)
\psset{fillstyle=none}
\psline(10.59,6.47)(10.68,6.48)(10.78,6.50)(10.88,6.51)(10.98,6.53)(11.08,6.55)(11.18,6.57)(11.28,6.60)(11.38,6.64)(11.47,6.68)(11.57,6.72)(11.67,6.78)(11.77,6.84)(11.87,6.91)(11.97,6.99)(12.07,7.09)(12.17,7.19)(12.27,7.31)(12.37,7.44)(12.47,7.59)(12.56,7.76)(12.66,7.94)(12.76,8.14)(12.86,8.35)(12.96,8.59)(13.06,8.84)(13.16,9.11)(13.26,9.39)(13.36,9.70)(13.46,10.01)(13.55,10.34)(13.65,10.69)(13.75,11.04)(13.85,11.40)(13.95,11.76)(14.05,12.13)(14.15,12.49)(14.25,12.85)(14.35,13.20)(14.44,13.54)(14.54,13.86)(14.64,14.16)(14.74,14.44)(14.84,14.69)(14.94,14.91)(15.04,15.10)(15.14,15.25)(15.24,15.37)(15.34,15.45)(15.43,15.49)(15.53,15.49)(15.63,15.45)(15.73,15.37)(15.83,15.25)(15.93,15.10)(16.03,14.91)(16.13,14.69)(16.23,14.44)(16.33,14.16)(16.43,13.86)(16.52,13.54)(16.62,13.20)(16.72,12.85)(16.82,12.49)(16.92,12.13)(17.02,11.76)(17.12,11.40)(17.22,11.04)(17.32,10.69)(17.41,10.34)(17.51,10.01)(17.61,9.70)(17.71,9.39)(17.81,9.11)(17.91,8.84)(18.01,8.59)(18.11,8.35)(18.21,8.14)(18.30,7.94)(18.40,7.76)(18.50,7.59)(18.60,7.44)(18.70,7.31)(18.80,7.19)(18.90,7.09)(19.00,6.99)(19.10,6.91)(19.20,6.84)(19.30,6.78)(19.40,6.72)(19.49,6.68)(19.59,6.64)(19.69,6.60)(19.79,6.57)(19.89,6.55)(19.99,6.53)(20.09,6.51)(20.19,6.50)(20.28,6.48)(20.38,6.47)
\psline(10.19,6.11)(20.78,6.11)(20.78,15.85)(10.19,15.85)(10.19,6.11)
\rput(15.48,16.12){Distribución de la media muestral (muestras grandes) $\bar x=N(\mu,\sigma/n)$}
\rput[l](15.40,4.42){$\bar x$}
\psline(15.40,4.65)(15.57,4.65)
\rput[l]{90}(8.83,8.61){Densidad de probabilidad $f(x)$}
\psline(12.96,6.11)(18.01,6.11)
\psline(12.96,6.11)(12.96,5.90)
\psline(18.01,6.11)(18.01,5.90)
\psline(15.48,6.11)(15.48,5.90)
\rput(12.96,5.35){$-z_{\alpha/2}$}
\rput(18.01,5.35){$z_{\alpha/2}$}
\rput(15.48,5.35){$\mu$}
\psline(10.19,6.43)(10.19,6.43)
\psline(10.19,6.43)(9.98,6.43)
\rput{90}(9.68,6.43){0}
\rput(12.51,6.76){$\alpha/2$}
\rput(18.46,6.76){$\alpha/2$}
\rput(15.48,6.76){$1-\alpha$}
\psset{linecolor=mycolor1}
\psline(10.19,6.43)(20.78,6.43)
\rput[c](15.48,3){$P\left(\mu-z_{\alpha/2}\dfrac{\sigma}{\sqrt n}\leq \bar x \leq \mu+z_{\alpha/2}\dfrac{\sigma}{\sqrt n} \right) = 1-\alpha$}
\end{pspicture}
%% End
}
\caption{Cálculo del intervalo de confianza para la media de una población
normal con varianza conocida, a partir de las distribución de la media muestral $\bar x\sim N(\mu,\sigma/\sqrt n)$ para muestras grandes ($n\geq 30$).} \label{g:intervalomedia}
\end{center}
\end{figure}


\subsubsection{Intervalo de confianza para la media de una población normal con varianza desconocida en muestras de
cualquier tamaño} 
\[ \left( \overline{x}-t_{\alpha /2}^{n-1}\cdot \dfrac{\hat s}{\sqrt{n}},\ \overline{x}+t_{\alpha
/2}^{n-1}\cdot \dfrac{\hat s}{\sqrt{n}}\right) 
\]
Si las muestras son grandes ($n\geq30$) el anterior intervalo puede aproximarse mediante:
\[
\left( \overline{x}-z_{\alpha /2}\cdot \dfrac{\hat s}{\sqrt{n}},\  \overline{x}+z_{\alpha /2}\cdot
\dfrac{\hat s}{\sqrt{n}}\right)
\]


\subsubsection{Intervalo de confianza para la media de una población no normal, varianza conocida y muestras grandes
($n\geq 30$)}
\[
\left( \overline{x}-z_{\alpha /2}\cdot \dfrac{\sigma }{\sqrt{n}},\ \overline{x}+z_{\alpha /2}\cdot \dfrac{\sigma
}{\sqrt{n}}\right)
\]


\subsubsection{Intervalo de confianza para la media de una población no normal, varianza desconocida y muestras grandes
($n\geq30$)} 
\[ 
\left( \overline{x}-t_{\alpha /2}^{n-1}\cdot \dfrac{\hat s}{\sqrt{n}},\ \overline{x}+t_{\alpha /2}^{n-1}\cdot
\dfrac{\hat s}{\sqrt{n}}\right) 
\]
Al tratarse de muestras grandes, el anterior intervalo puede aproximarse por:
\[
\left( \overline{x}-z_{\alpha /2}\cdot \dfrac{\hat s}{\sqrt{n}},\ \overline{x}+z_{\alpha /2}\cdot
\dfrac{\hat s}{\sqrt{n}}\right)
\]

Si la población de partida no es normal, y las muestras son pequeñas, no puede aplicarse el Teorema Central del Límite y
no se obtienen intervalos de confianza para la media.

Para cualquiera de los anteriores intervalos:
\begin{itemize}[label=--]
\item $n$ es el tamaño de la muestra.
\item $\overline{x}$ es la media muestral.
\item $\sigma $ es la desviación típica de la población.
\item $\hat s$ es la cuasidesviación típica muestral: $\hat s^{2}= \dfrac{\sum \left( x_{i}-\overline{x}\right)
^{2}}{n-1}$. 
\item $z_{\alpha /2}$ es el valor que deja a su derecha una probabilidad $\alpha /2 $ en una distribución Normal
tipificada.
\item $t_{\alpha /2}^{n-1}$ es el valor que deja a su derecha una probabilidad $\alpha/2$ en una distribución $t$ de
Student con $n-1$ grados de libertad.
\end{itemize}


\subsubsection {Intervalos de confianza para la proporción poblacional $p$}
Para muestras grandes ($n\geq30~$) y valores de $p$ (probabilidad de ``éxito'') cercanos a $0.5$, la distribución
Binomial puede aproximarse mediante una Normal de media $np$ y desviación típica $\sqrt {np(1-p)}$. En la práctica, para
que sea válida dicha aproximación, se toma el criterio de que tanto $np$ como $n(1-p)$ deben ser mayores que 5. Esto hace
que también podamos construir intervalos de confianza para proporciones tomando éstas como medias de variables
dicotómicas en las que la presencia o ausencia de la característica objeto de estudio (``éxito'' ó ``fracaso'') se
expresan mediante un 1 ó un 0 respectivamente.

De este modo, en muestras grandes y con distribuciones binomiales no excesivamente asimétricas (tanto $np$ como $n(1-p)$
deben ser mayores que 5), si denominamos $\widehat{p}$ a la proporción de individuos que presentan el atributo estudiado
en la muestra concreta, entonces el intervalo de confianza para la proporción con un nivel de significación $\alpha$
viene dado por:
\[
\left( \widehat{p}-z_{\alpha /2}\cdot \sqrt{\dfrac{\widehat{p}\cdot (1-%
\widehat{p})}{n}}\ ,\ \widehat{p}+z_{\alpha /2}\cdot \sqrt{\dfrac{\widehat{p}%
\cdot (1-\widehat{p})}{n}}\right)
\]
donde:
\begin{itemize}[label=--]
\item $n$ es el tamaño muestral.
\item $\widehat{p}$ a la proporción de individuos que presentan el atributo estudiado en la muestra concreta.
\item $z_{\alpha /2}$ es el valor que deja a su derecha una probabilidad $\alpha /2 $ en una distribución Normal
tipificada.
\end{itemize}

En muestras pequeñas o procedentes de una Binomial fuertemente asimétrica ($np \leq 5$ ó $n(1-p)\leq 5$) no puede
aplicarse el Teorema Central del Límite y la construcción de intervalos de confianza debe realizarse a partir de la
distribución Binomial.

\clearpage
\newpage

\section{Ejercicios resueltos}
\begin{enumerate}[leftmargin=*]
\item  Se analiza la concentración de principio activo en una muestra de 10 envases tomados de un lote de un fármaco,
obteniendo los siguientes resultados en mg/mm$^{3}$:
\[
17.6-19.2-21.3-15.1-17.6-18.9-16.2-18.3-19.0-16.4
\]

Se pide:
\begin{enumerate}
\item  Crear un conjunto de datos con la variable \variable{concentracion}.
\item  Calcular el intervalo de confianza para la media de la concentración del lote con nivel de confianza del 95\%
(nivel de significación $\alpha =0.05$).
\begin{indicacion}
\begin{enumerate}
\item Seleccionar el menú \menu{Teaching > Test paramétricos > Medias > Test t para una muestra}.
\item En el cuadro de diálogo que aparece seleccionar la variable \variable{concentracion} en el campo \campo{Variable}
y hacer clic sobre el botón \boton{Enviar}.
\end{enumerate}
\end{indicacion}

\item Calcular los intervalos de confianza para la media con niveles del 90\% y del 99\% (niveles de significación
$\alpha=0.1$ y $\alpha=0.01$).
\begin{indicacion}
Repetir los mismos pasos del apartado anterior, cambiando el nivel de confianza para cada intervalo en la solapa \opcion{Opciones de contraste}
\end{indicacion}

\item Si definimos la precisión del intervalo como la inversa de su amplitud, ¿cómo afecta a la precisión del intervalo de confianza el
tomar niveles de significación cada vez más altos? ¿Cuál puede ser la explicación?

\item ¿Qué tamaño muestral sería necesario para obtener una estimación del contenido medio de principio activo con un margen de error de
$\pm 0.5$ mg/mm$^3$ y una confianza del 95\%? 
\begin{indicacion}
\begin{enumerate}
\item Seleccionar el menú \menu{Teaching > Estadística descriptiva > Estadísticos}.
\item En el cuadro de diálogo que aparece seleccionar la variable \variable{concenctracion} en el campo
\campo{Variable}.
\item En la solapa \menu{Estadísticos básicos} marcar el estadístico \campo{Cuasidesviación típica} y hacer clic en el botón \boton{Enviar}.
\item Seleccionar el menú \menu{Teaching > Test paramétricos > Medias > Cálculo del tamaño muestral para la media}.
\item En el cuadro de diálogo que aparece introducir la cuasidesviación típica muestral en el campo \campo{Desviación
típica}, el nivel de confianza deseado, en este caso $0.95$, en el campo \campo{Nivel de confianza}, el margen de
error deseado, en este caso $0.5$, en el campo \campo{Error}, y hacer clic en el botón \boton{Enviar}.
\end{enumerate}
\end{indicacion}

\item Si, para que sea efectivo, el fármaco debe tener una concentración mínima de 16 mg/mm$^3$ de principio activo, ¿se
puede aceptar el lote como bueno? Justificar la respuesta.
\end{enumerate}

\item  Una central de productos lácteos recibe diariamente la leche de dos granjas $X$ e $Y$. Para analizar la calidad de
la leche, durante una temporada, se controla el contenido de materia grasa de la leche que proviene de ambas granjas, con
los siguientes resultados: 
\[
\begin{array}{ll|ll}
\multicolumn{2}{c|}{X} & \multicolumn{2}{c}{Y} \\
\hline
0.34 & 0.34 & 0.28 & 0.29 \\
0.32 & 0.35 & 0.30 & 0.32 \\
0.33 & 0.33 & 0.32 & 0.31 \\
0.32 & 0.32 & 0.29 & 0.29 \\
0.33 & 0.30 & 0.31 & 0.32 \\
0.31 & 0.32 & 0.29 & 0.31 \\
 &  & 0.33 & 0.32 \\
 &  & 0.32 & 0.33 \\
\end{array}
\]

\begin{enumerate}
\item Crear un conjunto de datos con las variables \variable{grasa} y \variable{granja}.
\item Calcular el intervalo de confianza con un 95\% de confianza para el contenido medio de materia grasa de la leche
sin tener en cuenta si la misma procede de una u otra granja.
\begin{indicacion}
\begin{enumerate}
\item Seleccionar el menú \menu{Teaching > Test paramétricos > Medias > Test t para una muestra}.
\item En el cuadro de diálogo que aparece seleccionar la variable \variable{grasa} en el campo \campo{Variable} y hacer
clic sobre el botón \boton{Enviar}.
\end{enumerate}
\end{indicacion}

\item Calcular los intervalos de confianza con un 95\% de confianza para el contenido medio de materia grasa de la leche
dividiendo los datos según la granja de procedencia de la leche. 
\begin{indicacion}
\begin{enumerate}
\item Seleccionar el menú \menu{Teaching > Test paramétricos > Medias > Test t para una muestra}.
\item En el cuadro de diálogo que aparece seleccionar la variable \variable{grasa} en el campo \campo{Variable}.
\item Seleccionar la casilla de \opcion{Filtro} e introducir la condición \comando{granja==``X''} hacer
clic sobre el botón \boton{Enviar}.
\item Repetir los mismos pasos para el intervalo de confianza de la granja $Y$, introduciendo la condición
\comando{granja==``Y''} en el campo \campo{Condición de selección}.
\end{enumerate}
\end{indicacion}

\item A la vista de los intervalos obtenidos en el punto anterior, ¿se puede concluir que existen diferencias
significativas en el contenido medio de grasa según la procedencia de la leche? Justificar la respuesta.
\end{enumerate}


\item En una encuesta realizada en una facultad, sobre si el alumnado utiliza habitualmente (al menos una vez a la
semana) la biblioteca de la misma, se han obtenido los siguientes resultados:
\begin{center}
\begin{tabular}{lcccccccccccc}
\hline
Alumno & 1 & 2 & 3 & 4 & 5 & 6 & 7 & 8 & 9 & 10 & 11 & 12  \\
Respuesta & 0 & 1 & 0 & 0 & 0 & 1 & 0 & 1 & 1 & 1 & 1 & 0  \\
\hline\\

\hline
Alumno    & 13 & 14 & 15 & 16 & 17 & 18 & 19 & 20 & 21 & 22 & 23 \\
Respuesta & 1  & 0  & 1  & 0  & 0  & 0  & 1  & 1  & 1  & 0  & 0  \\
\hline\\
\hline
Alumno 	  & 24 & 25 & 26 & 27 & 28 & 29 & 30 & 31 & 32 & 33 & 34 \\
Respuesta & 1 & 0 & 0 & 1 & 1 & 0 & 0 & 1 & 0 & 1 & 0 \\
\hline
\end{tabular}
\end{center}

\begin{enumerate}
\item Crear un conjunto de datos con la variable \variable{respuesta}.
\item Calcular el intervalo de confianza con $\alpha=0.01$ para la proporción del alumnado que utiliza habitualmente la biblioteca. 
\begin{indicacion}
\begin{enumerate}
\item Seleccionar el menú \menu{Teaching > Test paramétricos > Proporciones > Test para una proporción}.
\item En el cuadro de diálogo que aparece seleccionar la variable \variable{respuesta} en el campo \campo{Variable} e
introducir \texttt{si} en el campo \campo{Categoría}.
\item En la solapa \menu{Opciones de contraste} introducir $0.99$ en el campo \campo{Nivel de confianza} y hacer clic en
el botón \boton{Enviar}.
\end{enumerate}
\end{indicacion}

\item ¿Qué interpretación tiene dicho intervalo? ¿Cómo es su precisión?

\item ¿Qué tamaño muestral sería necesario para obtener una estimación del porcentaje de alumnos que utilizan regularmente la biblioteca 
con un margen de error de un 1\% y una confianza del 95\%? 
\begin{indicacion}
\begin{enumerate}
\item Seleccionar el menú \menu{Teaching > Test paramétricos > Proporciones > Cálculo del tamaño muestral para una proporción}.
\item En el cuadro de diálogo que aparece introducir la proporción muestral en el campo \campo{p}, el nivel de confianza
deseado, en este caso $0.95$, en el campo \campo{Nivel de confianza}, el margen de error deseado, en este caso
$0.01$, en el campo \campo{Error}, y hacer clic en el botón \boton{Enviar}.
\end{enumerate}
\end{indicacion}
\end{enumerate}



\item El Ministerio de Sanidad está interesado en la elaboración de un intervalo de confianza para la proporción de
personas mayores de 65 años con problemas respiratorios que han sido vacunadas en una determinada ciudad. Para ello,
después de preguntar a 200 pacientes mayores de 65 años con problemas respiratorios en los hospitales de dicha ciudad,
154 responden afirmativamente.
\begin{enumerate}
\item Calcular el intervalo de confianza al 95\% para la proporción de pacientes vacunados.
\begin{indicacion}
\begin{enumerate}
\item Seleccionar el menú \menu{Teaching > Test paramétricos > Proporciones > Test para una proporción}.
\item En el cuadro de diálogo que aparece marcar la opción \opcion{Introducción manual de frecuencias}, introducir 154 en el campo \campo{Frecuencia muestral}, introducir 200 en el campo \campo{Tamaño
muestral} y hacer clic en el botón \boton{Enviar}.
\end{enumerate}
\end{indicacion}

\item Si entre los objetivos del Ministerio se encontraba alcanzar una proporción del al menos un 70\% de vacunados en
dicho colectivo, ¿se puede concluir que se han cumplido los objetivos? Justificar la respuesta.
\end{enumerate}
\end{enumerate}


\section{Ejercicios propuestos}
\begin{enumerate}[leftmargin=*] \item  Para determinar el nivel medio de colesterol (en mg/dl) en la sangre de una población, se
realizaron análisis sobre una muestra de 8 personas, obteniéndose los siguientes resultados: 
\begin{center}
196\quad 212\quad 188\quad 206\quad 203\quad 210\quad 201\quad 198
\end{center}
Hallar los intervalos de confianza para la media del nivel de colesterol con niveles de significación $0.1$, $0.05$ y
$0.01$. ¿Se puede afirmar que el nivel de colesterol medio de la población está por debajo de 210 mg/dl?

\item Para tratar un determinado síndrome neurológico se utilizan dos técnicas $A$ y $B$. En un estudio se tomó una
muestra de 60 pacientes con dicho síndrome y se le aplicó la técnica $A$ a 25 de ellos y la técnica $B$ a los 35
restantes. De los pacientes tratados con la técnica $A$, 18 se curaron, mientras que de los tratados con la técnica $B$,
se curaron 21. Calcular un intervalo de confianza del 95\% para la proporción de curaciones con cada técnica. ¿Qué intervalo es más preciso?

\item A las siguientes elecciones locales en una ciudad se presentan tres partidos: A, B y C. Con el objetivo de hacer
una estimación sobre la proporción de voto que cada uno de ellos obtendrá, se realiza una encuesta en la que responden
300 personas, de las cuales 60 piensan votar a A, 80 a B, 90 a C, 15 en blanco y 55 abstenciones. Calcular un intervalo
de confianza para la proporción de votos, sobre el total del censo, de cada uno de los partidos que se presentan.

\item El fichero \texttt{nations.txt} contiene información sobre el desarrollo de distintos países (tasa de uso de
anticonceptivos (contraception), producto interior bruto per cápita (GDP), tasa de mortalidad infantil
(infant.mortality) y tasa de fertilidad (TFR)). Se pide:
\begin{enumerate}
\item Importar el fichero \texttt{nations.txt} en un conjunto de datos.
\item Calcular el intervalo de confianza de la tasa de uso de anticonceptivos y de la tasa de fertilidad para los países con un producto
interior bruto per cápita superior a 10000 US\$ e inferiores a dicha cantidad. Interpretar los intervalos. 
\end{enumerate}
\end{enumerate}
