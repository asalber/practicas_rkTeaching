% Author: Alfredo Sánchez Alberca (asalber@ceu.es)

\chapter[Contrastes de hipótesis no paramétricos]{Contrastes de Hipótesis\\ No Paramétricos}


\section{Fundamentos teóricos}
Gran parte de los procedimientos estadísticos diseñados para hacer inferencia (prueba $T$ para contrastar hipótesis sobre medias, prueba $F$
para contrastar hipótesis sobre varianzas...), presentan tres características comunes:

\begin{itemize}
\item Permiten contrastar hipótesis referidas a algún parámetro poblacional ($\mu$, $\sigma$, ...), o relaciones entre parámetros
poblacionales ($\mu_1-\mu_2$, $\sigma_1/\sigma_2$, ...).

\item Exigen el cumplimiento de determinados supuestos sobre las poblaciones originales de las que se extraen las muestras, como la
normalidad o la igualdad de varianzas (homogeneidad de varianzas u homocedasticidad), e incluso sobre la manera de obtener los datos, como
la aleatoriedad en las observaciones.

\item Están diseñados para trabajar con variables cuantitativas.
\end{itemize}

A los procedimientos estadísticos que presentan las tres características anteriores se les denomina \textbf{Contrastes Paramétricos} o
\textbf{Pruebas Paramétricas}, y son las técnicas estadísticas más utilizadas. No obstante, presentan dos inconvenientes que hacen que su
utilidad se vea reducida: por un lado, exigen el cumplimiento de supuestos que en determinadas ocasiones pueden resultar demasiado
restrictivos (a menudo las distribuciones no son normales, o no se tiene homogeneidad de varianzas...); por otro, no siempre se puede
trabajar con variables cuantitativas (en ciencias sociales y de la salud, en muchos casos serán cualitativas y en el mejor de los casos
cualitativas ordinales).

Afortunadamente, existen contrastes que permiten poner a prueba hipótesis que no se refieren específicamente a parámetros poblacionales
(como $\mu$ o $\sigma$), o que no exigen supuestos demasiado restrictivos en las poblaciones originales, e incluso que tampoco necesitan
trabajar con variables cuantitativas. Este tipo de contrastes que no cumplen alguna de las tres características de los contrastes
paramétricos reciben el nombre de \textbf{Contrastes No Paramétricos} o \textbf{Pruebas No Paramétricas}.

En realidad, en términos estrictos, para tener un contraste no paramétrico sería suficiente con que en el mismo no se plantease una
hipótesis sobre algún parámetro poblacional. Por eso, algunos autores distinguen entre contrastes no paramétricos y contrastes de
distribución libre (distribution-free), o exentos de distribución, que serían los que no impondrían supuestos restrictivos en la
distribución de la población original. No obstante, como casi todos los contrastes no paramétricos son a su vez de distribución libre, hoy
en día se denomina contraste no paramétrico a cualquiera que no cumpla alguna de las tres características señaladas.

Los mayores atractivos de los contrastes no paramétricos son:

\begin{itemize}
\item Al exigir condiciones menos restrictivas a la muestra sobre las características de la distribución de la población de la que se ha
extraído, son más generales que los paramétricos (se pueden aplicar tanto a situaciones en las que no se deberían aplicar paramétricos como
a situaciones en las que sí que se podrían aplicar).

\item Son especialmente útiles cuando hay que analizar muestras pequeñas, ya que, con muestras grandes ($n\geq 30$), aunque la población de
partida no siga una distribución normal, el Teorema Central del Límite garantiza que la variable media muestral sí que será normal, y no
habrá ningún problema con contrastes sobre la media poblacional basados en la media muestral cuyo comportamiento es normal (por ejemplo,
contrastes con la T de Student). Con muestras pequeñas ($n<30$), si la distribución de los datos de la muestra no es normal tampoco podemos
garantizar que lo sea la distribución de la media muestral, y no quedará más remedio que acudir a contrastes no paramétricos cuando se
pretenda ver si hay o no diferencias significativas entre distribuciones.

\item  Permiten trabajar tanto con variables cualitativas como cuantitativas. En éstas últimas, en lugar de trabajar con los valores
originales, generalmente se trabaja con su rangos, es decir, simplemente con el número de orden que ocupa cada valor, por lo que incluso se
simplifican los cálculos necesarios para aplicar los contrastes.
\end{itemize}

Por otro lado, también hay inconvenientes:

\begin{itemize}
\item Si en un problema concreto se puede aplicar tanto una prueba paramétrica como una no paramétrica, la paramétrica presentará mayor
potencia, es decir, mayor capacidad de detección de diferencias significativas; o lo que es lo mismo, la no paramétrica necesitará mayores
diferencias muestrales para concluir que hay diferencias poblacionales.

\item Con las pruebas paramétricas no sólo se pueden realizar los contrastes de hipótesis planteados sobre los adecuados parámetros
poblacionales para llegar al p-valor del contraste, sino que también se pueden generar intervalos de confianza con los que delimitar
adecuadamente el posible valor de dichos parámetros. Con el p-valor se genera un número que cuantifica si el efecto analizado ha sido o no
estadísticamente significativo, pero puede que el efecto sea tan pequeño que no resulte clínicamente relevante, mientras que con el
intervalo de confianza sí que se aprecia la magnitud del efecto, y por lo tanto se puede concluir si ha sido o no clínicamente relevante.
Igualmente, la aplicación de una prueba no paramétrica permitirá obtener un p-valor del contraste, pero en muy pocas de ellas hay
procedimientos desarrollados que permitan obtener intervalos de confianza.
\end{itemize}

En las técnicas no paramétricas juega un papel fundamental la ordenación de los datos, hasta el punto de que en gran cantidad de casos ni
siquiera es necesario hacer intervenir en los datos observados (variables cuantitativas observadas) más que para establecer una relación de
menor a mayor entre los mismos. El número de orden que cada dato ocupa recibe el nombre de rango del dato. Por ello, gran parte de los
contrastes no paramétricos, además de sobre variables cuantitativas, también pueden aplicarse sobre variables cualitativas ordinales, a
cuyos valores también se les podrá asignar un rango (un número de orden).

Por otra parte, el proceso para llevar a cabo un contraste no paramétrico es muy parecido al de uno paramétrico:
\begin{enumerate}
\item Planteamos una hipótesis nula $H_0$ y su correspondiente alternativa $H_1$.

\item Suponiendo cierta la $H_0$, se calcula el valor de un estadístico muestral de distribución conocida.

\item A partir del estadístico, se calcula el p-valor del contraste y se acepta o se rechaza $H_0$ dependiendo de que el p-valor obtenido
haya sido, respectivamente, mayor o menor que el nivel de significación marcado en la prueba (habitualmente $0,05$).
\end{enumerate}

\subsection{Contrastes no paramétricos más habituales}
Los contrastes no paramétricos más habituales son:
\begin{itemize}
\item Para analizar la \textbf{relación entre dos variables cualitativas nominales} (diferencia de proporciones en las tablas de
contingencia), el contraste no paramétrico más importante es el que se basa en el \textbf{Test de la Chi-Cuadrado}, al que se ha dedicado
toda una práctica previa (Contrastes basados en el estadístico $\chi^2$. Comparación de proporciones).

\item Para analizar la \textbf{aleatoriedad de una muestra}, sobre todo se utiliza el \textbf{Test de Rachas}.

\item Para analizar la \textbf{normalidad de los datos de una muestra}, sobre todo se utiliza el \textbf{Contraste de Kolmogorov-Smirnov},
aunque también se utilizan criterios basados en los estadísticos de Asimetría y Curtosis, el Contraste de Shapiro y Wilk, e incluso diversos
métodos que se apoyan en gráficos, como los gráficos P-P o Q-Q.

\item Para analizar las \textbf{diferencias entre dos muestras, con datos independientes, en variables cuantitativas u ordinales}, sobre
todo se utiliza el \textbf{Test de la U de Mann-Whitney}, cuyo correspondiente paramétrico para variables cuantitativas es la T de Student
de datos independientes.

\item Para analizar las \textbf{diferencias entre dos muestras, con datos pareados, en variables cuantitativas u ordinales}, sobre todo se
utiliza el \textbf{Test de Wilcoxon}, cuyo correspondiente paramétrico para variables cuantitativas es la T de Student de datos
independientes.

\item Para analizar las \textbf{diferencias entre varias muestras (más de dos), con datos independientes, en variables cuantitativas u
ordinales}, sobre todo se utiliza el \textbf{Test de Kruskal-Wallis}, cuyo correspondiente paramétrico para variables cuantitativas es el
ANOVA.

\item Para analizar las \textbf{diferencias entre varias muestras, con datos pareados, en variables cuantitativas u ordinales}, sobre todo
se utiliza el \textbf{Test de Friedman}, cuyo correspondiente paramétrico para variables cuantitativas es el ANOVA de Medidas Repetidas.

\item Para analizar la \textbf{homogeneidad de varianzas de varias muestras} (dos o más), sin tener en cuenta la normalidad de los datos,
sobre todo se utiliza el \textbf{Test de Levene}, cuyo correspondiente paramétrico cuando se comparan dos muestras es el test de igualdad de
varianzas mediante la distribución F de Fisher.

\item Por último, para analizar la \textbf{correlación entre dos variables cuantitativas u ordinales}, sobre todo se utiliza el
\textbf{coeficiente de correlación Rho de Spearman}, junto con su test asociado para ver si la correlación es o no significativa
\end{itemize}

A continuación se analizan todos los contrastes enumerados.

\subsection{Aleatoriedad de una muestra: Test de Rachas}
Frecuentemente suponemos que la muestra que utilizamos es aleatoria simple, es decir, que las  $n$  observaciones se han tomado de manera
aleatoria e independientemente unas de otras, y de la misma población. Pero, en general, esta hipótesis no siempre se puede admitir, siendo
por tanto necesario, en muchos casos, contrastar si realmente estamos frente a una muestra aleatoria simple o no.

El Test de Rachas es una prueba que mide hasta qué punto el valor de una observación en una variable puede influir en las observaciones
siguientes. Por ejemplo, si la variable analizada es el sexo y la muestra es aleatoria extraída de una población con aproximadamente el
mismo número de mujeres que de hombres, entonces resultará prácticamente imposible que después de preguntar a un individuo que resulta ser
hombre, los 10 individuos siguientes también sean hombres; en todo caso, podrían aparecer ``rachas" de 3 o 4 hombres seguidos, pero no de
10.

Consideremos una muestra de tamaño $n$ que ha sido dividida en dos categorías $X$ e $Y$ con $n_{1}$ y $n_{2}$ observaciones cada una. Se
denomina racha a una sucesión de valores de la misma categoría. Por ejemplo, si estudiamos una población de personas y anotamos el sexo: $X$
= Hombre e $Y$= Mujer, y obtenemos la secuencia: $X X X Y Y X Y Y Y$, tendríamos $n_{1}=4$, $n_{2}=5$, $n=n_{1}+n_{2}=9$, y el número de
rachas igual a 4. En función de las cantidades $n_{1}$ y $n_{2}$ se espera que el número de rachas no sea ni muy pequeño ni muy grande, lo
cual equivale a decir que no pueden aparecer rachas demasiado largas, ni demasiadas rachas cortas.

Si las observaciones son cantidades numéricas, pueden dividirse en dos categorías que posean aproximadamente el mismo tamaño sin más que
considerar la mediana de las observaciones como valor que sirve para dividir la muestra entre los valores que están por encima de la mediana
y los que están por debajo.

Con todo ello, el Test de Rachas permite determinar si la variable aleatoria número observado de rachas, $R$, en una muestra de tamaño $n$
(dividida en dos categorías de tamaños $n_1$ y $n_2$), es lo suficientemente grande o pequeño como para rechazar la hipótesis de
independencia (aleatoriedad) entre las observaciones. Para el contraste generalmente se utiliza el estadístico:
\[
Z = \frac{{R - E(R)}}{{\sigma _R }}
\]
donde:
\begin{align*}
E(R) &= \frac{{2n_1 n_2 }}{{n + 1}}\\
\sigma _R  &= \sqrt {\frac{{2n_1 n_2 \left( {2n_1 n_2  - n} \right)}}{{n^2 \left( {n + 1} \right)}}}
\end{align*}

En el supuesto de que se cumpla la hipótesis nula de aleatoreidad de la muestra, se puede demostrar que el estadístico $Z$ sigue una
distribución normal tipificada, que es la que se utiliza para dar el $p$ valor de la prueba.

Por último, conviene no confundir la hipótesis de aleatoriedad de una muestra con la hipótesis de bondad de ajuste de los datos de una
muestra mediante una distribución dada. Por ejemplo, si en 10 lanzamientos de una moneda obtenemos 5 caras y 5 cruces, sin duda los datos se
ajustan adecuadamente a una binomial con probabilidad de éxito igual a $0,5$, y para comprobarlo, por ejemplo, se podría utilizar un test de
Chi-cuadrado; pero si las 5 caras salen justo en los 5 primeros lanzamientos y las 5 cruces en los últimos, difícilmente se podría cumplir
la hipótesis de independencia o aleatoriedad.


\subsection{Pruebas de Normalidad}
Ya hemos comentado que muchos procedimientos estadísticos (intervalos de confianza, tests de hipótesis, análisis de la varianza...)
únicamente son aplicables si los datos provienen de distribuciones normales. Por ello, antes de su aplicación conviene comprobar si se
cumple la hipótesis de normalidad.

No obstante, las inferencias respecto a medias son en general robustas (no les afecta demasiado la hipótesis de normalidad), sea cual sea la
población base, si las muestras son grandes ($n\geq30$) ya que la distribución de la media muestral es asintóticamente normal. Por ello,
gran parte de los métodos paramétricos son válidos con muestras grandes incluso si las distribuciones de partida dejan de ser normales,
pero, aunque válidos, las inferencias respecto a la media dejan de ser óptimas; es decir los métodos paramétricos pierden precisión, y esto
se traduce en intervalos innecesariamente grandes o contrastes poco potentes.

Por otro lado, las inferencias respecto a varianzas son muy sensibles a la hipótesis de normalidad, por lo que no conviene construir
intervalos o contrastes para varianzas si no tenemos cierta seguridad de que la población es aproximadamente normal.

Entre los múltiples métodos que se utilizan para contrastar la hipótesis de normalidad destacan:
\begin{itemize}
\item El cálculo de los estadísticos de asimetría y curtosis de la distribución.
\item La prueba de Kolmogorov-Smirnov, especialmente en su versión corregida por Lilliefors.
\item La prueba de Saphiro-Wills.
\item Gráficos Q-Q y P-P de comparación con la distribución normal.
\end{itemize}

Conviene hacer la aclaración de que la prueba de Kolmogorov-Smirnov no tiene aplicación específica al contraste de normalidad, sino que,
como también sucede con la Chi-cuadrado, la prueba de Kolmogorov-Smirnov, ha sido desarrollada para establecer el ajuste de los datos
observados mediante un modelo teórico (una distribución de probabilidad) que no tiene que ser específicamente normal. La corrección de
Lilliefors adapta la prueba de Kolmogorov-Smirnov para un contraste específico de normalidad.

En cuanto a la conveniencia de una u otra prueba, hay que destacar que no existe un contraste óptimo para probar la hipótesis de normalidad.
La razón es que la potencia relativa de un contraste de normalidad depende del tamaño muestral y de la verdadera distribución que genera los
datos. Desde un punto de vista poco riguroso, el contraste de Shapiro-Wilks es, en términos generales, el más conveniente en pequeñas
muestras $(n<30)$, mientras que el de Kolmogorov-Smirnov, en la versión modificada de Lilliefors, es más adecuado para muestras grandes.

Pasamos a dar una pequeña explicación más detallada de cada uno de los contrastes citados.

\subsubsection{Estadísticos de asimetría y curtosis}
Una primera idea de si los datos provienen de una distribución normal, nos la pueden dar los estadísticos de asimetría, $g_1$, y de
curtosis, $g_2$, junto con sus correspondientes errores estándar, $EE_{g1}$ y $EE_{g2}$. Sabemos que las distribuciones normales son
simétricas, y por tanto $g_1=0$, y tienen un apuntamiento normal, $g_2=0$. Según esto, si una muestra proviene de una población normal sus
coeficientes de asimetría y de apuntamiento no deberían estar lejos de 0, y se acepta que no están demasiado lejos de 0 cuando:
\[
g_1<2EE_{g1}\quad \text{o}\quad g_2<2EE_{g2}.
\]

\subsubsection{Contraste de Kolmogorov-Smirnov y corrección de Lilliefors}
Este contraste general, válido para comprobar si los datos de una variable aleatoria continua se ajustan adecuadamente mediante un modelo de
distribución continuo, compara la función de distribución teórica con la empírica de una muestra. Tiene la ventaja de que no requiere
agrupar los datos. Utiliza el estadístico:
\[
D_n  = \sup \left| {F_n (x) - F(x)} \right|
\]
donde $F_n (x)$ es la función de distribución empírica muestral (la probabilidad acumulada hasta un cierto valor de la variable suponiendo
que en cada uno de los $n$ valores de la muestra acumulamos una probabilidad igual a $1/n$) y $F(x)$ es la función de distribución teórica
(probabilidad acumulada hasta un cierto valor de la variable, o lo que es lo mismo $\int\limits_{ - \infty }^x {f(t)dt}$, donde $f(t)$ es la
función de densidad teórica).

Con el estadístico $D_n$, se calcula:
\[
Z_{K - S}  = \frac{{D_n }}{{\sqrt n }}
\]
que sigue una distribución normal tipificada.

Para el caso específico en que se contrasta si la muestra proviene de una distribución normal, más que el estadístico $Z_{K-S}$ se utiliza
directamente la $D_n$ pero con una distribución tabulada corregida introducida por Lilliefors, por lo que a veces se habla del test de
Kolomogorov-Smirnov-Lilliefors, específico para el contraste de normalidad.

En general, el test de Kolmogorov-Smirnov sin corrección resulta menos potente, y por lo tanto será más difícil rechazar la normalidad de
los datos, que el corregido por Lilliefors, especialmente si la presenta falta de normalidad se produce porque la distribución de partida
presenta valores atípicos.

Ya sea mediante el método de Kolmogorov-Smirnov en general o mediante la corrección introducida por Lilliefors, obtenemos un $p$ valor del
contraste con el que aceptar o rechazar la hipótesis nula de normalidad de los datos de la muestra.

\subsubsection{Contraste de Shapiro-Wilk}
Es una prueba sólo válida para contrastar el ajuste de unos datos muestrales mediante una distribución normal. Además, por su potencia, se
considera que es la más adecuada para analizar muestras pequeñas ($n<30$).

En el proceso, se calcula un estadístico llamado $W$ de Shapiro-Wilk (la forma detallada de obtenerlo va más allá del nivel de esta
práctica), cuya distribución está tabulada. Con ello, de nuevo se genera un $p$ valor que se utiliza para aceptar o no la hipótesis nula de
normalidad de los datos de la muestra.

\subsubsection{Gráficos Q-Q y P-P de comparación con la distribución normal}
Son procedimientos gráficos que permiten llegar a conclusiones cualitativas sobre si los datos se ajustan adecuadamente mediante una
distribución normal.
\begin{itemize}
\item El gráfico Q-Q: presenta en el eje de abcisas los valores de la variable, y en el eje de ordenadas el valor que le correspondería en
una distribución normal tipificada según las probabilidades acumuladas obtenidas gracias a la función de distribución empírica. Junto con
los puntos obtenidos, también se representa la recta que se obtendría sin más que tipificar los datos teniendo en cuenta la media y la
desviación típica de los mismos, de manera que si los datos se ajustasen de forma perfecta mediante una distribución normal, los puntos
quedarían justo en la recta, mientras que si las distancias entre los puntos y la recta son grandes no cabría aceptar la normalidad del
conjunto de datos.

\item El gráfico P-P: es muy parecido al Q-Q pero directamente representa en un eje la probabilidad acumulada observada hasta cada valor de
la variable teniendo en cuenta la función de distribución empírica y en el otro la probabilidad acumulada teórica que le correspondería si
los datos se ajustasen perfectamente mediante una distribución normal. De nuevo se representa la recta que se produciría si el ajuste fuese
perfecto.
\end{itemize}

\subsection{Test de la U de Mann-Whitney}
Este test sustituye a la $T$ de Student para comparar las medias de dos grupos independientes cuando no se cumplen los supuestos en los que
se basa la prueba $T$. Como requiere ordenar los valores antes de hacer el test, no compara realmente las medias, sino los denominados
rangos o números de orden de los datos.

Se debe usar cuando:
\begin{itemize}
\item Alguna de las dos muestras contiene menos de 30 observaciones y no se puede asumir normalidad.
\item La comparación se realiza en una variable ordinal en vez de ser realmente cuantitativa.
\item La muestra es muy pequeña (menos de 10 observaciones en alguno de las dos grupos).
\end{itemize}

Intuitivamente, si se ordenan de menor a mayor conjuntamente los datos de dos muestras y se les asigna a cada uno de ellos su rango (es
decir su número de orden dentro de la lista obtenida al unir los datos de las dos muestras, teniendo en cuenta que si hay dos o más datos
iguales se les asigna a todos ellos el rango promedio de los que les corresponderían si fuesen distintos) es natural que, si se cumple que
las dos muestras originales están igualmente distribuidas, los rangos se mezclen al azar y no que los rangos de una de las muestras
aparezcan al principio y los de la otra al final, pues esto sería indicio de que una de las muestras tendría valores sistemáticamente
mayores o menores que la otra. Con esta idea, se plantean las hipótesis:
\begin{itemize}
\item $H_0$: Las poblaciones de las que provienen las muestras están igualmente distribuidas.
\item $H_1$: Las poblaciones difieren en su distribución.
\end{itemize}

Para llevar a cabo el contraste de hipótesis, se obtienen las sumas de rangos de la muestra 1, $R_1$, y la de la muestra 2, $R_2$. Con $R_1$
y $R_2$ se calculan los estadísticos:
\begin{align*}
U_1  &= n_1 n_2  + \frac{{n_1 \left( {n_1  + 1} \right)}}{2} - R_1\\
U_2  &= n_1 n_2  + \frac{{n_2 \left( {n_2  + 1} \right)}}{2} - R_2  = n_1 n_2  - U_1
\end{align*}
Y teniendo en cuenta $U_1$ y $U_2$, se toma $U$, que es el mínimo de $U_1$ y $U_2$, del que se conoce su distribución de probabilidad y con
el que se puede calcular el $p$ valor del contraste. No obstante, para muestras grandes, $(n\geq30)$, los cálculos con la distribución
exacta pueden ser complicados y por eso se suele trabajar con un estadístico $Z$ que sigue una distribución normal tipificada:
\[
Z = \dfrac{{U - \dfrac{{n_1 n_2 }}{2}}}{{\sqrt {\dfrac{{n_1 n_2 }}{{n\left( {n - 1} \right)}}\left( {\dfrac{{n^3  - n}}{{12}} -
\sum\limits_{i = 1}^k {\dfrac{{t_i ^3  - t_i  }}{{12}}} } \right)} }} 
\]
donde $k$ es el número de rangos distintos en los que existen empates y $t_i$ el número de puntuaciones distintas empatadas en el rango $i$.

\subsection{Test de Wilcoxon para datos emparejados}
Este test, también llamado prueba de rangos con signo de Wilcoxon, se debe utilizar en lugar de la $T$ para datos emparejados cuando:
\begin{itemize}
\item Los datos a comparar son ordinales.
\item Son datos cuantitativos pero la muestra es pequeña $(<30)$ y además no sigue una distribución normal en la variable diferencia entre
las dos mediciones emparejadas.
\end{itemize}

Intuitivamente, si calculamos las diferencias individuo a individuo en las dos variables emparejadas, suponiendo que no hay diferencia
global entre las dos variables se obtendrán aproximadamente tantas diferencias positivas como negativas. Además, sería conveniente trabajar
con un test que no sólo detecte si la diferencia ha sido positiva o negativa, sino que también debería tener en cuenta la cuantía de la
diferencia, o al menos el orden en la cuantía de la diferencia; con ello se podrían compensar situaciones en las que hay pocas diferencias
negativas de mucha cuantía frente a muchas positivas pero de poca cuantía, o a la inversa. Con esta idea, se plantean las hipótesis:
\begin{itemize}
\item $H_0$: No hay diferencia entre las observaciones emparejadas.
\item $H_1$: Sí que las hay.
\end{itemize}

Para realizar el contraste de hipótesis, en primer lugar se calculan las diferencias entre los datos emparejados de las variables $X$ e $Y$
para cada uno de los $n$ individuos:
\[
D_i  = X_i  - Y_i \quad i = 1,...,n
\]

Posteriormente se desechan la diferencias nulas y se toman los valores absolutos de todas las diferencias, $\left|{D_i}\right|$,
asignándoles rangos (órdenes), $R_i$ (si hay empates, se asignan rangos promedio). Después se suman por separado los rangos de las
diferencias que han resultado positivas, $S_{+}=\sum{R_{i}^{+}}$, y por otra parte los rangos de las diferencias que han resultado
negativas, $S_{-}=\sum{R_{i}^{-}}$. Y con ello, si $H_0$ fuese correcta, la suma de rangos positivos debería ser muy parecida a la suma de
rangos negativos: $S_+=S_-$.

Por último, teniendo en cuenta que se conoce la distribución de probabilidad de los estadísticos $S_+$ y $S_-$, se puede calcular el $p$
valor del contraste con la hipótesis nula de su igualdad. No obstante, para muestras grandes, los programas de estadística suelen utilizar
un nuevo estadístico $Z$ que sigue una distribución normal tipificada:
\[
Z = \dfrac{{S - \dfrac{{n\left( {n + 1} \right)}}{4}}}{{\sqrt {\dfrac{{n\left( {n + 1} \right)\left( {2n + 1} \right)}}{{24}} -
\sum\limits_{i = 1}^k {\dfrac{{t_i ^3  - t_{^i } }}{{48}}} } }} 
\]
$S$ es el mínimo de $S_+$ y $S_-$, $k$ el número de rangos distintos en los que existen empates, y $t_i$ el número de puntuaciones empatadas
en el rango $i$.


\subsection{Test de Kruskal-Wallis: comparación no paramétrica de k medias independientes}
Es el test no paramétrico equivalente en su uso al ANOVA, de tal forma que permite contrastar si $k$ muestras independientes (con $k\geq3$)
han sido obtenidas de una misma población y por lo tanto presentan igual distribución.

Se debe usar el Test de Kruskal-Wallis en lugar del ANOVA cuando:
\begin{itemize}
\item Los datos son ordinales.
\item No hay normalidad en alguna de las muestras.
\item No hay homogeneidad de varianzas (la homogeneidad de varianzas recibe el nombre de homocedasticidad).
\end{itemize}

Su uso está también especialmente indicado en el caso de muestras pequeñas y/o tamaños muestrales desiguales, ya que entonces es muy
arriesgado suponer normalidad y homocedasticidad de los datos.

En esencia, el Test de Kruskal-Wallis es una extensión del test de la U de Mann-Whitney, pero trabajando con $k$ muestras en lugar de 2.

Las hipótesis que se contrastan son:
\begin{itemize}
\item $H_0$: Las poblaciones de las que provienen las $k$ muestras están igualmente distribuidas.
\item $H_1$: Alguna de las poblaciones difiere en su distribución con respecto a las demás.
\end{itemize}

Para realizar el contraste, primero se ordenan de menor a mayor todos los valores observados en las k muestras. Luego se asigna el rango 1
al valor inferior, el rango 2 al 2º valor y así sucesivamente. En caso de empate entre dos valores se asigna la media de los números de
orden de los individuos empatados. Después se suman por separado los rangos asignados a las observaciones de cada grupo y se obtiene la suma
de rangos de cada grupo: $R_i, i=1,...,k$. Con la suma de rangos dentro de cada grupo se puede obtener el rango medio de cada grupo sin más
que dividir la suma entre el número total de observaciones en el grupo: $R_{i}/n_i, i=1,...,k$. Y por último, si la hipótesis nula fuera
cierta, los rangos medios de cada grupo serían muy parecidos entre sí y muy parecidos al rango medio total.

Basándose en lo anterior, el estadístico $H$ de Kruskall-Wallis se calcula mediante la fórmula:
\[
H = \frac{{12}}{{n\left( {n + 1} \right)}}\sum\limits_{i = 1}^k {\frac{{R_i ^2 }}{{n_i }}}  - 3\left( {n + 1} \right)
\]
Y se puede demostrar que, si la hipótesis nula es cierta, $H$ sigue una distribución Chi-cuadrado con $k-1$ grados de libertad, que es lo
que se utiliza para calcular el $p$-valor del contraste. Si el número de muestras es 3 y el número de observaciones en alguna no pasa de 5,
el estadístico $H$ no sigue una distribución Chi-cuadrado, pero su distribución está convenientemente tabulada y los programas estadísticos
aplican las adecuadas correcciones.

Si el resultado del test fuera significativo, para buscar entre qué grupos existen diferencias, se harán comparaciones por parejas con la
$U$ de Mann-Whitney, pero penalizando los $p$ valores obtenidos; esto quiere decir que los $p$ valores de cada una las parejas comparadas
deben multiplicarse por el número de comparaciones realizadas. Con ello se logra controlar el error de tipo I en el contraste global, es
decir la probabilidad de que haya aparecido, simplemente por azar y no porque realmente haya diferencias, algún $p$ valor menor que el nivel
de significación fijado. Por ejemplo, en un problema con 4 grupos, habría que hacer 6 comparaciones de parejas distintas, y trabajando con
un $p$-valor frontera de 0,05, la probabilidad de que apareciese alguna diferencia significativa entre las parejas simplemente por azar
viene dada por: $1-0,05^0\cdot0,95^6=0,265$ (probabilidad de algún éxito en una binomial de 6 intentos y probabilidad de éxito 0,05), que
queda muy alejado del 0,05 global con el que se suele trabajar; mientras que con un $p-valor$ frontera de $0,05/4=0,0125$,
$1-0,0125^0\cdot0,9875^5=0,073$, que es muy parecido $0,05$. Igualmente, con $n$ grupos, habría que exigir un $p$-valor en cada comparación
de $0,05/n$; o lo que es lo mismo, no considerar como significativas ninguna de las comparaciones entre parejas en las que el $p$-valor
obtenido multiplicado por $n$ sea mayor que $0,05$.


\subsection{Test de Friedman: equivalente no paramétrico del ANOVA con medidas repetidas}
qEs el test no paramétrico equivalente al ANOVA de medidas repetidas, y por lo tanto se aplica en situaciones en las que para cada individuo
tenemos varias medidas (3 o más) en diferentes tiempos, en una situación muy similar a la del Test de Wilcoxon para datos emparejados, pero en este último caso con sólo 2 medidas en cada individuo. Generalmente cada medida representa el resultado de un tratamiento, por lo que habitualmente se habla de que tenemos $n$ individuos con $k$ tratamientos, siendo $k\geq3$.

Se usa el Test de Friedman en lugar del ANOVA de medidas repetidas cuando:
\begin{itemize}
\item Se comparan medidas repetidas ordinales en lugar de cuantitativas.
\item Alguna de las variables diferencia, generadas mediante la diferencia de todos los tratamientos tomados dos a dos, no siga una
distribución normal.
\item Alguna de las variables diferencia, generadas mediante la diferencia de todos los tratamientos tomados dos a dos, no tenga la misma
varianza.
\end{itemize}

Cuando todas las variables diferencia sigan distribuciones normales con la misma varianza, se dice que los datos cumplen con el supuesto de
Esfericidad, necesario para aplicar el ANOVA de medidas repetidas. Para contrastar el supuesto de esfericidad de los datos se suele utilizar
el Test de Esfericidad de Mauchly.

De nuevo, con el test de Friedman, las hipótesis del contraste a realizar son:
\begin{itemize}
\item $H_0$: No hay diferencia entre los diferentes tratamientos.
\item $H_1$: Al menos alguno de los tratamientos presenta un comportamiento diferente del resto.
\end{itemize}

Para realizar el contraste, se reemplazan los datos de cada sujeto por su rango dentro de cada fila, es decir por su posición, una vez
ordenados de menor a mayor los datos correspondientes a las diferentes observaciones de cada uno de los sujetos. En el caso de empates se
asignará el rango promedio de los valores empatados. Con ello, tenemos una matriz de rangos $R_{ij}$, donde $i=1,...,n$ siendo $n$ el número
de individuos, y $j=1,...,k$, siendo $k$ el número de tratamientos. Después se suman los rangos correspondientes a cada una de las
mediciones realizadas:
\[
R_j  = \sum\limits_{i = 1}^n {R_{ij} } \quad j = 1,...,k
\]
y se calculan sus correspondientes promedios sin más que dividir entre $n$:
\[
\bar R_j  = \frac{{R_j }}{n}\quad j = 1,...,k
\]

En estas condiciones, si la hipótesis nula fuera cierta, los rangos promedio dentro de cada tratamiento deberían ser similares, por lo que
es posible plantearse de nuevo un contraste basado en la Chi-cuadrado. El estadístico, debido a Friedman, en el que se va a basar el
contraste es:
\[
\chi ^2  = \frac{{12}}{{n{\kern 1pt} k\left( {k + 1} \right)}}\sum\limits_{j = 1}^k {R_j ^2 }  - 3n\left( {k + 1} \right)
\]

Este estadístico sigue una distribución Chi-cuadrado con $k-1$ grados de libertad y es el que se utiliza para calcular el $p$ valor del
contraste. Si una vez realizado el contraste se rechaza la hipótesis nula, habrá que determinar qué tratamientos son los que presentan un
comportamiento diferente del resto. Para ello, habrá que aplicar un test de Wilcoxon para datos emparejados a cada una de las parejas que se
puedan formar teniendo en cuenta los diferentes tratamientos (los diferentes tiempos en los que se ha medido la respuesta). Por ejemplo, si
tenemos 3 tratamientos para cada individuo, se pueden generar tres comparaciones pareadas: 1-2, 1-3 y 2-3. Después de aplicar el test de
Wilcoxon, debemos corregir el $p$ valor obtenido en cada cruce multiplicándolo por el número de comparaciones realizado. Por ejemplo, si en
la comparación 1-2 hemos obtenido un $p$ valor igual a 0,03, teniendo en cuenta que hay 3 comparaciones posibles, el $p$ valor corregido
sería 0,09.


\subsection{Test de Levene para el contraste de homogenidad de varianzas}
Como ya se ha comentado, en algunos test paramétricos que realizan la comparación de medias de diferentes muestras se exige la condición de
que las muestras tengan igualdad de varianzas (homogeneidad de varianzas u homocedasticidad), especialmente en el ANOVA y también en algunos
tipos concretos de $T$ de Student. En realidad, se puede utilizar la distribución $F$ de Fisher para comprobar la homogeneidad de varianzas
de dos poblaciones de las que se han extraído muestras aleatorias, pero tiene el problema de que exige la normalidad de los datos y además
está limitado a la comparación de dos varianzas. Por lo tanto, sería conveniente disponer de un test no paramétrico que permita realizar el
contraste de homogeneidad de varianzas sin suponer la normalidad de los datos, y además también permitir el contraste de homogeneidad de
varianzas entre más de dos muestras. Eso mismo es lo que realiza el Test de Levene, cuyas hipótesis vinculadas son:
\begin{itemize}
\item $H_0$: Las varianzas poblacionales de las que se han extraído las $k$ muestras son iguales ($\sigma_1=\sigma_2=...=\sigma_k$).
\item $H_1$: Al menos alguna de las varianzas es diferente al resto.
\end{itemize}

Para hacer el contraste se utiliza el estadístico de Levene:
\[
W = \frac{{\left( {N - k} \right)}}{{\left( {k - 1} \right)}}\frac{{\sum\limits_{i = 1}^k {N_i \left( {Z_i . - Z..} \right)} ^2
}}{{\sum\limits_{i = 1}^k {\sum\limits_{j = 1}^{N_i } {\left( {Z_{ij}  - Z_i .} \right)^2 } } }} 
\]
donde:
\begin{itemize}
\item $k$ es el número de grupos diferentes.
\item $N$ es el número total de individuos.
\item $N_i$ es el número de individuos en el grupo $i$.
\item $Y_{ij}$ es el valor del individuo $j$ en el grupo $i$.
\item Para $Z_{ij}$ se utiliza habitualmente uno de los dos siguientes criterios:
\[
\renewcommand{\arraystretch}{1.8}
Z_{ij}  = \left\{ {\begin{array}{*{20}c}
   {\left| {Yij - \bar Y_i .} \right|}  \\
   {\left| {Yij - \tilde Y_i .} \right|}  \\
\end{array}} \right.
\]
donde $\bar Y_i .$ es la media del grupo $i$ y $\tilde Y_i .$ es la mediana del grupo $i$. Cuando se utilizan las medianas en lugar de las
medias, el Test de Levene suele recibir el nombre de Test de Brown-Forsythe.
\item $Z..$ es la media de todas las $Z_{ij}$:
\[
Z.. = \frac{1}{N}\sum\limits_{i = 1}^k {\sum\limits_{j = 1}^{N_i } {Z_{ij} } }
\]
\item $Z_i.$ es la media de los $Z_{ij}$ en el grupo $i$:
\[
Z_i . = \frac{1}{{N_i }}\sum\limits_{j = 1}^{N_i } {Z_{ij} }
\]
\end{itemize}

Al final, si se cumple la hipótesis nula de igualdad de varianzas se podría demostrar que el estadístico $W$ debería seguir una distribución
$F(k-1,N-k)$ de Fisher con $k-1$ y $N-k$ grados de libertad, que es la que se utiliza para calcular el $p$ valor del contraste.


\subsection{El coeficiente de correlación de Spearman}
El coeficiente de correlación de Spearman, $\rho$, es el equivalente no paramétrico al coeficiente de correlación lineal, $r$. Se utiliza en
lugar de $r$ cuando:
\begin{itemize}
\item Las variables entre las que se analiza la correlación no siguen distribuciones normales.
\item Cuando se quiere poner de manifiesto la relación entre variables ordinales.
\end{itemize}

Además, a diferencia del coeficiente de correlación lineal de Pearson, $r$, el coeficiente de Spearman no estima específicamente una
asociación lineal entre las variables, sino que es capaz de detectar asociación en general. De hecho, las hipótesis del contraste de
asociación de dos variables mediante $\rho$ serían:
\begin{itemize}
\item $H_0$: No hay asociación entre las dos variables.
\item $H_1$: Sí que hay asociación.
\end{itemize}

Para calcular el coeficiente $\rho$ se utilizan los rangos de los valores (el orden de cada valor), tomando rangos promedio cuando tengamos
dos o más valores iguales en alguna de las variables. Si tenemos dos variables $X$ e $Y$, entre las que queremos ver si hay relación, con
$n$ valores, el primer paso es obtenemos los rangos: $R_{x_i}$ y $R_{y_i}$, $i=1,...,n$. Después se calcula la diferencia entre rangos:
$d_i=R_{x_i}-R_{y_i}$, $i=1,...,n$, y posteriormente se calcula $\rho$ mediante la fórmula:
\[
\rho  = 1 - \frac{{6\sum\limits_{i = 1}^n {d_i ^2 } }}{{n\left( {n^2  - 1} \right)}}
\]

Otra forma de obtener el coeficiente $\rho$, que conduce exactamente a los mismos resultados que la fórmula anterior, es calcular el
coeficiente de correlación lineal $r$ pero de los rangos en lugar del $r$ de los valores de partida. El resultado obtenido para $\rho$,
igual que con $r$, siempre está comprendido entre $-1$ y $1$. Si $\rho$ está cercano o es igual a 1, quiere decir que los rangos crecen a la
ven en las dos variables, mientras que si su valor está cercano o es igual a $-1$, quiere decir que cuando en una variable crecen los rangos
en la otra decrecen. Si está cercano a 0 quiere decir que no hay correlación entre los rangos.

Por último, una vez obtenido $\rho$ el contraste de si existe o no asociación entre las variables se puede reformular en términos de si
$\rho$ puede o no ser igual a 0:
\begin{itemize}
\item $H_0$: $\rho=0$.
\item  $H_1$: $\rho\neq0$.
\end{itemize}

Para ello se utiliza el estadístico:
\[
t_\rho   = \frac{{\rho \sqrt {n - 2} }}{{\sqrt {1 - \rho ^2 } }}
\]

Que, bajo la hipótesis nula de no asociación entre las variables ($\rho=0$) y siempre que el tamaño muestral no sea demasiado pequeño (por
ejemplo, $n\geq10$), sigue una distribución $T$ de Student con $n-2$ grados de libertad, lo cual se puede utilizar para calcular el $p$
valor del contraste.

\clearpage
\newpage

\section{Ejercicios resueltos}
\begin{enumerate}[leftmargin=*]
% \item Sea una variable con distribución normal estándar. Se pide:
% \begin{enumerate}
% \item Generar una muestra aleatoria de tamaño 100.
% \begin{indicacion}
% \begin{enumerate}
% \item Seleccionar el menú \menu{Distribuciones > Distribuciones continuas > Distribución normal > Muestra de una
% distribución normal}.
% \item En el cuadro de diálogo que aparece introducir 100 en el campo \campo{Numero de muestras}, 1 en el campo \campo{Número de
% observaciones}, darle un nombre al conjunto de datos y hacer click en el botón \boton{Aceptar}. 
% \item Hacer click en el botón del \boton{Conjunto de datos} y seleccionar el conjunto de datos con la muestra generada.
% \end{enumerate}
% }
% \end{indicacion}
% 
% \item Comprobar la normalidad de los datos mediante un diagrama Q-Q. 
% \begin{indicacion}
% \begin{enumerate}
% \item Seleccionar el menú \menu{Gráficas > Gráfica de comparación de cuantiles}.
% \item En el cuadro de diálogo que aparece seleccionar la variable \variable{obs}, activar la opción correspondiente a la distribución normal
% y hacer click en el botón \boton{Aceptar}.
% \end{enumerate}
% Como los datos provienen de una población normal, los puntos del diagrama deberían quedar alineados alrededor de la línea recta de la
% diagonal.}
% \end{indicacion}
% 
% \item Comprobar la normalidad de los datos mediante un contraste de Shapiro-Wilk.
% \begin{indicacion}
% \begin{enumerate}
% \item Seleccionar el menú \menu{Estadísticos > Test no paramétricos > Test de normalidad de Shapiro-Wilk}.
% \item En el cuadro de diálogo que aparece, seleccionar la variable \variable{obs} en el campo \campo{Variable} y hacer click en el botón
% \boton{Aceptar}.
% \end{enumerate}
% }
% \end{indicacion}
% \end{enumerate}
% 
% \item Sea una variable con distribución uniforme continua $U(0,1)$. Se pide:
% \begin{enumerate}
% \item Generar una muestra aleatoria de tamaño 100.
% \begin{indicacion}
% \begin{enumerate}
% \item Seleccionar el menú \menu{Distribuciones > Distribuciones continuas > Distribución uniforme > Muestra de una
% distribución uniforme}.
% \item En el cuadro de diálogo que aparece introducir 100 en el campo \campo{Numero de muestras}, 1 en el campo \campo{Número de
% observaciones}, darle un nombre al conjunto de datos y hacer click en el botón \boton{Aceptar}. 
% \item Hacer click en el botón del \boton{Conjunto de datos} y seleccionar el conjunto de datos con la muestra generada.
% \end{enumerate}
% }
% \end{indicacion}
% 
% \item Comprobar la normalidad de los datos mediante un diagrama Q-Q. 
% \begin{indicacion}
% \begin{enumerate}
% \item Seleccionar el menú \menu{Gráficas > Gráfica de comparación de cuantiles}.
% \item En el cuadro de diálogo que aparece seleccionar la variable \variable{obs}, activar la opción correspondiente a la distribución normal
% y hacer click en el botón \boton{Aceptar}.
% \end{enumerate}
% Obsérvese que ahora los puntos del diagrama no están alineados sobre la linea recta de la diagonal ya que los datos no
% provienen de una población normal.}
% \end{indicacion}
% 
% \item Comprobar la normalidad de los datos mediante un contraste de Shapiro-Wilk.
% \begin{indicacion}
% \begin{enumerate}
% \item Seleccionar el menú \menu{Estadísticos > Test no paramétricos > Test de normalidad de Shapiro-Wilk}.
% \item En el cuadro de diálogo que aparece, seleccionar la variable \variable{obs} en el campo \campo{Variable} y hacer click en el botón
% \boton{Aceptar}.
% \end{enumerate}
% }
% \end{indicacion}
% \end{enumerate}

\item Las notas obtenidas en un examen en dos grupos de alumnos que han seguido metodologías de estudio distintas han sido:
\[
\begin{array}{lrrrrrrrrrr}
\text{Metodología A:} & 5.8 & 3.2 & 8.0 & 7.3 & 7.1 & 2.1 & 5.0 & 4.4 & 4.2 & 6.7\\
\text{Metodología B:} & 8.1 & 5.4 & 7.2 & 7.5 & 6.3 & 8.2 & 6.0 & 7.8 
\end{array}
\]
\begin{enumerate}
\item Crear un conjunto de datos con las variables \variable{nota} y \variable{metodologia}.

\item Comprobar la hipótesis de normalidad de los datos en cada grupo.
\begin{indicacion}
\begin{enumerate}
\item Seleccionar el menú \menu{Teaching > Test no paramétricos > Normalidad > Test de Shapiro-Wilk}.
\item En el cuadro de diálogo que aparece, seleccionar la variable \variable{nota} en el campo \campo{Variable},
seleccionar la opción \opcion{Contraste por grupos}, seleccionar la variable {metodologia} en el campo \campo{según} y
hacer click en el botón \boton{Aceptar}.
\end{enumerate}
\end{indicacion}

% \item Contrastar la hipótesis de homocedasticidad (igualdad de varianzas) entre las leches de las dos granjas.
% \begin{indicacion}
% \begin{enumerate}
% \item Seleccionar el menú \menu{Estadísticos > Varianzas > Test de Levene}
% \item En el cuadro de diálogo que aparece, seleccionar la variable \variable{Nota} en el campo \campo{Variable explicada}, la variable
% \variable{Metodología} en el campo \campo{Grupos} y hacer click en el botón \boton{Aceptar}.
% \end{enumerate}
% }
% \end{indicacion}

\item Utilizando el contraste más adecuado, ¿se puede concluir que existen diferencias en la nota media según cada
metodología?
\begin{indicacion}
Aunque según el análisis anterior, no podemos rechazar la hipótesis de normalidad en los grupos que queremos comparar,
como la muestra es muy pequeña e incluso uno de los grupos tiene menos de 10 observaciones, la más correcto sería
aplicar el contraste de la U de Mann Whitney.
\begin{enumerate}
\item Seleccionar el menú \menu{Teaching > Test no paramétricos > Test U de Mann-Whitney para dos muestras independientes}.
\item En el cuadro de diálogo que aparece seleccionar la variable \variable{nota} en el campo \campo{Comparar}, la variable
\variable{metodologia} en el campo \campo{Según} y hacer click en el botón \boton{Aceptar}.
\end{enumerate}
\end{indicacion}
\end{enumerate}

\item  Para ver si una campaña de publicidad sobre un fármaco ha influido en sus ventas, se tomó una muestra de 8
farmacias y se midió el número de unidades de dicho fármaco vendidas durante un mes, antes y después de la campaña,
obteniéndose los siguientes resultados:
\[
\begin{tabular}{|c||c|c|c|c|c|c|c|c|}
\hline Antes & 147 & 163 & 121 & 205 & 132 & 190 & 176 & 147 \\
\hline Después & 150 & 171 & 132 & 208 & 141 & 184 & 182 & 149
\\ \hline
\end{tabular}
\]
\begin{enumerate}
\item Crear un conjunto de datos \variable{publicidad} con las variables \variable{antes} y \variable{despues}.

\item Comprobar la hipótesis de normalidad de la variable diferencia.
\begin{indicacion}
Para calcular la variable diferencia:
\begin{enumerate}
\item Seleccionar el menú \menu{Teaching > Datos > Calcular variable}.
\item En el cuadro de diálogo que aparece, introducir la expresión \comando{antes-despues}.
\item En el campo \campo{Guardar nueva variable} introducir el nombre \variable{diferencia}, hacer click en el botón
\boton{Cambiar}, seleccionar el conjunto de datos \variable{publicidad} y hacer click en el botón \boton{Aceptar}.
\end{enumerate}
Para contrastar la normalidad:
\begin{enumerate}
\item Seleccionar el menú \menu{Teaching > Test no paramétricos > Normalidad > Test de Shapiro-Wilk}.
\item En el cuadro de diálogo que aparece, seleccionar la variable \variable{diferencia} en el campo \campo{Variable} y
hacer click en el botón \boton{Aceptar}.
\end{enumerate}
\end{indicacion}

\item Utilizando el contraste más adecuado, ¿se puede concluir que la campaña de publicidad ha aumentado las ventas?
\begin{indicacion}
Aunque, según el análisis anterior, no podemos rechazar la hipótesis de normalidad en la variable diferencia, como la
muestra es muy pequeña, lo más correcto sería aplicar el contraste de Wilcoxon.
\begin{enumerate}
\item Seleccionar el menú \menu{Teaching > Test no paramétricos > Test de Wilcoxon para dos muestras pareadas}.
\item En el cuadro de diálogo que aparece seleccionar la variable \variable{antes} en el campo \campo{Comparar} y la variable
\variable{despues} en el campo \campo{Con}.
\item En la solapa \opcion{Opciones del contraste}, seleccionar la opción \opcion{Unilateral menor} en el campo \campo{Hipótesis
alternativa} y hacer click en el botón \boton{Aceptar}.
\end{enumerate}
\end{indicacion}
\end{enumerate}


\item Se quiere contrastar la eficacia de tres fármacos para reducir la tensión arterial. Para ello se ha medido la
variación en la presión arterial sistólica (antes del tratamiento con el fármaco menos después del tratamiento, en mm
Hg) en quince pacientes, que se dividieron en tres grupos, aplicando a cada grupo un fármaco diferente. Los resultados
obtenidos tras un año de tratamiento fueron:
\[
\begin{tabular}{ccc}
\hline 
Fármaco A & Fármaco B & Fármaco C\\
12 & -3 & 1 \\
15 &  5 & 5 \\
16 & -8 & 19 \\
 6 & -2 & 45 \\
 8 &  4 &  3 \\
\hline
\end{tabular}
\]
\begin{enumerate}
\item Crear un conjunto de datos con las variables \variable{farmaco} y \variable{variacion.presion}.

\item Dibujar el diagrama de cajas de la variación de la presión sistólica según el fármaco recibido.
A la vista del diagrama, ¿crees que los datos presentan homogeneidad de varianzas? 
¿crees que hay algún grupo con un cambio en la presión arterial diferente del resto?
\begin{indicacion}
\begin{enumerate}
\item Seleccionar el menú \menu{Teaching > Gráficos > Diagrama de cajas}.
\item En el cuadro de diálogo que aparece, seleccionar la opcion \opcion{Cajas agrupadas}, seleccionar la variable
\variable{variacion.presion} en el campo \campo{Variable}, seleccionar la variable \variable{farmaco} en el campo
\campo{Variable de agrupación} y hacer click sobre el botón \boton{Aceptar}.
\end{enumerate}
\end{indicacion}

\item Analizar la normalidad de los datos de los tres grupos. 
\begin{indicacion}
\begin{enumerate}
\item Seleccionar el menú \menu{Teaching > Test no paramétricos > Normalidad > Test de Shapiro-Wilk}.
\item En el cuadro de diálogo que aparece, seleccionar la variable \variable{variacion.presion} en el campo \campo{Variable},
seleccionar la opción \opcion{Contraste por grupos}, seleccionar la variable {farmaco} en el campo \campo{según} y
hacer click en el botón \boton{Aceptar}.
\end{enumerate}
\end{indicacion}

\item Analizar la homogeneidad de varianzas de los tres grupos.
\begin{indicacion}
\begin{enumerate}
\item Seleccionar el menú \menu{Teaching > Test paramétricos > Varianzas > Test de Levene}.
\item En el cuadro de diálogo que aparece, seleccionar la variable \variable{variacion.presion} en el campo \campo{Comparar}, la
variable \variable{farmaco} en el campo \campo{Según} y hacer click en el botón \boton{Aceptar}.
\end{enumerate}
\end{indicacion}

\item Utilizando el contraste más adecuado, ¿se puede concluir que existen diferencias en los cambios de la presión
sistólica en función del fármaco recibido?
\begin{indicacion}
En este caso no se cumple la homogeneidad de varianzas, por lo que no se puede aplicar una ANOVA y tendremos que
recurrir al test de Kruskal-Wallis.
\begin{enumerate}
\item  Seleccionar el menú \menu{Teaching > Estadísticos > Test no paramétricos > Test de Kruskal-Wallis}.
\item En el cuadro de diálogo que aparece, seleccionar la variable \variable{variacion.presion} en el campo \campo{Comparar}, la
variable \variable{farmaco} en el campo \campo{Según} y hacer click en el botón \boton{Aceptar}.
\end{enumerate}
\end{indicacion}

\item ¿Entre qué grupos se dan las diferencias significativas?
\begin{indicacion}
\begin{enumerate}
\item  Seleccionar el menú \menu{Teaching > Estadísticos > Test no paramétricos > Test de Kruskal-Wallis}.
\item En el cuadro de diálogo que aparece, seleccionar la variable \variable{variacion.presion} en el campo \campo{Comparar} y la
variable \variable{farmaco} en el campo \campo{Según}.
\item En la solapa \menu{Comparación por pares} seleccionar la opción \opcion{Comparación por pares} y hacer click en el botón \boton{Aceptar}.
\end{enumerate}
\end{indicacion}
\end{enumerate}


\item Se quiere contrastar la dificultad de cuatro modelos de examen que se van a poner en la convocatoria ordinaria de
la asignatura de bioestadística. Para ello se pide a cinco profesores diferentes que valoren cada uno de los modelos de
0 a 10, y los resultados fueron:
\[
\begin{tabular}{|c|c|c|c|c|}
\cline{2-5}
\multicolumn{1}{c|}{}& Modelo1 & Modelo2 & Modelo3 & Modelo4  \\
\hline Profesor 1 & 6 & 8 & 5 & 8 \\
\hline Profesor 2 & 5 & 4 & 7 & 9 \\
\hline Profesor 3 & 5 & 4 & 5 & 6 \\
\hline Profesor 4 & 7 & 4 & 6 & 7 \\
\hline Profesor 5 & 6 & 3 & 7 & 8 \\
\hline
\end{tabular}
\]

\begin{enumerate}
\item Crear las variables \textsf{modelo1}, \textsf{modelo2}, \textsf{modelo3} y \textsf{modelo4} e introducir los datos
de la muestra.

\item ¿Podemos afirmar que el grado de dificultad de los modelos es diferente?
\begin{indicacion}
Puesto que el tamaño muestral es muy pequeño, debe utilizarse el test de Friedman ya que las puntuaciones en los
modelos son medidas repetidas en los profesores.
\begin{enumerate}
\item Seleccionar el menú \menu{Teaching > Test no paramétricos > Test de Friedman para medidas repetidas}.
\item En el cuadro de diálogo que aparece seleccionar las variables \textsf{modelo1}, \textsf{modelo2}, \textsf{modelo3} y \textsf{modelo4}
en el campo \texttt{Variables a comparar (medidas repetidas)} y hacer click sobre el botón \texttt{Aceptar}.
\end{enumerate}
\end{indicacion}
\end{enumerate}


\item El test de Apgar es un examen clínico de neonatología en donde el médico realiza una prueba medida en 3 estándares
sobre el recién nacido para obtener una primera valoración simple (macroscópica) y clínica sobre el estado general del
neonato después del parto. El recién nacido es evaluado de acuerdo a cinco parámetros fisioanatómicos simples, que son:
color de la piel, frecuencia cardíaca, reflejos, tono muscular y respiración. A cada parámetro se le asigna una
puntuación entre 0 y 2, y sumando las cinco puntuaciones se obtiene el resultado del test. El test se realiza al minuto,
a los cinco minutos y, en ocasiones, a los diez minutos de nacer. La puntuación al primer minuto evalúa el nivel de
tolerancia del recién nacido al proceso del nacimiento y su posible sufrimiento, mientras que la puntuación obtenida a
los 5 minutos evalúa el nivel de adaptabilidad del recién nacido al medio ambiente y su capacidad de recuperación.

En la siguiente tabla se refleja la puntuación obtenida por 22 recién nacidos en el test de Apgar al minuto y a los cinco minutos de haber
nacido:
\begin{center}
\begin{tabular}{|l|l|l|l|l|l|l|l|l|l|l|l|l|l|l|l|l|l|l|l|l|l|l|l|l|}
\hline
Apgar 1 & 10 & 3 & 8 & 9 & 8 & 9 & 8 & 8 & 8 & 8 & 7 & 8 & 6 & 8 & 9 & 9 & 9 & 9 & 8 & 9 & 3 & 9 \\
\hline
Apgar 5 & 10 & 6 & 9 & 10 & 9 & 10 & 9 & 9 & 9 & 9 & 9 & 9 & 6 & 8 & 9 & 9 & 9 & 9 & 8 & 9 & 3 & 9 \\
\hline
\end{tabular}
\end{center}

Con los datos anteriores se pretende realizar un contraste de hipótesis para analizar si existe o no relación entre las dos puntuaciones.
Para ello, se pide:
\begin{enumerate}
\item Crear un conjunto de datos con las variables \variable{apgar1} y \variable{apgar5}.

\item Comprobar si las variables siguen distribuciones normales.
\begin{indicacion}
\begin{enumerate}
\item Seleccionar el menú \menu{Teaching > Test no paramétricos > Normalidad > Test de Shapiro-Wilk}.
\item En el cuadro de diálogo que aparece, seleccionar la variable \variable{apgar1} en el campo \campo{Variable} y
hacer click en el botón \boton{Aceptar}.
\item Repetir lo mismo para la variable \variable{apgar5}.
\end{enumerate}
\end{indicacion}

\item Realizar el contraste de correlación.
\begin{indicacion}
Como las variables analizadas no siguen distribuciones normales, para realizar un contraste de relación entre ambas hay
que obtener el coeficiente de correlación de Spearman y ver si es o no significativamente diferente de $0$.
\begin{enumerate}
\item Seleccionar el menú \menu{Teaching > Regresión > Correlación}.
\item En el cuadro de diálogo que aparece, seleccionar las variables \variable{apgar1} y \variable{apgar5} en el campo \campo{Variables}.
\item En la solapa \menu{Opciones de correlación}, seleccionar la opción \opcion{Ro de Spearman} en el campo
\campo{Método} y hacer click en el botón \boton{Aceptar}.
\end{enumerate}
\end{indicacion}
\end{enumerate}

\end{enumerate}


\section{Ejercicios propuestos}
\begin{enumerate}[leftmargin=*]
\item  Se ha realizado un estudio para investigar el efecto del ejercicio físico en el nivel de colesterol en la sangre.
En el estudio participaron once personas, a las que se les midió el nivel de colesterol antes y después de desarrollar
un programa de ejercicios. 
Los resultados obtenidos fueron los siguientes:
\begin{center}
\begin{tabular}{|l|l|l|l|l|l|l|l|l|l|l|}
\hline
Nivel Previo, 223, 212, 221, 210, 225, 202, 198, 200, 185, 220\\
\hline
Nivel Posterior, 226, 211, 222, 212, 225, 201, 196, 217, 130, 220  \\
\hline
\end{tabular}
\end{center}

Utilizando el contraste más adecuado, ¿se puede concluir que el ejercicio físico disminuye el nivel de colesterol?


\item Dos químicos $A$ y $B$ realizan respectivamente 14 y 16 determinaciones de la actividad radiactiva de una muestra
de material. 
Sus resultados en Curios:
\begin{center}
\begin{tabular}{ll|ll}
\multicolumn{2}{c|}{A} & \multicolumn{2}{c}{B} \\
\hline
263.10 & 262.60 & 286.53 & 254.54 \\
262.10 & 259.60 & 284.55 & 286.30 \\
257.60 & 262.20 & 272.52 & 282.90 \\
261.70 & 261.20 & 283.85 & 253.75 \\
260.70 & 259.20 & 252.01 & 245.26 \\
269.13 & 268.63 & 275.08 & 266.08 \\
268.13 & 217.00 & 267.53 & 252.05 \\
 &  & 253.82 & 269.81 \\
\end{tabular}
\end{center}

Utilizando el contraste más adecuado, ¿se puede concluir que existen diferencias significativas en la actividad
detectada por cada químico?


\item En un hospital se están evaluando dos tratamientos diferentes para ver si existen diferencias entre ellos, para lo
cual se seleccionaron dos grupos de 32 pacientes cada uno y se aplicó un tratamiento a cada grupo.
Los resultados fueron:
\begin{center}
\begin{tabular}{|l|c|c|c|c|}
\cline{2-5}
\multicolumn{1}{c|}{} & Empeoraron & Igual & Mejoraron & Curaron\\
\hline
A & 9 & 12 & 5 & 6 \\
\hline
B & 5 & 6 & 11 & 10 \\
\hline
\end{tabular}
\end{center}

Utilizando el contraste más adecuado, ¿se puede concluir que existen diferencias significativas entre ambos
tratamientos?


\item Queremos comparar las notas iniciales de un grupo de 20 alumnos, con las obtenidas al final del curso, pera ver si
existen diferencias, las notas fueron (SS suspenso, A aprobado, N notable y SB sobresaliente):
\begin{center}
\begin{tabular}{|l|l|l|l|l|l|l|l|l|l|l|}
\hline
 Alumno & 1 & 2 & 3 & 4 & 5 & 6 & 7 & 8 & 9 & 10 \\
\hline
Nota Inicial & SS & A & A & SS & N & SS & SS & SB & A & A  \\
\hline
Nota Final & A & A & SS & A & SB & N & A & N & SS & A  \\
\hline
\end{tabular}
\end{center}

\begin{center}
\begin{tabular}{|l|l|l|l|l|l|l|l|l|l|l|}
\hline
 Alumno & 11 & 12 & 13 & 14 & 15 & 16 & 17 & 18 & 19 & 10 \\
\hline
Nota Inicial & SS & N & A & SS & SB & A & N & SS & A & SB  \\
\hline
Nota Final & SB & A & N & A & SB & A & N & A & N & SB  \\
\hline
\end{tabular}
\end{center}

Utilizando el contraste más adecuado, ¿se puede concluir que existen diferencias significativas entre las notas al
comienzo y al final del curso?


\item Disponemos de la evaluación que han obtenido tres grupos de prácticas de las asignatura de bioestadística (MM muy
mal, M mal, R regular, B bien y MB muy bien):
\begin{center}
\begin{tabular}{|l|l|l|l|l|l|l|l|l|l|l|l|l|l|l|l|}
\hline
Grupo 01 & R & B & R & M & MM & B & MB & R & M & B & M & R & R & MM & M\\
\hline
Grupo 02 & B & R & M & B & R & M & B & MB & M & R & M & R & & &  \\
\hline
Grupo 03 & MB & B & M & R & B & MB & B & R & B & MB & B & R & MB & &  \\
\hline
\end{tabular}
\end{center}

Utilizando el contraste más adecuado, ¿se puede concluir que existen diferencias significativas en la evaluación de los
diferentes grupos?


\item Para comparar las dificultades presentados por un grupo de problemas de lógica, se han seleccionado aleatoriamente
a ocho individuos a los que se les ha planteado tres pruebas iguales, a cada uno y se han anotado los tiempos, en
minutos, que han tardado en resolverlos. 
Los resultados obtenidos son:
\begin{center}
\begin{tabular}{ccc}
\hline
Prueba 1 & Prueba 2 & Prueba 3 \\
38 & 6 & 35 \\
22 & 4 & 9 \\
14 & 8 & 8 \\
8 & 2 & 4 \\
6 & 4 & 8 \\
10 & 14 & 10 \\
14 & 2 & 5 \\
8 & 6 & 3 \\
\hline
\end{tabular}
\end{center}

Utilizando el contraste más adecuado, ¿se puede concluir que existen diferencias significativas en los tiempos de
resolución de las tres pruebas?


\item La siguiente tabla muestra los datos de 9 pacientes con anemia aplástica:
\begin{center}
\begin{tabular}{|l|l|l|l|l|l|l|l|l|l|}
\hline
\multicolumn{1}{|c|}{Reticulocitos (\%)} & \multicolumn{1}{c|}{3,6} & \multicolumn{1}{c|}{2,0} & \multicolumn{1}{c|}{0,3} & \multicolumn{1}{c|}{0,3} & \multicolumn{1}{c|}{0,2} & \multicolumn{1}{c|}{3,0} & \multicolumn{1}{c|}{0,0} & \multicolumn{1}{c|}{1,0} & \multicolumn{1}{c|}{2,2} \\
\hline
\multicolumn{1}{|c|}{Linfocitos (por mm$^2$)} & \multicolumn{1}{c|}{1700} & \multicolumn{1}{c|}{3078} & \multicolumn{1}{c|}{1820} & \multicolumn{1}{c|}{2706} & \multicolumn{1}{c|}{2086} & \multicolumn{1}{c|}{2299} & \multicolumn{1}{c|}{676} & \multicolumn{1}{c|}{2088} & \multicolumn{1}{c|}{2013} \\
\hline
\end{tabular}
\end{center}

Mediante el adecuado contraste de hipótesis basado en el coeficiente de correlación de Spearman, ¿existe relación entre
ambas variables?

\end{enumerate}