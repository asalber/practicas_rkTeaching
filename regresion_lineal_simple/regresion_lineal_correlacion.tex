\chapter{Regresión Lineal Simple y Correlación}

\section{Fundamentos teóricos}
\subsection{Regresión}
La \emph{regresión} es la parte de la estadística que trata de determinar la posible relación entre una variable
numérica $Y$, que suele llamarse \emph{variable dependiente}, y otro conjunto de variables numéricas, $X_1,
X_2,\ldots,X_n$, conocidas como \emph{variables independientes}, de una misma población. 
Dicha relación se refleja mediante un modelo funcional $y=f(x_1,\ldots,x_n)$.

El caso más sencillo se da cuando sólo hay una variable independiente $X$, y entonces se habla de \emph{regresión
simple}. 
En este caso el modelo que explica la relación entre $X$ e $Y$ es una función de una variable $y=f(x)$.

Dependiendo de la forma de esta función, existen muchos tipos de regresión simple.
Los más habituales son los que aparecen en la siguiente tabla:
\begin{center}
\begin{tabular}{|l|c|}
\hline
 Modelo      &     Ecuación genérica      \\
\hline\hline
 Lineal                  &          $y=a+bx$          \\
\hline
 Parabólico              &       $y=a+bx+cx^2$        \\
\hline
 Polinómico de grado $n$ & $y=a_0+a_1x+\cdots+a_nx^n$ \\
\hline
 Potencial               &       $y=ax^b$       \\
\hline
 Exponencial             &     $y=e^{a+bx}$      \\
\hline
 Logarítmico             &       $y=a+b\log x$        \\
\hline
Inverso & $y=a+b/x$ \\
\hline
Curva S & $y= e^{a+b/x}$ \\
\hline
\end{tabular}
\end{center}

Para elegir un tipo de modelo u otro, se suele representar el \emph{diagrama de dispersión}, que consiste en dibujar
sobre unos ejes cartesianos correspondientes a las variables $X$ e $Y$, los pares de valores $(x_i,y_j)$ observados en
cada individuo de la muestra.

\begin{ejemplo}
En la figura la figura \ref{g:estatura-peso} aparece el diagrama de dispersión correspondiente a una muestra de 30
individuos en los que se ha medido la estatura en cm ($X$) y el peso en kg ($Y$).
En este caso la forma de la nube de puntos refleja una relación lineal entre la estatura y el peso.

\begin{figure}[h!]
\centering
\scalebox{0.75}{%% Input file name: diagrama_dispersion_estatura_peso.fig
%% FIG version: 3.2
%% Orientation: Landscape
%% Justification: Flush Left
%% Units: Inches
%% Paper size: A4
%% Magnification: 100.0
%% Resolution: 1200ppi
%% Include the following in the preamble:
%% \usepackage{textcomp}
%% End

\begin{pspicture}(5.94cm,3.48cm)(16.66cm,13.45cm)
\psset{unit=0.8cm}
%%
%% Depth: 2147483647
%%
\newrgbcolor{mycolor0}{1.00 0.50 0.31}\definecolor{mycolor0}{rgb}{1.00,0.50,0.31}
\newgray{mycolor1}{0.74}\definecolor{mycolor1}{gray}{0.74}
%%
%% Depth: 100
%%
\psset{linestyle=solid,linewidth=0.03175,linecolor=mycolor0}
\qdisk(16.02,11.64){0.1}
\qdisk(14.90,8.87){0.1}
\qdisk(16.39,9.70){0.1}
\qdisk(14.34,8.87){0.1}
\qdisk(12.10,6.94){0.1}
\qdisk(15.09,9.01){0.1}
\qdisk(14.71,8.46){0.1}
\qdisk(13.59,8.18){0.1}
\qdisk(18.82,12.33){0.1}
\qdisk(17.14,10.25){0.1}
\qdisk(12.85,7.49){0.1}
\qdisk(17.51,10.67){0.1}
\qdisk(19.56,14.95){0.1}
\qdisk(15.65,8.32){0.1}
\qdisk(15.83,9.56){0.1}
\qdisk(13.41,7.91){0.1}
\qdisk(11.35,6.80){0.1}
\qdisk(16.77,12.74){0.1}
\qdisk(13.59,6.94){0.1}
\qdisk(14.53,8.87){0.1}
\qdisk(15.27,9.56){0.1}
\qdisk(16.58,8.18){0.1}
\qdisk(13.78,8.04){0.1}
\qdisk(14.15,8.46){0.1}
\qdisk(14.71,9.56){0.1}
\qdisk(17.32,9.70){0.1}
\qdisk(14.71,7.35){0.1}
\qdisk(15.46,9.29){0.1}
\qdisk(13.97,9.15){0.1}
\qdisk(17.51,10.95){0.1}
\psset{linecolor=black,fillstyle=none}
\psline(10.61,6.47)(19.94,6.47)
\psline(10.61,6.47)(10.61,6.26)
\psline(12.47,6.47)(12.47,6.26)
\psline(14.34,6.47)(14.34,6.26)
\psline(16.21,6.47)(16.21,6.26)
\psline(18.07,6.47)(18.07,6.26)
\psline(19.94,6.47)(19.94,6.26)
\rput(10.61,5.71){150}
\rput(12.47,5.71){160}
\rput(14.34,5.71){170}
\rput(16.21,5.71){180}
\rput(18.07,5.71){190}
\rput(19.94,5.71){200}
\psline(10.23,6.80)(10.23,15.09)
\psline(10.23,6.80)(10.02,6.80)
\psline(10.23,8.18)(10.02,8.18)
\psline(10.23,9.56)(10.02,9.56)
\psline(10.23,10.95)(10.02,10.95)
\psline(10.23,12.33)(10.02,12.33)
\psline(10.23,13.71)(10.02,13.71)
\psline(10.23,15.09)(10.02,15.09)
\rput{90}(9.73,6.80){50}
\rput{90}(9.73,8.18){60}
\rput{90}(9.73,9.56){70}
\rput{90}(9.73,10.95){80}
\rput{90}(9.73,12.33){90}
\rput{90}(9.73,13.71){100}
\rput{90}(9.73,15.09){110}
\psline(10.23,6.47)(20.31,6.47)(20.31,15.28)(10.23,15.28)(10.23,6.47)
\rput(15.27,15.99){Diagrama de dispersión de Estaturas y Pesos}
\rput(15.27,4.86){Estatura (cm)}
\rput{90}(8.88,10.88){Peso (Kg)}
\psset{linestyle=dashed,linecolor=mycolor1}
\psline(16.02,6.47)(16.02,11.64)
\psline(10.23,11.64)(16.02,11.64)
\rput(16.02,12){$(179,85)$}
\end{pspicture}
%% End
}
\caption{Diagrama de dispersión. El punto (179,85) indicado corresponde a un individuo de la muestra que mide 179 cm y
pesa 85 Kg.}
\label{g:estatura-peso}
\end{figure}
\end{ejemplo}

Según la forma de la nube de puntos del diagrama, se elige el modelo más apropiado (figura~\ref{g:tiposrelaciones}), y
se determinan los parámetros de dicho modelo para que la función resultante se ajuste lo mejor posible a la nube de
puntos.

\begin{figure}[h!]
\centering 
\subfigure[Sin relación.]{\scalebox{0.5}{%% Input file name: diagrama_dispersion_sin_relacion.fig
%% FIG version: 3.2
%% Orientation: Landscape
%% Justification: Flush Left
%% Units: Inches
%% Paper size: A4
%% Magnification: 100.0
%% Resolution: 1200ppi
%% Include the following in the preamble:
%% \usepackage{textcomp}
%% End

\begin{pspicture}(7.06cm,3.29cm)(16.36cm,13.56cm)
\psset{unit=0.8cm}
%%
%% Depth: 2147483647
%%
\newrgbcolor{mycolor0}{1.00 0.50 0.31}\definecolor{mycolor0}{rgb}{1.00,0.50,0.31}
%%
%% Depth: 100
%%
\psset{linestyle=solid,linewidth=0.03175,linecolor=mycolor0}
\qdisk(18.77,12.67){0.1}
\qdisk(10.15,6.95){0.1}
\qdisk(16.47,7.53){0.1}
\qdisk(16.52,14.13){0.1}
\qdisk(15.03,13.90){0.1}
\qdisk(14.24,9.90){0.1}
\qdisk(16.29,7.29){0.1}
\qdisk(13.52,14.98){0.1}
\qdisk(12.57,8.44){0.1}
\qdisk(16.53,12.51){0.1}
\qdisk(16.20,7.88){0.1}
\qdisk(17.61,8.79){0.1}
\qdisk(11.06,12.08){0.1}
\qdisk(16.83,6.87){0.1}
\qdisk(10.82,11.76){0.1}
\qdisk(13.80,12.41){0.1}
\qdisk(19.55,8.34){0.1}
\qdisk(12.96,12.68){0.1}
\qdisk(16.39,8.39){0.1}
\qdisk(12.97,7.50){0.1}
\qdisk(14.32,11.36){0.1}
\qdisk(16.15,10.56){0.1}
\qdisk(17.20,10.70){0.1}
\qdisk(12.78,14.11){0.1}
\qdisk(17.35,14.54){0.1}
\qdisk(18.21,14.21){0.1}
\qdisk(11.13,9.04){0.1}
\qdisk(15.69,10.55){0.1}
\qdisk(10.89,11.18){0.1}
\qdisk(10.54,13.67){0.1}
\qdisk(17.70,14.68){0.1}
\qdisk(14.67,7.80){0.1}
\qdisk(12.12,15.21){0.1}
\qdisk(19.37,9.75){0.1}
\qdisk(11.19,7.40){0.1}
\qdisk(17.36,13.53){0.1}
\qdisk(14.87,10.39){0.1}
\qdisk(15.55,9.67){0.1}
\qdisk(14.23,13.73){0.1}
\qdisk(12.79,11.91){0.1}
\qdisk(16.15,11.59){0.1}
\qdisk(17.07,10.73){0.1}
\qdisk(16.68,6.09){0.1}
\qdisk(12.60,8.26){0.1}
\qdisk(15.69,10.66){0.1}
\qdisk(16.30,8.04){0.1}
\qdisk(14.04,12.02){0.1}
\qdisk(15.77,14.30){0.1}
\qdisk(12.94,14.05){0.1}
\qdisk(18.20,12.14){0.1}
\qdisk(19.35,6.51){0.1}
\qdisk(11.05,15.07){0.1}
\qdisk(19.06,10.84){0.1}
\qdisk(15.49,12.85){0.1}
\qdisk(14.62,14.33){0.1}
\qdisk(15.09,14.14){0.1}
\qdisk(19.12,6.00){0.1}
\qdisk(15.20,8.40){0.1}
\qdisk(12.89,12.55){0.1}
\qdisk(15.00,7.98){0.1}
\qdisk(17.34,8.76){0.1}
\qdisk(12.23,12.88){0.1}
\qdisk(12.52,7.20){0.1}
\qdisk(15.35,14.11){0.1}
\qdisk(11.78,15.00){0.1}
\qdisk(15.50,6.87){0.1}
\qdisk(14.43,13.71){0.1}
\qdisk(15.10,14.89){0.1}
\qdisk(19.08,12.63){0.1}
\qdisk(11.38,13.02){0.1}
\qdisk(14.59,8.29){0.1}
\qdisk(15.79,11.59){0.1}
\qdisk(17.09,11.97){0.1}
\qdisk(16.88,11.38){0.1}
\qdisk(17.89,6.70){0.1}
\qdisk(13.24,14.50){0.1}
\qdisk(14.26,11.53){0.1}
\qdisk(17.08,5.85){0.1}
\qdisk(11.02,9.44){0.1}
\qdisk(12.08,7.01){0.1}
\qdisk(18.41,7.62){0.1}
\qdisk(17.13,11.49){0.1}
\qdisk(18.09,11.24){0.1}
\qdisk(14.21,11.19){0.1}
\qdisk(13.15,14.78){0.1}
\qdisk(14.75,8.74){0.1}
\qdisk(16.34,15.26){0.1}
\qdisk(12.56,13.18){0.1}
\qdisk(12.58,14.69){0.1}
\qdisk(13.45,8.17){0.1}
\qdisk(15.59,11.11){0.1}
\qdisk(17.54,7.86){0.1}
\qdisk(13.91,9.09){0.1}
\qdisk(12.17,11.26){0.1}
\qdisk(13.35,8.11){0.1}
\qdisk(10.81,8.24){0.1}
\qdisk(19.43,8.62){0.1}
\qdisk(14.42,11.36){0.1}
\qdisk(14.01,10.76){0.1}
\qdisk(14.65,6.91){0.1}
\rput(14.85,16.12){Sin relación}
\rput[l](14.71,4.63){$X$}
\rput[l]{90}(9.35,10.42){$Y$}
\psset{linecolor=black,fillstyle=none}
\psline(9.77,5.48)(19.93,5.48)(19.93,15.64)(9.77,15.64)(9.77,5.48)
\end{pspicture}
%% End
}}\qquad
\subfigure[Relación lineal.]{\scalebox{0.5}{%% Input file name: diagrama_dispersion_relacion_lineal.fig
%% FIG version: 3.2
%% Orientation: Landscape
%% Justification: Flush Left
%% Units: Inches
%% Paper size: A4
%% Magnification: 100.0
%% Resolution: 1200ppi
%% Include the following in the preamble:
%% \usepackage{textcomp}
%% End

\begin{pspicture}(7.06cm,3.29cm)(16.36cm,13.56cm)
\psset{unit=0.8cm}
%%
%% Depth: 2147483647
%%
\newrgbcolor{mycolor0}{1.00 0.50 0.31}\definecolor{mycolor0}{rgb}{1.00,0.50,0.31}
%%
%% Depth: 100
%%
\psset{linestyle=solid,linewidth=0.03175,linecolor=mycolor0}
\qdisk(16.95,12.63){0.1}
\qdisk(18.43,15.26){0.1}
\qdisk(19.26,13.83){0.1}
\qdisk(14.59,10.24){0.1}
\qdisk(18.55,12.88){0.1}
\qdisk(14.48,8.38){0.1}
\qdisk(12.46,9.00){0.1}
\qdisk(14.75,11.02){0.1}
\qdisk(12.80,8.51){0.1}
\qdisk(10.41,7.61){0.1}
\qdisk(18.44,14.54){0.1}
\qdisk(17.23,12.12){0.1}
\qdisk(14.85,10.75){0.1}
\qdisk(13.93,8.79){0.1}
\qdisk(13.42,9.30){0.1}
\qdisk(13.43,9.51){0.1}
\qdisk(17.02,11.54){0.1}
\qdisk(17.17,13.10){0.1}
\qdisk(18.27,12.81){0.1}
\qdisk(14.32,8.29){0.1}
\qdisk(11.13,7.42){0.1}
\qdisk(18.63,13.65){0.1}
\qdisk(11.06,8.06){0.1}
\qdisk(15.23,11.47){0.1}
\qdisk(17.89,12.31){0.1}
\qdisk(18.27,12.17){0.1}
\qdisk(18.73,14.79){0.1}
\qdisk(19.54,14.89){0.1}
\qdisk(12.54,9.33){0.1}
\qdisk(17.19,12.52){0.1}
\qdisk(15.95,11.91){0.1}
\qdisk(18.41,13.01){0.1}
\qdisk(16.70,10.51){0.1}
\qdisk(16.80,11.76){0.1}
\qdisk(10.99,7.56){0.1}
\qdisk(11.26,8.94){0.1}
\qdisk(16.32,13.21){0.1}
\qdisk(12.58,8.97){0.1}
\qdisk(15.02,12.00){0.1}
\qdisk(11.85,7.59){0.1}
\qdisk(18.32,12.93){0.1}
\qdisk(11.07,7.63){0.1}
\qdisk(12.12,9.34){0.1}
\qdisk(18.07,13.64){0.1}
\qdisk(14.07,11.70){0.1}
\qdisk(15.00,9.86){0.1}
\qdisk(18.52,13.72){0.1}
\qdisk(16.90,13.34){0.1}
\qdisk(17.26,11.52){0.1}
\qdisk(10.15,6.52){0.1}
\qdisk(14.05,10.53){0.1}
\qdisk(18.30,13.55){0.1}
\qdisk(11.06,8.80){0.1}
\qdisk(18.12,13.08){0.1}
\qdisk(16.10,12.30){0.1}
\qdisk(10.17,6.67){0.1}
\qdisk(15.01,11.19){0.1}
\qdisk(12.28,8.04){0.1}
\qdisk(11.91,9.12){0.1}
\qdisk(12.05,9.09){0.1}
\qdisk(16.09,11.58){0.1}
\qdisk(15.98,11.74){0.1}
\qdisk(19.55,13.47){0.1}
\qdisk(14.89,10.67){0.1}
\qdisk(16.54,12.68){0.1}
\qdisk(15.80,11.33){0.1}
\qdisk(11.00,7.52){0.1}
\qdisk(14.16,9.83){0.1}
\qdisk(15.05,10.03){0.1}
\qdisk(15.52,11.93){0.1}
\qdisk(16.90,12.77){0.1}
\qdisk(17.52,12.28){0.1}
\qdisk(11.00,6.67){0.1}
\qdisk(12.15,8.79){0.1}
\qdisk(13.90,9.63){0.1}
\qdisk(18.07,12.87){0.1}
\qdisk(11.68,8.49){0.1}
\qdisk(13.01,9.73){0.1}
\qdisk(17.27,11.74){0.1}
\qdisk(17.70,13.12){0.1}
\qdisk(11.27,7.63){0.1}
\qdisk(14.84,11.82){0.1}
\qdisk(13.69,11.74){0.1}
\qdisk(14.07,10.20){0.1}
\qdisk(11.16,7.21){0.1}
\qdisk(15.88,11.23){0.1}
\qdisk(18.83,14.75){0.1}
\qdisk(14.09,10.28){0.1}
\qdisk(11.60,8.57){0.1}
\qdisk(14.04,9.73){0.1}
\qdisk(14.74,10.51){0.1}
\qdisk(10.70,7.36){0.1}
\qdisk(18.75,14.09){0.1}
\qdisk(15.46,10.73){0.1}
\qdisk(13.99,9.88){0.1}
\qdisk(10.73,5.85){0.1}
\qdisk(15.85,9.70){0.1}
\qdisk(11.50,9.04){0.1}
\qdisk(11.36,7.49){0.1}
\qdisk(18.17,13.03){0.1}
\rput(14.85,16.12){Relación lineal}
\rput[l](14.71,4.63){$X$}
\rput[l]{90}(9.35,10.42){$Y$}
\psset{linecolor=black,fillstyle=none}
\psline(9.77,5.48)(19.93,5.48)(19.93,15.64)(9.77,15.64)(9.77,5.48)
\end{pspicture}
%% End
}}\qquad
\subfigure[Relación polinómica.]{\scalebox{0.5}{%% Input file name: diagrama_dispersion_relacion_parabolica.fig
%% FIG version: 3.2
%% Orientation: Landscape
%% Justification: Flush Left
%% Units: Inches
%% Paper size: A4
%% Magnification: 100.0
%% Resolution: 1200ppi
%% Include the following in the preamble:
%% \usepackage{textcomp}
%% End

\begin{pspicture}(7.06cm,3.29cm)(16.36cm,13.56cm)
\psset{unit=0.8cm}
%%
%% Depth: 2147483647
%%
\newrgbcolor{mycolor0}{1.00 0.50 0.31}\definecolor{mycolor0}{rgb}{1.00,0.50,0.31}
%%
%% Depth: 100
%%
\psset{linestyle=solid,linewidth=0.03175,linecolor=mycolor0}
\qdisk(16.95,7.05){0.1}
\qdisk(18.43,10.98){0.1}
\qdisk(19.26,14.20){0.1}
\qdisk(14.59,5.85){0.1}
\qdisk(18.55,15.26){0.1}
\qdisk(14.48,5.89){0.1}
\qdisk(12.46,9.12){0.1}
\qdisk(14.75,5.86){0.1}
\qdisk(12.80,6.85){0.1}
\qdisk(10.41,15.08){0.1}
\qdisk(18.44,13.64){0.1}
\qdisk(17.23,9.07){0.1}
\qdisk(14.85,5.87){0.1}
\qdisk(13.93,6.16){0.1}
\qdisk(13.42,6.71){0.1}
\qdisk(13.43,6.89){0.1}
\qdisk(17.02,8.43){0.1}
\qdisk(17.17,8.85){0.1}
\qdisk(18.27,10.34){0.1}
\qdisk(14.32,5.90){0.1}
\qdisk(11.13,13.56){0.1}
\qdisk(18.63,10.78){0.1}
\qdisk(11.06,11.35){0.1}
\qdisk(15.23,5.87){0.1}
\qdisk(17.89,8.99){0.1}
\qdisk(18.27,11.78){0.1}
\qdisk(18.73,13.44){0.1}
\qdisk(19.54,14.06){0.1}
\qdisk(12.54,8.14){0.1}
\qdisk(17.19,8.27){0.1}
\qdisk(15.95,6.50){0.1}
\qdisk(18.41,11.06){0.1}
\qdisk(16.70,8.24){0.1}
\qdisk(16.80,6.07){0.1}
\qdisk(10.99,10.09){0.1}
\qdisk(11.26,11.93){0.1}
\qdisk(16.32,6.47){0.1}
\qdisk(12.58,7.72){0.1}
\qdisk(15.02,5.89){0.1}
\qdisk(11.85,9.82){0.1}
\qdisk(18.32,9.56){0.1}
\qdisk(11.07,13.06){0.1}
\qdisk(12.12,7.74){0.1}
\qdisk(18.07,9.09){0.1}
\qdisk(14.07,6.14){0.1}
\qdisk(15.00,5.98){0.1}
\qdisk(18.52,11.42){0.1}
\qdisk(16.90,7.06){0.1}
\qdisk(17.26,7.30){0.1}
\qdisk(10.15,13.82){0.1}
\qdisk(14.05,6.81){0.1}
\qdisk(18.30,10.44){0.1}
\qdisk(11.06,12.20){0.1}
\qdisk(18.12,10.97){0.1}
\qdisk(16.10,7.26){0.1}
\qdisk(10.17,13.42){0.1}
\qdisk(15.01,5.85){0.1}
\qdisk(12.28,8.81){0.1}
\qdisk(11.91,9.30){0.1}
\qdisk(12.05,8.93){0.1}
\qdisk(16.09,6.29){0.1}
\qdisk(15.98,6.44){0.1}
\qdisk(19.55,13.43){0.1}
\qdisk(14.89,5.96){0.1}
\qdisk(16.54,7.46){0.1}
\qdisk(15.80,6.47){0.1}
\qdisk(11.00,10.22){0.1}
\qdisk(14.16,5.95){0.1}
\qdisk(15.05,5.86){0.1}
\qdisk(15.52,6.41){0.1}
\qdisk(16.90,8.42){0.1}
\qdisk(17.52,7.97){0.1}
\qdisk(11.00,12.87){0.1}
\qdisk(12.15,8.94){0.1}
\qdisk(13.90,7.48){0.1}
\qdisk(18.07,10.16){0.1}
\qdisk(11.68,8.26){0.1}
\qdisk(13.01,8.48){0.1}
\qdisk(17.27,8.16){0.1}
\qdisk(17.70,10.53){0.1}
\qdisk(11.27,12.32){0.1}
\qdisk(14.84,5.85){0.1}
\qdisk(13.69,6.86){0.1}
\qdisk(14.07,6.32){0.1}
\qdisk(11.16,13.67){0.1}
\qdisk(15.88,6.31){0.1}
\qdisk(18.83,11.79){0.1}
\qdisk(14.09,6.34){0.1}
\qdisk(11.60,9.10){0.1}
\qdisk(14.04,6.02){0.1}
\qdisk(14.74,5.92){0.1}
\qdisk(10.70,13.05){0.1}
\qdisk(18.75,11.38){0.1}
\qdisk(15.46,6.61){0.1}
\qdisk(13.99,6.04){0.1}
\qdisk(10.73,10.42){0.1}
\qdisk(15.85,6.30){0.1}
\qdisk(11.50,12.58){0.1}
\qdisk(11.36,9.86){0.1}
\qdisk(18.17,11.08){0.1}
\rput(14.85,16.12){Relación parabólica}
\rput[l](14.71,4.63){$X$}
\rput[l]{90}(9.35,10.42){$Y$}
\psset{linecolor=black,fillstyle=none}
\psline(9.77,5.48)(19.93,5.48)(19.93,15.64)(9.77,15.64)(9.77,5.48)
\end{pspicture}
%% End
}}\\
\subfigure[Relación exponencial.]{\scalebox{0.5}{%% Input file name: diagrama_dispersion_relacion_exponencial.fig
%% FIG version: 3.2
%% Orientation: Landscape
%% Justification: Flush Left
%% Units: Inches
%% Paper size: A4
%% Magnification: 100.0
%% Resolution: 1200ppi
%% Include the following in the preamble:
%% \usepackage{textcomp}
%% End

\begin{pspicture}(7.06cm,3.29cm)(16.36cm,13.56cm)
\psset{unit=0.8cm}
%%
%% Depth: 2147483647
%%
\newrgbcolor{mycolor0}{1.00 0.50 0.31}\definecolor{mycolor0}{rgb}{1.00,0.50,0.31}
%%
%% Depth: 100
%%
\psset{linestyle=solid,linewidth=0.03175,linecolor=mycolor0}
\qdisk(16.95,8.66){0.1}
\qdisk(18.43,11.23){0.1}
\qdisk(19.26,13.47){0.1}
\qdisk(14.59,6.74){0.1}
\qdisk(18.55,10.19){0.1}
\qdisk(14.48,6.49){0.1}
\qdisk(12.46,7.19){0.1}
\qdisk(14.75,7.06){0.1}
\qdisk(12.80,6.75){0.1}
\qdisk(10.41,6.21){0.1}
\qdisk(18.44,10.17){0.1}
\qdisk(17.23,8.05){0.1}
\qdisk(14.85,7.38){0.1}
\qdisk(13.93,7.62){0.1}
\qdisk(13.42,7.14){0.1}
\qdisk(13.43,6.77){0.1}
\qdisk(17.02,7.22){0.1}
\qdisk(17.17,8.48){0.1}
\qdisk(18.27,9.59){0.1}
\qdisk(14.32,6.61){0.1}
\qdisk(11.13,6.01){0.1}
\qdisk(18.63,9.95){0.1}
\qdisk(11.06,6.06){0.1}
\qdisk(15.23,6.62){0.1}
\qdisk(17.89,9.32){0.1}
\qdisk(18.27,9.78){0.1}
\qdisk(18.73,10.58){0.1}
\qdisk(19.54,15.26){0.1}
\qdisk(12.54,6.34){0.1}
\qdisk(17.19,7.91){0.1}
\qdisk(15.95,7.10){0.1}
\qdisk(18.41,9.80){0.1}
\qdisk(16.70,7.53){0.1}
\qdisk(16.80,8.43){0.1}
\qdisk(10.99,7.07){0.1}
\qdisk(11.26,7.47){0.1}
\qdisk(16.32,7.47){0.1}
\qdisk(12.58,6.68){0.1}
\qdisk(15.02,6.57){0.1}
\qdisk(11.85,7.13){0.1}
\qdisk(18.32,11.28){0.1}
\qdisk(11.07,7.25){0.1}
\qdisk(12.12,6.78){0.1}
\qdisk(18.07,9.43){0.1}
\qdisk(14.07,6.73){0.1}
\qdisk(15.00,7.07){0.1}
\qdisk(18.52,10.64){0.1}
\qdisk(16.90,7.94){0.1}
\qdisk(17.26,8.80){0.1}
\qdisk(10.15,6.25){0.1}
\qdisk(14.05,6.85){0.1}
\qdisk(18.30,10.67){0.1}
\qdisk(11.06,7.26){0.1}
\qdisk(18.12,8.97){0.1}
\qdisk(16.10,7.24){0.1}
\qdisk(10.17,5.85){0.1}
\qdisk(15.01,6.49){0.1}
\qdisk(12.28,6.77){0.1}
\qdisk(11.91,6.84){0.1}
\qdisk(12.05,6.44){0.1}
\qdisk(16.09,7.07){0.1}
\qdisk(15.98,6.97){0.1}
\qdisk(19.55,15.03){0.1}
\qdisk(14.89,6.85){0.1}
\qdisk(16.54,7.48){0.1}
\qdisk(15.80,7.49){0.1}
\qdisk(11.00,6.37){0.1}
\qdisk(14.16,6.58){0.1}
\qdisk(15.05,7.74){0.1}
\qdisk(15.52,7.71){0.1}
\qdisk(16.90,7.84){0.1}
\qdisk(17.52,8.51){0.1}
\qdisk(11.00,6.20){0.1}
\qdisk(12.15,6.05){0.1}
\qdisk(13.90,6.49){0.1}
\qdisk(18.07,9.70){0.1}
\qdisk(11.68,6.94){0.1}
\qdisk(13.01,6.99){0.1}
\qdisk(17.27,8.46){0.1}
\qdisk(17.70,9.07){0.1}
\qdisk(11.27,7.33){0.1}
\qdisk(14.84,7.40){0.1}
\qdisk(13.69,6.25){0.1}
\qdisk(14.07,7.73){0.1}
\qdisk(11.16,6.96){0.1}
\qdisk(15.88,7.00){0.1}
\qdisk(18.83,11.16){0.1}
\qdisk(14.09,6.52){0.1}
\qdisk(11.60,6.04){0.1}
\qdisk(14.04,6.19){0.1}
\qdisk(14.74,6.96){0.1}
\qdisk(10.70,6.17){0.1}
\qdisk(18.75,11.05){0.1}
\qdisk(15.46,6.32){0.1}
\qdisk(13.99,6.06){0.1}
\qdisk(10.73,6.62){0.1}
\qdisk(15.85,7.01){0.1}
\qdisk(11.50,6.76){0.1}
\qdisk(11.36,7.04){0.1}
\qdisk(18.17,9.43){0.1}
\rput(14.85,16.12){Relación exponencial}
\rput[l](14.71,4.63){$X$}
\rput[l]{90}(9.35,10.42){$Y$}
\psset{linecolor=black,fillstyle=none}
\psline(9.77,5.48)(19.93,5.48)(19.93,15.64)(9.77,15.64)(9.77,5.48)
\end{pspicture}
%% End
}}\qquad
\subfigure[Relación logarítmica.]{\scalebox{0.5}{%% Input file name: diagrama_dispersion_relacion_logaritmica.fig
%% FIG version: 3.2
%% Orientation: Landscape
%% Justification: Flush Left
%% Units: Inches
%% Paper size: A4
%% Magnification: 100.0
%% Resolution: 1200ppi
%% Include the following in the preamble:
%% \usepackage{textcomp}
%% End

\begin{pspicture}(7.06cm,3.29cm)(16.36cm,13.56cm)
\psset{unit=0.8cm}
%%
%% Depth: 2147483647
%%
\newrgbcolor{mycolor0}{1.00 0.50 0.31}\definecolor{mycolor0}{rgb}{1.00,0.50,0.31}
%%
%% Depth: 100
%%
\psset{linestyle=solid,linewidth=0.03175,linecolor=mycolor0}
\qdisk(10.25,6.45){0.1}
\qdisk(17.97,14.25){0.1}
\qdisk(10.24,6.13){0.1}
\qdisk(16.74,13.98){0.1}
\qdisk(13.45,11.24){0.1}
\qdisk(11.98,10.24){0.1}
\qdisk(14.03,12.49){0.1}
\qdisk(16.30,12.92){0.1}
\qdisk(12.87,11.39){0.1}
\qdisk(17.66,13.86){0.1}
\qdisk(11.35,9.63){0.1}
\qdisk(16.71,13.56){0.1}
\qdisk(19.53,14.55){0.1}
\qdisk(15.06,12.76){0.1}
\qdisk(15.65,13.66){0.1}
\qdisk(13.18,11.82){0.1}
\qdisk(13.05,11.37){0.1}
\qdisk(16.89,13.60){0.1}
\qdisk(17.06,13.28){0.1}
\qdisk(13.15,11.19){0.1}
\qdisk(15.50,12.48){0.1}
\qdisk(16.41,12.87){0.1}
\qdisk(11.42,9.02){0.1}
\qdisk(15.96,13.66){0.1}
\qdisk(11.49,10.18){0.1}
\qdisk(12.35,9.70){0.1}
\qdisk(19.45,15.26){0.1}
\qdisk(17.93,14.26){0.1}
\qdisk(18.72,14.44){0.1}
\qdisk(17.62,14.22){0.1}
\qdisk(12.71,10.52){0.1}
\qdisk(17.08,14.05){0.1}
\qdisk(15.11,13.01){0.1}
\qdisk(12.50,10.64){0.1}
\qdisk(16.19,13.66){0.1}
\qdisk(15.82,13.18){0.1}
\qdisk(12.90,11.88){0.1}
\qdisk(18.47,14.95){0.1}
\qdisk(12.06,10.13){0.1}
\qdisk(14.62,12.91){0.1}
\qdisk(15.37,12.94){0.1}
\qdisk(10.84,8.18){0.1}
\qdisk(16.16,13.44){0.1}
\qdisk(11.27,8.31){0.1}
\qdisk(14.40,11.97){0.1}
\qdisk(18.34,14.56){0.1}
\qdisk(12.47,10.96){0.1}
\qdisk(16.79,13.98){0.1}
\qdisk(11.01,8.05){0.1}
\qdisk(11.88,10.62){0.1}
\qdisk(17.92,14.09){0.1}
\qdisk(17.86,13.72){0.1}
\qdisk(19.50,14.67){0.1}
\qdisk(11.65,8.98){0.1}
\qdisk(15.13,13.30){0.1}
\qdisk(11.51,8.82){0.1}
\qdisk(16.11,13.46){0.1}
\qdisk(15.11,13.21){0.1}
\qdisk(12.18,11.15){0.1}
\qdisk(10.99,9.12){0.1}
\qdisk(11.28,9.20){0.1}
\qdisk(16.69,13.79){0.1}
\qdisk(12.10,10.21){0.1}
\qdisk(12.22,10.72){0.1}
\qdisk(18.31,14.93){0.1}
\qdisk(17.86,14.36){0.1}
\qdisk(17.02,14.18){0.1}
\qdisk(15.83,13.26){0.1}
\qdisk(12.33,10.39){0.1}
\qdisk(15.52,13.68){0.1}
\qdisk(14.22,12.24){0.1}
\qdisk(14.79,12.44){0.1}
\qdisk(10.71,6.89){0.1}
\qdisk(12.82,10.84){0.1}
\qdisk(12.78,10.72){0.1}
\qdisk(13.79,11.86){0.1}
\qdisk(15.02,12.49){0.1}
\qdisk(18.93,14.52){0.1}
\qdisk(15.15,13.33){0.1}
\qdisk(16.76,14.09){0.1}
\qdisk(10.15,5.85){0.1}
\qdisk(18.14,14.45){0.1}
\qdisk(19.50,14.79){0.1}
\qdisk(19.34,14.92){0.1}
\qdisk(17.47,14.14){0.1}
\qdisk(18.13,14.28){0.1}
\qdisk(12.24,10.52){0.1}
\qdisk(18.14,14.37){0.1}
\qdisk(18.97,14.74){0.1}
\qdisk(15.14,13.44){0.1}
\qdisk(14.97,13.41){0.1}
\qdisk(18.14,14.79){0.1}
\qdisk(17.23,14.34){0.1}
\qdisk(16.82,14.06){0.1}
\qdisk(19.40,14.69){0.1}
\qdisk(19.38,14.66){0.1}
\qdisk(14.72,12.57){0.1}
\qdisk(19.41,14.63){0.1}
\qdisk(16.66,13.91){0.1}
\qdisk(19.55,15.10){0.1}
\rput(14.85,16.12){Relación logarímica}
\rput[l](14.71,4.63){$X$}
\rput[l]{90}(9.35,10.42){$Y$}
\psset{linecolor=black,fillstyle=none}
\psline(9.77,5.48)(19.93,5.48)(19.93,15.64)(9.77,15.64)(9.77,5.48)
\end{pspicture}
%% End
}}\qquad
\subfigure[Relación inversa.]{\scalebox{0.5}{%% Input file name: diagrama_dispersion_relacion_inversa.fig
%% FIG version: 3.2
%% Orientation: Landscape
%% Justification: Flush Left
%% Units: Inches
%% Paper size: A4
%% Magnification: 100.0
%% Resolution: 1200ppi
%% Include the following in the preamble:
%% \usepackage{textcomp}
%% End

\begin{pspicture}(7.06cm,3.29cm)(16.36cm,13.56cm)
\psset{unit=0.8cm}
%%
%% Depth: 2147483647
%%
\newrgbcolor{mycolor0}{1.00 0.50 0.31}\definecolor{mycolor0}{rgb}{1.00,0.50,0.31}
%%
%% Depth: 100
%%
\psset{linestyle=solid,linewidth=0.03175,linecolor=mycolor0}
\qdisk(10.72,9.92){0.1}
\qdisk(11.70,9.02){0.1}
\qdisk(12.50,8.54){0.1}
\qdisk(13.59,7.21){0.1}
\qdisk(15.91,7.19){0.1}
\qdisk(12.47,8.08){0.1}
\qdisk(18.88,6.95){0.1}
\qdisk(13.58,7.44){0.1}
\qdisk(11.98,8.31){0.1}
\qdisk(17.32,6.94){0.1}
\qdisk(18.72,6.19){0.1}
\qdisk(14.93,6.96){0.1}
\qdisk(11.56,9.00){0.1}
\qdisk(15.27,7.49){0.1}
\qdisk(15.63,7.87){0.1}
\qdisk(12.52,8.00){0.1}
\qdisk(11.70,9.08){0.1}
\qdisk(16.82,6.28){0.1}
\qdisk(19.14,5.85){0.1}
\qdisk(10.84,10.82){0.1}
\qdisk(10.92,10.52){0.1}
\qdisk(12.81,7.77){0.1}
\qdisk(12.93,7.81){0.1}
\qdisk(14.78,6.65){0.1}
\qdisk(10.24,15.26){0.1}
\qdisk(10.15,13.68){0.1}
\qdisk(14.47,6.35){0.1}
\qdisk(17.31,6.11){0.1}
\qdisk(17.63,6.60){0.1}
\qdisk(17.37,7.05){0.1}
\qdisk(14.01,7.43){0.1}
\qdisk(15.83,6.39){0.1}
\qdisk(13.08,7.62){0.1}
\qdisk(14.01,7.50){0.1}
\qdisk(17.34,6.06){0.1}
\qdisk(11.58,9.45){0.1}
\qdisk(12.12,8.35){0.1}
\qdisk(13.95,6.41){0.1}
\qdisk(15.85,6.37){0.1}
\qdisk(12.82,7.39){0.1}
\qdisk(15.74,7.45){0.1}
\qdisk(12.18,8.88){0.1}
\qdisk(14.16,7.25){0.1}
\qdisk(15.33,6.41){0.1}
\qdisk(16.55,7.40){0.1}
\qdisk(13.29,8.16){0.1}
\qdisk(13.61,7.80){0.1}
\qdisk(15.76,6.47){0.1}
\qdisk(13.60,7.92){0.1}
\qdisk(10.97,11.36){0.1}
\qdisk(11.15,9.53){0.1}
\qdisk(14.71,6.82){0.1}
\qdisk(14.41,6.13){0.1}
\qdisk(15.64,6.69){0.1}
\qdisk(14.47,7.06){0.1}
\qdisk(17.79,7.19){0.1}
\qdisk(13.02,7.99){0.1}
\qdisk(12.05,7.84){0.1}
\qdisk(16.36,6.09){0.1}
\qdisk(14.64,7.33){0.1}
\qdisk(16.29,6.80){0.1}
\qdisk(12.65,8.67){0.1}
\qdisk(12.30,8.99){0.1}
\qdisk(12.92,8.20){0.1}
\qdisk(11.42,9.83){0.1}
\qdisk(14.74,7.50){0.1}
\qdisk(11.11,9.50){0.1}
\qdisk(17.04,6.90){0.1}
\qdisk(16.72,6.65){0.1}
\qdisk(19.55,6.65){0.1}
\qdisk(18.26,6.51){0.1}
\qdisk(15.61,7.02){0.1}
\qdisk(17.54,6.60){0.1}
\qdisk(17.95,6.72){0.1}
\qdisk(10.44,10.95){0.1}
\qdisk(14.77,8.23){0.1}
\qdisk(12.74,7.76){0.1}
\qdisk(19.49,6.88){0.1}
\qdisk(17.28,6.55){0.1}
\qdisk(13.17,7.79){0.1}
\qdisk(13.52,7.44){0.1}
\qdisk(18.61,6.44){0.1}
\qdisk(13.68,7.60){0.1}
\qdisk(19.05,6.48){0.1}
\qdisk(12.66,8.42){0.1}
\qdisk(17.37,6.01){0.1}
\qdisk(16.33,6.27){0.1}
\qdisk(16.52,6.81){0.1}
\qdisk(15.16,7.02){0.1}
\qdisk(11.12,9.67){0.1}
\qdisk(14.81,7.22){0.1}
\qdisk(10.42,12.66){0.1}
\qdisk(11.34,9.55){0.1}
\qdisk(19.27,6.17){0.1}
\qdisk(11.62,9.62){0.1}
\qdisk(16.55,6.94){0.1}
\qdisk(15.32,6.90){0.1}
\qdisk(15.82,6.83){0.1}
\qdisk(17.60,6.66){0.1}
\qdisk(18.39,6.51){0.1}
\rput(14.85,16.12){Relación inversa}
\rput[l](14.71,4.63){$X$}
\rput[l]{90}(9.35,10.42){$Y$}
\psset{linecolor=black,fillstyle=none}
\psline(9.77,5.48)(19.93,5.48)(19.93,15.64)(9.77,15.64)(9.77,5.48)
\end{pspicture}
%% End
}}\\
\caption{Diagramas de dispersión correspondientes a distintos tipos de relaciones
entre variables.} \label{g:tiposrelaciones}
\end{figure}

\clearpage

El criterio que suele utilizarse para obtener la función óptima, es que la distancia de cada punto a la curva, medida en
el eje Y, sea lo menor posible.
A estas distancias se les llama \emph{residuos} o \emph{errores} en $Y$ (figura~\ref{g:residuos}).
La función que mejor se ajusta a la nube de puntos será, por tanto, aquella que hace mínima la suma de los cuadrados de
los residuos.\footnote{Se elevan al cuadrado para evitar que en la suma se compensen los residuos positivos con los
negativos.}

\begin{figure}[h!]
\centering
\scalebox{0.8}{%% Input file name: residuos_y.fig
%% FIG version: 3.2
%% Orientation: Landscape
%% Justification: Flush Left
%% Units: Inches
%% Paper size: A4
%% Magnification: 100.0
%% Resolution: 1200ppi

\begin{pspicture}(7.13cm,3.92cm)(16.36cm,13.49cm)
\psset{unit=0.8cm}
%%
%% Depth: 2147483647
%%
\newrgbcolor{mycolor0}{1.00 0.50 0.31}\definecolor{mycolor0}{rgb}{1.00,0.50,0.31}
\newrgbcolor{mycolor1}{0.28 0.46 1.00}\definecolor{mycolor1}{rgb}{0.28,0.46,1.00}
\newgray{mycolor2}{0.74}\definecolor{mycolor2}{gray}{0.74}
%%
%% Depth: 100
%%
\psset{linestyle=solid,linewidth=0.03175,linecolor=mycolor0}
\qdisk(15.82,12.35){0.1}
\qdisk(14.74,9.32){0.1}
\qdisk(16.18,10.23){0.1}
\qdisk(14.21,9.32){0.1}
\qdisk(12.06,7.20){0.1}
\qdisk(14.92,9.47){0.1}
\qdisk(14.57,8.86){0.1}
\qdisk(13.49,8.56){0.1}
\qdisk(18.50,13.11){0.1}
\qdisk(16.89,10.83){0.1}
\qdisk(12.78,7.80){0.1}
\qdisk(17.25,11.29){0.1}
\qdisk(19.22,15.98){0.1}
\qdisk(15.46,8.71){0.1}
\qdisk(15.64,10.08){0.1}
\qdisk(13.32,8.26){0.1}
\qdisk(11.35,7.05){0.1}
\qdisk(17.43,13.56){0.1}
\qdisk(13.49,7.20){0.1}
\qdisk(14.39,9.32){0.1}
\qdisk(15.10,10.08){0.1}
\qdisk(16.36,8.56){0.1}
\qdisk(13.67,8.41){0.1}
\qdisk(14.03,8.86){0.1}
\qdisk(14.57,10.08){0.1}
\qdisk(17.07,10.23){0.1}
\qdisk(14.57,7.65){0.1}
\qdisk(15.28,9.77){0.1}
\qdisk(13.85,9.62){0.1}
\qdisk(17.25,11.59){0.1}
\rput[l](14.96,5.42){$X$}
\rput[l](9,11.37){$Y$}
\psset{linecolor=black,fillstyle=none}
\psline(10.28,6.69)(19.93,6.69)(19.93,16.34)(10.28,16.34)(10.28,6.69)
\psset{linewidth=0.0635}
\psline(11.87,6.69)(19.93,14.36)
\psset{linestyle=dashed,linewidth=0.03175,linecolor=mycolor2}
\psline(17.43,6.69)(17.43,13.56)
\psline(10.28,11.98)(17.43,11.98)
\psline(10.28,13.56)(17.43,13.56)
\psset{linewidth=0.0635, linestyle=solid,linecolor=mycolor1}
\psline{<->}(17.43,11.98)(17.43,13.56)
\rput[r](10.13,11.87){$f(x_i)$}
\rput[t](17.43,6.5){$x_i$}
\rput[r](10,13.53){$y_j$}
\rput[r](16.89,12.62){$e_{ij}=y_j-f(x_i)$}
\rput[l](16.8,13.9){$(x_i,y_j)$}
\end{pspicture}
%% End
}
\caption{Residuos o errores en $Y$. El residuo correspondiente a un punto $(x_i,y_j)$ es la diferencia entre el valor
$y_j$ observado en la muestra, y el valor teórico del modelo $f(x_i)$, es decir, $e_{ij}=y_j-f(x_i)$.}
\label{g:residuos}
\end{figure}

\subsubsection{Rectas de regresión}
En el caso de que la nube de puntos tenga forma lineal y optemos por explicar la relación entre $X$ e $Y$ mediante una
recta $y=a+bx$, los parámetros a determinar son $a$ (punto de corte con el eje de ordenadas) y $b$ (pendiente de la
recta).
Los valores de estos parámetros que hacen mínima la suma de residuos al cuadrado, determinan la recta óptima. Esta recta
se conoce como \emph{recta de regresión de $Y$ sobre $X$} y explica la variable $Y$ en función de la variable
$X$. Su ecuación es
\[ y= \bar{y}+\frac{s_{xy}}{s_x^2}(x-\bar{x}),\]
donde $s_{xy}$ es un estadístico llamado \emph{covarianza} que mide el grado de relación lineal, y cuya fórmula es
\[s_{xy}=\frac{1}{n}\sum_{i,j} (x_i-\bar{x}) (y_j-\bar{y}) n_{ij}.\]

\begin{ejemplo}
En la figura~\ref{g:rectas-estatura-peso} aparecen las rectas de regresión de Estatura sobre Peso y de Peso sobre Estatura del ejemplo anterior.

\begin{figure}[h!]
\centering
\scalebox{0.8}{\input{regresion_lineal_simple/img/rectas_regresion}}
\caption{Rectas de regresión de Estatura sobre Peso y de Peso sobre Estatura. 
Las rectas de regresión siempre se cortan en el punto de medias $(\bar x, \bar y)$}
\label{g:rectas-estatura-peso}
\end{figure}
\end{ejemplo}

La pendiente de la recta de regresión de $Y$ sobre $X$ se conoce como \emph{coeficiente de regresión de $Y$ sobre $X$},
y mide el incremento que sufrirá la variable $Y$ por cada unidad que se incremente la variable $X$, según la recta.

Cuanto más pequeños sean los residuos, en valor absoluto, mejor se ajustará el modelo a la nube de puntos, y por tanto,
mejor explicará la relación entre $X$ e $Y$.
Cuando todos los residuos son nulos, la recta pasa por todos los puntos de la nube, y la relación es perfecta.
En este caso ambas rectas, la de $Y$ sobre $X$ y la de $X$ sobre $Y$ coinciden (figura~\ref{g:dependenciafuncional}).

Por contra, cuando no existe relación lineal entre las variables, la recta de regresión de $Y$ sobre $X$ tiene pendiente
nula, y por tanto la ecuación es $y=\bar y$, en la que, efectivamente no aparece $x$, o $x=\bar x$ en el caso de la
recta de regresión $X$ sobre $Y$, de manera que ambas rectas se cortan perpendicularmente
(figura~\ref{g:independencialineal}).

\begin{figure}[htbp]
\centering 
\subfigure[Dependencia funcional lineal.] {\label{g:dependenciafuncional}
\scalebox{0.7}{%% Input file name: rectas_dependencia_lineal_perfecta.fig
%% FIG version: 3.2
%% Orientation: Landscape
%% Justification: Flush Left
%% Units: Inches
%% Paper size: A4
%% Magnification: 100.0
%% Resolution: 1200ppi
%% Include the following in the preamble:
%% \usepackage{textcomp}
%% End

\begin{pspicture}(7.06cm,3.29cm)(16.36cm,13.56cm)
\psset{unit=0.8cm}
%%
%% Depth: 2147483647
%%
\newrgbcolor{mycolor0}{1.00 0.50 0.31}\definecolor{mycolor0}{rgb}{1.00,0.50,0.31}
%%
%% Depth: 100
%%
\psset{linestyle=solid,linewidth=0.03175,linecolor=mycolor0}
\qdisk(11.95,8.54){0.1}
\qdisk(11.84,8.46){0.1}
\qdisk(19.55,13.83){0.1}
\qdisk(16.48,11.69){0.1}
\qdisk(17.77,12.59){0.1}
\qdisk(12.99,9.26){0.1}
\qdisk(15.08,10.72){0.1}
\qdisk(14.45,10.28){0.1}
\qdisk(17.63,12.49){0.1}
\qdisk(19.36,13.70){0.1}
\qdisk(12.36,8.82){0.1}
\qdisk(14.07,10.02){0.1}
\qdisk(14.87,10.57){0.1}
\qdisk(16.37,11.62){0.1}
\qdisk(10.49,7.52){0.1}
\qdisk(19.23,13.61){0.1}
\qdisk(13.17,9.39){0.1}
\qdisk(16.36,11.61){0.1}
\qdisk(18.39,13.03){0.1}
\qdisk(18.61,13.18){0.1}
\qdisk(18.99,13.44){0.1}
\qdisk(15.54,11.04){0.1}
\qdisk(19.25,13.62){0.1}
\qdisk(17.47,12.38){0.1}
\qdisk(13.14,9.37){0.1}
\qdisk(18.57,13.15){0.1}
\qdisk(16.59,11.77){0.1}
\qdisk(18.99,13.44){0.1}
\qdisk(19.05,13.48){0.1}
\qdisk(19.34,13.68){0.1}
\qdisk(17.73,12.56){0.1}
\qdisk(12.21,8.72){0.1}
\qdisk(19.09,13.51){0.1}
\qdisk(19.53,13.82){0.1}
\qdisk(10.50,7.53){0.1}
\qdisk(13.27,9.46){0.1}
\qdisk(12.90,9.20){0.1}
\qdisk(10.18,7.30){0.1}
\qdisk(10.15,7.28){0.1}
\qdisk(14.76,10.49){0.1}
\qdisk(15.92,11.31){0.1}
\qdisk(17.66,12.52){0.1}
\qdisk(18.46,13.07){0.1}
\qdisk(14.35,10.21){0.1}
\qdisk(13.63,9.71){0.1}
\qdisk(10.32,7.41){0.1}
\qdisk(19.36,13.70){0.1}
\qdisk(14.73,10.48){0.1}
\qdisk(12.94,9.23){0.1}
\qdisk(13.98,9.95){0.1}
\qdisk(19.55,13.83){0.1}
\qdisk(14.46,10.29){0.1}
\qdisk(14.41,10.25){0.1}
\qdisk(17.91,12.69){0.1}
\qdisk(14.87,10.57){0.1}
\qdisk(12.97,9.25){0.1}
\qdisk(14.48,10.30){0.1}
\qdisk(15.61,11.08){0.1}
\qdisk(18.44,13.06){0.1}
\qdisk(13.92,9.91){0.1}
\qdisk(10.90,7.81){0.1}
\qdisk(16.54,11.73){0.1}
\qdisk(10.82,7.75){0.1}
\qdisk(11.73,8.38){0.1}
\qdisk(12.02,8.59){0.1}
\qdisk(13.27,9.46){0.1}
\qdisk(13.20,9.41){0.1}
\qdisk(17.86,12.65){0.1}
\qdisk(17.05,12.09){0.1}
\qdisk(14.88,10.58){0.1}
\qdisk(16.77,11.89){0.1}
\qdisk(14.26,10.15){0.1}
\qdisk(11.68,8.35){0.1}
\qdisk(13.77,9.81){0.1}
\qdisk(10.45,7.49){0.1}
\qdisk(16.97,12.03){0.1}
\qdisk(11.75,8.40){0.1}
\qdisk(17.22,12.21){0.1}
\qdisk(13.91,9.91){0.1}
\qdisk(18.12,12.84){0.1}
\qdisk(19.39,13.72){0.1}
\qdisk(14.97,10.64){0.1}
\qdisk(12.90,9.20){0.1}
\qdisk(13.09,9.33){0.1}
\qdisk(11.98,8.56){0.1}
\qdisk(14.01,9.98){0.1}
\qdisk(15.78,11.21){0.1}
\qdisk(15.26,10.85){0.1}
\qdisk(14.42,10.26){0.1}
\qdisk(11.24,8.05){0.1}
\qdisk(15.58,11.07){0.1}
\qdisk(11.99,8.57){0.1}
\qdisk(10.24,7.35){0.1}
\qdisk(16.92,12.00){0.1}
\qdisk(17.51,12.41){0.1}
\qdisk(15.24,10.83){0.1}
\qdisk(11.73,8.38){0.1}
\qdisk(15.33,10.89){0.1}
\qdisk(14.29,10.17){0.1}
\qdisk(11.55,8.26){0.1}
\rput(14.85,16.12){Relación lineal perfecta}
\rput[l](14.71,4.63){$X$}
\rput[l]{90}(9.35,10.42){$Y$}
\psset{linecolor=black,fillstyle=none}
\psline(9.77,5.48)(19.93,5.48)(19.93,15.64)(9.77,15.64)(9.77,5.48)
\psset{linewidth=0.0635}
\psline(9.77,7.02)(19.93,14.10)
\rput[l](14.93,9.49){$X$ sobre $Y$ $=$ $Y$ sobre $X$}
\end{pspicture}
%% End
}}\qquad
\subfigure[Independencia lineal.]{\label{g:independencialineal}
\scalebox{0.7}{%% Input file name: rectas_independencia_lineal.fig
%% FIG version: 3.2
%% Orientation: Landscape
%% Justification: Flush Left
%% Units: Inches
%% Paper size: A4
%% Magnification: 100.0
%% Resolution: 1200ppi
%% Include the following in the preamble:
%% \usepackage{textcomp}
%% End

\begin{pspicture}(7.06cm,3.29cm)(16.36cm,13.56cm)
\psset{unit=0.8cm}
%%
%% Depth: 2147483647
%%
\newrgbcolor{mycolor0}{1.00 0.50 0.31}\definecolor{mycolor0}{rgb}{1.00,0.50,0.31}
%%
%% Depth: 100
%%
\psset{linestyle=solid,linewidth=0.03175,linecolor=mycolor0}
\qdisk(16.63,9.96){0.1}
\qdisk(18.53,10.92){0.1}
\qdisk(13.60,12.64){0.1}
\qdisk(12.83,13.91){0.1}
\qdisk(16.21,6.16){0.1}
\qdisk(16.04,9.07){0.1}
\qdisk(17.89,6.74){0.1}
\qdisk(13.39,11.97){0.1}
\qdisk(13.66,9.79){0.1}
\qdisk(18.77,11.43){0.1}
\qdisk(18.79,10.20){0.1}
\qdisk(15.90,13.76){0.1}
\qdisk(16.33,11.13){0.1}
\qdisk(18.47,9.84){0.1}
\qdisk(11.42,6.20){0.1}
\qdisk(10.15,9.35){0.1}
\qdisk(15.48,12.85){0.1}
\qdisk(12.40,9.99){0.1}
\qdisk(13.81,13.59){0.1}
\qdisk(17.18,6.11){0.1}
\qdisk(17.94,8.41){0.1}
\qdisk(17.76,13.78){0.1}
\qdisk(13.97,8.58){0.1}
\qdisk(18.60,5.92){0.1}
\qdisk(10.92,14.58){0.1}
\qdisk(16.44,12.66){0.1}
\qdisk(18.74,14.83){0.1}
\qdisk(12.98,11.21){0.1}
\qdisk(10.70,8.89){0.1}
\qdisk(10.33,7.27){0.1}
\qdisk(17.96,11.10){0.1}
\qdisk(12.36,15.04){0.1}
\qdisk(17.98,13.34){0.1}
\qdisk(19.55,8.17){0.1}
\qdisk(10.25,6.88){0.1}
\qdisk(12.06,15.26){0.1}
\qdisk(15.21,8.56){0.1}
\qdisk(11.72,8.30){0.1}
\qdisk(12.25,10.12){0.1}
\qdisk(12.58,12.44){0.1}
\qdisk(11.91,12.42){0.1}
\qdisk(14.86,11.21){0.1}
\qdisk(13.75,13.21){0.1}
\qdisk(15.09,15.10){0.1}
\qdisk(12.35,7.22){0.1}
\qdisk(10.73,9.81){0.1}
\qdisk(10.43,14.15){0.1}
\qdisk(11.32,8.59){0.1}
\qdisk(16.15,12.81){0.1}
\qdisk(19.51,7.91){0.1}
\qdisk(11.44,13.64){0.1}
\qdisk(11.21,12.44){0.1}
\qdisk(18.86,6.46){0.1}
\qdisk(10.31,10.08){0.1}
\qdisk(14.02,13.31){0.1}
\qdisk(17.26,8.80){0.1}
\qdisk(13.32,8.87){0.1}
\qdisk(11.63,6.47){0.1}
\qdisk(18.64,9.07){0.1}
\qdisk(11.00,7.62){0.1}
\qdisk(11.52,5.85){0.1}
\qdisk(15.30,7.81){0.1}
\qdisk(10.55,6.39){0.1}
\qdisk(17.38,12.21){0.1}
\qdisk(15.61,13.11){0.1}
\qdisk(10.75,12.83){0.1}
\qdisk(12.50,8.51){0.1}
\qdisk(17.62,13.47){0.1}
\qdisk(12.53,11.83){0.1}
\qdisk(12.42,6.73){0.1}
\qdisk(16.71,12.34){0.1}
\qdisk(13.96,13.40){0.1}
\qdisk(12.70,14.84){0.1}
\qdisk(18.84,8.00){0.1}
\qdisk(10.69,9.75){0.1}
\qdisk(14.34,8.62){0.1}
\qdisk(18.00,11.21){0.1}
\qdisk(14.23,6.39){0.1}
\qdisk(11.13,8.85){0.1}
\qdisk(16.74,7.02){0.1}
\qdisk(13.69,13.22){0.1}
\qdisk(11.43,10.96){0.1}
\qdisk(14.30,6.05){0.1}
\qdisk(18.77,13.78){0.1}
\qdisk(12.15,6.08){0.1}
\qdisk(10.41,12.01){0.1}
\qdisk(16.28,10.93){0.1}
\qdisk(19.51,12.10){0.1}
\qdisk(10.27,7.83){0.1}
\qdisk(17.47,8.75){0.1}
\qdisk(15.26,9.19){0.1}
\qdisk(17.82,6.24){0.1}
\qdisk(16.99,15.21){0.1}
\qdisk(10.35,7.39){0.1}
\qdisk(15.29,8.17){0.1}
\qdisk(19.06,6.05){0.1}
\qdisk(17.96,8.48){0.1}
\qdisk(15.95,8.06){0.1}
\qdisk(15.13,6.57){0.1}
\qdisk(14.13,10.01){0.1}
\rput(14.85,16.12){Sin relación lineal}
\rput[l](14.71,4.63){$X$}
\rput[l]{90}(9,10.42){$Y$}
\psset{linecolor=black,fillstyle=none}
\psline(9.77,5.48)(19.93,5.48)(19.93,15.64)(9.77,15.64)(9.77,5.48)
\psset{linewidth=0.0635}
\psline(9.77,10.16)(19.93,10.16)
\rput[r](9.5,10.16){$\bar y$}
\psline(14.57,5.48)(14.57,15.64)
\rput[t](14.57,5.2){$\bar x$}
\rput[l](14.72,6.85){$X$ sobre $Y$}
\rput[l](17.82,9.57){$Y$ sobre $X$}
\end{pspicture}
%% End
}}
\caption{Distintos grados de dependencia. En el primer caso, la relación es perfecta
y los residuos son nulos. En el segundo caso no existe relación lineal y la
pendiente de la recta es nula.}
\end{figure}


\subsection{Correlación}
El principal objetivo de la regresión simple es construir un modelo funcional $y=f(x)$ que explique lo mejor posible la
relación entre dos variables $X$ (variable independiente) e $Y$ (variable dependiente) medidas en una misma muestra.
Generalmente, el modelo construido se utiliza para realizar inferencias predictivas de $Y$ en función de $X$ en el resto
de la población.
Pero aunque la regresión garantiza que el modelo construido es el mejor posible, dentro del tipo de modelo elegido
(lineal, polinómico, exponencial, logarítmico, etc.), puede que aún así, no sea un buen modelo para hacer predicciones,
precisamente porque no haya relación de ese tipo entre $X$ e $Y$.
Así pues, con el fin de validar un modelo para realizar predicciones fiables, se necesitan medidas que nos hablen del
grado de dependencia entre $X$ e $Y$, con respecto a un modelo de regresión construido. Estas medidas se conocen como
medidas de \emph{correlación}.

Dependiendo del tipo de modelo ajustado, habrá distintos tipos de medidas de correlación.
Así, si el modelo de regresión construido es una recta, hablaremos de correlación lineal; si es un polinomio, hablaremos
de correlación polinómica; si es una función exponencial, hablaremos de correlación exponencial, etc.
En cualquier caso, estas medidas nos hablarán de lo bueno que es el modelo construido, y como consecuencia, de si
podemos fiarnos de las predicciones realizadas con dicho modelo.

La mayoría de las medidas de correlación surgen del estudio de los residuos o errores en $Y$, que son las distancias de
los puntos del diagrama de dispersión a la curva de regresión construida, medidas en el eje $Y$, tal y como se muestra
en la figura ~(\ref{g:residuos}).
Estas distancias, son en realidad, los errores predictivos del modelo sobre los propios valores de la muestra.

Cuanto más pequeños sean los residuos, mejor se ajustará el modelo a la nube de puntos, y por tanto, mejor explicará la
relación entre $X$ e $Y$.
Cuando todos los residuos son nulos, la curva de regresión pasa por todos los puntos de la nube, y entonces se dice que
la relación es perfecta, o bien que existe una dependencia funcional entre $X$ e $Y$
(figura~\ref{g:dependenciafuncional}).
Por contra, cuando los residuos sean grandes, el modelo no explicará bien la relación entre $X$ e $Y$, y por tanto, sus
predicciones no serán fiables (figura~\ref{g:independencialineal}).


\subsubsection{Varianza residual}
Una primera medida de correlación, construida a partir de los residuos es la \emph{varianza residual}, que se define
como el promedio de los residuos al cuadrado:
\[
s^2_{ry}=\frac{\sum_{i,j} e_{ij}^2 n_{ij}}{n}= \frac{\sum_{i,j} (y_j-f(x_i))^2
n_{ij}}{n}.
\]

Cuando los residuos son nulos, entonces $s^2_{ry}=0$ y eso indica que hay dependencia funcional.
Por otro lado, cuando las variables son independientes, con respecto al modelo de regresión ajustado, entonces los
residuos se convierten en las desviaciones de los valores de $Y$ con respecto a su media, y se cumple que
$s^2_{ry}=s_y^2$.
Así pues, se cumple que \[  0 \leq s^2_{ry}\leq s_y^2. \] Según esto, cuanto menor sea la varianza residual, mayor será
la dependencia entre $X$ e $Y$, de acuerdo al modelo ajustado.
No obstante, la varianza tiene como unidades las unidades de $Y$ al cuadrado, y eso dificulta su interpretación.


\subsubsection{Coeficiente de determinación}
Puesto que el valor máximo que puede tomar la varianza residual es la varianza de $Y$, se puede definir fácilmente un
coeficiente a partir de la comparación de ambas medidas. 
Surge así el \emph{coeficiente de determinación} que se define como
\[
R^2=1-\frac{s^2_{ry}}{s_y^2}.
\]

Se cumple que
\[ 0\leq R^2\leq 1,\]
y además no tiene unidades, por lo que es más fácil de interpretar que la varianza residual:
\begin{itemize}
\item $R^2=0$ indica que existe independencia según el tipo de relación planteada por el modelo de regresión.
\item $R^2=1$ indica dependencia funcional.
\end{itemize}
Por tanto, cuanto mayor sea $R^2$, mejor será el modelo de regresión.

Si multiplicamos el coeficiente de determinación por 100, se obtiene el porcentaje de variabilidad de $Y$ que explica el
modelo de regresión.
El porcentaje restante corresponde a la variabilidad que queda por explicar y se corresponde con el error predictivo del
modelo.
Así, por ejemplo, si tenemos un coeficiente de determinación $R^2=0.5$, el modelo de regresión explicaría la mitad de la
variabilidad de $Y$, y en consecuencia, si se utiliza dicho modelo para hacer predicciones, estas tendrían la mitad de
error que si no se utilizase, y se tomase como valor de la predicción el valor de la media de $Y$.

\subsubsection{Coeficiente de determinación lineal}
En el caso de que el modelo de regresión sea lineal, la fórmula del coeficiente de determinación se simplifica y se
convierte en
\[
r^2=\frac{s_{xy}^2}{s_x^2 s_y^2},
\]
que se conoce como \emph{coeficiente de determinación lineal}.

\subsubsection{Coeficiente de correlación}
Otra medida de dependencia bastante habitual es el \emph{coeficiente de correlación}, que se define como la raíz
cuadrada del coeficiente de determinación:
\[
R=\pm\sqrt{1-\frac{s^2_{ry}}{s_y^2}},
\]
tomando la raíz del mismo signo que la covarianza.

La única ventaja del coeficiente de correlación con respecto al coeficiente de determinación, es que tiene signo, y por
tanto, además del grado de dependencia entre $X$ e $Y$, también nos habla de si la relación es directa (signo +) o
inversa (signo -). 
Su interpretación es:
\begin{itemize}
\item $R=0$ indica independencia con respecto al tipo de relación planteada por el modelo de
regresión.
\item $R=-1$ indica dependencia funcional inversa.
\item $R=1$ indica dependencia funcional directa.
\end{itemize}
Por consiguiente, cuanto más próximo esté a -1 o a 1, mejor será el modelo de regresión.


\subsubsubsection{Coeficiente de correlación lineal}
Al igual que ocurría con el coeficiente de determinación, cuando el modelo de regresión es lineal, la fórmula del
coeficiente de correlación se convierte en
\[
r=\frac{s_{xy}}{s_x s_y},
\]
y se llama \emph{coeficiente de correlación lineal}.

Por último, conviene remarcar que un coeficiente de determinación o de correlación nulo, indica que hay independencia
según el modelo de regresión construido, pero puede haber dependencia de otro tipo.
Esto se ve claramente en el ejemplo de  la figura~\ref{g:dependenciaparabolica}.

\begin{figure}[h!]
\centering \subfigure[Dependencia lineal débil.]
{\label{g:dependencialinealdebil}
\scalebox{0.7}{%% Input file name: recta_regresion_relacion_parabolica.fig
%% FIG version: 3.2
%% Orientation: Landscape
%% Justification: Flush Left
%% Units: Inches
%% Paper size: A4
%% Magnification: 100.0
%% Resolution: 1200ppi

\begin{pspicture}(7.40cm,3.92cm)(16.36cm,13.22cm)
\psset{unit=0.8cm}
%%
%% Depth: 2147483647
%%
\newrgbcolor{mycolor0}{1.00 0.50 0.31}\definecolor{mycolor0}{rgb}{1.00,0.50,0.31}
%%
%% Depth: 100
%%
\psset{linestyle=solid,linewidth=0.03175,linecolor=mycolor0}
\qdisk(16.93,9.07){0.1}
\qdisk(17.69,9.82){0.1}
\qdisk(19.49,14.73){0.1}
\qdisk(17.50,9.41){0.1}
\qdisk(13.34,10.56){0.1}
\qdisk(15.42,7.80){0.1}
\qdisk(15.70,7.66){0.1}
\qdisk(13.21,10.16){0.1}
\qdisk(11.75,13.95){0.1}
\qdisk(15.70,8.06){0.1}
\qdisk(14.62,8.46){0.1}
\qdisk(18.89,12.73){0.1}
\qdisk(17.59,9.99){0.1}
\qdisk(18.59,12.78){0.1}
\qdisk(13.40,10.23){0.1}
\qdisk(19.50,14.99){0.1}
\qdisk(16.42,8.05){0.1}
\qdisk(18.44,12.02){0.1}
\qdisk(11.97,13.38){0.1}
\qdisk(11.39,15.36){0.1}
\qdisk(17.05,9.28){0.1}
\qdisk(19.03,13.17){0.1}
\qdisk(14.05,9.22){0.1}
\qdisk(16.24,8.28){0.1}
\qdisk(18.86,13.54){0.1}
\qdisk(12.61,11.22){0.1}
\qdisk(18.07,11.47){0.1}
\qdisk(16.86,8.97){0.1}
\qdisk(17.74,10.03){0.1}
\qdisk(11.83,14.36){0.1}
\qdisk(11.37,15.67){0.1}
\qdisk(18.14,11.14){0.1}
\qdisk(18.43,12.09){0.1}
\qdisk(14.08,8.49){0.1}
\qdisk(18.28,11.60){0.1}
\qdisk(17.24,9.89){0.1}
\qdisk(11.81,14.46){0.1}
\qdisk(16.52,8.96){0.1}
\qdisk(14.97,8.02){0.1}
\qdisk(16.01,8.14){0.1}
\qdisk(13.73,9.25){0.1}
\qdisk(19.15,13.49){0.1}
\qdisk(15.11,8.21){0.1}
\qdisk(18.50,12.14){0.1}
\qdisk(15.68,7.94){0.1}
\qdisk(13.78,9.47){0.1}
\qdisk(12.23,13.13){0.1}
\qdisk(18.33,11.13){0.1}
\qdisk(12.08,13.20){0.1}
\qdisk(15.95,7.99){0.1}
\qdisk(18.72,12.91){0.1}
\qdisk(14.88,7.79){0.1}
\qdisk(13.25,9.80){0.1}
\qdisk(15.46,7.44){0.1}
\qdisk(12.10,13.13){0.1}
\qdisk(16.51,8.26){0.1}
\qdisk(17.46,9.96){0.1}
\qdisk(16.61,8.56){0.1}
\qdisk(16.79,8.51){0.1}
\qdisk(13.94,9.33){0.1}
\qdisk(17.27,9.33){0.1}
\qdisk(19.60,15.65){0.1}
\qdisk(12.74,11.14){0.1}
\qdisk(12.02,14.59){0.1}
\qdisk(11.74,15.21){0.1}
\qdisk(17.22,8.44){0.1}
\qdisk(19.46,15.33){0.1}
\qdisk(14.31,8.07){0.1}
\qdisk(11.73,14.29){0.1}
\qdisk(14.67,8.20){0.1}
\qdisk(16.41,8.52){0.1}
\qdisk(16.33,8.46){0.1}
\qdisk(18.29,10.67){0.1}
\qdisk(17.68,9.92){0.1}
\qdisk(18.88,13.36){0.1}
\qdisk(18.65,12.88){0.1}
\qdisk(16.94,9.35){0.1}
\qdisk(15.91,7.83){0.1}
\qdisk(14.25,8.58){0.1}
\qdisk(17.19,9.13){0.1}
\qdisk(12.42,11.66){0.1}
\qdisk(18.44,12.02){0.1}
\qdisk(13.71,9.47){0.1}
\qdisk(18.68,12.62){0.1}
\qdisk(14.16,8.37){0.1}
\qdisk(13.42,9.68){0.1}
\qdisk(16.46,8.68){0.1}
\qdisk(15.98,8.04){0.1}
\qdisk(14.72,8.59){0.1}
\qdisk(17.51,9.78){0.1}
\qdisk(16.88,9.04){0.1}
\qdisk(11.47,15.26){0.1}
\qdisk(14.89,8.11){0.1}
\qdisk(17.65,10.07){0.1}
\qdisk(18.39,11.22){0.1}
\qdisk(13.64,10.12){0.1}
\qdisk(12.73,11.24){0.1}
\qdisk(19.12,14.52){0.1}
\qdisk(13.01,10.44){0.1}
\qdisk(18.67,13.05){0.1}
\psset{linecolor=black,fillstyle=none}
\psline(11.35,7.11)(19.65,7.11)
\psline(11.35,7.11)(11.35,6.90)
\psline(13.01,7.11)(13.01,6.90)
\psline(14.67,7.11)(14.67,6.90)
\psline(16.33,7.11)(16.33,6.90)
\psline(17.99,7.11)(17.99,6.90)
\psline(19.65,7.11)(19.65,6.90)
\rput(11.35,6.35){0}
\rput(13.01,6.35){2}
\rput(14.67,6.35){4}
\rput(16.33,6.35){6}
\rput(17.99,6.35){8}
\rput(19.65,6.35){10}
\psline(11.04,8.19)(11.04,15.72)
\psline(11.04,8.19)(10.83,8.19)
\psline(11.04,9.44)(10.83,9.44)
\psline(11.04,10.70)(10.83,10.70)
\psline(11.04,11.95)(10.83,11.95)
\psline(11.04,13.21)(10.83,13.21)
\psline(11.04,14.46)(10.83,14.46)
\psline(11.04,15.72)(10.83,15.72)
\rput{90}(10.53,8.19){2}
\rput{90}(10.53,9.44){3}
\rput{90}(10.53,10.70){4}
\rput{90}(10.53,11.95){5}
\rput{90}(10.53,13.21){6}
\rput{90}(10.53,14.46){7}
\rput{90}(10.53,15.72){8}
\psline(11.04,7.11)(19.93,7.11)(19.93,16.00)(11.04,16.00)(11.04,7.11)
\rput[l](15.34,5.42){$X$}
\rput[l]{90}(9.77,11.42){$Y$}
\psset{linewidth=0.0635}
\psline(11.04,10.80)(19.93,10.61)
\rput(15.50,14.38){$y= -0.02 x + 4.07$}
\rput(15.50,13.72){$r^2 = 0$}
\end{pspicture}
%% End
}}\qquad
\subfigure[Dependencia parabólica fuerte.] {\label{g:dependenciaparabolicafuerte}
\scalebox{0.7}{%% Input file name: regresion_parabolica.fig
%% FIG version: 3.2
%% Orientation: Landscape
%% Justification: Flush Left
%% Units: Inches
%% Paper size: A4
%% Magnification: 100.0
%% Resolution: 1200ppi

\begin{pspicture}(7.40cm,3.92cm)(16.36cm,13.22cm)
\psset{unit=0.8cm}
%%
%% Depth: 2147483647
%%
\newrgbcolor{mycolor0}{1.00 0.50 0.31}\definecolor{mycolor0}{rgb}{1.00,0.50,0.31}
%%
%% Depth: 100
%%
\psset{linestyle=solid,linewidth=0.03175,linecolor=mycolor0}
\qdisk(16.93,9.07){0.1}
\qdisk(17.69,9.82){0.1}
\qdisk(19.49,14.73){0.1}
\qdisk(17.50,9.41){0.1}
\qdisk(13.34,10.56){0.1}
\qdisk(15.42,7.80){0.1}
\qdisk(15.70,7.66){0.1}
\qdisk(13.21,10.16){0.1}
\qdisk(11.75,13.95){0.1}
\qdisk(15.70,8.06){0.1}
\qdisk(14.62,8.46){0.1}
\qdisk(18.89,12.73){0.1}
\qdisk(17.59,9.99){0.1}
\qdisk(18.59,12.78){0.1}
\qdisk(13.40,10.23){0.1}
\qdisk(19.50,14.99){0.1}
\qdisk(16.42,8.05){0.1}
\qdisk(18.44,12.02){0.1}
\qdisk(11.97,13.38){0.1}
\qdisk(11.39,15.36){0.1}
\qdisk(17.05,9.28){0.1}
\qdisk(19.03,13.17){0.1}
\qdisk(14.05,9.22){0.1}
\qdisk(16.24,8.28){0.1}
\qdisk(18.86,13.54){0.1}
\qdisk(12.61,11.22){0.1}
\qdisk(18.07,11.47){0.1}
\qdisk(16.86,8.97){0.1}
\qdisk(17.74,10.03){0.1}
\qdisk(11.83,14.36){0.1}
\qdisk(11.37,15.67){0.1}
\qdisk(18.14,11.14){0.1}
\qdisk(18.43,12.09){0.1}
\qdisk(14.08,8.49){0.1}
\qdisk(18.28,11.60){0.1}
\qdisk(17.24,9.89){0.1}
\qdisk(11.81,14.46){0.1}
\qdisk(16.52,8.96){0.1}
\qdisk(14.97,8.02){0.1}
\qdisk(16.01,8.14){0.1}
\qdisk(13.73,9.25){0.1}
\qdisk(19.15,13.49){0.1}
\qdisk(15.11,8.21){0.1}
\qdisk(18.50,12.14){0.1}
\qdisk(15.68,7.94){0.1}
\qdisk(13.78,9.47){0.1}
\qdisk(12.23,13.13){0.1}
\qdisk(18.33,11.13){0.1}
\qdisk(12.08,13.20){0.1}
\qdisk(15.95,7.99){0.1}
\qdisk(18.72,12.91){0.1}
\qdisk(14.88,7.79){0.1}
\qdisk(13.25,9.80){0.1}
\qdisk(15.46,7.44){0.1}
\qdisk(12.10,13.13){0.1}
\qdisk(16.51,8.26){0.1}
\qdisk(17.46,9.96){0.1}
\qdisk(16.61,8.56){0.1}
\qdisk(16.79,8.51){0.1}
\qdisk(13.94,9.33){0.1}
\qdisk(17.27,9.33){0.1}
\qdisk(19.60,15.65){0.1}
\qdisk(12.74,11.14){0.1}
\qdisk(12.02,14.59){0.1}
\qdisk(11.74,15.21){0.1}
\qdisk(17.22,8.44){0.1}
\qdisk(19.46,15.33){0.1}
\qdisk(14.31,8.07){0.1}
\qdisk(11.73,14.29){0.1}
\qdisk(14.67,8.20){0.1}
\qdisk(16.41,8.52){0.1}
\qdisk(16.33,8.46){0.1}
\qdisk(18.29,10.67){0.1}
\qdisk(17.68,9.92){0.1}
\qdisk(18.88,13.36){0.1}
\qdisk(18.65,12.88){0.1}
\qdisk(16.94,9.35){0.1}
\qdisk(15.91,7.83){0.1}
\qdisk(14.25,8.58){0.1}
\qdisk(17.19,9.13){0.1}
\qdisk(12.42,11.66){0.1}
\qdisk(18.44,12.02){0.1}
\qdisk(13.71,9.47){0.1}
\qdisk(18.68,12.62){0.1}
\qdisk(14.16,8.37){0.1}
\qdisk(13.42,9.68){0.1}
\qdisk(16.46,8.68){0.1}
\qdisk(15.98,8.04){0.1}
\qdisk(14.72,8.59){0.1}
\qdisk(17.51,9.78){0.1}
\qdisk(16.88,9.04){0.1}
\qdisk(11.47,15.26){0.1}
\qdisk(14.89,8.11){0.1}
\qdisk(17.65,10.07){0.1}
\qdisk(18.39,11.22){0.1}
\qdisk(13.64,10.12){0.1}
\qdisk(12.73,11.24){0.1}
\qdisk(19.12,14.52){0.1}
\qdisk(13.01,10.44){0.1}
\qdisk(18.67,13.05){0.1}
\psset{linecolor=black,fillstyle=none}
\psline(11.35,7.11)(19.65,7.11)
\psline(11.35,7.11)(11.35,6.90)
\psline(13.01,7.11)(13.01,6.90)
\psline(14.67,7.11)(14.67,6.90)
\psline(16.33,7.11)(16.33,6.90)
\psline(17.99,7.11)(17.99,6.90)
\psline(19.65,7.11)(19.65,6.90)
\rput(11.35,6.35){0}
\rput(13.01,6.35){2}
\rput(14.67,6.35){4}
\rput(16.33,6.35){6}
\rput(17.99,6.35){8}
\rput(19.65,6.35){10}
\psline(11.04,8.19)(11.04,15.72)
\psline(11.04,8.19)(10.83,8.19)
\psline(11.04,9.44)(10.83,9.44)
\psline(11.04,10.70)(10.83,10.70)
\psline(11.04,11.95)(10.83,11.95)
\psline(11.04,13.21)(10.83,13.21)
\psline(11.04,14.46)(10.83,14.46)
\psline(11.04,15.72)(10.83,15.72)
\rput{90}(10.53,8.19){2}
\rput{90}(10.53,9.44){3}
\rput{90}(10.53,10.70){4}
\rput{90}(10.53,11.95){5}
\rput{90}(10.53,13.21){6}
\rput{90}(10.53,14.46){7}
\rput{90}(10.53,15.72){8}
\psline(11.04,7.11)(19.93,7.11)(19.93,16.00)(11.04,16.00)(11.04,7.11)
\rput[l](15.34,5.42){$X$}
\rput[l]{90}(9.77,11.42){$Y$}
\psset{linewidth=0.0635}
\psline(11.37,15.73)(11.45,15.42)(11.53,15.12)(11.62,14.82)(11.70,14.53)(11.78,14.25)(11.86,13.97)(11.94,13.70)(12.03,13.44)(12.11,13.18)(12.19,12.93)(12.27,12.69)(12.36,12.44)(12.44,12.21)(12.52,11.98)(12.60,11.76)(12.69,11.55)(12.77,11.34)(12.85,11.14)(12.93,10.94)(13.02,10.75)(13.10,10.57)(13.18,10.39)(13.26,10.22)(13.34,10.06)(13.43,9.90)(13.51,9.75)(13.59,9.60)(13.67,9.46)(13.76,9.32)(13.84,9.19)(13.92,9.07)(14.00,8.96)(14.08,8.85)(14.17,8.75)(14.25,8.65)(14.33,8.56)(14.41,8.47)(14.50,8.39)(14.58,8.32)(14.66,8.26)(14.74,8.20)(14.83,8.14)(14.91,8.10)(14.99,8.06)(15.07,8.02)(15.16,7.99)(15.24,7.97)(15.32,7.95)(15.40,7.94)(15.48,7.94)(15.57,7.94)(15.65,7.95)(15.73,7.96)(15.81,7.98)(15.90,8.01)(15.98,8.04)(16.06,8.08)(16.14,8.13)(16.22,8.18)(16.31,8.24)(16.39,8.30)(16.47,8.37)(16.55,8.45)(16.64,8.53)(16.72,8.62)(16.80,8.71)(16.88,8.82)(16.97,8.92)(17.05,9.04)(17.13,9.16)(17.21,9.28)(17.30,9.41)(17.38,9.55)(17.46,9.70)(17.54,9.85)(17.63,10.01)(17.71,10.17)(17.79,10.34)(17.87,10.51)(17.95,10.70)(18.04,10.88)(18.12,11.08)(18.20,11.28)(18.28,11.49)(18.37,11.70)(18.45,11.92)(18.53,12.14)(18.61,12.37)(18.69,12.61)(18.78,12.85)(18.86,13.10)(18.94,13.36)(19.02,13.62)(19.11,13.89)(19.19,14.16)(19.27,14.45)(19.35,14.73)(19.44,15.03)(19.52,15.32)(19.60,15.63)
\rput(15.50,14.38){$y= 0.25 x^2 -2.51 x + 8.05$}
\rput(15.50,13.72){$r^2 = 0.97$}
\end{pspicture}
%% End
}}
\caption{En la figura de la izquierda se ha ajustado un modelo lineal y se ha obtenido un $R^2=0$, lo que indica que el
modelo no explica nada de la relación entre $X$ e $Y$, pero no podemos afirmar que $X$ e $Y$ son independientes. De
hecho, en la figura de la derecha se observa que al ajustar un modelo parabólico, $R^2=0.97$, lo que indica que casi hay
una dependencia funcional parabólica entre $X$ e $Y$.}
\label{g:dependenciaparabolica}
\end{figure}


\subsubsection{Fiabilidad de las predicciones}
Aunque el coeficiente de determinación o de correlación nos hablan de la bondad de un modelo de regresión, no es el
único dato que hay que tener en cuenta a la hora de hacer predicciones.

La fiabilidad de las predicciones que hagamos con un modelo de regresión depende de varias cosas:
\begin{itemize}
\item El coeficiente de determinación: Cuando mayor sea, menores serán los errores predictivos y mayor la fiabilidad de
las predicciones.
\item La variablidad de la población: Cuanto más variable es una población, más difícil es predecir y por tanto menos
fiables serán las predicciones del modelo.
\item El tamaño muestral: Cuanto mayor sea, más información tendremos y, en consecuencia, más fiables serán las
predicciones.
\end{itemize} 

Además, hay que tener en cuenta que un modelo de regresión es válido para el rango de valores observados en la muestra,
pero fuera de ese rango no tenemos información del tipo de relación entre las variables, por lo que no deberíamos hacer
predicciones para valores que estén lejos de los observados en la muestra.

\clearpage
\newpage



\section{Ejercicios resueltos}
\begin{enumerate}[leftmargin=*]
\item Se han medido dos variables $X$ e $Y$ en 10 individuos obteniendo los siguientes resultados:
\[
\begin{array}{lrrrrrrrrrr}
\hline
X & 0 & 1 & 2 & 3 & 4 & 5 & 6 & 7 & 8 & 9 \\
Y & 2 & 5 & 8 & 11 & 14 & 17 & 20 & 23 & 26 & 29\\
\hline
\end{array}
\]

Se pide:

\begin{enumerate}
\item  Crear un conjunto de datos con las variables \variable{X} y \variable{Y} e introducir estos datos.
\item  Dibujar el diagrama de dispersión correspondiente.
\begin{indicacion}{
\begin{enumerate}
\item Seleccionar el menú \menu{Teaching>Gráficos>Diagrama de Dispersión}.
\item En el cuadro de diálogo que aparece, seleccionar la variable \variable{Y} en el campo \campo{Variable Y}, la
variable \variable{X} en el campo \campo{Variable X}, y hacer clic en el botón \boton{Enviar}.
\end{enumerate}}
\end{indicacion}

En vista del diagrama, ¿qué tipo de modelo crees que explicará mejor la relación entre  \variable{X} y \variable{Y}?

\item Calcular la recta de regresión de $Y$ sobre $X$.
\begin{indicacion}{
\begin{enumerate}
\item Seleccionar el menú \menu{Teaching>Regresión>Regresión lineal}.
\item En el cuadro de diálogo que aparece, seleccionar la variable \variable{Y} en el campo \campo{Variable
dependiente} y la variable \variable{X} en el campo \campo{Variable independiente}, y hacer clic sobre el botón
\boton{Enviar}.
\end{enumerate}}
\end{indicacion}

\item Dibujar dicha recta sobre el diagrama de dispersión.
\begin{indicacion}{
\begin{enumerate}
\item Seleccionar el menú \menu{Teaching>Gráficos>Diagrama de Dispersión}.
\item En el cuadro de diálogo que aparece, seleccionar la variable \variable{Y} en el campo \campo{Variable Y}, la
variable \variable{X} en el campo \campo{Variable X}, y hacer clic en el botón \boton{Enviar}.
\item En la solapa \menu{Línea de ajuste}, seleccionar \opcion{Dibujar recta de regresión} y hacer clic en el botón
\boton{Enviar}.
\end{enumerate}}
\end{indicacion}

\item Calcular la recta de regresión de $X$ sobre $Y$ y dibujarla sobre el correspondiente diagrama de dispersión.
\begin{indicacion}{
Repetir los pasos de los apartados anteriores pero escogiendo como \campo{Variable dependiente} la variable \variable{X},
y como \campo{Variable independiente} la variable \variable{Y}}
\end{indicacion}

\item ¿Son grandes los residuos? Comentar los resultados.
\end{enumerate}


\item  En una licenciatura se quiere estudiar la relación entre el número medio de horas de estudio diarias y el número
de asignaturas suspensas. Para ello se obtuvo la siguiente muestra:
\[
\begin{array}{cccccccc}
\text{Horas} & \text{Suspensos} &  & \text{Horas} & \text{Suspensos} & & \text{Horas} & \text{Suspensos}  \\
\cline{1-2}\cline{4-5}\cline{7-8}
3.5 & 1 & & 2.2 & 2 & & 1.3 & 4 \\
0.6 & 5 & & 3.3 & 0 & & 3.1 & 0 \\
2.8 & 1 & & 1.7 & 3 & & 2.3 & 2 \\
2.5 & 3 & & 1.1 & 3 & & 3.2 & 2 \\
2.6 & 1 & & 2.0 & 3 & & 0.9 & 4 \\
3.9 & 0 & & 3.5 & 0 & & 1.7 & 2 \\
1.5 & 3 & & 2.1 & 2 & & 0.2 & 5 \\
0.7 & 3 & & 1.8 & 2 & & 2.9 & 1 \\
3.6 & 1 & & 1.1 & 4 & & 1.0 & 3 \\
3.7 & 1 & & 0.7 & 4 & & 2.3 & 2 \\
\end{array}
\]

Se pide:
\begin{enumerate}
\item  Crear un conjunto de datos con las variables \variable{horas.estudio} y \variable{suspensos} e introducir estos
datos.

\item Construir la tabla de frecuencias bidimensional de las variables \variable{horas.estudio} y \variable{suspensos}. 
\begin{indicacion}{
\begin{enumerate}
\item Seleccionar el menú \menu{Teaching>Distribución de frecuencias>Tabla de frecuencias bidimensional}.
\item En el cuadro de diálogo que aparece, seleccionar la variable \variable{horas.estudio} en el campo \campo{Variable
a tabular en filas}, la variable \variable{suspensos} en el campo \campo{Variable a tabular en columnas}, y hacer clic
sobre el botón \boton{Enviar}. 
\end{enumerate}}
\end{indicacion}

\item  Calcular la recta de regresión de \variable{suspensos} sobre \variable{horas.estudio} y dibujarla.
\begin{indicacion}{
Para calcular la recta de regresión:
\begin{enumerate}
\item Seleccionar el menú \menu{Teaching>Regresión>Regresión lineal}.
\item En el cuadro de diálogo que aparece, seleccionar la variable \variable{suspensos} en el campo \campo{Variable
dependiente} y la variable \variable{horas.estudio} en el campo \campo{Variable independiente}, seleccionar
\opcion{Guardar el modelo}, introducir un nombre para el modelo y hacer clic sobre el botón \boton{Enviar}.
\end{enumerate}
Para dibujar la recta de regresión:
\begin{enumerate}
\item Seleccionar el menú \menu{Teaching>Gráficos>Diagrama de Dispersión}.
\item En el cuadro de diálogo que aparece, seleccionar la variable \variable{suspensos} en el campo \campo{Variable Y} y
la variable \variable{horas.estudio} en el campo \campo{Variable X}.
\item En la solapa \menu{Línea de ajuste}, seleccionar \opcion{Lineal} y hacer clic en el botón
\boton{Enviar}.
\end{enumerate}}
\end{indicacion}

\item Indicar el coeficiente de regresión de \variable{suspensos} sobre \variable{horas.estudio}. 
¿Cómo lo interpretarías?
\begin{indicacion}{
El coeficiente de regresión es la pendiente de la recta de regresión.}
\end{indicacion}

\item La relación lineal entre estas dos variables, ¿es mejor o peor que la del ejercicio anterior? 
Comentar los resultados a partir las gráficas de las rectas de regresión y sus residuos.

\item Calcular los coeficientes de correlación y de determinación lineal. 
¿Es un buen modelo la recta de regresión?
¿Qué porcentaje de la variabilidad del número de suspensos está explicada por el modelo?
\begin{indicacion}{
El coeficiente de determinación aparece en la ventana de resultados como \resultado{R$^2$ ajustado}, y el
coeficiente de correlación es su raíz cuadrada.}
\end{indicacion}

\item Utilizar la recta de regresión para predecir el número de suspensos correspondiente a 3 horas de estudio diarias.
¿Es fiable esta predicción? \begin{indicacion}{
\begin{enumerate}
\item Seleccionar el menú \menu{Teaching>Regresión>Predicciones}.
\item En el cuadro de diálogo que aparece seleccionar como modelo de regresión la recta calculada en el segundo
apartado, introducir los valores para los que se desea la predicción en el campo \campo{Predicciones para} y hacer clic
sobre el botón \boton{Enviar}.
\end{enumerate}}
\end{indicacion}

\item Según el modelo lineal, ¿cuántas horas diarias tendrá que estudiar como mínimo un alumno si quiere aprobarlo
todo?
\begin{indicacion}{
Seguir los mismos pasos de los apartados anteriores, pero escogiendo como variable dependiente \variable{horas.estudio},
y como independiente \variable{suspensos}, y haciendo la predicción para 0 suspensos.}
\end{indicacion}
\end{enumerate}


\item Después de tomar un litro de vino se ha medido la concentración de alcohol en la sangre en distintos instantes,
obteniendo:
\[
\begin{array}{lrrrrrrr}
\hline 
\mbox{Tiempo después (minutos)} & 30 & 60 & 90 & 120 & 150 & 180 & 210\\ 
\mbox{Concentración (gramos/litro)} & 1.6 & 1.7 & 1.5 & 1.1 & 0.7 & 0.2 & 2.1\\
\hline
\end{array}
\]

Se pide:
\begin{enumerate}
\item Crear las variables \variable{tiempo} y \variable{alcohol} e introducir estos datos.
\item Calcular el coeficiente de correlación lineal entre el alcohol y el tiempo e interpretarlo. ¿Es bueno el modelo
lineal? 
\begin{indicacion}{
\begin{enumerate}
\item Seleccionar el menú \menu{Teaching>Regresión>Regresión lineal}.
\item En el cuadro de diálogo que aparece, seleccionar la variable \variable{alcohol} en el campo \campo{Variable
dependiente} y la variable \variable{tiempo} en el campo \campo{Variable independiente}, y hacer clic sobre el botón
\boton{Enviar}.
\end{enumerate}}
\end{indicacion}

\item Dibujar la recta de regresión del alcohol sobre el tiempo. 
¿Existe algún individuo con un residuo demasiado grande? 
Si es así, eliminar dicho individuo de la muestra y volver a calcular el coeficiente de correlación. 
¿Ha mejorado el modelo?
\begin{indicacion}{
\begin{enumerate}
\item Seleccionar el menú \menu{Teaching>Gráficos>Diagrama de Dispersión}.
\item En el cuadro de diálogo que aparece, seleccionar la variable \variable{alcohol} en el campo \campo{Variable Y} y
la variable \variable{tiempo} en el campo \campo{Variable X}.
\item En la solapa \menu{Línea de ajuste}, seleccionar \opcion{Lineal} y hacer clic en el botón \boton{Enviar}.
\end{enumerate}
Se observa que hay un residuo atípico para el punto que corresponde al los 210 minutos. 
Para eliminarlo:
En la ventana de edición del conjunto de datos hacer clic con el botón derecho del ratón sobre la fila correspondiente
al dato con el residuo atípico y seleccionar \opcion{Borrar esta fila}.}
\end{indicacion}

\item Si la concentración máxima de alcohol en la sangre que permite la ley para poder conducir es $0.3$ g/l, ¿cuánto
tiempo habrá que esperar después de tomarse un litro de vino para poder conducir sin infringir la ley? 
¿Es fiable esta predicción?
\begin{indicacion}{
Para construir la recta de regresión:
\begin{enumerate}
\item Seleccionar el menú \menu{Teaching>Regresión>Regresión lineal}.
\item En el cuadro de diálogo que aparece, seleccionar la variable \variable{tiempo} en el campo \campo{Variable
dependiente} y la variable \variable{alcohol} en el campo \campo{Variable independiente}.
\item Seleccionar \opcion{Guardar el modelo}, introducir un nombre para el modelo y hacer clic sobre el botón \boton{Enviar}.
\end{enumerate}
Para hacer la predicción:
\begin{enumerate}
\item Seleccionar el menú \menu{Teaching>Regresión>Predicciones}.
\item En el cuadro de diálogo que aparece seleccionar como modelo de regresión la recta calculada e introducir los
valores para los que se desea la predicción en el campo \campo{Predicciones para} y hacer clic sobre el botón
\boton{Enviar}.
\end{enumerate}}
\end{indicacion}
\end{enumerate}


\item El conjunto de datos \variable{edad.estatura} del paquete \variable{rk.Teaching} contine la edad y la estatura
de 30 personas. 
Se pide:
\begin{enumerate}
\item Cargar datos del conjunto de datos \variable{edad.estatura} desde el paquete \variable{rk.Teaching}.

\item Calcular la recta de regresión de la estatura sobre la edad. ¿Es un buen modelo la recta de regresión?
\begin{indicacion}{
\begin{enumerate}
\item Seleccionar el menú \menu{Teaching>Regresión>Regresión lineal}.
\item En el cuadro de diálogo que aparece, seleccionar la variable \variable{estatura} en el campo \campo{Variable
dependiente} y la variable \variable{edad} en el campo \campo{Variable independiente}, y hacer clic en el botón
\boton{Enviar}.
\end{enumerate}}
\end{indicacion}

\item Dibujar el diagrama de dispersión de la estatura sobre la edad. 
¿Alrededor de qué edad se observa un cambio en la tendencia? 
\begin{indicacion}{
\begin{enumerate}
\item Seleccionar el menú \menu{Teaching>Gráficos>Diagrama de Dispersión}.
\item En el cuadro de diálogo que aparece, seleccionar la variable \variable{estatura} en el campo \campo{Variable Y},
la variable \variable{edad} en el campo \campo{Variable X}, y hacer clic en el botón \boton{Enviar}.
\end{enumerate}}
\end{indicacion}

\item Recodificar la variable edad en dos grupos para mayores y menores de 20 años.
\begin{indicacion}{
\begin{enumerate}
\item Seleccionar el menú \menu{Teaching> Datos>Recodificar variable}.
\item En el cuadro de diálogo que aparece seleccionar en el campo \vampo{Variable a recodificar} la variable
\variable{edad}.
\item En el campo \campo{Reglas de recodificación} introducir
\begin{quote}
\lstinline{lo:20 = ``menores''}\\
\lstinline{20:hi = ``mayores''}
\end{quote}
\item En el cuadro \campo{Guardar nueva variable} hacer clic sobre el botón \boton{Cambiar}.
\item En el cuadro de diálogo que aparece seleccionar como objeto padre la el conjunto de datos \variable{edad\_estatura} y hacer clic sobre el botón \boton{Aceptar}.
\item Introducir el nombre de la nueva variable \variable{grupo.edad} y hacer clic sobre el botón \boton{Enviar}.
\end{enumerate}}
\end{indicacion}

\item Calcular la recta de regresión de la estatura sobre la edad para cada grupo de edad. 
¿En qué grupo explica mejor la recta de regresión la relación entre la estatura y la edad? 
Justificar la respuesta.
\begin{indicacion}{
\begin{enumerate}
\item Seleccionar el menú \menu{Teaching>Regresión>Regresión lineal}.
\item En el cuadro de diálogo que aparece, seleccionar la variable \variable{estatura} en el campo \campo{Variable
dependiente} y la variable \variable{edad} como \campo{Variable independiente}.
\item Seleccionar la opición \opcion{Ajuste por grupos}, introducir la variable \variable{grupo.edad} en el campo
\campo{Variable de agrupación}, y hacer clic en el \boton{Enviar}.
\end{enumerate}}
\end{indicacion}

\item Dibujar las rectas de regresión anteriores.
\begin{indicacion}{
\begin{enumerate}
\item Seleccionar el menú \menu{Teaching>Gráficos>Diagrama de Dispersión}.
\item En el cuadro de diálogo que aparece, seleccionar la variable \variable{estatura} en el campo \campo{Variable Y} y
la variable \variable{edad} en el campo \campo{Variable X}.
\item Seleccionar la opción \opcion{Dibujar por grupos} e introducir la variable \variable{grupo.edad} en el campo
\campo{Variable de agrupación}.
\item En la solapa \menu{Línea de ajuste}, seleccionar \opcion{Lineal} y hacer clic en el botón
\boton{Enviar}.
\end{enumerate}}
\end{indicacion}

\item ¿Qué estatura se espera que tenga una persona de 14 años? ¿Y una de 38?
\begin{indicacion}{
Para predecir la estatura de la persona de 14 años:
\begin{enumerate}
\item Seleccionar el menú \menu{Teaching>Regresión>Predicciones}.
\item En el cuadro de diálogo que aparece seleccionar como modelo de regresión la recta calculada para los menores e
introducir 14 en el campo \campo{Predicciones para} y hacer clic sobre el botón
\boton{Enviar}.
\end{enumerate}
para predecir la estatura de la persona de 38 años, repetir lo mismo pero seleccionando la recta de regresión para los
mayores e introducidento 38 en el campo \campo{Predicciones para}.
}
\end{indicacion}
\end{enumerate}

\opt{largo}{
\item La siguiente tabla recoge la información de las calificaciones obtenidas por un grupo de alumnos en dos
asignaturas $X$ e $Y$.
\begin{center}
\begin{tabular}{lcccccccccccc}
Alumno & 1 & 2 & 3 & 4 & 5 & 6 & 7 & 8 & 9 & 10 & 11 & 12\\
\hline
$X$ & NT & AP & SS & SS & AP & AP & SS & NT & SB & SS & AP & AP\\
$Y$ & SB & SS & AP & SS & AP & NT & SS & NT & NT & AP & AP & NT
\end{tabular}
\end{center}
Se pide:
\begin{enumerate}
\item Crear un conjunto de datos con las variables \varaible{X} e \variable{Y} e introducir los datos.

\item ¿Existe relación entre las calificaciones de $X$ e $Y$? Justificar la respuesta.
\begin{indicacion}{
\begin{enumerate}
\item Seleccionar el menú \menu{Teaching>Regresión>Correlación}.
\item En el cuadro de diálogo que aparece seleccionar la variables \variable{X} e \variable{Y} en el campo
\campo{Variables}.
\item En la solapa \menu{Opciones de correlación} seleccionar el método de \opcion{Ro de Spearman} y hacer clic sobre
el botón \boton{Enviar}.
\end{enumerate}}
\end{indicacion}
\end{enumerate}
}

\end{enumerate}


\section{Ejercicios propuestos}
\begin{enumerate}[leftmargin=*]
\item  Se determina la pérdida de actividad que experimenta un medicamento desde el momento de su fabricación a lo
largo del tiempo, obteniéndose el siguiente resultado:
\begin{center}
\begin{tabular}{|l|c|c|c|c|c|}
\hline 
Tiempo (en años) & 1 & 2 & 3 & 4 & 5 \\ 
\hline 
Actividad restante (\%) & 96 & 84 & 70 & 58 & 52 \\ 
\hline
\end{tabular}
\end{center}
Se desea calcular:
\begin{enumerate}
\item  La relación fundamental (recta de regresión) entre actividad restante y tiempo transcurrido.
\item ¿En qué porcentaje disminuye la actividad cada año que pasa?
\item ¿Cuándo tiempo debe pasar para que el fármaco tenga una actividad del 80\%? ¿Cuándo será nula la actividad?
¿Son igualmente fiables estas predicciones?
\end{enumerate}

\item Al realizar un estudio sobre la dosificación de un cierto medicamento, se trataron 6 pacientes con dosis diarias
de 2 mg, 7 pacientes con 3 mg y otros 7 pacientes con 4 mg. De los pacientes tratados con 2 mg, 2 curaron al cabo de 5
días, y 4 al cabo de 6 días. De los pacientes tratados con 3 mg diarios, 2 curaron al cabo de 3 días, 4 al cabo de 5
días y 1 al cabo de 6 días. Y de los pacientes tratados con 4 mg diarios, 5 curaron al cabo de 3 días y 2 al cabo de 4
días. Se pide: 
\begin{enumerate}
\item Calcular la recta de regresión del tiempo de curación con respecto a la dosis suministrada.
\item Calcular el coeficiente de regresión del tiempo de curación con respecto a la dosis e interpretarlo.
\item Calcular el coeficiente de correlación lineal e interpretarlo.
\item Determinar el tiempo esperado de curación para una dosis de 5 mg diarios. ¿Es fiable esta predicción?
\item ¿Qué dosis debe aplicarse si queremos que el paciente tarde 4 días en curarse? ¿Es fiable la predicción?
\end{enumerate}

\item El fichero \variable{estaturas.pesos.alumnos} del paquete \variable{rk.Teaching}, contiene la estatura, el peso y
el sexo de una muestra de alumnos universitarios.
Se pide:
\begin{enumerate}
\item Cargar el conjunto de datos \variable{estaturas.pesos.alumnos} desde el paquete \variable{rk.Teaching}.
\item Calcular la recta de regresión del peso sobre la estatura y dibujarla.
\item Calcular las rectas de regresión del peso sobre la estatura para cada sexo y dibujarlas.
\item Calcular los coeficientes de determinación de ambas rectas. ¿Qué recta es mejor modelo? Justificar la respuesta.
\item ¿Qué peso tendrá un hombre que mida 170 cm? ¿Y una mujer de la misma estatura?
\end{enumerate}

\item El conjunto de datos \variable{neonatos} del paquete \variable{rk.Teaching}, contiene información sobre una
muestra de 320 recién nacidos en un hospital durante un año que cumplieron el tiempo normal de gestación. 
Se pide:
\begin{enumerate}
\item Construir la tabla de frecuencias bidimensional del Agpar al minuto de nacer frente a si la madre ha fumado o no
durante el embarazo. ¿Qué conclusiones se pueden sacar?
\item Construir la tabla de frecuencias bidimensional del peso de los recién nacidos frente a la edad de la madre. ¿Qué
conclusiones se pueden sacar?
\item Construir la recta de regresión del peso de los recién nacidos sobre el número de cigarros fumados al día por las
madres. ¿Existe una relación lineal fuerte entre el peso y el número de cigarros?
\item Dibujar la recta de regresión calculada en el apartado anterior. ¿Por qué la recta no se ajusta bien a la nube de
puntos?
\item Calcular y dibujar la recta de regresión del peso de los recién nacidos sobre el número de cigarros fumados al día
por las madres en el grupo de las madres que si fumaron durante el embarazo. ¿Es este modelo mejor o pero que la recta
de los apartados anteriores? 

Según este modelo, ¿cuánto disminuirá el peso del recién nacido por cada cigarro más diario
que fume la madre? 
\item Según el modelo anterior, ¿qué peso tendrá un recién nacido de una madre que ha fumado 5 cigarros diarios durante
el embarazo? ¿Y si la madre ha fumado 30 cigarros diarios durante el embarazo? ¿Son fiables estas predicciones?
\item ¿Existe la misma relación lineal entre el peso de los recién nacidos y el número de cigarros fumados al día por
las madres que fumaron durante el embarazo en el grupo de las madres menores de 20 y en el grupo de las madres mayores de
20? ¿Qué se puede concluir?
\end{enumerate}
\end{enumerate}
