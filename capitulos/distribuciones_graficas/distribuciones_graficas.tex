% Author: Alfredo Sánchez Alberca (asalber@ceu.es)

\chapter[Distribuciones de Frecuencias y Representaciones Gráficas]{Distribuciones de Frecuencias\\ y Representaciones Gráficas}\label{cap:distribuciones-frecuencias}

% \section{Fundamentos teóricos}
% Uno de los primeros pasos en cualquier estudio estadístico es el resumen y la descripción de la información contenida en
% una muestra. Para ello se van a aplicar algunos métodos de análisis descriptivo, que nos permitirán clasificar y
% estructurar la información al igual que representarla gráficamente.

% Las características que estudiamos pueden ser o no susceptibles de medida; en este sentido definiremos una
% \emph{variable} como un carácter susceptible de ser medido, es decir, cuantitativo y cuantificable mediante la
% observación, (por ejemplo el peso de las personas, la edad, etc...), y definiremos un \emph{atributo} como un carácter
% no susceptible de ser medido, y en consecuencia observable tan sólo cualitativamente (por ejemplo el color de ojos,
% estado de un paciente, etc...). Se llaman modalidades a las posibles observaciones de un atributo.

% Dentro de los atributos, podemos hablar de \emph{atributos ordinales}, los que presentan algún tipo de orden entre las
% distintas modalidades, y de \emph{atributos nominales}, en los que no existe ningún orden entre ellas.

% Dentro de las variables podemos diferenciar entre \emph{discretas}, si sus valores posibles son valores aislados, y
% \emph{continuas}, si pueden tomar cualquier valor dentro de un intervalo.

% En algunos textos no se emplea el término \emph{atributo} y se denominan a todos los caracteres \emph{variables}. En ese
% caso se distinguen \emph{variables cuantitativas} para designar las que aquí hemos definido como \emph{variables}, y
% \emph{variables cualitativas} para las que aquí se han llamado \emph{atributos}. En lo sucesivo se aplicará este
% criterio para simplificar la exposición.


% \subsection{Cálculo de Frecuencias}
% Para estudiar cualquier característica, lo primero que deberemos hacer es un recuento de las observaciones, y el número
% de repeticiones de éstas. Para cada valor $x_i$ de la muestra se define:
% \begin{description}
% \item[Frecuencia absoluta] Es el número de veces que aparece cada uno de los valores $x_i$ y se denota por $n_i$.

% \item [Frecuencia relativa] Es el número de veces que aparece cada valor $x_i$ dividido entre el tamaño muestral y se
% denota por $f_i$

% \[f_i=\frac{n_i}{n}\]

% Generalmente las frecuencias relativas se multiplican por $100$ para que representen el tanto por ciento.
% \end{description}

% En el caso de que exista un orden entre los valores de la variable, a veces nos interesa no sólo conocer el número de
% veces que se repite un determinado valor, sino también el número de veces que aparece dicho valor y todos los menores. A
% este tipo de frecuencias se le denomina \emph{frecuencias acumuladas}.

% \begin{description}
% \item [Frecuencia absoluta acumulada] Es la suma de las frecuencias absolutas de los valores menores que $x_i$ más la frecuencia
% absoluta de $x_i$, y se denota por $N_i$

% \[N_i=n_1+n_2+\ldots+n_i\]

% \item [Frecuencia relativa acumulada] Es la suma de las frecuencias relativas de los valores menores que $x_i$ más la frecuencia
% relativa de $x_i$, y se denota por $F_i$

% \[F_i=f_1+f_2+\ldots+f_i\]
% \end{description}

% Los resultados de las observaciones de los valores de una variable estadística en una muestra suelen representarse en
% forma de tabla.
% En la primera columna se representan los valores $x_i$ de la variable colocados en orden creciente, y en la siguiente
% columna los valores de las frecuencias absolutas correspondientes $n_i$.

% Podemos completar la tabla con otras columnas, correspondientes a las frecuencias relativas, $f_i$, y a las frecuencias
% acumuladas, $N_i$ y $F_i$. Al conjunto de los valores de la variable observados en la muestra junto con sus frecuencias
% se le conoce como \emph{distribución de frecuencias muestral}.

% \begin{ejemplo}
% En una encuesta a 25 matrimonios, sobre el número de hijos que tienen, se obtienen los siguientes datos:
% \begin{center}
% 1, 2, 4, 2, 2, 2, 3, 2, 1, 1, 0, 2, 2, 0, 2, 2, 1, 2, 2, 3, 1, 2,
% 2, 1, 2.
% \end{center}

% Los valores distintos de la variable son: 0, 1, 2, 3 y 4. Así la tabla será:
%   \[\begin{array}{|c|l|r|}
%     \hline
%     x_i & \mbox{Recuento} & n_i \\ \hline
%     0 & \mbox{II} & 2 \\
%     1 & \mbox{IIIII I} & 6 \\
%     2 & \mbox{IIIII IIIII IIII} & 14 \\
%     3 & \mbox{II} & 2 \\
%     4 & \mbox{I} & 1 \\ \hline

%   \end{array}
%   \]

% La distribución de las frecuencias quedaría:
%   \[\begin{array}{|c|c|c|c|c|}
%     \hline
%     x_i & n_i & f_i & N_i & F_i \\ \hline
%     0 & 2 & 0.08 & 2 & 0.08 \\ \hline
%     1 & 6 & 0.24 & 8 & 0.32 \\ \hline
%     2 & 14 & 0.56 & 22 & 0.88 \\ \hline
%     3 & 2 & 0.08 & 24 & 0.96 \\ \hline
%     4 & 1 & 0.04 & 25 & 1 \\ \hline
%     \mbox{Suma} & 25 & 1 & \multicolumn{2}{c}{} \\
%     \cline{1-3}
%   \end{array}
%   \]
% \end{ejemplo}

% Cuando el tamaño de la muestra es grande en el caso de variables discretas con muchos valores distintos de la variable,
% y en cualquier caso si se trata de variables continuas, se agrupan las observaciones en \emph{clases}, que son
% intervalos contiguos, preferiblemente de la misma amplitud.

% Para decidir el número de clases a considerar, una regla frecuentemente utilizada es tomar el entero más próximo a
% $\sqrt{n}$ donde $n$ es el número de observaciones en la muestra. Pero conviene probar con distintos números de clases y
% escoger el que proporcione una descripción más clara. Así se prefijan los intervalos $(a_{i-1},a_i] , i=1,2,\ldots,l$
% siendo $a=a_0<a_1<....<a_l=b$ de tal modo que todos los valores observados estén dentro del intervalo $(a, b]$, y sin
% que exista ambig\"{u}edad a la hora de decidir a qué intervalo pertenece cada dato.

% Llamaremos \emph{marca de clase} al punto medio de cada intervalo.
% Así la \emph{marca de la clase} $(a_{i-1},a_i]$ es el punto medio $x_i$ de dicha clase, es decir

% \[  x_i=\frac{a_{i-1}+a_i}{2} \]

% En el tratamiento estadístico de los datos agrupados, todos los valores que están en una misma clase se consideran
% iguales a la marca de la clase. De esta manera si en la clase $(a_{i-1},a_i]$ hay $n_i$ valores observados, se puede
% asociar la marca de la clase $x_i$ con esta frecuencia $n_i$.



% \subsection{Representaciones Gráficas}
% Hemos visto que la tabla estadística resume los datos de una muestra, de forma que ésta se puede analizar de una manera
% más sistemática y resumida. Para conseguir una percepción visual de las carac\-terísticas de la población resulta muy
% útil el uso de gráficas y diagramas. Dependiendo del tipo de variable y de si trabajamos con datos agrupados o no, se
% utilizarán distintos tipos.


% \subsubsection{Diagrama de barras y polígono de frecuencias}
% Consiste en representar sobre el eje de abscisas de un sistema de ejes coordenados los distintos valores de la variable
% $X$, y levantar sobre cada uno de esos puntos una barra cuya altura sea igual a la frecuencia absoluta o relativa
% correspondiente a ese valor, tal y como se muestra en la figura \ref{g:diagramaabsolutas}.
% Esta representación se utiliza para distribuciones de frecuencias con pocos valores distintos de la variable, tanto
% cuantitativas como cualitativas, y en este último caso se suele representar con rectángulos de altura igual a la
% frecuencia de cada modalidad.

% En el caso de variables cuantitativas se puede representar también el diagrama de barras de las frecuencias acumuladas,
% tal y como se muestra en la figura \ref{g:diagramaacumuladas}.

% Otra representación habitual es el \emph{polígono de frecuencias} que consiste en la línea poligonal cuyos vertices son
% los puntos $(x_i,n_i)$, tal y como se ve en la figura \ref{g:poligonoabsolutas}, y si en vez de considerar las
% frecuencias absolutas o relativas se consideran las absolutas o relativas acumuladas, se obtiene el \emph{polígono de
% frecuencias acumuladas}, como se ve en la figura \ref{g:poligonoacumuladas}.

% \begin{figure}[h!]
% \centering
% \subfigure[Diagrama de barras de frecuencias absolutas.]{\label{g:diagramaabsolutas}
% \scalebox{0.65}{\input{distribuciones_graficas/img/diagrama_barras_frecuencia_absoluta}}}\qquad
% \subfigure[Diagrama de barras de frecuencias absolutas acumuladas.]{\label{g:diagramaacumuladas}
% \scalebox{0.65}{\input{distribuciones_graficas/img/diagrama_barras_frecuencia_acumulada}}}\\
% \subfigure[Polígono de frecuencias absolutas.]{\label{g:poligonoabsolutas}
% \scalebox{0.65}{\input{distribuciones_graficas/img/poligono_frecuencia_absoluta}}}\qquad
% \subfigure[Polígono de frecuencias absolutas acumuladas]{\label{g:poligonoacumuladas}
% \scalebox{0.65}{\input{distribuciones_graficas/img/poligono_frecuencia_acumulada}}}
% \caption{Diagramas de barras y polígonos asociados para datos no
% agrupados.}
% \end{figure}


% \subsubsection{Histogramas}
% Este tipo de representaciones se utiliza en variables continuas y en variables discretas en que se ha realizado una
% agrupación de las observaciones en clases. Un \emph{histograma} es un conjunto de rectángulos, cuyas bases son los
% intervalos de clase $(a_{i-1},a_i]$ sobre el eje $OX$ y su altura la correspondiente frecuencia absoluta , relativa,
% absoluta acumulada, o relativa acumulada, tal y como se muestra en la figuras~\ref{g:histogramaabsolutas} y
% \ref{g:histogramaacumuladas}.

% Si unimos los puntos medios de las bases superiores de los rectángulos del histograma, se obtiene el \emph{polígono de
% frecuencias} correspondiente a datos agrupados (figura~\ref{g:poligonoabsolutasagrupado}).

% El polígono de frecuencias también se puede utilizar para representar las frecuencias acumuladas, tanto absolutas como
% relativas. En este caso la línea poligonal se traza uniendo los extremos derechos de las bases superiores de los
% rectángulos del histograma de frecuencias acumuladas, en lugar de los puntos centrales
% (figura~\ref{g:poligonoacumuladasagrupado}).

% \begin{figure}[h!]
% \centering
% \subfigure[Histograma de frecuencias absolutas.]{\label{g:histogramaabsolutas}
% \scalebox{0.65}{\input{distribuciones_graficas/img/histograma_frecuencia_absoluta}}}\qquad
% \subfigure[Histograma de frecuencias absolutas acumuladas.]{\label{g:histogramaacumuladas}
% \scalebox{0.65}{\input{distribuciones_graficas/img/histograma_frecuencia_acumulada}}}\\
% \subfigure[Polígono de frecuencias absolutas.]{\label{g:poligonoabsolutasagrupado}
% \scalebox{0.65}{\input{distribuciones_graficas/img/poligono_frecuencia_absoluta_agrupado}}}\qquad
% \subfigure[Polígono de frecuencias absolutas acumuladas]{\label{g:poligonoacumuladasagrupado}
% \scalebox{0.65}{\input{distribuciones_graficas/img/poligono_frecuencia_acumulada_agrupado}}}
% \caption{Histograma y polígonos asociados para datos agrupados.}
% \end{figure}

% Para variables cualitativas y cuantitativas discretas también se pueden usar las superficies representativas; de éstas,
% las más empleadas son los \emph{sectores circulares}.


% % \subsubsection{Sectores circulares o diagrama de sectores}
% % Es una representación en la que un círculo se divide en sectores, de forma que los ángulos, y por tanto las áreas
% % respectivas, sean proporcionales a la frecuencia.

% % \begin{ejemplo}
% % Se está haciendo un estudio en una población del grupo sanguíneo de sus ciudadanos. Para ello disponemos
% % de una muestra de 30 personas, con los siguientes resultados: 5 personas con grupo 0, 14 con grupo A, 8 con grupo B y  3 con grupo AB.
% % El el diagrama de sectores de frecuencias relativas correspondiente aparece en la figura~\ref{g:diagramasectoresgruposanguineo}.

% % \begin{figure}[h!]
% % \centering
% % \scalebox{0.7}{\input{distribuciones_graficas/img/diagrama_sectores_grupo_sanguineo}}
% % \caption{Diagrama de sectores de frecuencias relativas del grupo sanguíneo.}
% % \label{g:diagramasectoresgruposanguineo}
% % \end{figure}
% % \end{ejemplo}


% % \subsubsection{Diagrama de cajas y datos atípicos}
% % Los datos extremadamente altos o bajos, en comparación con los del resto de la muestra, reciben el nombre de datos
% % influyentes o \emph{datos atípicos}. Tales datos que, como su propio nombre indica, pueden modificar las conclusiones de
% % un estudio, deben ser considerados atentamente antes de aceptarlos, pues no pocas veces podrán ser, simplemente, datos
% % erróneos. La representación gráfica más apropiada para detectar estos datos es el \emph{diagrama de cajas}. Este
% % diagrama está formado por una caja que contiene el 50\% de los datos centrales de la distribución, y unos segmentos que
% % salen de la caja, que indican los límites a partir de los cuales los datos se consideran atípicos. En la figura
% % \ref{g:cajas} se puede observar un ejemplo en el que aparecen dos datos atípicos.

% % \begin{figure}[h!]
% % \begin{center}
% % \scalebox{0.8}{\input{distribuciones_graficas/img/diagrama_caja}}
% % \caption{Diagrama de cajas para una muestra de recién nacidos.
% % Existen dos niños con pesos atípicos, uno con peso extremadamente
% % bajo $1.9$ kg, y otro con peso extremadamente alto $4.3$ kg.}
% % \label{g:cajas}
% % \end{center}
% % \end{figure}

% \clearpage
% \newpage

\section{Ejercicios resueltos}
\begin{enumerate}[leftmargin=*]

\item  En una encuesta a 25 matrimonios sobre el número de hijos que tenían se obtuvieron los siguientes datos:
\begin{center}
1, 2, 4, 2, 2, 2, 3, 2, 1, 1, 0, 2, 2, 0, 2, 2, 1, 2, 2, 3, 1, 2, 2, 1, 2
\end{center}
Se pide:
\begin{enumerate}
\item Crear un conjunto de datos con la variable \variable{hijos} e introducir los datos.

\item Construir la tabla de frecuencias.
\begin{indicacion}
\begin{enumerate}
\item Seleccionar el menú \menu{Teaching > Distribución de frecuencias > Tabla de frecuencias} .
\item En el cuadro de diálogo que aparece, seleccionar la variable \variable{hijos} en el campo \field{Variable a
tabular} y hacer clic en el botón \button{Enviar}.
\end{enumerate}
\end{indicacion}

\item  Dibujar el diagrama de barras de las frecuencias absolutas.
\begin{indicacion}
\begin{enumerate}
\item Seleccionar el menú \menu{Teaching > Gráficos > Diagrama de barras}.
\item En el cuadro de diálogo que aparece, seleccionar la variable \variable{hijos} en el campo \field{Variable} y hacer
clic en el botón \button{Enviar}.
\end{enumerate}
\end{indicacion}

\item Para la misma tabla de frecuencias anterior, dibujar también el diagrama de barras de las frecuencias relativas,
el de absolutas acumuladas y el de relativas acumuladas, además de sus correspondientes polígonos.
\begin{indicacion}Repetir los pasos del apartado anterior activando, en la solapa de \option{Opciones de las barras},
la opción \option{Frecuencias relativas} si se desea el diagrama de barras de frecuencias relativas, activando la opción
\option{Frecuencias acumuladas} si se desea el diagrama de barras de frecuencias acumuladas y activando la opción
\option{Polígono} para obtener el polígono asociado.
\end{indicacion}
\end{enumerate}

\item En un hospital se realizó un estudio sobre el número de personas que ingresaron en urgencias cada día del mes de
noviembre. Los datos observados fueron:
\begin{center}
15, 23, 12, 10, 28, 50, 12, 17, 20, 21, 18, 13, 11, 12, 26 \\
30, 6, 16, 19, 22, 14, 17, 21, 28, 9, 16, 13, 11, 16, 20
\end{center}
Se pide:

\begin{enumerate}
\item  Crear un conjunto de datos con la variable \variable{urgencias} e introducir los datos.

\item  Dibujar el diagrama de cajas. ¿Existe algún dato atípico? En el caso de que exista, eliminarlo y proceder con los
siguientes apartados.
\begin{indicacion}
\begin{enumerate}
\item Seleccionar el menú \menu{Teaching > Gráficos > Diagrama de cajas}.
\item En el cuadro de diálogo que aparece, seleccionar la variable \variable{urgencias} en el campo \field{Variables} y
hacer clic en el botón \button{Enviar}.
\item En la ventana que aparece con el diagrama de cajas identificar el dato atípico.
\item Ir a la ventana de edición de datos y eliminar la fila del dato atípico haciendo clic con el botón derecho del
ratón en la cabecera de la fila y seleccionando \menu{Borrar esta fila}. 
\end{enumerate}
\end{indicacion}

\item Construir la tabla de frecuencias agrupando en 5 clases.
\begin{indicacion}
\begin{enumerate}
\item Seleccionar el menú \menu{Teaching > Distribución de frecuencias > Tabla de frecuencias}.
\item En el cuadro de diálogo que aparece seleccionar la variable \variable{urgencias}.
\item En la solapa de \option{Clases} activar la casilla \option{Agrupar en intervalos}, marcar la opción \option{Número
de intervalos} e introducir el número deseado de intervalos en el campo \field{Intervalos sugeridos} y hacer clic sobre el botón
\button{Enviar}.
\end{enumerate}
\end{indicacion}

\item  Dibujar el histograma de frecuencias absolutas correspondiente a la tabla anterior.
\begin{indicacion}
\begin{enumerate}
\item Seleccionar el menú \menu{Teaching > Gráficos > Histograma}.
\item En el cuadro de diálogo que aparece seleccionar la variable \variable{urgencias} en el campo \field{Variable}.
\item En la solapa de \option{Clases} activar la casilla \option{Agrupar en intervalos}, marcar la opción \option{Número
de intervalos} e introducir el número deseado de intervalos en el campo \field{Intervalos sugeridos} y hacer clic sobre el botón
\button{Enviar}.
\end{enumerate}
\end{indicacion}

\item Para la misma tabla de frecuencias anterior, dibujar también el histograma de las frecuencias relativas, el de
absolutas acumuladas y el de relativas acumuladas, además de sus correspondientes polígonos.
\begin{indicacion}Repetir los pasos del apartado anterior activando, en la solapa de \option{Opciones del histograma},
la opción \option{Frecuencias relativas} si se desea el histograma de frecuencias relativas, activando la opción
\option{Frecuencias acumuladas} si se desea el histograma de frecuencias acumuladas y activando la opción
\option{Polígono} para obtener el polígono asociado.
\end{indicacion}
\end{enumerate}

\item Los grupos sanguíneos de una muestra de 30 personas son:
\begin{center}
A, B, B, A, AB, 0, 0, A, B, B, A, A, A, A, AB,\\
A, A, A, B, 0, B, B, B, A, A, A, 0, A, AB, 0. 
\end{center}
Se pide:
\begin{enumerate}
\item Crear un conjunto de datos con la variable \variable{grupo.sanguineo} e introducir los datos.

\item Construir la tabla de frecuencias.
\begin{indicacion}
\begin{enumerate}
\item Seleccionar el menú \menu{Teaching > Distribución de frecuencias > Tabla de frecuencias} .
\item En el cuadro de diálogo que aparece, seleccionar la variable \variable{grupo.sanguineo} en el campo
\field{Variable a tabular} y hacer clic en el botón \button{Enviar}.
\end{enumerate}
\end{indicacion}

\item Dibujar el diagrama de sectores.
\begin{indicacion}
\begin{enumerate}
\item Seleccionar el menú \menu{Teaching > Gráficos > Diagrama de sectores}.
\item En el cuadro de diálogo que aparece, seleccionar la variable \variable{grupo.sanguineo} en el campo
\field{Variables} y hacer clic sobre el botón \button{Enviar}.
\end{enumerate}
\end{indicacion}
\end{enumerate}

\item  En un estudio de población se tomó una muestra de 27 personas, y se les preguntó por su edad y estado civil,
obteniendo los siguientes resultados:
\begin{center}
\begin{tabular}{|l|rrrrrrrrr|}
\hline
Estado civil & \multicolumn{9}{c|}{Edad}\\
\hline
Soltero    & 31 & 45 & 35 & 65 & 21 & 38 & 62 & 22 & 31 \\
Casado     & 62 & 39 & 62 & 59 & 21 & 62 &    &    &    \\
Viudo      & 80 & 68 & 65 & 40 & 78 & 69 & 75 &    &    \\
Divorciado & 31 & 65 & 59 & 49 & 65 &    &    &    &    \\
\hline
\end{tabular}
\end{center}

Se pide:
\begin{enumerate}
\item Crear un conjunto de datos con la variables \variable{estado.civil} y \variable{edad} e introducir los datos.
\item Construir la tabla de frecuencias de la variable \variable{edad} para cada categoría de la
variable \variable{estado.civil}.
\begin{indicacion}
\begin{enumerate}
\item Seleccionar el menú \menu{Teaching > Distribución de frecuencias > Tabla de frecuencias}.
\item En el cuadro de diálogo que aparece, seleccionar la variable \variable{edad} en el campo \field{Variable a
tabular}, activar la casilla \option{Tabular por grupos}, seleccionar la variable \variable{estado.civil} en el campo
\field{Variable de agrupación} y hacer clic en el botón \button{Enviar}.
\end{enumerate}
\end{indicacion}

\item Dibujar los diagramas de cajas de la edad según el estado civil. ¿Existen datos atípicos? ¿En qué grupo hay mayor
dispersión?
\begin{indicacion}
\begin{enumerate}
\item Seleccionar el menú \menu{Teaching > Gráficos > Diagrama de cajas}.
\item En el cuadro de diálogo que aparece, seleccionar la variable \variable{edad} en el campo \field{Variables},
activar la casilla \option{Dibujar por grupos}, seleccionar la variable \variable{estado.civil} en el campo
\field{Variable de agrupación} y hacer clic en el botón \button{Enviar}.
\end{enumerate}
\end{indicacion}
\end{enumerate}

\end{enumerate}


\section{Ejercicios propuestos}
\begin{enumerate}[leftmargin=*]

\item  El número de lesiones padecidas durante una temporada por cada jugador de un equipo de fútbol fue el siguiente:
\begin{center}
0, 1, 2, 1, 3, 0, 1, 0, 1, 2, 0, 1, 1, 1, 2, 0, 1, 3, 2, 1, 2, 1, 0, 1
\end{center}

Se pide:
\begin{enumerate}
\item Construir la tabla de frecuencias.
\item Dibujar el diagrama de barras de las frecuencias relativas y de frecuencias relativas acumuladas.
\item Dibujar el diagrama de sectores.
\end{enumerate}

\item Para realizar un estudio sobre la estatura de los estudiantes universitarios, seleccionamos, mediante un proceso
de muestreo aleatorio, una muestra de 30 estudiantes, obteniendo los siguientes resultados (medidos en centímetros):
\begin{center}
179, 173, 181, 170, 158, 174, 172, 166, 194, 185,\\
162, 187, 198, 177, 178, 165, 154, 188, 166, 171,\\
175, 182, 167, 169, 172, 186, 172, 176, 168, 187.
\end{center}

Se pide:
\begin{enumerate}
\item Dibujar el histograma de las frecuencias absolutas agrupando desde 150 a 200 en clases de amplitud 10.
\item Dibujar el diagrama de cajas. ¿Existe algún dato atípico?.
\end{enumerate}

\item El conjunto de datos \variable{neonatos} del paquete \variable{rk.Teaching}, contiene información sobre una
muestra de 320 recién nacidos en un hospital durante un año que cumplieron el tiempo normal de gestación. 
Se pide:
\begin{enumerate}
\item Construir la tabla de frecuencias de la puntuación Apgar al minuto de nacer. 
Si se considera que una puntuación Apgar de 3 o menos indica que el neonato está deprimido, ¿qué porcentaje de niños está deprimido en la muestra?
\item Comparar las distribuciones de frecuencias de las puntuaciones Apgar al minuto de nacer según si la madre es mayor
o menor de 20 años.
¿En qué grupo hay más neonatos deprimidos?
\item Construir la tabla de frecuencias para el peso de los neonatos, agrupando en clases de amplitud $0.5$ desde el
$2$ hasta el $4.5$. ¿En qué intervalo de peso hay más niños?
\item Comparar la distribución de frecuencias relativas del peso de los neonatos según si la madre fuma o no. Si se
considera como peso bajo un peso menor de $2.5$ kg, ¿En qué grupo hay un mayor porcentaje de niños con peso bajo?
\item Si en los recién nacidos se considera como peso bajo un peso menor de $2.5$ kg, calcular la prevalencia del bajo
peso de recién nacidos en el grupo de madres fumadoras y en el de no fumadoras. 
\item Calcular el riesgo relativo de que un recién nacido tenga bajo peso cuando la madre fuma, frente a cuando la madre
no fuma. 
\item Construir el diagrama de barras de la puntuación Apgar al minuto. ¿Qué puntuación Apgar es la más frecuente? 
\item Construir el diagrama de frecuencias relativas acumuladas de la puntuación Apgar al minuto. ¿Por debajo de que puntuación estarán la mitad de los niños?
\item Comparar mediante diagramas de barras de frecuencias relativas las distribuciones de las puntuaciones Apgar al
minuto según si la madre ha fumado o no durante el embarazo. ¿Qué se puede concluir?
\item Construir el histograma de pesos, agrupando en clases de amplitud $0.5$ desde el $2$ hasta el $4.5$. ¿En qué
intervalo de peso hay más niños?
\item Comparar la distribución de frecuencias relativas del peso de los neonatos según si la madre fuma o no. ¿En qué
grupo se aprecia menor peso de los niños de la muestra?
\item Comparar la distribución de frecuencias relativas del peso de los neonatos según si la madre fumaba o no antes del
embarazo. ¿Qué se puede concluir?
\item Construir el diagrama de caja y bigotes del peso. ¿Entre qué valores se considera que el peso de un neonato es
normal? ¿Existen datos atípicos?
\item Comparar el diagrama de cajas y bigotes del peso, según si la madre fumó o no durante el embarazo y si era mayor o
no de 20 años. ¿En qué grupo el peso tiene más dispersión central? ¿En qué grupo pesan menos los niños de la
muestra?
\item Comparar el diagrama de cajas de la puntuación Apgar al minuto y a los cinco minutos. ¿En qué variable hay más
dispersión central?
\end{enumerate}  

\end{enumerate}
