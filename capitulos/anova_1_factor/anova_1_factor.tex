% Author: Alfredo Sánchez Alberca (asalber@ceu.es)

\chapter{Análisis de la Varianza de 1 Factor}

% \section{Fundamentos teóricos}

% El \emph{Análisis de la Varianza con un Factor} es una técnica
% estadística de contraste de hipótesis, cuyo propósito es estudiar
% el efecto de la aplicación de varios \emph{niveles}, también
% llamados \emph{tratamientos}, de una variable aleatoria
% cualitativa, llamada \emph{factor}, en una variable cuantitativa,
% llamada \emph{respuesta}.

% Por ejemplo, supongamos que estamos interesados en conocer si el sueldo medio de los médicos que entran a formar parte
% de la plantilla de un hospital, depende de la comunidad autónoma en la que trabajan.
% En este problema, la variable factor es la comunidad autónoma, con sus distintos niveles que son las distintas
% comunidades, mientras que la variable respuesta es el sueldo cobrado.
% A diferencia de un análisis de regresión simple, en el que se intenta explicar la variable respuesta mediante otra
% variable cuantitativa (como por ejemplo, el sueldo en función de las horas de permanencia en el hospital, o de la
% antigüedad en el puesto de trabajo), en el análisis de la varianza el factor, que es la variable independiente, es una
% variable cualitativa.

% Por otro lado, el análisis de la varianza de 1 factor se parece a un contraste de comparación de medias, sólo que en
% dicho contraste se comparan las medias de dos poblaciones, mientras que en el análisis de la varianza se comparan las
% medias de las $k$ poblaciones correspondientes a los $k$ niveles del factor.

% Para comparar las medias de la variable respuesta según los diferentes niveles del factor, se realiza un contraste de
% hipótesis en el que la hipótesis nula, $H_0$, es que la variable respuesta tiene igual media en todos los niveles,
% mientras que la hipótesis alternativa, $H_1$, es que hay diferencias estadísticamente significativas en al menos dos de
% las medias; y dicho contraste de hipótesis se basa en la comparación de dos estimadores de la varianza total de los
% datos de la variable respuesta; de ahí procede el nombre de esta técnica: \emph{ANOVA} (Analysis of Variance).

% \subsection{Notación, Modelo y Contraste}
% La notación habitual en ANOVA es la siguiente:
% \begin{description}
% \item[$k$] es el número de niveles del factor.
% \item[$n_i$] es el tamaño de la muestra aleatoria correspondiente al nivel $i$-ésimo del factor.
% \item[$n = \sum_{i = 1}^k {n_i}$] es el número total de observaciones.
% \item[$X_{ij}\ (i = 1,...,k;\,j = 1,...,n_i)$] es una variable aleatoria que indica la respuesta de la $j$-ésima unidad
% experimental al $i$-ésimo nivel del factor.
% \item [$x_{ij}$] es el valor concreto, en una muestra dada, de la variable $X_{ij}$.

% \begin{center}
% \begin{tabular}{|l|l|l|l|}
% \hline
% \multicolumn{4}{|c|}{Nivel del Factor} \\
% \hline
% \multicolumn{1}{|c|}{$1$} & \multicolumn{1}{c|}{$2$} & \multicolumn{1}{c|}{$\cdots$} & \multicolumn{1}{c|}{$k$} \\
% \hline
% \multicolumn{1}{|c|}{$X_{11}$} & \multicolumn{1}{c|}{$X_{21}$} & \multicolumn{1}{c|}{$\cdots$} &
% \multicolumn{1}{c|}{$X_{k1}$}
% \\
% \hline
% \multicolumn{1}{|c|}{$X_{12}$} & \multicolumn{1}{c|}{$X_{22}$} & \multicolumn{1}{c|}{$\cdots$} & \multicolumn{1}{c|}{$X_{k2}$} \\
% \hline
% \multicolumn{1}{|c|}{$\cdots$} & \multicolumn{1}{c|}{$\cdots$} & \multicolumn{1}{c|}{$\cdots$} & \multicolumn{1}{c|}{$\cdots$} \\
% \hline
% \multicolumn{1}{|c|}{$X_{1n_1}$} & \multicolumn{1}{c|}{$X_{2n_2}$} & \multicolumn{1}{c|}{$\cdots$} &
% \multicolumn{1}{c|}{$X_{kn_k}$}
% \\
% \hline
% \multicolumn{1}{c|}{} & \multicolumn{1}{c|}{$X_{2n_2}$} & \multicolumn{1}{c}{} & \multicolumn{1}{c}{} \\
% \cline{2-2}
% \end {tabular}
% \end{center}

% \item[$\mu_i$] es la media de la población del nivel $i$.
% \item [$\overline X_i = \sum_{j = 1}^{n_i} X_{ij}/n_i$] es la variable media muestral del nivel $i$, y
% estimador de $\mu_i$.
% \item [$\overline x_i = \sum_{j = 1}^{n_i} x_{ij}/n_i$] es la estimación concreta para una muestra dada de la
% variable media muestral del nivel $i$.
% \item [$\mu$] es la media de la población incluidos todos los niveles.
% \item [$\overline X  = \sum_{i = 1}^k \overline X_i/k = \sum_{i = 1}^k \sum_{j = 1}^{n_i } X_{ij}/n$] es la variable
% media muestral de todas las respuestas, y estimador de $\mu$.
% \item [$\overline x  = \sum_{i = 1}^k \overline x_i/k = \sum_{i = 1}^k \sum_{j = 1}^{n_i } x_{ij}/n$] es la estimación
% concreta para una muestra dada de la variable media muestral.
% \end{description}

% Con esta notación podemos expresar la variable respuesta mediante un modelo matemático que la descompone en componentes
% atribuibles a distintas causas:
% \[
% X_{ij}  = \mu  + \left( {\mu _i  - \mu } \right) + \left(
% {X_{ij} - \mu _i } \right),
% \]
% es decir, la respuesta $j$-ésima en el nivel $i$-ésimo puede descomponerse como resultado de una media global, más la
% desviación con respecto a la media global debida al hecho de que recibe el tratamiento $i$-ésimo, más una nueva
% desviación con respecto a la media del nivel debida a influencias aleatorias.

% Sobre este modelo se plantea la hipótesis nula: las medias correspondientes a todos los niveles son iguales; y su
% correspondiente alternativa: al menos hay dos medias de nivel que son diferentes.
% \[
% \begin{cases}
% H_0 &: & \mu _1  = \mu _2  = ... = \mu _k\\
% H_1 &: & \mu _i  \neq  \mu _j \;\textrm{para algún}\, i\,\textrm{y}\, j
% \]

% Para poder realizar el contraste con este modelo es necesario plantear ciertas hipótesis estructurales (supuestos del
% modelo):

% \begin{itemize}
% \item Las $k$ muestras, correspondientes a los $k$ niveles del factor, representan muestras aleatorias independientes de
% $k$ poblaciones con medias $\mu _1  = \mu _2  = ... = \mu _k$ desconocidas.
% \item Cada una de las $k$ poblaciones es normal.
% \item Cada una de las $k$ poblaciones tiene la misma varianza, $\sigma^2$.
% \end{itemize}

% Teniendo en cuenta la hipótesis nula y los supuestos del modelo, podemos construir un estadístico del contraste con
% distribución conocida, tal que permite aceptar o rechazar $H_0$; pero hasta poder dar el valor de dicho estadístico, aún
% debemos seguir ampliando la notación habitual en los test de ANOVA.

% Si sustituimos en el modelo las medias poblacionales por sus correspondientes estimadores muestrales tenemos
% \[
% X_{ij}  = \overline X  + \left( {\overline X_i  - \overline X }
% \right) + \left( {X_{ij}  - \overline X_i } \right),
% \]
% o lo que es lo mismo,
% \[
% X_{ij}- \overline X =   \left( {\overline X_i  - \overline X }
% \right) + \left( {X_{ij}  - \overline X_i } \right).
% \]

% Elevando al cuadrado y teniendo en cuenta las propiedades de los sumatorios, se llega a la ecuación que recibe el nombre
% de \emph{identidad de la suma de cuadrados}:
% \[
% \sum\limits_{i = 1}^k {\sum\limits_{j = 1}^{n_i } {\left( {X_{ij}
% - \overline X } \right)^2 } }  = \sum\limits_{i = 1}^k {n_i\left(
% {\overline X_i  - \overline X } \right)^2 }  + \sum\limits_{i =
% 1}^k {\sum\limits_{j = 1}^{n_i } {\left( {X_{ij}  - \overline X_i
% } \right)^2 } },
% \]
% donde:
% \begin{description}
% \item [$\sum_{i = 1}^k \sum_{j = 1}^{n_i } (X_{ij}- \overline X )^2$] recibe el nombre de \emph{suma total de
% cuadrados}, ($STC$), y es la suma de cuadrados de las desviaciones con respecto a la media global; por lo tanto, una
% medida de la variabilidad total de los datos.
% \item [$\sum_{j = 1}^k n_i (\overline X_i  - \overline X)^2$] recibe el nombre de \emph{suma de cuadrados de los
% tratamientos o suma de cuadrados intergrupos}, ($SCInter$), y es la suma ponderada de cuadrados de las desviaciones de
% la media de cada nivel con respecto a la media global; por lo tanto, una medida de la variabilidad atribuida al hecho de
% que se utilizan diferentes niveles o tratamientos.
% \item [$\sum_{i = 1}^k \sum_{j = 1}^{n_i } (X_{ij}- \overline X_i )^2$] recibe el nombre de \emph{suma de cuadrados
% residual o suma de cuadrados intragrupos}, ($SCIntra$), y es la suma de cuadrados de las desviaciones de las
% observaciones con respecto a las medias de los sus respectivos niveles o tratamientos; por lo tanto, una medida de la
% variabilidad en los datos atribuida a las fluctuaciones aleatorias dentro del mismo nivel.
% \end{description}

% Con esta notación la identidad de suma de cuadrados se expresa:
% \[
% SCT=SCInter+SCIntra
% \]
% Y un último paso para llegar al estadístico que permitirá contrastar $H_0$, es la definición de los \emph{Cuadrados
% Medios}, que se obtienen al dividir cada una de las sumas de cuadrados por sus correspondientes grados de libertad.
% Para $SCT$ el número de grados de libertad es $n-1$; para $SCInter$ es $k-1$; y para $SCIntra$ es $n-k$.
% Por lo tanto,
% \begin{align*}
% CMT &= \frac{{SCT}}{{n - 1}}\\
% CMInter &= \frac{{SCInter}}{{k - 1}}\\
% CMIntra &= \frac{{SCIntra}}{{n -k}}
% \end{align*}

% Y se podría demostrar que, en el supuesto de ser cierta la hipótesis nula y los supuestos del modelo, el cociente:
% \[
% \frac{{CMInter}}{{CMIntra}}
% \] 
% sigue una distribución $F$ de Fisher con $k-1$ y $n-k$ grados de libertad.

% De forma que, si $H_0$ es cierta, el valor del cociente para un conjunto de muestras dado, estará próximo a 1 (aún
% siendo siempre mayor que 1); pero si no se cumple $H_0$ crece la variabilidad intergrupos y la estimación del
% estadístico crece.
% En definitiva realizaremos un contraste de hipótesis unilateral con cola a la derecha de igualdad de varianzas, y para
% ello calcularemos el $p$-valor de la estimación de $F$ obtenida y aceptaremos o rechazaremos en función del nivel de
% significación fijado.

% \subsubsection{Tabla de ANOVA}

% Todos los estadísticos planteados en el punto anterior se recogen en una tabla denominada Tabla de ANOVA, en la que se
% ponen los resultados de las estimaciones de dichos estadísticos en las muestras concretas objeto de estudio.
% Esas tablas también son las que aportan como resultado de cualquier ANOVA los programas estadísticos, que suelen añadir
% al final de la tabla el $p$-valor del $F$ calculado, y que permite aceptar o rechazar la hipótesis nula de que las
% medias correspondientes a todos los niveles del factor son iguales.
% \begin{center}
% \renewcommand{\arraystretch}{2}
% \begin{tabular}{|l|l|l|l|l|l|}
% \cline{2-6}
% \multicolumn{1}{c|}{} & \multicolumn{1}{p{1.5cm}|}{\centering Suma de \newline cuadrados} & \multicolumn{1}{p{1.8cm}|}{\centering Grados de\newline libertad} & \multicolumn{1}{p{3.5cm}|}{\centering\ \newline Cuadrados medios} & \multicolumn{1}{p{2.5cm}|}{\centering\ \newline Estadístico $F$} & \multicolumn{1}{p{1.5cm}|}{\centering\ \newline $p$-valor} \\

% \hline
% \multicolumn{1}{|l|}{Intergrupos} & \multicolumn{1}{c|}{$SCInter$} & \multicolumn{1}{c|}{$k-1$} & \multicolumn{1}{c|}{$CMInter = \dfrac{{SCInter}}{{k - 1}}$} & \multicolumn{1}{c|}{$f_0=\dfrac{{CMInter}}{{CMIntra}}$} & \multicolumn{1}{c|}{$P\left( {F > f_0} \right)$} \\
% \hline
% \multicolumn{1}{|l|}{Intragrupos} & \multicolumn{1}{c|}{$SCIntra$} & \multicolumn{1}{c|}{$n-k$} & \multicolumn{1}{c|}{$CMIntra = \dfrac{{SCIntra}}{{n -k}}$} & \multicolumn{1}{c|}{} & \multicolumn{1}{c|}{} \\
% \hline
% \multicolumn{1}{|l|}{Total} & \multicolumn{1}{c|}{$SCT$} & \multicolumn{1}{c|}{$n-1$} & \multicolumn{1}{c|}{} & \multicolumn{1}{c|}{} &  \\
% \hline
% \end{tabular}
% \end{center}


% \subsection*{Test de Comparaciones Múltiples y por Parejas} 
% Una vez realizado el ANOVA de un factor para comparar las
% $k$ medias correspondientes a los $k$ niveles o tratamientos del factor, nos encontramos en una de las dos siguientes
% situaciones:
% \begin{itemize}
% \item No hemos podido rechazar $H_0$. En este caso se da por concluido el análisis de los datos en cuanto a detección de
% diferencias entre los niveles.
% \item Tenemos razones estadísticas para rechazar $H_0$. En este caso es natural continuar con el análisis para tratar de
% localizar con precisión dónde está la diferencia, cuáles son el nivel o niveles cuyas respuestas son estadísticamente
% diferentes.
% \end{itemize}

% En el segundo supuesto, hay varios métodos que permiten detectar las diferencias entre las medias de los diferentes
% niveles, y que reciben el nombre de \emph{Test de Comparaciones Múltiples}.
% A su vez este tipo de test se suelen clasificar en:
% \begin{itemize}
% \item \emph{Test de comparaciones por parejas}, cuyo objetivo es la comparación una a una de todas las posibles parejas
% de medias que se pueden tomar al considerar los diferentes niveles.
% Su resultado es una tabla en la que se reflejan las diferencias entre todas las posibles parejas y los intervalos de
% confianza para dichas diferencias, con la indicación de si hay o no diferencias significativas entre las mismas.
% Hay que aclarar que los intervalos obtenidos no son los mismos que resultarían si considerásemos cada pareja de medias
% por separado, ya que el rechazo de $H_0$ en el contraste general de ANOVA implica la aceptación de una hipótesis
% alternativa en la que están involucrados varios contrastes individuales a su vez; y si queremos mantener un nivel de
% significación $\alpha$ en el general, en los individuales debemos utilizar un $\alpha '$ considerablemente más pequeño.
% \item \emph{Test de rango múltiple}, cuyo objetivo es la identificación de subconjuntos homogéneos de medias que no se
% diferencian entre sí.
% \end{itemize}
% Entre otros, para los primeros, el test de Bonferroni; para los segundos, el test de Duncan; y para ambas categorías a
% la vez los test HSD de Tukey y Scheffé.

% \clearpage
% \newpage

\section{Ejercicios resueltos}
\begin {enumerate}[leftmargin=*]

\item Se realiza un estudio para comparar la eficacia de tres programas terapéuticos para el tratamiento del acné. Se
emplean tres métodos:
\begin{enumerate}
\item Lavado, dos veces al día, con cepillo de polietileno y un jabón abrasivo, junto con el uso diario de 250 mg de
tetraciclina.
\item Aplicación de crema de tretinoína, evitar el sol, lavado dos veces al día con un jabón emulsionante y agua, y
utilización dos veces al día de 250 mg de tetraciclina.
\item Evitar el agua, lavado dos veces al día con un limpiador sin lípidos y uso de crema de tretinoína y de peróxido
benzoílico.
\end{enumerate}
En el estudio participan 35 pacientes. Se separó aleatoriamente a estos pacientes en tres subgrupos de tamaños 10, 12 y
13, a los que se asignó respectivamente los tratamientos I, II, y III.
Después de 16 semanas se anotó para cada paciente el porcentaje de mejoría en el número de lesiones.
\[
\begin{array}{ll|ll|ll}
\multicolumn{6}{c}{\text{Tratamiento}} \\
\hline
\multicolumn{2}{c}{\text{I}} & \multicolumn{2}{c}{\text{II}} & \multicolumn{2}{c}{\text{III}} \\
\hline
48.6 & 50.8 & 68.0 & 71.9 & 67.5 & 61.4 \\
49.4 & 47.1 & 67.0 & 71.5 & 62.5 & 67.4 \\
50.1 & 52.5 & 70.1 & 69.9 & 64.2 & 65.4 \\
49.8 & 49.0 & 64.5 & 68.9 & 62.5 & 63.2 \\
50.6 & 46.7 & 68.0 & 67.8 & 63.9 & 61.2 \\
     &      & 68.3 & 68.9 & 64.8 & 60.5 \\
     &      &      &      & 62.3 &      \\
\hline
\end{array}
\]
\begin{enumerate}
\item Crear un conjunto de datos con las variables \variable{tratamiento} y \variable{mejora} e introducir los datos de
la muestra.

\item Dibujar el diagrama de puntos. ¿Se observan diferencias entre los tratamientos en el diagrama?
\begin{indicacion}
\begin{enumerate}
\item Seleccionar el menú \menu{Teaching > Gráficos > Diagrama de puntos}.
\item En el cuadro de diálogo que aparece, seleccionar la variable \variable{mejora} en el campo \field{Variable}, seleccionar la variable \variable{tratamiento} en el campo \field{Grupos}, y hacer click sobre el botón
\button{Aceptar}.
\end{enumerate}
\end{indicacion}

\item Realizar el contraste de ANOVA. 
¿Se puede concluir que los tres tratamientos tienen el mismo efecto medio con un nivel de significación de $0.05$?
\begin{indicacion}
\begin{enumerate}
\item Seleccionar el menú \menu{Teaching > Test paramétricos > Medias > ANOVA}.
\item En el cuadro de diálogo que aparece, seleccionar el conjunto de datos \variable{acne} en el campo \field{Conjunto de datos}.
\item Seleccionar la variable \variable{mejora} en el campo \field{Variable dependiente}, la variable
\variable{tratamiento} en el campo \field{Factores entre individuos} y hacer click sobre el botón \button{Aceptar}.
\end{enumerate}
\end{indicacion}

\item Obtener la tabla de ANOVA correspondiente al problema pero que además muestre los intervalos de confianza de
comparación de los tres tratamientos con una significación de $0.05$.
¿Entre qué parejas de tratamientos hay diferencias estadísticamente significativas?
\begin{indicacion} 
Repetir los mismos pasos del apartado anterior activando la opción \option{Comparaciones de medias por pares} de la solapa \menu{Comparación por pares}.
\end{indicacion}

\item Comprobar que se cumple la hipótesis de igualdad de varianzas entre los tratamientos.
\begin{indicacion} 
Repetir los mismos pasos del apartado anterior activando la opción \option{Test de Levene} de la solapa \menu{Opciones}.
\end{indicacion}

\item Dibujar el gráfico de los intervalos de confianza para la media de cada tratamiento. 
\begin{indicacion}
\begin{enumerate}
\item Seleccionar el menú \menu{Teaching  > Gráficos > Diagrama de medias}.
\item En el cuadro de diálogo que aparece, seleccionar la variable \variable{mejora} en el campo \field{Variable} y seleccionar la variable \variable{tratamiento} en el campo \field{Grupos}.
\item Seleccionar la opción \option{Intervalos de confianza} y hacer click sobre el botón \button{Aceptar}.
\end{enumerate}
\end{indicacion}
\end{enumerate}


\item Se sospecha que hay diferencias en la preparación del examen de selectividad entre los diferentes centros de
bachillerato de una ciudad.
Con el fin de comprobarlo, de cada uno de los 5 centros, se eligieron 8 alumnos al azar, con la condición de que
hubieran cursado las mismas asignaturas, y se anotaron las notas que obtuvieron en el examen de selectividad. 
Los resultados fueron:
\[
\begin{array}{lllll}
\multicolumn{5}{c}{\text{Centros}} \\
\hline
1 & 2 & 3 & 4 & 5 \\
\hline
5.5 & 6.1 & 4.9 & 3.2 & 6.7 \\
5.2 & 7.2 & 5.5 & 3.3 & 5.8 \\
5.9 & 5.5 & 6.1 & 5.5 & 5.4 \\
7.1 & 6.7 & 6.1 & 5.7 & 5.5 \\
6.2 & 7.6 & 6.2 & 6.0 & 4.9 \\
5.9 & 5.9 & 6.4 & 6.1 & 6.2 \\
5.3 & 8.1 & 6.9 & 4.7 & 6.1 \\
6.2 & 8.3 & 4.5 & 5.1 & 7.0 \\
\hline
\end{array}
\]

\begin{enumerate}
\item Crear un conjunto de datos con las variables \variable{nota} y \variable{centro} e introducir los datos de la muestra.

\item Dibujar el diagrama de puntos. ¿Se observan diferencias entre los centros en el diagrama?
\begin{indicacion}
\begin{enumerate}
\item Seleccionar el menú \menu{Teaching  > Gráficos > Diagrama de puntos}.
\item En el cuadro de diálogo que aparece, seleccionar la variable \variable{nota} en el campo \field{Variable}, seleccionar la
variable \variable{centro} en el campo \field{Grupos} y hacer click sobre el botón \button{Aceptar}.
\end{enumerate}
\end{indicacion}

\item Realizar el contraste de ANOVA. 
¿Se puede confimar la sospecha de que hay diferencias entre las notas medias de los centros?
\begin{indicacion}
\begin{enumerate}
\item Seleccionar el menú \menu{Teaching > Test paramétricos > Medias > ANOVA}.
\item En el cuadro de diálogo que aparece, seleccionar el conjunto de datos \variable{selectividad} en el campo \field{Conjunto de datos}.
\item Seleccionar la variable \variable{notas} en el campo \field{Variable dependiente}, la variable
\variable{centro} en el campo \field{Factores entre individuos} y hacer click sobre el botón \button{Aceptar}.
\end{enumerate}
\end{indicacion}

\item ¿Qué centros son los mejores en la preparación de la selectividad?
\begin{indicacion} 
Repetir los mismos pasos del apartado anterior activando la opción \option{Comparaciones de medias por pares} de la solapa \menu{Comparación por pares}.
\end{indicacion}
\end{enumerate}

\end{enumerate}


\section{Ejercicios propuestos}
\begin{enumerate}[leftmargin=*]

\item Se midió la frecuencia cardíaca (latidos por minuto) en cuatro grupos de adultos; controles normales (A),
pacientes con angina (B), individuos con arritmias cardíacas (C) y pacientes recuperados del infarto de miocardio (D).
Los resultados son los siguientes:

\begin{center}
\begin{tabular}{llll}
A & B & C & D \\
\hline
83 & 81 & 75 & 61 \\
61 & 65 & 68 & 75 \\
80 & 77 & 80 & 78 \\
63 & 87 & 80 & 80 \\
67 & 95 & 74 & 68 \\
89 & 89 & 78 & 65 \\
71 & 103 & 69 & 68 \\
73 & 89 & 72 & 69 \\
70 & 78 & 76 & 70 \\
66 & 83 & 75 & 79 \\
57 & 91 & 69 & 61 \\
\hline
\end{tabular}
\end{center}

¿Proporcionan estos datos la suficiente evidencia para indicar una diferencia en la frecuencia cardiaca media entre esos
cuatro tipos de pacientes?. 
Considerar $\alpha=0.05$.


\item Se midió la frecuencia respiratoria (inspiraciones por minuto) en ocho animales de laboratorio y con tres niveles
diferentes de exposición al monóxido de carbono. 
Los resultados son los siguientes:
\begin{center}
\begin{tabular}{lll}
\multicolumn{3}{c}{Nivel de exposición} \\
\hline
\multicolumn{1}{c}{Bajo} & \multicolumn{1}{c}{Moderado} & \multicolumn{1}{c}{Alto} \\
\hline
\multicolumn{1}{c}{36} & \multicolumn{1}{c}{43} & \multicolumn{1}{c}{45} \\
\multicolumn{1}{c}{33} & \multicolumn{1}{c}{38} & \multicolumn{1}{c}{39} \\
\multicolumn{1}{c}{35} & \multicolumn{1}{c}{41} & \multicolumn{1}{c}{33} \\
\multicolumn{1}{c}{39} & \multicolumn{1}{c}{34} & \multicolumn{1}{c}{39} \\
\multicolumn{1}{c}{41} & \multicolumn{1}{c}{28} & \multicolumn{1}{c}{33} \\
\multicolumn{1}{c}{41} & \multicolumn{1}{c}{44} & \multicolumn{1}{c}{26} \\
\multicolumn{1}{c}{44} & \multicolumn{1}{c}{30} & \multicolumn{1}{c}{39} \\
\multicolumn{1}{c}{45} & \multicolumn{1}{c}{31} & \multicolumn{1}{c}{29} \\
\hline
\end{tabular}
\end{center}

Con base en estos datos, ¿es posible concluir que los tres niveles de exposición, en promedio, tienen un efecto
diferente sobre la frecuencia respiratoria? 
Tomar $\alpha=0,05$.
\end{enumerate}

