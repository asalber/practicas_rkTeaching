% Author: Alfredo Sánchez Alberca (asalber@ceu.es)

\chapter{Regresión no lineal}

\section{Fundamentos teóricos}
La regresión simple tiene por objeto la construcción de un modelo funcional $y=f(x)$ que explique lo mejor posible la
relación entre dos variables $Y$ (variable dependiente) y $X$ (variable independiente) medidas en una misma muestra.

Ya vimos que, dependiendo de la forma de esta función, existen muchos tipos de regresión simple. 
Entre los más habituales están:
\begin{center}
\begin{tabular}{|l|c|}
\hline
 Modelo      &     Ecuación genérica      \\
\hline\hline
 Lineal                  &          $y=a+bx$          \\
\hline
 Parabólico              &       $y=a+bx+cx^2$        \\
\hline
 Polinómico de grado $n$ & $y=a_0+a_1x+\cdots+a_nx^n$ \\
\hline
 Potencial               &       $y=ax^b$       \\
\hline
 Exponencial             &     $y=e^{a+bx}$      \\
\hline
 Logarítmico             &       $y=a+b\log x$        \\
\hline
Inverso & $y=a+b/x$ \\
\hline
Curva S & $y= e^{a+b/x}$ \\
\hline
\end{tabular}
\end{center}

La elección de un tipo de modelo u otro suele hacerse según la forma de la nube de puntos del diagrama de dispersión.
A veces estará claro qué tipo de modelo se debe construir, tal y como ocurre en los diagramas de dispersión de la
figura~\ref{g:tiposrelaciones2}.
Pero otras veces no estará tan claro, y en estas ocasiones, lo normal es ajustar los dos o tres modelos que nos parezcan
más convincentes, para luego quedarnos con el que mejor explique la relación entre $Y$ y $X$, mirando el coeficiente de
determinación\footnote{Ver la práctica de regresión lineal y correlación.} de cada modelo.

\begin{figure}[h!]
\centering 
\subfigure[Sin relación.]{\scalebox{0.5}{%% Input file name: diagrama_dispersion_sin_relacion.fig
%% FIG version: 3.2
%% Orientation: Landscape
%% Justification: Flush Left
%% Units: Inches
%% Paper size: A4
%% Magnification: 100.0
%% Resolution: 1200ppi
%% Include the following in the preamble:
%% \usepackage{textcomp}
%% End

\begin{pspicture}(7.06cm,3.29cm)(16.36cm,13.56cm)
\psset{unit=0.8cm}
%%
%% Depth: 2147483647
%%
\newrgbcolor{mycolor0}{1.00 0.50 0.31}\definecolor{mycolor0}{rgb}{1.00,0.50,0.31}
%%
%% Depth: 100
%%
\psset{linestyle=solid,linewidth=0.03175,linecolor=mycolor0}
\qdisk(18.77,12.67){0.1}
\qdisk(10.15,6.95){0.1}
\qdisk(16.47,7.53){0.1}
\qdisk(16.52,14.13){0.1}
\qdisk(15.03,13.90){0.1}
\qdisk(14.24,9.90){0.1}
\qdisk(16.29,7.29){0.1}
\qdisk(13.52,14.98){0.1}
\qdisk(12.57,8.44){0.1}
\qdisk(16.53,12.51){0.1}
\qdisk(16.20,7.88){0.1}
\qdisk(17.61,8.79){0.1}
\qdisk(11.06,12.08){0.1}
\qdisk(16.83,6.87){0.1}
\qdisk(10.82,11.76){0.1}
\qdisk(13.80,12.41){0.1}
\qdisk(19.55,8.34){0.1}
\qdisk(12.96,12.68){0.1}
\qdisk(16.39,8.39){0.1}
\qdisk(12.97,7.50){0.1}
\qdisk(14.32,11.36){0.1}
\qdisk(16.15,10.56){0.1}
\qdisk(17.20,10.70){0.1}
\qdisk(12.78,14.11){0.1}
\qdisk(17.35,14.54){0.1}
\qdisk(18.21,14.21){0.1}
\qdisk(11.13,9.04){0.1}
\qdisk(15.69,10.55){0.1}
\qdisk(10.89,11.18){0.1}
\qdisk(10.54,13.67){0.1}
\qdisk(17.70,14.68){0.1}
\qdisk(14.67,7.80){0.1}
\qdisk(12.12,15.21){0.1}
\qdisk(19.37,9.75){0.1}
\qdisk(11.19,7.40){0.1}
\qdisk(17.36,13.53){0.1}
\qdisk(14.87,10.39){0.1}
\qdisk(15.55,9.67){0.1}
\qdisk(14.23,13.73){0.1}
\qdisk(12.79,11.91){0.1}
\qdisk(16.15,11.59){0.1}
\qdisk(17.07,10.73){0.1}
\qdisk(16.68,6.09){0.1}
\qdisk(12.60,8.26){0.1}
\qdisk(15.69,10.66){0.1}
\qdisk(16.30,8.04){0.1}
\qdisk(14.04,12.02){0.1}
\qdisk(15.77,14.30){0.1}
\qdisk(12.94,14.05){0.1}
\qdisk(18.20,12.14){0.1}
\qdisk(19.35,6.51){0.1}
\qdisk(11.05,15.07){0.1}
\qdisk(19.06,10.84){0.1}
\qdisk(15.49,12.85){0.1}
\qdisk(14.62,14.33){0.1}
\qdisk(15.09,14.14){0.1}
\qdisk(19.12,6.00){0.1}
\qdisk(15.20,8.40){0.1}
\qdisk(12.89,12.55){0.1}
\qdisk(15.00,7.98){0.1}
\qdisk(17.34,8.76){0.1}
\qdisk(12.23,12.88){0.1}
\qdisk(12.52,7.20){0.1}
\qdisk(15.35,14.11){0.1}
\qdisk(11.78,15.00){0.1}
\qdisk(15.50,6.87){0.1}
\qdisk(14.43,13.71){0.1}
\qdisk(15.10,14.89){0.1}
\qdisk(19.08,12.63){0.1}
\qdisk(11.38,13.02){0.1}
\qdisk(14.59,8.29){0.1}
\qdisk(15.79,11.59){0.1}
\qdisk(17.09,11.97){0.1}
\qdisk(16.88,11.38){0.1}
\qdisk(17.89,6.70){0.1}
\qdisk(13.24,14.50){0.1}
\qdisk(14.26,11.53){0.1}
\qdisk(17.08,5.85){0.1}
\qdisk(11.02,9.44){0.1}
\qdisk(12.08,7.01){0.1}
\qdisk(18.41,7.62){0.1}
\qdisk(17.13,11.49){0.1}
\qdisk(18.09,11.24){0.1}
\qdisk(14.21,11.19){0.1}
\qdisk(13.15,14.78){0.1}
\qdisk(14.75,8.74){0.1}
\qdisk(16.34,15.26){0.1}
\qdisk(12.56,13.18){0.1}
\qdisk(12.58,14.69){0.1}
\qdisk(13.45,8.17){0.1}
\qdisk(15.59,11.11){0.1}
\qdisk(17.54,7.86){0.1}
\qdisk(13.91,9.09){0.1}
\qdisk(12.17,11.26){0.1}
\qdisk(13.35,8.11){0.1}
\qdisk(10.81,8.24){0.1}
\qdisk(19.43,8.62){0.1}
\qdisk(14.42,11.36){0.1}
\qdisk(14.01,10.76){0.1}
\qdisk(14.65,6.91){0.1}
\rput(14.85,16.12){Sin relación}
\rput[l](14.71,4.63){$X$}
\rput[l]{90}(9.35,10.42){$Y$}
\psset{linecolor=black,fillstyle=none}
\psline(9.77,5.48)(19.93,5.48)(19.93,15.64)(9.77,15.64)(9.77,5.48)
\end{pspicture}
%% End
}}\qquad
\subfigure[Relación lineal.]{\scalebox{0.5}{%% Input file name: diagrama_dispersion_relacion_lineal.fig
%% FIG version: 3.2
%% Orientation: Landscape
%% Justification: Flush Left
%% Units: Inches
%% Paper size: A4
%% Magnification: 100.0
%% Resolution: 1200ppi
%% Include the following in the preamble:
%% \usepackage{textcomp}
%% End

\begin{pspicture}(7.06cm,3.29cm)(16.36cm,13.56cm)
\psset{unit=0.8cm}
%%
%% Depth: 2147483647
%%
\newrgbcolor{mycolor0}{1.00 0.50 0.31}\definecolor{mycolor0}{rgb}{1.00,0.50,0.31}
%%
%% Depth: 100
%%
\psset{linestyle=solid,linewidth=0.03175,linecolor=mycolor0}
\qdisk(16.95,12.63){0.1}
\qdisk(18.43,15.26){0.1}
\qdisk(19.26,13.83){0.1}
\qdisk(14.59,10.24){0.1}
\qdisk(18.55,12.88){0.1}
\qdisk(14.48,8.38){0.1}
\qdisk(12.46,9.00){0.1}
\qdisk(14.75,11.02){0.1}
\qdisk(12.80,8.51){0.1}
\qdisk(10.41,7.61){0.1}
\qdisk(18.44,14.54){0.1}
\qdisk(17.23,12.12){0.1}
\qdisk(14.85,10.75){0.1}
\qdisk(13.93,8.79){0.1}
\qdisk(13.42,9.30){0.1}
\qdisk(13.43,9.51){0.1}
\qdisk(17.02,11.54){0.1}
\qdisk(17.17,13.10){0.1}
\qdisk(18.27,12.81){0.1}
\qdisk(14.32,8.29){0.1}
\qdisk(11.13,7.42){0.1}
\qdisk(18.63,13.65){0.1}
\qdisk(11.06,8.06){0.1}
\qdisk(15.23,11.47){0.1}
\qdisk(17.89,12.31){0.1}
\qdisk(18.27,12.17){0.1}
\qdisk(18.73,14.79){0.1}
\qdisk(19.54,14.89){0.1}
\qdisk(12.54,9.33){0.1}
\qdisk(17.19,12.52){0.1}
\qdisk(15.95,11.91){0.1}
\qdisk(18.41,13.01){0.1}
\qdisk(16.70,10.51){0.1}
\qdisk(16.80,11.76){0.1}
\qdisk(10.99,7.56){0.1}
\qdisk(11.26,8.94){0.1}
\qdisk(16.32,13.21){0.1}
\qdisk(12.58,8.97){0.1}
\qdisk(15.02,12.00){0.1}
\qdisk(11.85,7.59){0.1}
\qdisk(18.32,12.93){0.1}
\qdisk(11.07,7.63){0.1}
\qdisk(12.12,9.34){0.1}
\qdisk(18.07,13.64){0.1}
\qdisk(14.07,11.70){0.1}
\qdisk(15.00,9.86){0.1}
\qdisk(18.52,13.72){0.1}
\qdisk(16.90,13.34){0.1}
\qdisk(17.26,11.52){0.1}
\qdisk(10.15,6.52){0.1}
\qdisk(14.05,10.53){0.1}
\qdisk(18.30,13.55){0.1}
\qdisk(11.06,8.80){0.1}
\qdisk(18.12,13.08){0.1}
\qdisk(16.10,12.30){0.1}
\qdisk(10.17,6.67){0.1}
\qdisk(15.01,11.19){0.1}
\qdisk(12.28,8.04){0.1}
\qdisk(11.91,9.12){0.1}
\qdisk(12.05,9.09){0.1}
\qdisk(16.09,11.58){0.1}
\qdisk(15.98,11.74){0.1}
\qdisk(19.55,13.47){0.1}
\qdisk(14.89,10.67){0.1}
\qdisk(16.54,12.68){0.1}
\qdisk(15.80,11.33){0.1}
\qdisk(11.00,7.52){0.1}
\qdisk(14.16,9.83){0.1}
\qdisk(15.05,10.03){0.1}
\qdisk(15.52,11.93){0.1}
\qdisk(16.90,12.77){0.1}
\qdisk(17.52,12.28){0.1}
\qdisk(11.00,6.67){0.1}
\qdisk(12.15,8.79){0.1}
\qdisk(13.90,9.63){0.1}
\qdisk(18.07,12.87){0.1}
\qdisk(11.68,8.49){0.1}
\qdisk(13.01,9.73){0.1}
\qdisk(17.27,11.74){0.1}
\qdisk(17.70,13.12){0.1}
\qdisk(11.27,7.63){0.1}
\qdisk(14.84,11.82){0.1}
\qdisk(13.69,11.74){0.1}
\qdisk(14.07,10.20){0.1}
\qdisk(11.16,7.21){0.1}
\qdisk(15.88,11.23){0.1}
\qdisk(18.83,14.75){0.1}
\qdisk(14.09,10.28){0.1}
\qdisk(11.60,8.57){0.1}
\qdisk(14.04,9.73){0.1}
\qdisk(14.74,10.51){0.1}
\qdisk(10.70,7.36){0.1}
\qdisk(18.75,14.09){0.1}
\qdisk(15.46,10.73){0.1}
\qdisk(13.99,9.88){0.1}
\qdisk(10.73,5.85){0.1}
\qdisk(15.85,9.70){0.1}
\qdisk(11.50,9.04){0.1}
\qdisk(11.36,7.49){0.1}
\qdisk(18.17,13.03){0.1}
\rput(14.85,16.12){Relación lineal}
\rput[l](14.71,4.63){$X$}
\rput[l]{90}(9.35,10.42){$Y$}
\psset{linecolor=black,fillstyle=none}
\psline(9.77,5.48)(19.93,5.48)(19.93,15.64)(9.77,15.64)(9.77,5.48)
\end{pspicture}
%% End
}}\qquad
\subfigure[Relación polinómica.]{\scalebox{0.5}{%% Input file name: diagrama_dispersion_relacion_parabolica.fig
%% FIG version: 3.2
%% Orientation: Landscape
%% Justification: Flush Left
%% Units: Inches
%% Paper size: A4
%% Magnification: 100.0
%% Resolution: 1200ppi
%% Include the following in the preamble:
%% \usepackage{textcomp}
%% End

\begin{pspicture}(7.06cm,3.29cm)(16.36cm,13.56cm)
\psset{unit=0.8cm}
%%
%% Depth: 2147483647
%%
\newrgbcolor{mycolor0}{1.00 0.50 0.31}\definecolor{mycolor0}{rgb}{1.00,0.50,0.31}
%%
%% Depth: 100
%%
\psset{linestyle=solid,linewidth=0.03175,linecolor=mycolor0}
\qdisk(16.95,7.05){0.1}
\qdisk(18.43,10.98){0.1}
\qdisk(19.26,14.20){0.1}
\qdisk(14.59,5.85){0.1}
\qdisk(18.55,15.26){0.1}
\qdisk(14.48,5.89){0.1}
\qdisk(12.46,9.12){0.1}
\qdisk(14.75,5.86){0.1}
\qdisk(12.80,6.85){0.1}
\qdisk(10.41,15.08){0.1}
\qdisk(18.44,13.64){0.1}
\qdisk(17.23,9.07){0.1}
\qdisk(14.85,5.87){0.1}
\qdisk(13.93,6.16){0.1}
\qdisk(13.42,6.71){0.1}
\qdisk(13.43,6.89){0.1}
\qdisk(17.02,8.43){0.1}
\qdisk(17.17,8.85){0.1}
\qdisk(18.27,10.34){0.1}
\qdisk(14.32,5.90){0.1}
\qdisk(11.13,13.56){0.1}
\qdisk(18.63,10.78){0.1}
\qdisk(11.06,11.35){0.1}
\qdisk(15.23,5.87){0.1}
\qdisk(17.89,8.99){0.1}
\qdisk(18.27,11.78){0.1}
\qdisk(18.73,13.44){0.1}
\qdisk(19.54,14.06){0.1}
\qdisk(12.54,8.14){0.1}
\qdisk(17.19,8.27){0.1}
\qdisk(15.95,6.50){0.1}
\qdisk(18.41,11.06){0.1}
\qdisk(16.70,8.24){0.1}
\qdisk(16.80,6.07){0.1}
\qdisk(10.99,10.09){0.1}
\qdisk(11.26,11.93){0.1}
\qdisk(16.32,6.47){0.1}
\qdisk(12.58,7.72){0.1}
\qdisk(15.02,5.89){0.1}
\qdisk(11.85,9.82){0.1}
\qdisk(18.32,9.56){0.1}
\qdisk(11.07,13.06){0.1}
\qdisk(12.12,7.74){0.1}
\qdisk(18.07,9.09){0.1}
\qdisk(14.07,6.14){0.1}
\qdisk(15.00,5.98){0.1}
\qdisk(18.52,11.42){0.1}
\qdisk(16.90,7.06){0.1}
\qdisk(17.26,7.30){0.1}
\qdisk(10.15,13.82){0.1}
\qdisk(14.05,6.81){0.1}
\qdisk(18.30,10.44){0.1}
\qdisk(11.06,12.20){0.1}
\qdisk(18.12,10.97){0.1}
\qdisk(16.10,7.26){0.1}
\qdisk(10.17,13.42){0.1}
\qdisk(15.01,5.85){0.1}
\qdisk(12.28,8.81){0.1}
\qdisk(11.91,9.30){0.1}
\qdisk(12.05,8.93){0.1}
\qdisk(16.09,6.29){0.1}
\qdisk(15.98,6.44){0.1}
\qdisk(19.55,13.43){0.1}
\qdisk(14.89,5.96){0.1}
\qdisk(16.54,7.46){0.1}
\qdisk(15.80,6.47){0.1}
\qdisk(11.00,10.22){0.1}
\qdisk(14.16,5.95){0.1}
\qdisk(15.05,5.86){0.1}
\qdisk(15.52,6.41){0.1}
\qdisk(16.90,8.42){0.1}
\qdisk(17.52,7.97){0.1}
\qdisk(11.00,12.87){0.1}
\qdisk(12.15,8.94){0.1}
\qdisk(13.90,7.48){0.1}
\qdisk(18.07,10.16){0.1}
\qdisk(11.68,8.26){0.1}
\qdisk(13.01,8.48){0.1}
\qdisk(17.27,8.16){0.1}
\qdisk(17.70,10.53){0.1}
\qdisk(11.27,12.32){0.1}
\qdisk(14.84,5.85){0.1}
\qdisk(13.69,6.86){0.1}
\qdisk(14.07,6.32){0.1}
\qdisk(11.16,13.67){0.1}
\qdisk(15.88,6.31){0.1}
\qdisk(18.83,11.79){0.1}
\qdisk(14.09,6.34){0.1}
\qdisk(11.60,9.10){0.1}
\qdisk(14.04,6.02){0.1}
\qdisk(14.74,5.92){0.1}
\qdisk(10.70,13.05){0.1}
\qdisk(18.75,11.38){0.1}
\qdisk(15.46,6.61){0.1}
\qdisk(13.99,6.04){0.1}
\qdisk(10.73,10.42){0.1}
\qdisk(15.85,6.30){0.1}
\qdisk(11.50,12.58){0.1}
\qdisk(11.36,9.86){0.1}
\qdisk(18.17,11.08){0.1}
\rput(14.85,16.12){Relación parabólica}
\rput[l](14.71,4.63){$X$}
\rput[l]{90}(9.35,10.42){$Y$}
\psset{linecolor=black,fillstyle=none}
\psline(9.77,5.48)(19.93,5.48)(19.93,15.64)(9.77,15.64)(9.77,5.48)
\end{pspicture}
%% End
}}\\
\subfigure[Relación exponencial.]{\scalebox{0.5}{%% Input file name: diagrama_dispersion_relacion_exponencial.fig
%% FIG version: 3.2
%% Orientation: Landscape
%% Justification: Flush Left
%% Units: Inches
%% Paper size: A4
%% Magnification: 100.0
%% Resolution: 1200ppi
%% Include the following in the preamble:
%% \usepackage{textcomp}
%% End

\begin{pspicture}(7.06cm,3.29cm)(16.36cm,13.56cm)
\psset{unit=0.8cm}
%%
%% Depth: 2147483647
%%
\newrgbcolor{mycolor0}{1.00 0.50 0.31}\definecolor{mycolor0}{rgb}{1.00,0.50,0.31}
%%
%% Depth: 100
%%
\psset{linestyle=solid,linewidth=0.03175,linecolor=mycolor0}
\qdisk(16.95,8.66){0.1}
\qdisk(18.43,11.23){0.1}
\qdisk(19.26,13.47){0.1}
\qdisk(14.59,6.74){0.1}
\qdisk(18.55,10.19){0.1}
\qdisk(14.48,6.49){0.1}
\qdisk(12.46,7.19){0.1}
\qdisk(14.75,7.06){0.1}
\qdisk(12.80,6.75){0.1}
\qdisk(10.41,6.21){0.1}
\qdisk(18.44,10.17){0.1}
\qdisk(17.23,8.05){0.1}
\qdisk(14.85,7.38){0.1}
\qdisk(13.93,7.62){0.1}
\qdisk(13.42,7.14){0.1}
\qdisk(13.43,6.77){0.1}
\qdisk(17.02,7.22){0.1}
\qdisk(17.17,8.48){0.1}
\qdisk(18.27,9.59){0.1}
\qdisk(14.32,6.61){0.1}
\qdisk(11.13,6.01){0.1}
\qdisk(18.63,9.95){0.1}
\qdisk(11.06,6.06){0.1}
\qdisk(15.23,6.62){0.1}
\qdisk(17.89,9.32){0.1}
\qdisk(18.27,9.78){0.1}
\qdisk(18.73,10.58){0.1}
\qdisk(19.54,15.26){0.1}
\qdisk(12.54,6.34){0.1}
\qdisk(17.19,7.91){0.1}
\qdisk(15.95,7.10){0.1}
\qdisk(18.41,9.80){0.1}
\qdisk(16.70,7.53){0.1}
\qdisk(16.80,8.43){0.1}
\qdisk(10.99,7.07){0.1}
\qdisk(11.26,7.47){0.1}
\qdisk(16.32,7.47){0.1}
\qdisk(12.58,6.68){0.1}
\qdisk(15.02,6.57){0.1}
\qdisk(11.85,7.13){0.1}
\qdisk(18.32,11.28){0.1}
\qdisk(11.07,7.25){0.1}
\qdisk(12.12,6.78){0.1}
\qdisk(18.07,9.43){0.1}
\qdisk(14.07,6.73){0.1}
\qdisk(15.00,7.07){0.1}
\qdisk(18.52,10.64){0.1}
\qdisk(16.90,7.94){0.1}
\qdisk(17.26,8.80){0.1}
\qdisk(10.15,6.25){0.1}
\qdisk(14.05,6.85){0.1}
\qdisk(18.30,10.67){0.1}
\qdisk(11.06,7.26){0.1}
\qdisk(18.12,8.97){0.1}
\qdisk(16.10,7.24){0.1}
\qdisk(10.17,5.85){0.1}
\qdisk(15.01,6.49){0.1}
\qdisk(12.28,6.77){0.1}
\qdisk(11.91,6.84){0.1}
\qdisk(12.05,6.44){0.1}
\qdisk(16.09,7.07){0.1}
\qdisk(15.98,6.97){0.1}
\qdisk(19.55,15.03){0.1}
\qdisk(14.89,6.85){0.1}
\qdisk(16.54,7.48){0.1}
\qdisk(15.80,7.49){0.1}
\qdisk(11.00,6.37){0.1}
\qdisk(14.16,6.58){0.1}
\qdisk(15.05,7.74){0.1}
\qdisk(15.52,7.71){0.1}
\qdisk(16.90,7.84){0.1}
\qdisk(17.52,8.51){0.1}
\qdisk(11.00,6.20){0.1}
\qdisk(12.15,6.05){0.1}
\qdisk(13.90,6.49){0.1}
\qdisk(18.07,9.70){0.1}
\qdisk(11.68,6.94){0.1}
\qdisk(13.01,6.99){0.1}
\qdisk(17.27,8.46){0.1}
\qdisk(17.70,9.07){0.1}
\qdisk(11.27,7.33){0.1}
\qdisk(14.84,7.40){0.1}
\qdisk(13.69,6.25){0.1}
\qdisk(14.07,7.73){0.1}
\qdisk(11.16,6.96){0.1}
\qdisk(15.88,7.00){0.1}
\qdisk(18.83,11.16){0.1}
\qdisk(14.09,6.52){0.1}
\qdisk(11.60,6.04){0.1}
\qdisk(14.04,6.19){0.1}
\qdisk(14.74,6.96){0.1}
\qdisk(10.70,6.17){0.1}
\qdisk(18.75,11.05){0.1}
\qdisk(15.46,6.32){0.1}
\qdisk(13.99,6.06){0.1}
\qdisk(10.73,6.62){0.1}
\qdisk(15.85,7.01){0.1}
\qdisk(11.50,6.76){0.1}
\qdisk(11.36,7.04){0.1}
\qdisk(18.17,9.43){0.1}
\rput(14.85,16.12){Relación exponencial}
\rput[l](14.71,4.63){$X$}
\rput[l]{90}(9.35,10.42){$Y$}
\psset{linecolor=black,fillstyle=none}
\psline(9.77,5.48)(19.93,5.48)(19.93,15.64)(9.77,15.64)(9.77,5.48)
\end{pspicture}
%% End
}}\qquad
\subfigure[Relación logarítmica.]{\scalebox{0.5}{%% Input file name: diagrama_dispersion_relacion_logaritmica.fig
%% FIG version: 3.2
%% Orientation: Landscape
%% Justification: Flush Left
%% Units: Inches
%% Paper size: A4
%% Magnification: 100.0
%% Resolution: 1200ppi
%% Include the following in the preamble:
%% \usepackage{textcomp}
%% End

\begin{pspicture}(7.06cm,3.29cm)(16.36cm,13.56cm)
\psset{unit=0.8cm}
%%
%% Depth: 2147483647
%%
\newrgbcolor{mycolor0}{1.00 0.50 0.31}\definecolor{mycolor0}{rgb}{1.00,0.50,0.31}
%%
%% Depth: 100
%%
\psset{linestyle=solid,linewidth=0.03175,linecolor=mycolor0}
\qdisk(10.25,6.45){0.1}
\qdisk(17.97,14.25){0.1}
\qdisk(10.24,6.13){0.1}
\qdisk(16.74,13.98){0.1}
\qdisk(13.45,11.24){0.1}
\qdisk(11.98,10.24){0.1}
\qdisk(14.03,12.49){0.1}
\qdisk(16.30,12.92){0.1}
\qdisk(12.87,11.39){0.1}
\qdisk(17.66,13.86){0.1}
\qdisk(11.35,9.63){0.1}
\qdisk(16.71,13.56){0.1}
\qdisk(19.53,14.55){0.1}
\qdisk(15.06,12.76){0.1}
\qdisk(15.65,13.66){0.1}
\qdisk(13.18,11.82){0.1}
\qdisk(13.05,11.37){0.1}
\qdisk(16.89,13.60){0.1}
\qdisk(17.06,13.28){0.1}
\qdisk(13.15,11.19){0.1}
\qdisk(15.50,12.48){0.1}
\qdisk(16.41,12.87){0.1}
\qdisk(11.42,9.02){0.1}
\qdisk(15.96,13.66){0.1}
\qdisk(11.49,10.18){0.1}
\qdisk(12.35,9.70){0.1}
\qdisk(19.45,15.26){0.1}
\qdisk(17.93,14.26){0.1}
\qdisk(18.72,14.44){0.1}
\qdisk(17.62,14.22){0.1}
\qdisk(12.71,10.52){0.1}
\qdisk(17.08,14.05){0.1}
\qdisk(15.11,13.01){0.1}
\qdisk(12.50,10.64){0.1}
\qdisk(16.19,13.66){0.1}
\qdisk(15.82,13.18){0.1}
\qdisk(12.90,11.88){0.1}
\qdisk(18.47,14.95){0.1}
\qdisk(12.06,10.13){0.1}
\qdisk(14.62,12.91){0.1}
\qdisk(15.37,12.94){0.1}
\qdisk(10.84,8.18){0.1}
\qdisk(16.16,13.44){0.1}
\qdisk(11.27,8.31){0.1}
\qdisk(14.40,11.97){0.1}
\qdisk(18.34,14.56){0.1}
\qdisk(12.47,10.96){0.1}
\qdisk(16.79,13.98){0.1}
\qdisk(11.01,8.05){0.1}
\qdisk(11.88,10.62){0.1}
\qdisk(17.92,14.09){0.1}
\qdisk(17.86,13.72){0.1}
\qdisk(19.50,14.67){0.1}
\qdisk(11.65,8.98){0.1}
\qdisk(15.13,13.30){0.1}
\qdisk(11.51,8.82){0.1}
\qdisk(16.11,13.46){0.1}
\qdisk(15.11,13.21){0.1}
\qdisk(12.18,11.15){0.1}
\qdisk(10.99,9.12){0.1}
\qdisk(11.28,9.20){0.1}
\qdisk(16.69,13.79){0.1}
\qdisk(12.10,10.21){0.1}
\qdisk(12.22,10.72){0.1}
\qdisk(18.31,14.93){0.1}
\qdisk(17.86,14.36){0.1}
\qdisk(17.02,14.18){0.1}
\qdisk(15.83,13.26){0.1}
\qdisk(12.33,10.39){0.1}
\qdisk(15.52,13.68){0.1}
\qdisk(14.22,12.24){0.1}
\qdisk(14.79,12.44){0.1}
\qdisk(10.71,6.89){0.1}
\qdisk(12.82,10.84){0.1}
\qdisk(12.78,10.72){0.1}
\qdisk(13.79,11.86){0.1}
\qdisk(15.02,12.49){0.1}
\qdisk(18.93,14.52){0.1}
\qdisk(15.15,13.33){0.1}
\qdisk(16.76,14.09){0.1}
\qdisk(10.15,5.85){0.1}
\qdisk(18.14,14.45){0.1}
\qdisk(19.50,14.79){0.1}
\qdisk(19.34,14.92){0.1}
\qdisk(17.47,14.14){0.1}
\qdisk(18.13,14.28){0.1}
\qdisk(12.24,10.52){0.1}
\qdisk(18.14,14.37){0.1}
\qdisk(18.97,14.74){0.1}
\qdisk(15.14,13.44){0.1}
\qdisk(14.97,13.41){0.1}
\qdisk(18.14,14.79){0.1}
\qdisk(17.23,14.34){0.1}
\qdisk(16.82,14.06){0.1}
\qdisk(19.40,14.69){0.1}
\qdisk(19.38,14.66){0.1}
\qdisk(14.72,12.57){0.1}
\qdisk(19.41,14.63){0.1}
\qdisk(16.66,13.91){0.1}
\qdisk(19.55,15.10){0.1}
\rput(14.85,16.12){Relación logarímica}
\rput[l](14.71,4.63){$X$}
\rput[l]{90}(9.35,10.42){$Y$}
\psset{linecolor=black,fillstyle=none}
\psline(9.77,5.48)(19.93,5.48)(19.93,15.64)(9.77,15.64)(9.77,5.48)
\end{pspicture}
%% End
}}\qquad
\subfigure[Relación inversa.]{\scalebox{0.5}{%% Input file name: diagrama_dispersion_relacion_inversa.fig
%% FIG version: 3.2
%% Orientation: Landscape
%% Justification: Flush Left
%% Units: Inches
%% Paper size: A4
%% Magnification: 100.0
%% Resolution: 1200ppi
%% Include the following in the preamble:
%% \usepackage{textcomp}
%% End

\begin{pspicture}(7.06cm,3.29cm)(16.36cm,13.56cm)
\psset{unit=0.8cm}
%%
%% Depth: 2147483647
%%
\newrgbcolor{mycolor0}{1.00 0.50 0.31}\definecolor{mycolor0}{rgb}{1.00,0.50,0.31}
%%
%% Depth: 100
%%
\psset{linestyle=solid,linewidth=0.03175,linecolor=mycolor0}
\qdisk(10.72,9.92){0.1}
\qdisk(11.70,9.02){0.1}
\qdisk(12.50,8.54){0.1}
\qdisk(13.59,7.21){0.1}
\qdisk(15.91,7.19){0.1}
\qdisk(12.47,8.08){0.1}
\qdisk(18.88,6.95){0.1}
\qdisk(13.58,7.44){0.1}
\qdisk(11.98,8.31){0.1}
\qdisk(17.32,6.94){0.1}
\qdisk(18.72,6.19){0.1}
\qdisk(14.93,6.96){0.1}
\qdisk(11.56,9.00){0.1}
\qdisk(15.27,7.49){0.1}
\qdisk(15.63,7.87){0.1}
\qdisk(12.52,8.00){0.1}
\qdisk(11.70,9.08){0.1}
\qdisk(16.82,6.28){0.1}
\qdisk(19.14,5.85){0.1}
\qdisk(10.84,10.82){0.1}
\qdisk(10.92,10.52){0.1}
\qdisk(12.81,7.77){0.1}
\qdisk(12.93,7.81){0.1}
\qdisk(14.78,6.65){0.1}
\qdisk(10.24,15.26){0.1}
\qdisk(10.15,13.68){0.1}
\qdisk(14.47,6.35){0.1}
\qdisk(17.31,6.11){0.1}
\qdisk(17.63,6.60){0.1}
\qdisk(17.37,7.05){0.1}
\qdisk(14.01,7.43){0.1}
\qdisk(15.83,6.39){0.1}
\qdisk(13.08,7.62){0.1}
\qdisk(14.01,7.50){0.1}
\qdisk(17.34,6.06){0.1}
\qdisk(11.58,9.45){0.1}
\qdisk(12.12,8.35){0.1}
\qdisk(13.95,6.41){0.1}
\qdisk(15.85,6.37){0.1}
\qdisk(12.82,7.39){0.1}
\qdisk(15.74,7.45){0.1}
\qdisk(12.18,8.88){0.1}
\qdisk(14.16,7.25){0.1}
\qdisk(15.33,6.41){0.1}
\qdisk(16.55,7.40){0.1}
\qdisk(13.29,8.16){0.1}
\qdisk(13.61,7.80){0.1}
\qdisk(15.76,6.47){0.1}
\qdisk(13.60,7.92){0.1}
\qdisk(10.97,11.36){0.1}
\qdisk(11.15,9.53){0.1}
\qdisk(14.71,6.82){0.1}
\qdisk(14.41,6.13){0.1}
\qdisk(15.64,6.69){0.1}
\qdisk(14.47,7.06){0.1}
\qdisk(17.79,7.19){0.1}
\qdisk(13.02,7.99){0.1}
\qdisk(12.05,7.84){0.1}
\qdisk(16.36,6.09){0.1}
\qdisk(14.64,7.33){0.1}
\qdisk(16.29,6.80){0.1}
\qdisk(12.65,8.67){0.1}
\qdisk(12.30,8.99){0.1}
\qdisk(12.92,8.20){0.1}
\qdisk(11.42,9.83){0.1}
\qdisk(14.74,7.50){0.1}
\qdisk(11.11,9.50){0.1}
\qdisk(17.04,6.90){0.1}
\qdisk(16.72,6.65){0.1}
\qdisk(19.55,6.65){0.1}
\qdisk(18.26,6.51){0.1}
\qdisk(15.61,7.02){0.1}
\qdisk(17.54,6.60){0.1}
\qdisk(17.95,6.72){0.1}
\qdisk(10.44,10.95){0.1}
\qdisk(14.77,8.23){0.1}
\qdisk(12.74,7.76){0.1}
\qdisk(19.49,6.88){0.1}
\qdisk(17.28,6.55){0.1}
\qdisk(13.17,7.79){0.1}
\qdisk(13.52,7.44){0.1}
\qdisk(18.61,6.44){0.1}
\qdisk(13.68,7.60){0.1}
\qdisk(19.05,6.48){0.1}
\qdisk(12.66,8.42){0.1}
\qdisk(17.37,6.01){0.1}
\qdisk(16.33,6.27){0.1}
\qdisk(16.52,6.81){0.1}
\qdisk(15.16,7.02){0.1}
\qdisk(11.12,9.67){0.1}
\qdisk(14.81,7.22){0.1}
\qdisk(10.42,12.66){0.1}
\qdisk(11.34,9.55){0.1}
\qdisk(19.27,6.17){0.1}
\qdisk(11.62,9.62){0.1}
\qdisk(16.55,6.94){0.1}
\qdisk(15.32,6.90){0.1}
\qdisk(15.82,6.83){0.1}
\qdisk(17.60,6.66){0.1}
\qdisk(18.39,6.51){0.1}
\rput(14.85,16.12){Relación inversa}
\rput[l](14.71,4.63){$X$}
\rput[l]{90}(9.35,10.42){$Y$}
\psset{linecolor=black,fillstyle=none}
\psline(9.77,5.48)(19.93,5.48)(19.93,15.64)(9.77,15.64)(9.77,5.48)
\end{pspicture}
%% End
}}\\
\caption{Diagramas de dispersión correspondientes a distintos tipos de relaciones
entre variables.} \label{g:tiposrelaciones2}
\end{figure}

Ya vimos en la práctica sobre regresión lineal simple, cómo construir rectas de regresión.
En el caso de que optemos por ajustar un modelo no lineal, la construcción del mismo puede realizarse siguiendo los
mismos pasos que en el caso lineal.
Básicamente se trata de determinar los parámetros del modelo que minimizan la suma de los cuadrados de los residuos en
$Y$.
En los modelos multiplicativo y exponencial, el sistema aplica transformaciones logarítmicas a las variables y después
ajusta un modelo lineal a los datos transformados.
En el modelo recíproco, el sistema sustituye la variable dependiente por su recíproco antes de estimar la ecuación de
regresión.

\clearpage
\newpage

\section{Ejercicios resueltos}
El procedimiento más sencillo para construir un modelo no lineal, siempre que sea posible, es transformar las variables
para convertirlo en un modelo lineal. En el caso de los modelos de regresión simple más comunes las transformaciones que
convierten cada modelo en un modelo lineal aparecen en la tabla siguiente:
\begin{center}
\begin{tabular}{|l|c|c|p{5cm}|}
\hline
 Modelo      &     Modelo no lineal     & Modelo lineal & Transformación  \\
\hline
 Potencial               &       $y=ax^b$       & $\log(y) = \log(a)+b\log(x)$ & Se toma el logaritmo de ambas
 variables\\
\hline
 Exponencial             &     $y=e^{a+bx}$     & $\log(y) = a+bx$ & Se toma el logaritmo de la variable dependiente\\
\hline
 Logarítmico             &       $y=a+b\log x$  &  $y=a+b\log x$ & Se toma el logaritmo de la variable independiente   
 \\
\hline
Inverso & $y=a+b/x$ & $y=a+b\frac{1}{x}$ & Se toma el inverso de la variable independiente \\
\hline
Curva S & $y= e^{a+b/x}$ & $\log(y)=a+b\frac{1}{x}$ & Se toma el logaritmo de la variable dependiente y el inverso de
la independiente\\
\hline
\end{tabular}
\end{center}

\begin{enumerate}[leftmargin=*]
\item En un experimento se ha medido el número de bacterias por unidad de volumen en un cultivo, cada hora transcurrida,
obteniendo los siguientes resultados:
\begin{center}
\begin{tabular}{lrrrrrrrrr}
\hline
Horas & 0 & 1 & 2 & 3 & 4 & 5 & 6 & 7 & 8  \\
Nº Bacterias & 25 & 28 & 47 & 65 & 86 & 121 & 190 & 290 & 362\\
\hline
\end{tabular}
\end{center}

Se pide:
\begin{enumerate}
\item Crear un conjunto de datos con las variables \variable{horas} y \variable{bacterias} e introducir estos datos.

\item Dibujar el diagrama de dispersión correspondiente. En vista del diagrama, ¿qué tipo de modelo crees que explicará
mejor la relación entre el número de bacterias y el tiempo transcurrido? 
\begin{indicacion}{
\begin{enumerate}
\item Seleccionar el menú \menu{Teaching>Gráficos>Diagrama de dispersión}.
\item En el cuadro de diálogo que aparece, seleccionar la variable \variable{bacterias} en el campo \campo{Variable Y}
y la variable \variable{horas} en el campo \campo{Variable X}, y hacer clic en el botón \boton{Enviar}.
\end{enumerate}}
\end{indicacion}

\item Calcular los modelos exponencial y cuadrático de las bacterias sobre las horas. ¿Qué tipo de modelo es el mejor?
\begin{indicacion}{
Para el modelo exponencial:
\begin{enumerate}
\item Seleccionar el menú \menu{Teaching>Regresión>Regresión no lineal}.
\item En el cuadro de diálogo que aparece, seleccionar la variable \variable{bacterias} en el campo \campo{Variable
dependiente} y la variable \variable{horas} en el campo \campo{Variable independiente}.
\item En la solapa de \opcion{Modelo de regresión} seleccionar el modelo \opcion{Exponencial}.
\item Seleccionar \opcion{Guardar modelo} e introducir un nombre para el modelo y hacer clic sobre el botón \boton{Enviar}.
\end{enumerate}
Para el modelo cuadrático repetir los pasos pero seleccionando como modelo el \opcion{Cuadrático}. 
El modelo mejor será aquel que tenga un coeficiente de determinación mayor.
}
\end{indicacion}

\item Dibujar la curva del mejor de los modelos anteriores.
\begin{indicacion}{
\begin{enumerate}
\item Seleccionar el menú \menu{Teaching>Gráficos>Diagrama de dispersión}.
\item En el cuadro de diálogo que aparece, seleccionar la variable \variable{bacterias} en el campo \campo{Variable Y}
y la variable \variable{horas} en el campo \campo{Variable X}.
\item En la solapa \menu{Línea de ajuste} seleccionar la opción \opcion{Exponencial} y hacer clic sobre el botón \boton{Enviar}.
\end{enumerate}}
\end{indicacion}

\item Según el modelo anterior, ¿cuántas bacterias habrá al cabo de 3 horas y media del inicio del cultivo? 
¿Y al cabo de 10 horas? 
¿Son fiables estas predicciones?
\begin{indicacion}{
\begin{enumerate}
\item Seleccionar el menú \menu{Teaching>Regresión>Predicciones}.
\item En el cuadro de diálogo que aparece seleccionar el modelo de regresión exponencial construido antes.
\item Introducir los valores $3.5, 10$ en el campo \campo{Predicciones para} y hacer clic sobre el botón \boton{Enviar}.
\item Como se trata de un modelo exponencial, las predicciones obtenidas corresponden al logaritmo de bacterias. 
Para obtener la predicción de bacterias basta con aplicar la función exponencial a los valores obtenidos.
\end{enumerate}}
\end{indicacion}

\item Dar una predicción lo más fiable posible del tiempo que tendría que transcurrir para que en el cultivo hubiese 100
bacterias.
\begin{indicacion}{
Para construir el modelo logarítmico:
\begin{enumerate}
\item Seleccionar el menú \menu{Teaching>Regresión>Regresión no lineal}.
\item En el cuadro de diálogo que aparece, seleccionar la variable \variable{horas} en el campo \campo{Variable
dependiente} y la variable \variable{bacterias} en el campo \campo{Variable independiente}.
\item Seleccionar como modelo el \opcion{Logarítmico}.
\item Seleccionar \opcion{Guardar modelo} e introducir un nombre para el modelo y hacer clic sobre el botón \boton{Enviar}.
\end{enumerate}
Para hacer la predicción:
\begin{enumerate}
\item Seleccionar el menú \menu{Teaching>Regresión>Predicciones}.
\item En el cuadro de diálogo que aparece seleccionar el modelo de regresión logarítmico construido antes.
\item Introducir el valor $100$ en el campo \campo{Predicciones para} y hacer clic sobre el botón \boton{Enviar}.
\end{enumerate}}
\end{indicacion}
\end{enumerate}

\item El conjunto de datos \variable{dieta} del paquete \variable{rk.Teaching} contiene los datos de un estudio llevado
a cabo por un centro dietético para probar una nueva dieta de adelgazamiento.
Para cada individuo se ha medido el número de días que lleva con la dieta, el número de kilos perdidos desde entonces y
si realizó o no un programa de ejercicios.
Se pide:
\begin{enumerate}
\item Cargar el conjunto de datos \variable{dieta} desde el paquete \variable{rk.Teaching}.

\item Dibujar el diagrama de dispersión. 
Según la nube de puntos, ¿qué tipo de modelo explicaría mejor la relación entre los kilos perdidos y los días de dieta?
\begin{indicacion}{
\begin{enumerate}
\item Seleccionar el menú \menu{Teaching>Gráficos>Diagrama de dispersión}.
\item En el cuadro de diálogo que aparece, seleccionar la variable \variable{kilos} en el campo \campo{Variable Y}, la
variable \variable{dias} en el campo \campo{Variable X}, y hacer clic en el botón \boton{Enviar}.
\end{enumerate}}
\end{indicacion}

\item Construir el modelo de regresión que mejor explique la relación entre los kilos perdidos y los días de dieta.
\begin{indicacion}{
\begin{enumerate}
\item Seleccionar el menú \menu{Teaching>Regresión>Comparación de modelos}.
\item En el cuadro de diálogo que aparece, seleccionar la variable \variable{kilos} en el campo \campo{Variable
dependiente} y la variable \variable{dias} en el campo \campo{Variable independiente}.
\item En la solapa \menu{Modelos de regresión} seleccionar todos los modelos y hacer clic sobre el botón \boton{Enviar}.
\item El mejor modelo aparece en primer lugar y es el que tenga el coeficiente de determinación mayor.
\end{enumerate}}
\end{indicacion}

\item Dibujar el modelo del apartado anterior.
\begin{indicacion}{
\begin{enumerate}
\item Seleccionar el menú \menu{Teaching>Gráficos>Diagrama de Dispersión}.
\item En el cuadro de diálogo que aparece, seleccionar la variable \variable{kilos} en el campo \campo{Variable Y}
y la variable \variable{dias} en el campo \campo{Variable X}.
\item En la solapa \menu{Línea de ajuste} seleccionar la opción correspondiente al mejor modelo y hacer
clic sobre el botón \boton{Enviar}.
\end{enumerate}}
\end{indicacion}

\item Construir el modelo de regresión que mejor explique la relación entre los kilos perdidos y los días de dieta para
los que no hacen ejercicio.
\begin{indicacion}{
Para ver qué modelo es mejor:
\begin{enumerate}
\item Seleccionar el menú \menu{Teaching>Regresión>Comparación de modelos}.
\item En el cuadro de diálogo que aparece, seleccionar la variable \variable{kilos} en el campo \campo{Variable
dependiente} y la variable \variable{dias} en el campo \campo{Variable independiente}.
\item Seleccionar la opción \opcion{Filtro} e introducir la condición \lstinline{ejercicio=="no"} en el campo \campo{Condición de selección}.
\item En la solapa \menu{Modelos de regresión} seleccionar todos los modelos y hacer clic sobre el botón \boton{Enviar}.
\item El mejor modelo aparece en primer lugar y es el que tenga el coeficiente de determinación mayor.
\end{enumerate}
Para construir el modelo:
\begin{enumerate}
\item Seleccionar el menú \menu{Teaching>Regresión>Regresión no lineal}.
\item En el cuadro de diálogo que aparece, seleccionar la variable \variable{kilos} en el campo \campo{Variable
dependiente} y la variable \variable{dias} en el campo \campo{Variable independiente}.
\item Seleccionar la opción \opcion{Filtro} e introducir la condición \lstinline{ejercicio=="no"} en el campo \campo{Condición de selección}.
\item Seleccionar \opcion{Guardar modelo} e introducir un nombre para el modelo y hacer clic sobre el botón \boton{Enviar}.
\end{enumerate}}
\end{indicacion}

\item Construir el modelo de regresión que mejor explique la relación entre los kilos perdidos y los días de dieta para
los que si hacen ejercicio.
\begin{indicacion}{
Para ver qué modelo es mejor:
\begin{enumerate}
\item Seleccionar el menú \menu{Teaching>Regresión>Comparación de modelos}.
\item En el cuadro de diálogo que aparece, seleccionar la variable \variable{kilos} en el campo \campo{Variable
dependiente} y la variable \variable{dias} en el campo \campo{Variable independiente}.
\item Seleccionar la opción \opcion{Filtro} e introducir la condición \lstinline{ejercicio=="si"} en el campo \campo{Condición de selección}.
\item En la solapa \menu{Modelos de regresión} seleccionar todos los modelos y hacer clic sobre el botón \boton{Enviar}.
\item El mejor modelo aparece en primer lugar y es el que tenga el coeficiente de determinación mayor.
\end{enumerate}
Para construir el modelo:
\begin{enumerate}
\item Seleccionar el menú \menu{Teaching>Regresión>Regresión no lineal}.
\item En el cuadro de diálogo que aparece, seleccionar la variable \variable{kilos} en el campo \campo{Variable
dependiente} y la variable \variable{dias} en el campo \campo{Variable independiente}.
\item Seleccionar la opción \opcion{Filtro} e introducir la condición \lstinline{ejercicio=="si"} en el campo \campo{Condición de selección}.
\item Seleccionar \opcion{Guardar modelo} e introducir un nombre para el modelo y hacer clic sobre el botón \boton{Enviar}.
\end{enumerate}}
\end{indicacion}

\opt{largo}{
\item Utilizar el modelo construido para predecir el número de kilos perdidos tras 40 y 500 días de dieta, tanto para
los que hacen ejercicio como para los que no. ¿Son fiables estas predicciones?
\begin{indicacion}{
\begin{enumerate}
\item Seleccionar el menú \menu{Teaching>Regresión>Predicciones}.
\item En el cuadro de diálogo que aparece seleccionar el modelo de regresión construido antes para los que no hacen ejercicio.
\item Introducir los valores $40, 500$ en el campo \campo{Predicciones para} y hacer clic sobre el botón \boton{Enviar}.
\end{enumerate}
Repetir los pasos anteriores seleccionando el modelo de regresión construido antes para los que si hacen ejercicio.}
\end{indicacion}
}
\end{enumerate}

\end{enumerate}


\section{Ejercicios propuestos}
\begin{enumerate}[leftmargin=*]
\item La concentración de un fármaco en sangre, $C$ en mg/dl, es función del tiempo, $t$ en horas, y viene dada por la
siguiente tabla: 
\[
\begin{array}{|l|r|r|r|r|r|r|r|}
\hline
\text{t} & 2 & 3 & 4 & 5 & 6 & 7 & 8\\
\hline
\text{C} & 25 & 36 & 48 & 64 & 86 & 114 & 168\\
\hline
\end{array}
\]
Se pide: 
\begin{enumerate}
\item Según el modelo exponencial, ¿qué concentración de fármaco habría a las $4.8$ horas? ¿Es fiable la predicción?
Justificar adecuadamente la respuesta.
\item Según el modelo logarítmico, ¿qué tiempo debe pasar para que la concentración sea de 100 mg/dl?
\end{enumerate}

\item El fichero \texttt{naciones.txt} contiene información sobre el desarrollo de distintos países (tasa de fertilidad, tasa de uso de
anticonceptivos, tasa de mortalidad infantil, producto interior bruto per cápita y continente). 
Se pide:
\begin{enumerate}
\item Importar el fichero \texttt{naciones.txt} en un conjunto de datos.
\item Construir el mejor modelo de regresión de la tasa de fertilidad sobre el producto interior bruto. ¿Cómo
explicarías esta relación?
\item Dibujar el modelo del apartado anterior.
\item ¿Qué tasa de fertilidad le corresponde a una mujer que viva en un país con un producto interior bruto per cápita
de 10000 \$? ¿Y si la mujer vive en Europa?
\end{enumerate}

\end{enumerate}