% Author: Alfredo Sánchez Alberca (asalber@ceu.es)

\chapter[ANOVA de múltiples factores y medidas repetidas]{ANOVA de Múltiples Factores y ANOVA de Medidas Repetidas}

\medskip
\section{Fundamentos teóricos}
Como ya se vio en una práctica anterior, el \emph{Análisis de la Varianza de un Factor}, \emph{ANOVA} o también
\emph{ANOVA de una Vía}, es una técnica estadística de contraste de hipótesis cuyo propósito es estudiar el efecto de la
aplicación de varios \emph{niveles} (también llamados \emph{tratamientos}) de una variable aleatoria cualitativa,
llamada \emph{factor} o \emph{vía}, en una variable cuantitativa, llamada \emph{respuesta}.
Si se supone que la variable cualitativa independiente, es decir el factor, presenta $k$ niveles diferentes, entonces
para comparar las $k$ medias de la variable respuesta según los diferentes niveles del factor se realiza un contraste de
hipótesis, cuya hipótesis nula, $H_0$, es que la variable respuesta tiene igual media en todos los niveles, mientras que
la hipótesis alternativa, $H_1$, es que hay diferencias estadísticamente significativas en al menos dos de las medias.
 Dicho contraste de
hipótesis se basa en la comparación de dos estimadores de la varianza total de los datos de la variable respuesta; de
ahí procede el nombre de esta técnica: \emph{ANOVA} (Analysis of Variance).

No obstante, en muchos problemas aparece no ya un único factor que permite clasificar los individuos de la muestra en
$k$ diferentes niveles, sino que pueden presentarse dos o más factores que permiten clasificar a los individuos de la
muestra en múltiples grupos según diferentes criterios, que se pueden analizar para ver si hay o no diferencias
significativas entre las medias de la variable respuesta.
Para tratar con este tipo de problemas surge el \emph{ANOVA Múltiples Factores} (o también \emph{ANOVA de Varias Vías})
como una generalización del proceso de un factor, que además de permitir el análisis de la influencia de cada uno de los
factores por separado también hace posible el estudio de la \emph{interacción} entre ellos.

Por otra parte, también son frecuentes los problemas en los que se toma más de una medida de una variable cuantitativa
(respuesta) en cada sujeto de la muestra, y se procede al análisis de las diferencias entre las diferentes medidas.
Si sólo se toman dos, el procedimiento adecuado es la T de Student de datos pareados, o su correspondiente no
paramétrico, el test de Wilcoxon; pero si se han tomado tres o más medidas, el test paramétrico correspondiente a la T
de Student de datos pareados es el \emph{ANOVA de Medidas Repetidas}.

Incluso también se puede dar el caso de un problema en el que se analice una misma variable cuantitativa medida en
varias ocasiones en cada sujeto de la muestra pero teniendo en cuenta a la vez la influencia de uno, dos o más factores
que permiten clasificar a los individuos en varios subgrupos diferentes.
En definitiva, pueden aparecer problemas donde a la par que un ANOVA de medidas repetidas se requiera realizar un ANOVA
de dos o más vías.

Por último, la situación más compleja que se puede plantear en el análisis de una respuesta cuantitativa se presenta
cuando, añadida a medidas repetidas y dos o más vías o factores de clasificación, se tienen una o más variables
cuantitativas, llamadas \emph{Covariables}, que se piensa que pueden influir en la variable respuesta.
Se procede entonces a realizar un \emph{ANCOVA} o \emph{Análisis de Covarianza}, con el que se pretende analizar la
influencia de los factores y también ver si hay diferencias entre las medidas repetidas pero habiendo eliminado
previamente la influencia (variabilidad) debida a la presencia de las covariables que se pretenden controlar.


\subsection{ANOVA de múltiples factores}
\subsubsection{ANOVA de dos factores con dos niveles cada factor}
Para entender qué es un ANOVA de múltiples factores, conviene partir de un caso sencillo con dos factores y dos niveles
en cada factor. Por ejemplo, se puede plantear un experimento con individuos que siguen o no una dieta (primer factor:
dieta, con dos niveles: sí y no), y que a su vez toman o no un determinado fármaco (segundo factor: fármaco, con dos
niveles: sí y no) para reducir su peso corporal (variable respuesta numérica: reducción del peso corporal expresada en
Kg). En esta situación, se generan cuatro grupos diferentes: los que no hacen dieta ni toman fármaco (No-No), los que no
hacen dieta pero sí toman fármaco (No-Sí), los que hacen dieta y no toman fármaco (Sí-No), y los que hacen dieta y toman
fármaco (Sí-Sí). Y se pueden plantear tres efectos diferentes:

\begin{itemize}
\item El de la dieta: viendo si hay o no diferencias significativas en los Kg perdidos entre los individuos que la han
seguido y los que no.
\item El del fármaco: viendo si hay o no diferencias significativas en los Kg perdidos entre los individuos que lo han
tomado y los que no.
\item El de la interacción: viendo si el efecto combinado de dieta y fármaco es diferente del que tendrían sumando sus
efectos por separado, y entonces se diría que sí que hay interacción; o si por el contrario el efecto de la combinación
de dieta y fármaco es el mismo que la suma de los efectos por separado, y entonces se diría que no hay interacción. A su
vez, si hay interacción se puede dar en dos sentidos: si la combinación de dieta y fármaco ha hecho perder más kilos a
los pacientes de los que cabría esperar con la suma de dieta y fármaco por separado, entonces la interacción de ambos
factores ha actuado en sinergia con los mismos, mientras que si la combinación ha hecho perder menos kilos de los que
cabría esperar con dieta y fármaco por separado, entonces la interacción ha actuado en antagonismo con ambos.
\end{itemize}

Siguiendo con el ejemplo, supongamos que la tabla que aparece a continuación refleja la media de Kg perdidos dentro de
cada uno de los grupos comentados. Por simplificar el ejemplo, no se reflejan los Kg en cada individuo con la
consiguiente variabilidad de los mismos, pero el ANOVA de dos vías sí que tendría en cuenta esa variabilidad para poder
hacer inferencia estadística, plantear contrastes de hipótesis y calcular sus correspondientes p-valores.

\begin{center}
\begin{tabular}{|l|c|c|}
\cline{2-3}
\multicolumn{1}{c|}{} & Fármaco No & Fármaco Sí \\
\hline
Dieta No & 0 & 5 \\
\hline
Dieta Sí & 3 & 8 \\
\hline
\end{tabular}
\end{center}

Si los resultados obtenidos fuesen los de la tabla anterior, se diría que no hay interacción entre fármaco y dieta, ya
que el efecto del fármaco en el grupo de los que no hacen dieta ha hecho perder 5 Kg en media a los individuos, el
efecto de la dieta en el grupo de los que no toman fármaco les ha hecho perder 3 Kg en media, y el efecto combinado de
dieta y fármaco ha hecho perder 8 Kg con respecto a los que no hacen dieta y tampoco toman fármaco. Estos 8 Kg son
iguales a la suma de 3 y 5, es decir iguales a la suma de los efectos de los factores por separado, sin ningún tipo de
interacción (de término añadido) que cambie el resultado de la suma.

Con las medias de los cuatro grupos que se generan en el cruce de los dos factores, cada uno con dos niveles
($2\times2$), se representan los gráficos de medias que aparecen más adelante. En estos gráficos, cuando no hay
interacción las rectas que unen las medias correspondientes a un mismo nivel de uno de los factores son paralelas dentro
de cierto margen de variabilidad.

\begin{figure}[h!]
\begin{center}
\scalebox{0.8}{%% Input file name: anova2/medias_sin_interaccion.fig
%% FIG version: 3.2
%% Orientation: Landscape
%% Justification: Flush Left
%% Units: Inches
%% Paper size: A4
%% Magnification: 100.0
%% Resolution: 1200ppi

\begin{pspicture}(5.84cm,3.48cm)(16.66cm,13.45cm)
\psset{unit=0.8cm}
%%
%% Depth: 2147483647
%%
\newrgbcolor{mycolor0}{1.00 0.50 0.31}\definecolor{mycolor0}{rgb}{1.00,0.50,0.31}
\newrgbcolor{mycolor1}{0.25 0.41 0.88}\definecolor{mycolor1}{rgb}{0.25,0.41,0.88}
%%
%% Depth: 100
%%
\psset{linestyle=dashed,linewidth=0.03175,linecolor=mycolor0,fillstyle=none}
\psline(11.50,6.88)(18.07,9.77)
\psset{linestyle=solid,linecolor=black,fillstyle=solid,fillcolor=mycolor0}
\pspolygon(11.22,6.72)(11.38,6.72)(11.38,6.88)(11.22,6.88)(11.22,6.72)
\pspolygon(18.19,9.78)(18.35,9.78)(18.35,9.94)(18.19,9.94)(18.19,9.78)
\psset{linecolor=black,fillstyle=none}
\psline(10.23,6.80)(10.23,14.95)
\psline(10.23,6.80)(10.02,6.80)
\psline(10.23,8.84)(10.02,8.84)
\psline(10.23,10.88)(10.02,10.88)
\psline(10.23,12.91)(10.02,12.91)
\psline(10.23,14.95)(10.02,14.95)
\rput{90}(9.73,6.80){0}
\rput{90}(9.73,8.84){2}
\rput{90}(9.73,10.88){4}
\rput{90}(9.73,12.91){6}
\rput{90}(9.73,14.95){8}
\psline(10.23,6.47)(20.31,6.47)(20.31,15.28)(10.23,15.28)(10.23,6.47)
\rput(15.27,15.99){Medias de peso perdido}
\rput(15.27,4.86){Dieta}
\rput{90}(8.88,10.88){Kilogramos}
\psset{linecolor=mycolor1}
\psline(11.50,11.98)(18.07,14.87)
\psset{linecolor=black,fillstyle=solid,fillcolor=mycolor1}
\pspolygon(11.22,11.90)(11.31,11.97)(11.38,11.90)(11.31,11.82)(11.22,11.90)
\pspolygon(18.19,14.95)(18.27,15.03)(18.35,14.95)(18.27,14.87)(18.19,14.95)
\rput(11.31,5.71){no}
\rput(18.27,5.71){si}
\rput[l](18.61,14.42){Fármaco}
\psset{linecolor=black,fillstyle=none}
\pspolygon(18.61,14.14)(18.61,12.87)(20.21,12.87)(20.21,14.14)(18.61,14.14)
\psset{linecolor=mycolor1}
\psline(18.71,13.71)(19.34,13.71)
\psset{linestyle=dashed,linecolor=mycolor0}
\psline(18.71,13.29)(19.34,13.29)
\psset{linestyle=solid,linecolor=black,fillstyle=solid,fillcolor=mycolor1}
\pspolygon(18.95,13.71)(19.03,13.79)(19.11,13.71)(19.03,13.63)(18.95,13.71)
\psset{fillcolor=mycolor0}
\pspolygon(18.95,13.21)(19.11,13.21)(19.11,13.37)(18.95,13.37)(18.95,13.21)
\rput[l](19.66,13.59){si}
\rput[l](19.66,13.16){no}
\end{pspicture}
%% End
}
\caption{Gráfico de medias de dos factores sin interacción}
\end{center}
\end{figure}

Por el contrario, también podría obtenerse una tabla en la que la suma de los efectos por separado fuese menor que el efecto combinado de dieta y fármaco:

\begin{center}
\begin{tabular}{|l|c|c|}
\cline{2-3}
\multicolumn{1}{c|}{} & Fármaco No & Fármaco Sí \\
\hline
Dieta No & 0 & 5 \\
\hline
Dieta Sí & 3 & 12 \\
\hline
\end{tabular}
\end{center}

En este caso, dejando al margen las variabilidad dentro de cada uno de los grupos y suponiendo que la misma es lo
suficientemente pequeña como para que las diferencias sean significativas, los 8 Kg en media que se perderían al sumar
los efectos por separado de dieta y fármaco son menores que los 12 que, en media, han perdido los individuos que han
tomado el fármaco y han seguido la dieta a la vez. Por lo tanto, se ha producido una interacción de los dos factores
que, al unirlos, ha servido para potenciar sus efectos por separado. Dicho de otra forma, para explicar el resultado
final de los individuos que han tomado el fármaco y también han seguido la dieta habría que introducir un nuevo término
en la suma, el término de interacción, que contribuiría con 4 Kg de pérdida añadidos a los 8 Kg que se perderían
considerando simplemente la suma de dieta y fármaco. Como este nuevo término contribuye a aumentar la pérdida que se
obtendría al sumar los efectos por separado de ambos factores, se trataría de un caso de interacción en sinergia con los
dos factores de partida.

\begin{figure}[h!]
\begin{center}
\scalebox{0.8}{%% Input file name: anova2/medias_con_interaccion_sinergica.fig
%% FIG version: 3.2
%% Orientation: Landscape
%% Justification: Flush Left
%% Units: Inches
%% Paper size: A4
%% Magnification: 100.0
%% Resolution: 1200ppi

\begin{pspicture}(5.84cm,3.48cm)(16.66cm,13.45cm)
\psset{unit=0.8cm}
%%
%% Depth: 2147483647
%%
\newrgbcolor{mycolor0}{1.00 0.50 0.31}\definecolor{mycolor0}{rgb}{1.00,0.50,0.31}
\newrgbcolor{mycolor1}{0.25 0.41 0.88}\definecolor{mycolor1}{rgb}{0.25,0.41,0.88}
%%
%% Depth: 100
%%
\psset{linestyle=dashed,linewidth=0.03175,linecolor=mycolor0,fillstyle=none}
\psline(11.51,6.86)(18.06,8.78)
\psset{linestyle=solid,linecolor=black,fillstyle=solid,fillcolor=mycolor0}
\pspolygon(11.22,6.72)(11.38,6.72)(11.38,6.88)(11.22,6.88)(11.22,6.72)
\pspolygon(18.19,8.76)(18.35,8.76)(18.35,8.92)(18.19,8.92)(18.19,8.76)
\psset{linecolor=black,fillstyle=none}
\psline(10.23,6.80)(10.23,14.95)
\psline(10.23,6.80)(10.02,6.80)
\psline(10.23,8.16)(10.02,8.16)
\psline(10.23,9.52)(10.02,9.52)
\psline(10.23,10.88)(10.02,10.88)
\psline(10.23,12.23)(10.02,12.23)
\psline(10.23,13.59)(10.02,13.59)
\psline(10.23,14.95)(10.02,14.95)
\rput{90}(9.73,6.80){0}
\rput{90}(9.73,8.16){2}
\rput{90}(9.73,9.52){4}
\rput{90}(9.73,10.88){6}
\rput{90}(9.73,12.23){8}
\rput{90}(9.73,13.59){10}
\rput{90}(9.73,14.95){12}
\psline(10.23,6.47)(20.31,6.47)(20.31,15.28)(10.23,15.28)(10.23,6.47)
\rput(15.27,15.99){Medias de peso perdido}
\rput(15.27,4.86){Dieta}
\rput{90}(8.88,10.88){Kilogramos}
\psset{linecolor=mycolor1}
\psline(11.48,10.32)(18.09,14.83)
\psset{linecolor=black,fillstyle=solid,fillcolor=mycolor1}
\pspolygon(11.22,10.20)(11.31,10.28)(11.38,10.20)(11.31,10.12)(11.22,10.20)
\pspolygon(18.19,14.95)(18.27,15.03)(18.35,14.95)(18.27,14.87)(18.19,14.95)
\rput(11.31,5.71){no}
\rput(18.27,5.71){si}
\rput[l](18.61,14.42){Fármaco}
\psset{linecolor=black,fillstyle=none}
\pspolygon(18.61,14.14)(18.61,12.87)(20.21,12.87)(20.21,14.14)(18.61,14.14)
\psset{linecolor=mycolor1}
\psline(18.71,13.71)(19.34,13.71)
\psset{linestyle=dashed,linecolor=mycolor0}
\psline(18.71,13.29)(19.34,13.29)
\psset{linestyle=solid,linecolor=black,fillstyle=solid,fillcolor=mycolor1}
\pspolygon(18.95,13.71)(19.03,13.79)(19.11,13.71)(19.03,13.63)(18.95,13.71)
\psset{fillcolor=mycolor0}
\pspolygon(18.95,13.21)(19.11,13.21)(19.11,13.37)(18.95,13.37)(18.95,13.21)
\rput[l](19.66,13.59){si}
\rput[l](19.66,13.16){no}
\end{pspicture}
%% End
}
\caption{Gráfico de medias de dos factores con interacción sinérgica.}
\end{center}
\end{figure}

Por último, también se podría obtener una tabla en la que la suma de los efectos por separado fuese mayor que el efecto combinado de los dos
factores:

\begin{center}
\begin{tabular}{|l|c|c|}
\cline{2-3}
\multicolumn{1}{c|}{} & Fármaco No & Fármaco Sí \\
\hline
Dieta No & 0 & 5 \\
\hline
Dieta Sí & 3 & 4 \\
\hline
\end{tabular}
\end{center}

Igualmente, en este nuevo ejemplo los 8 Kg en media que se perderían al sumar los efectos por separado de los dos
factores son mayores que los 4 que en realidad pierden, en media, los individuos que han seguido la dieta y utilizado el
fármaco. Por lo tanto, para explicar el resultado obtenido en el grupo de los que toman el fármaco y siguen la dieta
habría que introducir un término añadido a la suma de efectos sin más, que se restaría a los 8 Kg hasta dejarlos en 4
Kg. Se trataría de un caso de interacción en antagonismo con los dos factores de partida.

\begin{figure}[h!]
\begin{center}
\scalebox{0.8}{%% Input file name: anova2/medias_con_interaccion_antagonica.fig
%% FIG version: 3.2
%% Orientation: Landscape
%% Justification: Flush Left
%% Units: Inches
%% Paper size: A4
%% Magnification: 100.0
%% Resolution: 1200ppi

\begin{pspicture}(5.84cm,3.48cm)(16.66cm,13.45cm)
\psset{unit=0.8cm}
%%
%% Depth: 2147483647
%%
\newrgbcolor{mycolor0}{1.00 0.50 0.31}\definecolor{mycolor0}{rgb}{1.00,0.50,0.31}
\newrgbcolor{mycolor1}{0.25 0.41 0.88}\definecolor{mycolor1}{rgb}{0.25,0.41,0.88}
%%
%% Depth: 100
%%
\psset{linestyle=dashed,linewidth=0.03175,linecolor=mycolor0,fillstyle=none}
\psline(11.48,6.92)(18.09,11.57)
\psset{linestyle=solid,linecolor=black,fillstyle=solid,fillcolor=mycolor0}
\pspolygon(11.22,6.72)(11.38,6.72)(11.38,6.88)(11.22,6.88)(11.22,6.72)
\pspolygon(18.19,11.61)(18.35,11.61)(18.35,11.77)(18.19,11.77)(18.19,11.61)
\psset{linecolor=black,fillstyle=none}
\psline(10.23,6.80)(10.23,14.95)
\psline(10.23,6.80)(10.02,6.80)
\psline(10.23,8.43)(10.02,8.43)
\psline(10.23,10.06)(10.02,10.06)
\psline(10.23,11.69)(10.02,11.69)
\psline(10.23,13.32)(10.02,13.32)
\psline(10.23,14.95)(10.02,14.95)
\rput{90}(9.73,6.80){0}
\rput{90}(9.73,8.43){1}
\rput{90}(9.73,10.06){2}
\rput{90}(9.73,11.69){3}
\rput{90}(9.73,13.32){4}
\rput{90}(9.73,14.95){5}
\psline(10.23,6.47)(20.31,6.47)(20.31,15.28)(10.23,15.28)(10.23,6.47)
\rput(15.27,15.99){Medias de peso perdido}
\rput(15.27,4.86){Dieta}
\rput{90}(8.88,10.88){Kilogramos}
\psset{linecolor=mycolor1}
\psline(11.51,14.90)(18.06,13.37)
\psset{linecolor=black,fillstyle=solid,fillcolor=mycolor1}
\pspolygon(11.22,14.95)(11.31,15.03)(11.38,14.95)(11.31,14.87)(11.22,14.95)
\pspolygon(18.19,13.32)(18.27,13.40)(18.35,13.32)(18.27,13.24)(18.19,13.32)
\rput(11.31,5.71){no}
\rput(18.27,5.71){si}
\rput[l](18.61,14.42){Fármaco}
\psset{linecolor=black,fillstyle=none}
\pspolygon(18.61,14.14)(18.61,12.87)(20.21,12.87)(20.21,14.14)(18.61,14.14)
\psset{linecolor=mycolor1}
\psline(18.71,13.71)(19.34,13.71)
\psset{linestyle=dashed,linecolor=mycolor0}
\psline(18.71,13.29)(19.34,13.29)
\psset{linestyle=solid,linecolor=black,fillstyle=solid,fillcolor=mycolor1}
\pspolygon(18.95,13.71)(19.03,13.79)(19.11,13.71)(19.03,13.63)(18.95,13.71)
\psset{fillcolor=mycolor0}
\pspolygon(18.95,13.21)(19.11,13.21)(19.11,13.37)(18.95,13.37)(18.95,13.21)
\rput[l](19.66,13.59){si}
\rput[l](19.66,13.16){no}
\end{pspicture}
%% End
}
\caption{Gráfico de medias de dos factores con interacción antagónica.}
\end{center}
\end{figure}

En realidad, la interacción también puede producirse en sinergia con uno de los factores y en antagonismo con el otro,
ya que a veces los dos factores pueden producir un efecto con signo contrario. Por ejemplo, al hablar del factor dieta,
se tiende a pensar que se trata de una dieta que sirve para bajar el peso, pero también cabe plantearse un experimento
con personas que siguen una dieta de alto contenido calórico que en principio debería hacerles subir peso y ver qué
evolución siguen cuando a la vez toman un fármaco para bajarlo.

Como puede deducirse fácilmente de las tablas y gráficas anteriores, la presencia de interacción implica que la
diferencia entre las medias de los dos grupos dentro de un mismo nivel de uno de los factores no es la misma que para el
otro nivel. Por ejemplo, en la segunda tabla, la diferencia entre las medias de Kg perdidos entre los que sí que toman
el fármaco y los que no lo toman vale: 5-0=5 Kg en los que no hacen dieta, y 12-3=9 Kg en los que sí que hacen dieta. Lo
cual gráficamente se traduce en que la pendiente de la recta que une las medias dentro del grupo de los que sí que toman
el fármaco es diferente de la pendiente que une las medias dentro del grupo de los que no lo toman. En las ideas
anteriores se basará el planteamiento del contraste de hipótesis para ver si la interacción ha resultado o no
significativa.

Como ya se ha comentado, en cualquiera de las tablas anteriores se podrían analizar tres efectos diferentes: el de la
dieta, el del fármaco y el de la interacción de dieta con fármaco; lo cual, en términos matemáticos, se traduce en tres
contrastes de hipótesis diferentes:

\begin{enumerate}

\item Efecto de la dieta sobre la cantidad de peso perdido:
\begin{align*}
H_0&: \mu_{\text{con dieta}}=\mu_{\text{sin dieta}}\\
H_1&: \mu_{\text{con dieta}}\neq\mu_{\text{sin dieta}}
\end{align*}
\item Efecto del fármaco sobre la cantidad de peso perdido:
\begin{align*}
H_0&: \mu_{\text{con fármaco}}=\mu_{\text{sin fármaco}}\\
H_1&: \mu_{\text{con fármaco}}\neq\mu_{\text{sin fármaco}}
\end{align*}
\item Efecto de la interacción entre dieta y fármaco, que a su vez se puede plantear de dos formas equivalentes:

\begin{enumerate}
\item Viendo si dentro dentro de los grupos definidos en función de la dieta la diferencia de Kg perdidos entre los que
toman fármaco y los que no lo toman es la misma:
\begin{align*}
H_0&: (\mu_{\text{con fármaco}}-\mu_{\text{sin fármaco}})_{\text{sin dieta}}=(\mu_{\text{con fármaco}}-\mu_{\text{sin fármaco}})_{\text{con
dieta}}\\
H_1&: (\mu_{\text{con fármaco}}-\mu_{\text{sin fármaco}})_{\text{sin dieta}}\neq(\mu_{\text{con fármaco}}-\mu_{\text{sin
fármaco}})_{\text{con dieta}}
\end{align*}
\item Viendo si dentro de los grupos definidos en función del fármaco la diferencia de Kg perdidos entre los que hacen
dieta y los que no la hacen es la misma:
\begin{align*}
H_0&: (\mu_{\text{con dieta}}-\mu_{\text{sin dieta}})_{\text{sin fármaco}}=(\mu_{\text{con dieta}}-\mu_{\text{sin dieta}})_{\text{con
fármaco}}\\ 
H_1&: (\mu_{\text{con dieta}}-\mu_{\text{sin dieta}})_{\text{sin fármaco}}\neq(\mu_{\text{con dieta}}-\mu_{\text{sin
dieta}})_{\text{con fármaco}}
\end{align*}
\end{enumerate}
\end{enumerate}

Aunque los detalles matemáticos más precisos sobre cómo el ANOVA de dos o más vías da respuesta a los contrastes
expuestos quedan fuera del nivel de esta práctica, la idea general es sencilla y muy parecida a la explicada con más
detalle en la práctica de ANOVA de una vía.  En el ANOVA de una vía, la variabilidad total de los datos, expresada como
suma de distancias al cuadrado con respecto a la media global (llamada Suma de Cuadrados Total), se descompone en dos
diferentes fuentes de variabilidad: las distancias al cuadrado de los datos de cada grupo con respecto a la media del
grupo, \emph{Suma de Cuadrados Intra}, más las distancias al cuadrado entre las diferentes medias de los grupos y la
media general, \emph{Suma de Cuadrados Inter}. La suma de cuadrados intra-grupos es también llamada \emph{Variabilidad
Residual} o \emph{Suma de Cuadrados Residual}, ya que su cuantía es una medida de la dispersión residual, remanente
incluso después de haber dividido los datos en grupos. Estas sumas de cuadrados, una vez divididas por sus
correspondientes grados de libertad, generan varianzas llamadas \emph{Cuadrados Medios}, y el cociente de cuadrados
medios (cuadrado medio inter dividido entre cuadrado medio intra) bajo la hipótesis nula de igualdad de medias en todos
los grupos sigue una distribución \emph{F} de Fisher que se puede utilizar para calcular un $p$-valor del contraste de
igualdad de medias. En el ANOVA de dos factores, en lugar de dos fuentes de variabilidad tenemos cuatro: una por el
primer factor, otra por el segundo, otra por la interacción y otra más que contempla la variabilidad residual o
variabilidad intragrupos. En el ejemplo anterior, las cuatro fuentes de variabilidad son:

\begin{enumerate}
\item La debida al primer factor: la dieta.
\item La debida al segundo factor: el fármaco.
\item La debida a la interacción entre ambos.
\item La residual.
\end{enumerate}

Las tres primeras fuentes de variabilidad llevan asociadas sus correspondientes sumas de cuadrados, similares a la suma
de cuadrados inter del ANOVA de una vía, mientras que la variabilidad residual lleva asociada su suma de cuadrados
residual, similar a la suma de cuadrados intra del ANOVA de una vía. Dividiendo las sumas de cuadrados entre sus
respectivos grados de libertad se obtienen varianzas, que divididas entre la varianza residual generan, bajo la
hipótesis nula de igualdad de medias, valores \emph{f} de la distribución \emph{F} de Fisher que pueden utilizarse para
calcular el p-valor del correspondiente contraste.

Lo anterior se resume en forma de tabla de un ANOVA de dos vías, considerando un primer factor con $k_1$ niveles, un
segundo factor con $k_2$ niveles y un total de datos $n$. Si se denomina $F_1$ al primer factor, $F_2$ al segundo, $I$ a
la interacción y $R$ al residual, la tabla de un ANOVA de dos vías tiene la siguiente forma:

\[
\resizebox{\textwidth}{!}{
\renewcommand{\arraystretch}{2}
\begin{array}{cccccc}
\hline
\text{Fuente} & \text{Suma Cuadrados} & \text {Grados Libertad} & \text{Cuadrados Medios} & \text{Estadístico $f$} & \text{$p$-valor}\\
\hline
F_1 & SF_1 & k_1-1 & CF_1=\frac{SF_1}{k_1-1} & f_1=\frac{CF_1}{CR} & P(F>f_1) \\
F_2 & SF_2 & k_2-1 & CF_2=\frac{SF_2}{k_2-1} & f_2=\frac{CF_2}{CR} & P(F>f_2) \\
\text{Interacción} & SI & (k_1-1)(k_2-1) & CI=\frac{SI}{(k_1-1)(k_2-1)} & f_I=\frac{CI}{CR} & P(F>f_I) \\
\text{Residual} & SR & n-k_1k_2 & CR=\frac{SR}{n-k_1k_2} &  &  \\
\hline
\text{Total} & ST & n-1 &  &  & 
\end{array}
}
\]

Una vez obtenida la tabla, habitualmente mediante un programa de estadística para evitar realizar la gran cantidad de
cálculos que conlleva (los distintos programas pueden proporcionar tablas ligeramente diferentes a la expuesta en esta
práctica, en las que pueden aparecer filas añadidas cuya interpretación dependerá del programa utilizado), el siguiente
paso es la interpretación de los $p$-valores obtenidos en cada uno de los factores y en la interacción. Para ello,
resulta clave el $p$-valor de la interacción porque condicionará completamente el análisis:
\begin{itemize}
\item Si la interacción no ha resultado significativa ($p$-valor de la interacción mayor que el nivel de significación,
habitualmente $0.05$), se puede considerar por separado la actuación de los dos factores y ver si hay o no diferencias
significativas en sus niveles atendiendo al $p$-valor que aparece en la tabla para cada uno de ellos. Por ejemplo, en la
primera de las tablas del análisis de Kg perdidos en función de la dieta y el fármaco, se obtendría que la interacción
no es significativa, lo cual implicaría que habría que analizar el efecto de los factores por separado. Para ello, se
acudiría al $p$-valor del factor dieta y si es menor que el nivel de significación fijado, entonces el factor dieta
habría resultado significativo, lo cual quiere decir que habría diferencias significativas (más allá de las asumibles
por azar) entre los Kg perdidos por los individuos que hacen dieta y los que no; y todo ello, independientemente de si
los individuos están tomando o no el fármaco, ya que no hay una interacción significativa que ligue los resultados de la
dieta con el fármaco.
Igualmente, con el factor fármaco, se acudiría a su $p$-valor y se vería si hay o no diferencias significativas entre
los Kg perdidos por los que toman el fármaco y los que no lo hacen, independientemente de si siguen o no la dieta.
\item Si la interacción ha resultado significativa ($p$-valor de la interacción menor que el nivel de significación,
habitualmente $0.05$), no se puede considerar por separado la actuación de los dos factores, la presencia de uno de los
factores condiciona lo que sucede en el otro y el análisis de diferencias debidas al segundo factor debe realizarse por
separado dentro de cada uno de los niveles del primero; y a la inversa, el análisis de diferencias debidas al primero
debe realizarse por separado dentro de cada uno de los niveles del segundo. Por ejemplo, en la segunda de las tablas del
análisis de Kg perdidos en función de la dieta y el fármaco, muy probablemente se obtendría que la interacción sí que es
significativa, con lo cual no habría un único efecto del fármaco: en el grupo de los que no toman el fármaco, la
diferencia de Kg perdidos entre los que sí que hacen dieta y los que no  la hacen no sería la misma que en el grupo de
los que sí que toman el fármaco. E igualmente, tampoco habría un único efecto de la dieta: en el grupo de los que no
hacen dieta, la diferencia de Kg perdidos entre los que sí que toman el fármaco y los que no lo hacen no sería la misma
que en el grupo de los que sí que hacen dieta.
\end{itemize}

Una aclaración final importante es que en ningún caso un ANOVA de dos factores con dos niveles en cada vía equivale a
hacer por separado una T de Student de datos independientes en cada uno de los factores. Ni siquiera en el caso de que
no haya interacción el $p$-valor que se obtiene en cada uno de los dos factores coincide con el que se obtendría en la
comparación de los niveles mediante la T de Student. El ANOVA de dos factores es una técnica multivariante que
cuantifica la influencia de cada una de las variables independientes en la variable dependiente después de haber
eliminado la parte de la variabilidad que se debe a las otras variables independientes que forman parte del modelo. En
el ejemplo de los Kg perdidos, no sería lo mismo analizar la influencia de la variable dieta después de eliminar la
variabilidad explicada mediante la variable fármaco e incluso la interacción entre dieta y fármaco, que es lo que haría
el ANOVA de dos factores, que analizar simplemente la influencia de la variable dieta sin más, o fármaco sin más, que es
lo que podríamos hacer mediante una T de Student de datos independientes. Tampoco el análisis de la interacción en el
ANOVA de dos factores equivale a realizar un ANOVA de una vía considerando una nueva variable independiente con cuatro
categorías diferentes (1:Sí-Sí, 2:Sí-No, 3:No-Sí, 4:No-No), por el mismo motivo:
las conclusiones del ANOVA de dos vías hay que entenderlas en el contexto de una técnica multivariante en que la
importancia de cada variable independiente se obtiene después de eliminar de los datos la variabilidad debida a las
demás.


\subsubsection{ANOVA de dos factores con tres o más niveles en algún factor}
El planteamiento y resolución de un ANOVA de dos factores con tres o más niveles en algún factor es muy parecido al ya
expuesto de dos niveles en cada factor. Únicamente cambian ligeramente las hipótesis nulas planteadas en los factores en
las que habría que incluir la igualdad de tantas medias como niveles tenga el factor analizado, y las alternativas en
las que se supone que alguna de las medias es diferente. En cuanto a las interacciones, también se contemplarían
diferencias de medias pero teniendo en cuenta que hay más diferencias posibles al tener más niveles dentro de cada
factor.

En cuanto a la interpretación final de los resultados de la tabla del ANOVA, si no hay interacción y sin embargo hay
diferencias significativas en cualquiera de los factores con 3 o más niveles, el siguiente paso sería ver entre qué
medias se dan esas diferencias. Por ejemplo, si no hay interacción y se ha rechazado la hipótesis nula de igualdad de
medias entre los tres niveles del factor 1, habría que ver si esas diferencias aparecen entre los niveles 1 y 2, o entre
el 1 y 3, e incluso entre el 2 y el 3, independientemente del factor 2; e igualmente con el factor 2. Para poder ver
entre qué niveles hay diferencias, habría que realizar \emph{Test de Comparaciones Múltiples y por Parejas}; por ejemplo
un test de Bonferroni o cualquier otro de los vistos en la práctica de ANOVA de una vía. Si la interacción saliese
significativa, habría que hacer lo mismo pero considerando las posibles diferencias entre los 3 niveles del factor 1
dentro de cada nivel del factor 2 y viceversa.

Como ya se ha comentado para el ANOVA de dos factores con dos niveles en cada factor y la T de Student de datos
independientes, igualmente el ANOVA de dos factores con tres o más niveles en algún factor no equivale a dos ANOVAS de
una vía. El $p$-valor que se obtiene en el de dos factores no es el mismo que que se obtendría en los ANOVAS de una vía
realizados teniendo en cuenta cada uno de los factores por separado, incluso si la interacción no es significativa.


\subsubsection{ANOVA de tres o más factores}
Aunque los fundamentos del ANOVA de tres o más factores son muy parecidos a los de dos y la tabla obtenida es muy
similar, la complejidad en la interpretación sube un escalón. Por ejemplo, en un ANOVA de tres factores la tabla
presentaría los tres efectos de cada uno de los factores por separado, las tres interacciones dobles (1 con 2, 1 con 3 y
2 con 3), e incluso también podría mostrar la interacción triple (los programas de estadística permiten considerar o no
las interacciones de cualquier orden). Si la interacción triple fuese significativa, entonces no se podría hablar del
efecto general del factor 1, sino que habría que analizar el efecto del factor 1 dentro de cada nivel del 2 y a su vez
dentro de cada nivel del 3, y así sucesivamente. Si la interacción triple no fuese significativa pero sí que lo fuese la
del factor 1 con el 2, entonces habría que analizar el efecto del factor 1 dentro de cada uno de los niveles del 2 pero
independientemente del factor 3. Y así hasta completar un conjunto muy grande de análisis posibles y de Test de
Comparaciones Múltiples aplicados. No obstante, es el propio experimentador el que debe limitar el conjunto de análisis
a realizar con un planteamiento muy claro del experimento, reduciendo en la medida de lo posible el número de factores
considerados y teniendo claro que no merece la pena considerar interacciones triples, o de órdenes superiores, si no hay
forma clara de interpretar su resultado.

En ningún caso un ANOVA de tres o más factores equivale a tres ANOVAS de una vía realizados teniendo en cuenta los
factores considerados por separado.


\subsubsection{Factores fijos y Factores aleatorios}
A la hora de realizar un ANOVA de varios factores, el tratamiento de la variabilidad debida a cada uno de ellos y
también las conclusiones que se pueden obtener después de realizarlo, son diferentes dependiendo de que los factores
sean fijos o aleatorios.

Se entiende como \emph{Factor Fijo o Factor de Efectos Fijos} aquel cuyos niveles los establece, los fija de antemano,
el investigador (por ejemplo, cantidades concretas de fármaco o de tiempo transcurrido), o vienen dados por la propia
naturaleza del factor (por ejemplo, el sexo o la dieta). Su variabilidad es más fácil de controlar y también resulta más
sencillo su tratamiento en los cálculos que hay que hacer para llegar a la tabla final del ANOVA, pero tienen el
problema de que los niveles concretos que toma el factor constituyen la población de niveles sobre los que se hace
inferencia. Es decir, no se pueden sacar conclusiones poblacionales que no se refieran a esos niveles fijos con los que
se ha trabajado.

Por contra, un \emph{Factor Aleatorio o Factor de Efectos Aleatorios} es aquel cuyos niveles son seleccionados de forma
aleatoria entre todos los posibles niveles del factor (por ejemplo, cantidad de fármaco, con niveles 23 mg, 132 mg y 245
mg, obtenidos al escoger 3 niveles de forma aleatoria entre 0 y 250 mg). Su tratamiento es más complicado, pero al
constituir una muestra aleatoria de niveles, se pretende sacar conclusiones extrapolables a todos los niveles posibles.


\subsubsection{Supuestos del modelo de ANOVA de dos o más vías}
Como ya sucedía con el ANOVA de una vía, el de dos o más vías es un test paramétrico que supone que:
\begin{itemize}
\item Los qdatos deben seguir distribuciones normales dentro de cada categoría, entendiendo por categorías todas las que
se forman del cruce de todos los niveles de todos los factores. Por ejemplo, en un ANOVA de 2 factores con 3 niveles en
cada factor, se tienen $3^2$ categorías diferentes.
\item Todas las distribuciones normales deben tener igualdad de varianzas (homocedasticidad).
\end{itemize}

Cuando no se cumplen las condiciones anteriores y además las muestras son pequeñas, no se debería aplicar el ANOVA de
dos o más vías, con el problema añadido de que no hay un test no paramétrico que lo sustituya. Mediante test no
paramétricos (sobre todo mediante el test de Kruskall-Wallis) se podría controlar la influencia de cada uno de los
factores por separado en los datos, pero nunca el importantísimo papel de la interacción.


\subsection{ANOVA de medidas repetidas}

\subsubsection{Concepto de ANOVA de medidas repetidas}
En muchos problemas se cuantifica el valor de una variable dependiente en varias ocasiones en el mismo sujeto (por
ejemplo: en un grupo de individuos que están siguiendo una misma dieta, se puede anotar el peso perdido al cabo de un
mes, al cabo de dos y al cabo de tres), y se intenta comparar la media de esa variable en las diferentes ocasiones en
que se ha medido, es decir, ver si ha habido una evolución de la variable a lo largo de las diferentes medidas (en el
ejemplo anterior, una evolución del peso perdido). Conceptualmente es una situación análoga a la estudiada al comparar
dos medias con datos emparejados mediante una T de Student de datos emparejados, o su correspondiente no paramétrico, el
test de Wilcoxon, pero ahora hay más de dos medidas emparejadas, realizadas en el mismo individuo. En estas situaciones
se utiliza el ANOVA de medidas repetidas.

El ANOVA de medidas repetidas, como también sucede con cualquier otro test que utilice datos emparejados, tiene la
ventaja de que las comparaciones que se realizan están basadas en lo que sucede dentro de cada sujeto (intra-sujetos),
lo cual reduce el ruido o variabilidad que se produce en comparaciones entre diferentes grupos de sujetos. Por ejemplo,
en el estudio sobre la evolución del peso perdido con personas que siguen la misma dieta, se podría haber cuantificado
la variable al cabo de uno, dos y tres meses, pero en tres grupos diferentes que hubiesen seguido la misma dieta, pero
con este diseño del estudio no se controlan otras variables que pueden influir en el resultado final, por ejemplo el
sexo, la edad, o la cantidad de ejercicio que se hace al día. Dicho de otra forma, en el diseño con grupos
independientes es posible que alguno de los grupos tenga una media de edad superior, o no haya igual número de hombres
que de mujeres, y todo ello tener su reflejo en el número de Kg perdidos. Mientras que, con el diseño de datos
emparejados, la segunda medida se compara con la primera que también se ha realizado en la misma persona, y por lo tanto
es igual su sexo, su edad y la cantidad de deporte que realiza; y así con todas las demás medidas que se comparan entre
sí pero dentro del mismo individuo. Eso permite controlar la variabilidad y detectar pequeñas diferencias que de otra
forma serían indetectables.

\subsubsection{ANOVA de medidas repetidas como ANOVA de dos vías sin interacción}
El ANOVA de medidas repetidas puede realizarse como un ANOVA de dos vías sin interacción sin más que realizar los
cálculos oportunos introduciendo adecuadamente los datos en un programa estadístico.

En la situación de partida, si suponemos que tenemos $k$ medidas emparejadas de una variable dependiente numérica y $n$
individuos en los que hemos tomado las medidas, los datos se pueden organizar como aparecen en la tabla siguientes:

\[
\begin{array}{|c|c|c|c|c|}
\cline{2-5}
\multicolumn{1}{c|}{} & \text{Medida 1} & \text{Medida 2} & \ldots & \text{Medida $k$} \\
\hline
\text{Individuo 1} & x_{1,1} & x_{1,2} & \cdots & x_{1,k} \\
\hline
\text{Individuo 2} & x_{2,1} & x_{2,2} & \cdots & x_{2,k} \\
\hline
\vdots & \vdots & \vdots & \ddots & \vdots\\
\hline
\text{Individuo $n$} & x_{n,1} & x_{n,2} & \cdots & x_{n,k} \\
\hline
\end{array}
\]

Pero esos mismos datos también se pueden ordenar en un formato de tabla mucho más conveniente para poderles aplicar un
ANOVA de dos vías:

\[
\begin{array}{|l|c|c|c|}
\cline{2-4}
\multicolumn{1}{c|}{} & \text{Variable Dependiente} & \text{Individuo} & \text{Medida} \\
\hline
\text{Fila 1} & x_{1,1} & 1 & 1 \\
\hline
\text{Fila 2} & x_{2,1} & 2 & 1 \\
\hline
\vdots & \vdots & \vdots & \vdots \\
\hline
\text{Fila $n$} & x_{n,1} & n & 1 \\
\hline
\text{Fila $n+1$} & x_{1,2} & 1 & 2 \\
\hline
\text{Fila $n+2$} & x_{2,2} & 2 & 2 \\
\hline
\vdots & \vdots & \vdots & \vdots \\
\hline
\text{Fila $2n$} & x_{n,2} & n & 2 \\
\hline
\vdots & \vdots & \vdots & \vdots \\
\hline
\text{Fila $(k-1)n+1$} & x_{1,k} & 1 & k \\
\hline
\text{Fila $(k-1)n+2$} & x_{2,k} & 2 & k \\
\hline
\vdots & \vdots & \vdots & \vdots \\
\hline
\text{Fila $kn$} & x_{n,k} & n & k \\
\hline
\end{array}
\]

Con ello, tanto Individuo como Medida son variables categóricas que dividen la muestra total ($n\cdot k$ datos de la
variable dependiente) en grupos: $n$ grupos en la variable Individuo y $k$ grupos en la variable Medida. Además,
considerando el cruce de ambas variables (Medida x Individuo) se forman $n\cdot k$ grupos con un único dato de la
variable dependiente en cada grupo.

Para explicar la variabilidad de los datos de la variable dependiente cuantitativa se pueden considerar tres fuentes: la
debida a la variable Medida, la debida a la variable Individuo, y la residual. Ahora no cabe hablar de la variabilidad
debida a la interacción entre Medida e Individuo ya que los grupos que surgen del cruce de los dos factores sólo tienen
un dato y no es viable calcular medias y dispersiones dentro de un grupo con un único dato. Y el análisis de la
influencia de cada uno de los factores se realiza mediante un ANOVA de dos factores sin interacción, que genera la
siguiente tabla:


\[
\resizebox{\textwidth}{!}{
\renewcommand{\arraystretch}{2}
\begin{array}{cccccc}
\hline
\text{Fuente} & \text{Suma Cuadrados} & \text {Grados Libertad} & \text{Cuadrados Medios} & \text{Estadístico $f$} & \text{$p$-valor}\\
\hline
F_1=\text{Medida} & SF_1 & k-1 & CF_1=\frac{SF_1}{k-1} & f_1=\frac{CF_1}{CR} & P(F>f_1) \\
F_2=\text{Individuo} & SF_2 & n-1 & CF_2=\frac{SF_2}{n-1} & f_2=\frac{CF_2}{CR} & P(F>f_2) \\
\text{Residual} & SR & nk-n-k+1 & CR=\frac{SR}{nk-n-k+1} &  &  \\
\hline
\text{Total} & ST & n-1 &  &  & 
\end{array}
}
\]


Y permite dar respuesta a los siguientes contrastes:

\begin{enumerate}
\item En la variable Medida:
\begin{align*}
H_0&: \mu_{\text{Medida 1}}=\mu_{\text{Medida 2}}=...=\mu_{\text{Medida k}}\\
H_1&: \text{Alguna de las medias es diferente.}
\end{align*}

Si el $p$-valor obtenido es menor que el nivel de significación fijado querrá decir que alguna de las medias es significativamente diferente del resto. Este es el contraste más importante del ANOVA de medidas repetidas y supone que la variabilidad dentro de cada individuo (intra-sujeto) es lo suficientemente grande como para que se descarte el azar como su causa. Por lo tanto la variable Medida ha tenido un efecto significativo.

\item En la variable Individuo:
\begin{align*}
H_0&: \mu_{\text{Individuo 1}}=\mu_{\text{Individuo 2}}=...=\mu_{\text{Individuo n}}\\
H_1&: \text{Alguna de las medias es diferente.}
\end{align*}

Si el $p$-valor obtenido es menor que el nivel de significación fijado querrá decir que alguna de las medias es significativamente diferente del resto, y por lo tanto alguno de los individuos analizados ha tenido un comportamiento en la variable dependiente diferente del resto. En realidad no es un contraste importante en el ANOVA de medidas repetidas ya que supone un análisis de la variabilidad entre individuos (inter-sujetos), pero es muy difícil que en un experimento dado esta variabilidad no esté presente.
\end{enumerate}

Si la conclusión del ANOVA es que hay que rechazar alguna de las dos hipótesis nulas, ya sea la de igualdad de medias en
los grupos formados por la variable Medida o la de igualdad de medias en los grupos formados por la variable Individuo,
entonces en el siguiente paso se podría aplicar un Test de Comparaciones Múltiples y por Parejas, por ejemplo un test de
Bonferroni, para ver qué medias son diferentes, especialmente para ver entre qué niveles del la variable Medida se dan
las diferencias.


\subsubsection{Supuestos del ANOVA de medidas repetidas}
Como en cualquier otro ANOVA, en el de medidas repetidas se exige que:

\begin{itemize}
\item Los datos de la variable dependiente deben seguir distribuciones normales dentro de cada grupo, ya sea formado por
la variable Medida o por la variable Individuo. Como el contraste más importante se realiza en la variable Medida,
resultará especialmente importante que sean normales las distribuciones de todas las Medidas .

\item Todas las distribuciones normales deben tener igualdad de varianzas (homocedasticidad), especialmente las de las
diferentes Medidas.
\end{itemize}

Cuando en un ANOVA de medidas repetidas se cumple la normalidad y la homocedasticidad de todas las distribuciones se
dice que se cumple la \emph{Esfericidad} de los datos, y hay tests estadísticos especialmente diseñados para contrastar
la esfericidad como la \emph{prueba de Mauchly}.

Cuando no se cumplen las condiciones anteriores y además las muestras son pequeñas, no se debería aplicar el ANOVA de
medidas repetidas, pero al menos sí que hay una prueba no paramétrica que permite realizar el contraste de si hay o no
diferencias significativas entre los distintos niveles de la variable Medida, que es el \emph{test de Friedman}.


\subsection{ANOVA de medidas repetidas + ANOVA de una o más vías}
No son pocos los problemas en los que, además de analizar el efecto intra-sujetos en una variable dependiente
cuantitativa medida varias veces en los mismos individuos para el que cabría plantear un ANOVA de medidas repetidas,
también aparecen variables cualitativas que se piensa que pueden estar relacionadas con la variable dependiente. Estas
últimas variables introducen un efecto que aunque habitualmente es catalogado como inter-sujetos más bien se trataría de
un efecto inter-grupos, ya que permiten definir grupos entre los que se podría plantear un ANOVA de una o más vías. Por
ejemplo, se podría analizar la pérdida de peso en una muestra de individuos al cabo de uno, dos y tres meses de
tratamiento (ANOVA de medidas repetidas), pero teniendo en cuenta que los individuos de la muestra han sido divididos en
seis grupos que se forman por el cruce de dos factores, Dieta y Ejercicio, con tres dietas diferentes: a, b y c, y dos
niveles de ejercicio físico diferentes: bajo y alto. Para analizar la influencia de estos dos factores inter-sujetos,
habría que plantear un ANOVA de dos vías con interacción. Para un ejemplo como el comentado, aunque los datos podrían
disponerse de una forma similar a la que permite realizar el ANOVA de medidas repetidas como un ANOVA de dos factores
(variables Medida e Individuo), y añadirle dos factores más (Dieta y Ejercicio), no resulta cómodo tener que introducir
en la matriz de datos varias filas para un mismo individuo (tantas como medidas repetidas diferentes se hayan
realizado). Por ello, determinados programas de estadística, como R, permiten realizar ANOVAS de medidas repetidas
introduciendo los datos en el formato clásico, una fila para cada individuo y una variable para cada una de las medidas
repetidas, definiendo factores intra-sujeto que en realidad estarían compuestos por todas las variables que forman parte
de las medidas repetidas. Además, a los factores intra-sujeto permiten añadirle nuevos factores inter-sujeto
(categorías) que pueden influir en las variables respuesta (las diferentes medidas), e incluso comprobar si hay o no
interacción entre los factores inter-sujeto entre sí y con los factores intra-sujeto. Por lo tanto, son procedimientos
que realizan a la vez un ANOVA de medidas repetidas y un ANOVA de una o más vías, con la ventaja de que se pueden
introducir los datos en la forma clásica: una fila para cada individuo.

El resultado de la aplicación de estos procedimientos es muy parecido a los comentados en apartados previos: se generan
tablas de ANOVA en las que se calcula un $p$-valor para cada uno de los factores, ya sean intra-sujeto (medidas
repetidas) o inter-sujeto (categorías), y también para la interacción, ya sea de los factores inter-sujeto entre sí o de
factores inter-sujeto con los intra-sujeto.

\clearpage
\newpage


\section{Ejercicios resueltos}
\begin{enumerate}[leftmargin=*] \item En un estudio diseñado para analizar la influencia de un tipo de dieta y de un
fármaco en el peso corporal perdido, expresado en Kg, se ha anotado el número de Kg perdidos en un grupo de personas al
cabo de 3 meses de dieta y de tomar el fármaco, obteniendo los siguientes resultados (si algún individuo presenta un
dato negativo significa que en lugar de perder Kg de peso los ha ganado):
\[
\begin{array}{|l|c|c|}
\cline{2-3}
\multicolumn{1}{l|}{} & \text{Fármaco NO} & \text{Fármaco SÍ} \\
\hline
\text{Dieta NO} & 1.5\quad 0.5\quad 0.0\quad -1.0\quad -1.0 & 6.5\quad 5.0\quad 7.0\quad 3.0\quad 4.5\quad 4.0 \\
\hline
\text{Dieta SÍ} & 3.5\quad 3.0\quad 4.0\quad 2.5\quad 2.0 & 9.5\quad 8.0\quad 7.5\quad 7.0\quad 8.5\quad 7.5 \\
\hline
\end{array}
\]

\begin{enumerate}
\item Crear un conjunto de datos \variable{dieta.farmaco} con las variables \variable{kilos\_perdidos}, \variable{dieta}
y \variable{farmaco}.

\item Mostrar el diagrama de interacción de los kilos perdidos para los distintos grupos según la dieta y el fármaco.
¿Qué conclusiones cualitativas pueden sacarse del gráfico obtenido? 
\begin{indicacion}
\begin{enumerate}
\item Seleccionar el menú \menu{Teaching > Gráficos > Diagrama de interacción}.
\item Seleccionar la variable \variable{kilos\_perdidos} en el campo \campo{Variable respuesta}, la variable
\variable{dieta} en el campo \campo{Grupos eje X} y la variable \variable{farmaco} en el campo \campo{Grupos de
trazado}, y hacer click en el botón \boton{Aceptar}.
\end{enumerate}
Se observa claramente que no hay interacción (líneas paralelas), que los dos puntos del grupo de los que no hacen dieta
están por debajo de los que sí que la hacen, lo cual hace sospechar que el factor dieta será significativo, e igualmente
los dos puntos de los que no toman fármaco están por debajo de los que sí que lo toman, lo cual hace sospechar que el
factor fármaco también será significativo.
\end{indicacion}  


\item Realizar un contraste de ANOVA de dos factores con los datos e interpretar la tabla de ANOVA obtenida.
\begin{indicacion}
\begin{enumerate}
\item Seleccionar el menú \texttt{Teaching > Tests paramétricos > Medias > ANOVA}.
\item En el cuadro de diálogo que aparece, seleccionar el conjunto de datos \variable{dieta.farmaco}.
\item Seleccionar la variable \variable{kilos\_perdidos} como \campo{Variable dependiente} y las variables
\variable{dieta} y \variable{farmaco} en el campo \campo{Factores entre individuos}, y hacer click sobre el botón
\boton{Aceptar}.
\end{enumerate}

Para la interpretación de la tabla de ANOVA, prestar especial atención a las siguientes líneas de la tabla:
\begin{enumerate}
\item \resultado{dieta}: muestra si la dieta resulta o no significativa para explicar la variabilidad del peso perdido.
\item \resultado{farmaco}: muestra si el fármaco resulta o no significativo.
\item \resultado{dieta:farmaco}: muestra si la interacción de dieta y fármaco resulta o no significativa.
\end{enumerate}
Una conclusión muy importante a la luz de los resultados es que no hay una interacción significativa entre dieta y
fármaco, es decir que el efecto del fármaco no dependerá de si una persona toma o no dieta, y a la inversa, que el
efecto de la dieta no dependerá de si se toma o no fármaco.
\end{indicacion}

%\item Calcular las medias y desviaciones típicas de los Kg perdidos en todos los grupos.
%\begin{indicacion}
%El procedimiento anterior para obtener el contraste de ANOVA también muestra las medias y desviaciones típicas para cada grupo.
%}
%\end{indicacion}


\item Teniendo en cuenta que no hay interacción significativa, calcular el intervalo de confianza para la diferencia de
medias en los kg perdidos según la variable dieta e igualmente con la variable fármaco.
\begin{indicacion}
\begin{enumerate}
\item Seleccionar el menú \texttt{Teaching > Tests paramétricos > Medias > ANOVA}.
\item En el cuadro de diálogo que aparece, seleccionar el conjunto de datos \variable{dieta.farmaco}.
\item Seleccionar la variable \variable{kilos\_perdidos} como \campo{Variable
dependiente} y las variables \variable{dieta} y \variable{farmaco} en el campo \campo{Factores entre individuos}.
\item En la solapa \menu{Comparación por pares} seleccionar la opción \opcion{Comparación de medias por pares} y hacer
click sobre el botón \boton{Aceptar}.
\end{enumerate}
\end{indicacion}
\end{enumerate}


\item En un estudio diseñado para analizar la influencia de un tipo de dieta y de un fármaco en el peso corporal
perdido, expresado en Kg, se ha anotado el número de Kg perdidos en un grupo de personas al cabo de 3 meses de dieta y
de tomar el fármaco, obteniendo los siguientes resultados (si algún individuo presenta un dato negativo significa que en
lugar de perder Kg de peso los ha ganado):

\[
\begin{array}{|l|c|c|}
\cline{2-3}
\multicolumn{1}{l|}{} & \text{Fármaco NO} & \text{Fármaco SÍ} \\
\hline
\text{Dieta NO} & 1.5\quad 0.5\quad 0.0\quad -1.0\quad -1.0 & 6.5\quad 5.0\quad 7.0\quad 3.0\quad 4.5\quad 4.0 \\
\hline
\text{Dieta SÍ} & 3.5\quad 3.0\quad 4.0\quad 2.5\quad 2.0 & 12.5\quad 12.0\quad 11.5\quad 13.5\quad 12.5\quad 10.0 \\
\hline
\end{array}
\]

\begin{enumerate}
\item Crear un conjunto de datos \variable{dieta.farmaco} con las variables \variable{kilos\_perdidos}, \variable{dieta}
y \variable{farmaco}.

\item Mostrar el gráfico de interacción de los kilos perdidos para los distintos grupos según la dieta y el fármaco.
¿Qué conclusiones cualitativas pueden sacarse del gráfico obtenido? 
\begin{indicacion}
\begin{enumerate}
\item Seleccionar el menú \menu{Teaching > Gráficos > Diagrama de interacción}.
\item Seleccionar la variable \variable{kilos\_perdidos} en el campo \campo{Variable respuesta}, la variable
\variable{dieta} en el campo \campo{Grupos eje X} y la variable \variable{farmaco} en el campo \campo{Grupos de
trazado}, y hacer click en el botón \boton{Aceptar}.
\end{enumerate}
Ahora se observa claramente que hay interacción (líneas no paralelas), que los dos puntos del grupo de los que no hacen
dieta están por debajo de los que sí que la hacen, lo cual hace sospechar que el factor dieta será significativo, e
igualmente los dos puntos de los que no toman fármaco están por debajo de los que sí que lo toman, lo cual hace
sospechar que el factor fármaco también será significativo.
\end{indicacion}  

\item Realizar un contraste de ANOVA de dos factores con los datos e interpretar la tabla de ANOVA obtenida. 
¿Hay interacción significativa?
¿Cómo se interpretaría? 
\begin{indicacion}
\begin{enumerate}
\item Seleccionar el menú \texttt{Teaching > Tests paramétricos > Medias > ANOVA}.
\item En el cuadro de diálogo que aparece, seleccionar el conjunto de datos \variable{dieta.farmaco}.
\item Seleccionar la variable \variable{kilos\_perdidos} como \campo{Variable dependiente} y las variables
\variable{dieta} y \variable{farmaco} en el campo \campo{Factores entre individuos}, y hacer click sobre el botón
\boton{Aceptar}.
\end{enumerate}
Ahora puede concluirse que sí hay interacción significativa, y eso implica que no hay la misma diferencia en Kg perdidos
entre los que hacen dieta y los que no si consideramos el grupo de los que no toman fármaco, que si consideramos el
grupo de los que sí lo toman.
\end{indicacion}

\item Teniendo en cuenta que hay interacción significativa, calcular el intervalo de confianza para la diferencia de
medias en los kg perdidos según la variable dieta y fármaco, así como entre los grupos que surgen de su interacción.
\begin{indicacion}
\begin{enumerate}
\item Seleccionar el menú \texttt{Teaching > Tests paramétricos > Medias > ANOVA}.
\item En el cuadro de diálogo que aparece, seleccionar el conjunto de datos \variable{dieta.farmaco}.
\item Seleccionar la variable \variable{kilos\_perdidos} como \campo{Variable
dependiente} y las variables \variable{dieta} y \variable{farmaco} en el campo \campo{Factores entre individuos}.
\item En la solapa \menu{Comparación por pares} seleccionar la opción \opcion{Comparación de medias por pares} y hacer
click sobre el botón \boton{Aceptar}.
\end{enumerate}
\end{indicacion}
\end{enumerate}


\item Se ha realizado un experimento que consiste en que se ha anotado el tiempo, en días, que han tardado en contestar
correctamente a un cuestionario 30 personas, 15 hombres y 15 mujeres, distribuidos en grupos que han seguido tres
métodos diferentes de aprendizaje de la materia del cuestionario. Los resultados aparecen en la siguiente tabla:

\begin{center}
\begin{tabular}{|c|c|c|c|}
\cline{2-4}
\multicolumn{1}{c|}{} & Método a & Método b & Método c\\
\hline
Hombre & 15, 16, 18, 19, 14 & 25, 27, 28, 23, 29 & 21, 22, 18, 17, 20 \\
\hline
Mujer & 24, 27, 29, 25, 23 & 17, 15, 13, 16, 18 & 20, 19, 22, 17, 23 \\
\hline
\end{tabular}
\end{center}

\begin{enumerate}
\item Crear un conjunto de datos \variable{metodo.aprendizaje} con las variables \variable{sexo}, \variable{metodo} y \variable{dias}.

\item Mostrar el gráfico de interacción de los del tiempo de aprendizaje para los distintos grupos según el sexo y el método
de aprendizaje. 
¿Qué se puede decir de la interacción de las variables?
\begin{indicacion}
\begin{enumerate}
\item Seleccionar el menú \menu{Teaching > Gráficos > Diagrama de interacción}.
\item Seleccionar la variable \variable{dias} en el campo \campo{Variable respuesta}, la variable
\variable{metodo} en el campo \campo{Grupos eje X} y la variable \variable{sexo} en el campo \campo{Grupos de
trazado}, y hacer click en el botón \boton{Aceptar}.
\end{enumerate}
Es evidente que las líneas se cruzan, lo cual indica que hay interacción.
\end{indicacion}

\item Realizar un contraste de ANOVA de dos vías con interacción e interpretar los resultados.
\begin{indicacion}
\begin{enumerate}
\item Seleccionar el menú \texttt{Teaching > Tests paramétricos > Medias > ANOVA}.
\item En el cuadro de diálogo que aparece, seleccionar el conjunto de datos \variable{metodo.aprendizaje}.
\item Seleccionar la variable \variable{dias} como \campo{Variable dependiente} y las variables
\variable{sexo} y \variable{metodo} en el campo \campo{Factores entre individuos}, y hacer click sobre el botón
\boton{Aceptar}.
\end{enumerate}
Se puede observar que no hay diferencias significativas asociadas al sexo ni al método. 
Sin embargo sí que hay interacción, es decir la diferencia en el tiempo de respuesta entre hombres y mujeres depende del
método seguido, e igualmente las diferencias entre los tiempos de respuesta según los diferentes métodos dependen del
sexo.
\end{indicacion}

\item Calcular los intervalos de confianza para la diferencia de medias en el tiempo de aprendizaje entre los grupos que
surgen de la interacción del sexo con el método de aprendizaje.
\begin{indicacion}
\begin{enumerate}
\item Seleccionar el menú \texttt{Teaching > Tests paramétricos > Medias > ANOVA}.
\item En el cuadro de diálogo que aparece, seleccionar el conjunto de datos \variable{metodo.aprendizaje}.
\item Seleccionar la variable \variable{dias} como \campo{Variable dependiente} y las variables
\variable{sexo} y \variable{metodo} en el campo \campo{Factores entre individuos}.
\item En la solapa \menu{Comparación por pares} seleccionar la opción \opcion{Comparación de medias por pares} y hacer
click sobre el botón \boton{Aceptar}.
\end{enumerate}
\end{indicacion}
\end{enumerate}


\item Se desea comparar la rapidez con la que aparece el efecto de tres nuevos agentes repigmentadores: $A$, $B$ y $C$.
Con esta intención, se aplican de manera tópica dosis equivalentes de los tres repigmentadores en zonas de la piel con
pérdida total de pigmentación en los mismos ocho pacientes con vitíligo. A continuación, se recoge el tiempo, en días,
que tardan en aparecer los primeros signos de repigmentación:
\begin{center}
\begin{tabular}{rrr}
\hline
$A$ & $B$ & $C$ \\
\hline
19 & 3 & 31 \\
11 & 2 & 9 \\
7 & 4 & 16 \\
4 & 1 & 6 \\
3 & 2 & 8 \\
5 & 7 & 18 \\
7 & 1 & 5 \\
4 & 3 & 9 \\
\hline
\end{tabular}
\end{center}

\begin{enumerate}
\item Crear un conjunto de datos \variable{repigmentación} con las variables \variable{individuo}, \variable{tiempo} y
\variable{repigmentador}.
\begin{indicacion}Aunque todos los repigmentadores se aplican a cada individuo, para hacer un ANOVA de medidas
repetidas hay que introducir el individuo como el factor inter-sujetos, mientras que el repigmentador sería el
factor-intrasujetos.
\end{indicacion}

\item Realizar un ANOVA de medidas repetidas e interpretar el resultado obtenido.
\begin{indicacion}
\begin{enumerate}
\item Seleccionar el menú \menu{Teaching > Tests paramétricos > Medias > ANOVA}.
\item En el cuadro de diálogo que aparece, seleccionar el conjunto de datos \variable{repigmentación}.
\item Seleccionar la opción de diseño \opcion{Dentro de individuos (medidas repetidas)}.
\item Seleccionar la variable \variable{tiempo} como \campo{Variable dependiente}, la variable \variable{repigmentador}
en el campo \campo{Factores dentro de los individuos}, la variable \variable{individuo} en el campo \campo{Identificador
de los individuos} y hacer click en el botón \boton{Aceptar}.
\end{enumerate}
\end{indicacion}

\item ¿Entre qué medidas del tratamiento repigmentador se dan diferencias estadísticamente significativas?
\begin{indicacion}
\begin{enumerate}
\item Seleccionar el menú \menu{Teaching > Tests paramétricos > Medias > ANOVA}.
\item En el cuadro de diálogo que aparece, seleccionar el conjunto de datos \variable{repigmentación}.
\item Seleccionar la opción de diseño \opcion{Dentro de individuos (medidas repetidas)}.
\item Seleccionar la variable \variable{tiempo} como \campo{Variable dependiente}, la variable \variable{repigmentador}
en el campo \campo{Factores dentro de los sujetos}, la variable \variable{individuo} en el campo \campo{Identificador de
los individuos}.
\item En la solapa \menu{Comparación por pares} seleccionar la opción \opcion{Comparación de medias por pares} y hacer
click sobre el botón \boton{Aceptar}.
\end{enumerate}
\end{indicacion}
\end{enumerate}

\item En un estudio se quiere analizar la eficacia de dos tipos de entrenamiento (A1: entrenamiento sólo físico, A2:
entrenamiento físico + entrenamiento psicológico) para mejorar el rendimiento físico.
Para ello, se dispone de una muestra de 8 individuos con los que se generan dos grupos de 4 asignados aleatoriamente, y
se mide su rendimiento físico mediante un test de rendimiento numérico que va de 0 a 15 puntos.
Los 8 individuos son sometidos al test en 4 momentos diferentes (B1: al cabo de una semana de entrenamiento, B2: al cabo
de dos, B3: al cabo de tres y B4: al cabo de 4).
Los datos obtenidos fueron:

\begin{center}
\begin{tabular}{|l|l|l|l|l|}
\cline{2-5}
\multicolumn{1}{c|}{} & \multicolumn{1}{c|}{B1} & \multicolumn{1}{c|}{B2} & \multicolumn{1}{c|}{B3} & \multicolumn{1}{c|}{B4} \\
\hline
\multicolumn{1}{|c|}{A1} & \multicolumn{1}{c|}{3} & \multicolumn{1}{c|}{4} & \multicolumn{1}{c|}{7} & \multicolumn{1}{c|}{7} \\
\cline{2-5}
\multicolumn{1}{|c|}{} & \multicolumn{1}{c|}{6} & \multicolumn{1}{c|}{5} & \multicolumn{1}{c|}{8} & \multicolumn{1}{c|}{8} \\
\cline{2-5}
\multicolumn{1}{|c|}{} & \multicolumn{1}{c|}{3} & \multicolumn{1}{c|}{4} & \multicolumn{1}{c|}{7} & \multicolumn{1}{c|}{9} \\
\cline{2-5}
\multicolumn{1}{|c|}{} & \multicolumn{1}{c|}{3} & \multicolumn{1}{c|}{3} & \multicolumn{1}{c|}{6} & \multicolumn{1}{c|}{8} \\
\hline
\multicolumn{1}{|c|}{A2} & \multicolumn{1}{c|}{1} & \multicolumn{1}{c|}{2} & \multicolumn{1}{c|}{5} & \multicolumn{1}{c|}{10} \\
\cline{2-5}
\multicolumn{1}{|c|}{} & \multicolumn{1}{c|}{2} & \multicolumn{1}{c|}{3} & \multicolumn{1}{c|}{6} & \multicolumn{1}{c|}{10} \\
\cline{2-5}
\multicolumn{1}{|c|}{} & \multicolumn{1}{c|}{2} & \multicolumn{1}{c|}{4} & \multicolumn{1}{c|}{5} & \multicolumn{1}{c|}{9} \\
\cline{2-5}
\multicolumn{1}{|c|}{} & \multicolumn{1}{c|}{2} & \multicolumn{1}{c|}{3} & \multicolumn{1}{c|}{6} & \multicolumn{1}{c|}{11} \\
\hline
\end{tabular}

\end{center}

\begin{enumerate}
\item Crear un conjunto de datos \variable{entrenamiento.psicologico} con las variables \variable{individuo}, \variable{rendimiento} y
\variable{entrenamiento} y \variable{semana}.
\begin{indicacion}La variable semana contiene varias varias medidas repetidas en cada individuo, de modo que para poder
aplicar el procedimiento de ANOVA de medidas repetidas debe introducirse una fila por cada observación en el conjunto de
datos.
\end{indicacion}

\item Mostrar el gráfico de interacción de los del redimiento para los distintos grupos según el tipo de entrenamiento y
la semana.
¿Qué se puede decir de la interacción de las variables?
\begin{indicacion}
\begin{enumerate}
\item Seleccionar el menú \menu{Teaching > Gráficos > Diagrama de interacción}.
\item Seleccionar la variable \variable{rendimiento} en el campo \campo{Variable respuesta}, la variable
\variable{semana} en el campo \campo{Grupos eje X} y la variable \variable{entrenamiento} en el campo \campo{Grupos de
trazado}, y hacer click en el botón \boton{Aceptar}.
\end{enumerate}
Es evidente que las líneas se cruzan, lo cual indica que hay interacción.
\end{indicacion}

\item Realizar un ANOVA mixto (un factor entre individuos y otro dentro de los individuosmedidas) e interpretar el resultado obtenido.
¿Influye significativamente la semana en la que se realiza el test en el resultado? 
¿Y el tipo de entrenamiento?
¿Es significativa la interacción entre tipo de entrenamiento y la semana en la que se realiza el test?
\begin{indicacion}
\begin{enumerate}
\item Seleccionar el menú \menu{Teaching > Tests paramétricos > Medias > ANOVA}.
\item En el cuadro de diálogo que aparece, seleccionar el conjunto de datos \variable{entrenamiento.psicologico}.
\item Seleccionar la opción de diseño \opcion{Mixto}.
\item Seleccionar la variable \variable{rendimiento} como \campo{Variable dependiente}, la variable
\variable{entrenamiento} en el campo \campo{Factores entre individuos}, la variable \variable{semana} en el campo
\campo{Factores dentro de los individuos}, la variable \variable{individuo} en el campo \campo{Identificador de los
individuos} y hacer click en el botón \boton{Aceptar}.
\end{enumerate}
\end{indicacion}

\item Entre qué medias hay diferencias estadísticamente significativas? 
\begin{indicacion}
\begin{enumerate}
\item Seleccionar el menú \menu{Teaching > Tests paramétricos > Medias > ANOVA}.
\item En el cuadro de diálogo que aparece, seleccionar el conjunto de datos \variable{entrenamiento.psicologico}.
\item Seleccionar la opción de diseño \opcion{Mixto}.
\item Seleccionar la variable \variable{rendimiento} como \campo{Variable dependiente}, la variable
\variable{entrenamiento} en el campo \campo{Factores entre individuos}, la variable \variable{semana} en el campo
\campo{Factores dentro de los individuos}, la variable \variable{individuo} en el campo \campo{Identificador de los
individuos}.
\item En la solapa \menu{Comparación por pares} seleccionar la opción \opcion{Comparación de medias por pares} y hacer
click sobre el botón \boton{Aceptar}.
\end{enumerate}
\end{indicacion}
\end{enumerate}
\end{enumerate}


\section{Ejercicios propuestos}
\begin{enumerate}[leftmargin=*]
\item En un estudio se quiere analizar la influencia sobre la ansiedad social, cuantificada mediante una escala numérica
que va de 0 a 10, de la edad, dividida en tres categorías, y si se fuma o no. 
Los datos obtenidos fueron:
\[
\begin{array}{|l|c|c|}
\cline{2-3}
\multicolumn{1}{c|}{} & \text{Fumar No} & \text{Fumar Sí} \\
\hline
\text{Edad 1} & 3.91; 5.01; 4.47; 3.33; 4.71 & 4.83; 3.95; 4.04; 3.66; 9.44 \\
\hline
\text{Edad 2} & 5.65; 6.49; 5.50; 5.72; 5.44 & 9.66; 7.68; 9.57; 7.98; 7.39 \\
\hline
\text{Edad 3} & 4.94; 7.13; 5.54;5.94; 6.16 & 5.92; 5.48; 5.19; 6.12; 4.45 \\
\hline
\end{array}
\]

\begin{enumerate}
\item Considerando la posibilidad de interacción entre las variables independientes, ¿se puede considerar que la edad,
expresada en forma de categorías, influye en la ansiedad? 
¿Y el fumar? 
¿Se puede considerar que el fumar o no influye de forma diferente en la ansiedad dependiendo de la categoría de edad
analizada?

\item Dependiendo de los resultados del apartado anterior, ¿entre qué medias habría diferencias estadísticamente
significativas? Calcular los intervalos de confianza para las diferencias.
\end{enumerate}


\item Se ha aplicado un dispositivo electrónico que mide la frecuencia cardíaca a 10 estudiantes.
Se realizó una primera medición un minuto antes de que comenzasen a hacer un examen, la segunda medición se hizo cuando
llevaban 15 minutos realizando el examen, la tercera un minuto después de entregarlo y la cuarta 15 minutos después de
terminar.
Los resultados fueron:

\begin{center}
\begin{tabular}{ccccc}
\hline
Estudiante & Medida1 & Medida2 & Medida3 & Medida4 \\
\hline
1 & 57 & 61 & 77 & 70 \\
2 & 73 & 87 & 88 & 83 \\
3 & 75 & 89 & 89 & 65 \\
4 & 75 & 60 & 67 & 68 \\
5 & 77 & 87 & 67 & 67 \\
6 & 88 & 96 & 84 & 55 \\
7 & 89 & 65 & 89 & 60 \\
8 & 101 & 80 & 77 & 60 \\
9 & 103 & 85 & 76 & 66 \\
10 & 107 & 73 & 69 & 60 \\
\hline
\end{tabular}
\end{center}

¿Son las mediciones significativamente distintas entre sí? 
Si hay diferencia, ¿entre qué mediciones se dan?
\end{enumerate}

