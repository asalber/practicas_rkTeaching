% Author: Alfredo Sánchez Alberca (asalber@ceu.es)

\chapter{Variables Aleatorias Discretas}

\section{Fundamentos teóricos}
\subsection{Variables Aleatorias}
Se define una \emph{variable aleatoria} asignando a cada resultado del experimento aleatorio un número. Esta asignación
puede realizarse de distintas maneras, obteniéndose de esta forma diferentes variables aleatorias.
Así, en el lanzamiento de dos monedas podemos considerar el número de caras o el número de cruces.
En general, si los resultados del experimento son numéricos, se tomarán dichos números como los valores de la variable,
y si los resultados son cualitativos, se hará corresponder a cada modalidad un número arbitrariamente.

Formalmente, una \emph{variable aleatoria} $X$ es una función real definida sobre los puntos del espacio muestral $E$ de
un experimento aleatorio. \[ X:E\rightarrow \mathbb{R}\]

De esta manera, la distribución de probabilidad del espacio muestral $E$, se transforma en una distribución de
probabilidad para los valores de $X$.

El conjunto formado por todos los valores distintos que puede tomar la variable aleatoria se llama \emph{Rango} o
\emph{Recorrido} de la misma.

Las variables aleatorias pueden ser de dos tipos: discretas o continuas. Una variable es \emph{discreta} cuando sólo
puede tomar valores aislados, mientras que es \emph{continua} si puede tomar todos los valores posibles de un intervalo.

\subsection{Variables Aleatorias Discretas (v.a.d.)}
Se considera una v.a.d. $X$ que puede tomar los valores $x_i$, $i=1,2,...,n$.

\subsubsection{Función de probabilidad}
La \emph{distribución de probabilidad} de $X$ se suele caracterizar mediante una función $f(x)$, conocida como
\emph{función de probabilidad}, que asigna a cada valor de la variable su probabilidad. 
Esto es 
\[
f(x_i)=P(X=x_i),\
i=1,..,n
\]
verificándose que 
\[
\sum_{i=1}^{n} f(x_i)=1
\]


\subsubsection{Función de distribución}
Otra forma equivalente de caracterizar la distribución de probabilidad de $X$ es mediante otra función $F(x)$, llamada
\emph{función de distribución}, que asigna a cada $x\in \mathbb{R}$ la probabilidad de que $X$ tome un valor menor o
igual que dicho número $x$. Así,

\[
F(x) = P(X \le x) = \sum\limits_{x_i  \le x} {f(x_i)}
\]

Tanto la función de probabilidad como la función de distribución pueden representarse de forma gráfica, tal y como se
muestra en la figura \ref{g:graficasvad}.

\begin{figure}[h!]
\centering \subfigure[Función de probabilidad.]{
\scalebox{0.6}{%% Input file name: funcion_probabilidad_lanzamiento_2_monedas.fig
%% FIG version: 3.2
%% Orientation: Landscape
%% Justification: Flush Left
%% Units: Inches
%% Paper size: A4
%% Magnification: 100.0
%% Resolution: 1200ppi

\begin{pspicture}(5.75cm,3.48cm)(16.66cm,13.45cm)
\psset{unit=0.8cm}
%%
%% Depth: 2147483647
%%
\newrgbcolor{mycolor0}{1.00 0.50 0.31}\definecolor{mycolor0}{rgb}{1.00,0.50,0.31}
\newgray{mycolor1}{0.74}\definecolor{mycolor1}{gray}{0.74}
%%
%% Depth: 100
%%
\psset{linestyle=solid,linewidth=0.254,linecolor=mycolor0,fillstyle=none}
\psline(10.61,6.80)(10.61,10.88)
\psline(15.27,6.80)(15.27,14.95)
\psline(19.94,6.80)(19.94,10.88)
\psset{linewidth=0.03175,linecolor=black}
\psline(10.61,6.47)(19.94,6.47)
\psline(10.61,6.47)(10.61,6.26)
\psline(12.94,6.47)(12.94,6.26)
\psline(15.27,6.47)(15.27,6.26)
\psline(17.60,6.47)(17.60,6.26)
\psline(19.94,6.47)(19.94,6.26)
\rput(10.61,5.71){0.0}
\rput(12.94,5.71){0.5}
\rput(15.27,5.71){1.0}
\rput(17.60,5.71){1.5}
\rput(19.94,5.71){2.0}
\psline(10.23,6.80)(10.23,14.95)
\psline(10.23,6.80)(10.02,6.80)
\psline(10.23,8.43)(10.02,8.43)
\psline(10.23,10.06)(10.02,10.06)
\psline(10.23,11.69)(10.02,11.69)
\psline(10.23,13.32)(10.02,13.32)
\psline(10.23,14.95)(10.02,14.95)
\rput{90}(9.73,6.80){0.0}
\rput{90}(9.73,8.43){0.1}
\rput{90}(9.73,10.06){0.2}
\rput{90}(9.73,11.69){0.3}
\rput{90}(9.73,13.32){0.4}
\rput{90}(9.73,14.95){0.5}
\psline(10.23,6.47)(20.31,6.47)(20.31,15.28)(10.23,15.28)(10.23,6.47)
\rput(15.27,15.99){Lanzamiento de dos monedas}
\rput(15.27,4.86){Nº de caras}
\rput{90}(8.88,10.88){Probabilidad}
\psset{linecolor=mycolor1}
\psline(10.23,6.80)(20.31,6.80)
\end{pspicture}
%% End
}}\qquad
\subfigure[Función de distribución.]{
\scalebox{0.6}{%% Input file name: /media/datos/ceu/docencia/materiales/estadistica/presentaciones/curso_estadistica/img/variables_aleatorias_discretas/funcion_distribucion_lanzamiento_2_monedas.fig
%% FIG version: 3.2
%% Orientation: Landscape
%% Justification: Flush Left
%% Units: Inches
%% Paper size: A4
%% Magnification: 100.0
%% Resolution: 1200ppi

\begin{pspicture}(5.87cm,3.48cm)(16.66cm,13.45cm)
\psset{unit=0.8cm}
%%
%% Depth: 2147483647
%%
\newrgbcolor{mycolor0}{1.00 0.50 0.31}\definecolor{mycolor0}{rgb}{1.00,0.50,0.31}
\newgray{mycolor1}{0.74}\definecolor{mycolor1}{gray}{0.74}
\newrgbcolor{mycolor2}{0.25 0.41 0.88}\definecolor{mycolor2}{rgb}{0.25,0.41,0.88}
%%
%% Depth: 100
%%
\psset{linestyle=none,fillstyle=solid,fillcolor=mycolor0}
\pscircle(10.61,8.84){0.1}
\pscircle(15.27,12.91){0.1}
\pscircle(19.94,14.95){0.1}
\psset{linestyle=solid,linecolor=black,fillstyle=none}
\psline(10.61,6.47)(19.94,6.47)
\psline(10.61,6.47)(10.61,6.26)
\psline(15.27,6.47)(15.27,6.26)
\psline(19.94,6.47)(19.94,6.26)
\rput(10.61,5.71){0.0}
\rput(15.27,5.71){1.0}
\rput(19.94,5.71){2.0}
\psline(10.23,6.80)(10.23,14.95)
\psline(10.23,6.80)(10.02,6.80)
\psline(10.23,8.43)(10.02,8.43)
\psline(10.23,10.06)(10.02,10.06)
\psline(10.23,11.69)(10.02,11.69)
\psline(10.23,13.32)(10.02,13.32)
\psline(10.23,14.95)(10.02,14.95)
\rput{90}(9.73,6.80){0.0}
\rput{90}(9.73,8.43){0.2}
\rput{90}(9.73,10.06){0.4}
\rput{90}(9.73,11.69){0.6}
\rput{90}(9.73,13.32){0.8}
\rput{90}(9.73,14.95){1.0}
\psline(10.23,6.47)(20.31,6.47)(20.31,15.28)(10.23,15.28)(10.23,6.47)
\rput(15.27,15.99){Lanzamiento de dos monedas}
\rput(15.27,4.86){Nº de caras}
\rput{90}(8.88,10.88){Probabilidad acumulada}
\psset{linecolor=mycolor1}
\psline(10.23,6.80)(20.31,6.80)
\psset{linewidth=0.0635,linecolor=mycolor2}
\psline(10.23,6.80)(10.61,6.80)(10.61,8.84)(15.27,8.84)(15.27,12.91)(19.94,12.91)(19.94,14.95)(20.31,14.95)
\end{pspicture}
%% End
}}
\caption{Función de probabilidad y función de distribución de la variable aleatoria $X$ que mide el número de caras obtenido en el lanzamiento de dos monedas.} \label{g:graficasvad}
\end{figure}


\subsubsection{Estadísticos poblacionales}
Los parámetros descriptivos más importantes de una v.a.d. $X$ son:
\begin{description}
\item [Media o Esperanza]
\[
E[X]=\mu  = \sum\limits_{i = 1}^n {x_i f(x_i )}
\]

\item [Varianza]
\[
V[X]=\sigma ^2  = \sum\limits_{i = 1}^n {(x_i  - \mu )^2 f(x_i ) = }
\sum\limits_{i = 1}^n {x_i ^2 f(x_i ) - \mu ^2 }
\]

\item [Desviación típica]
\[
D[X]=\sigma  =  + \sqrt {\sigma ^2 }
\]
\end{description}

La media es una medida de tendencia central, mientras que la varianza y la desviación típica son medidas de dispersión.

Entre las v.a.d. cabe destacar las denominadas \emph{Binomial} y de \emph{Poisson}.


\subsubsection{Variable Binomial}

Se considera un experimento aleatorio en el que puede ocurrir el suceso $A$ o su contrario $\overline{A}$, con
probabilidades $p$ y $1-p$ respectivamente.

Si se realiza el experimento anterior $n$ veces, la v.a.d. $X$ que recoge el número de veces que ha ocurrido el suceso
$A$, se denomina \emph{Variable Binomial} y se designa por $X\sim B(n,\ p)$.

El recorrido de la variable $X$ es $\{0,1,...,n\}$ y su función de probabilidad viene dada por 
\[
f(x)= \binom{n}{x} p^x \left( {1 - p} \right)^{n - x} 
\] 
cuya gráfica se puede apreciar en la figura~\ref{g:binomial}.

\begin{figure}[h!]
\centering
\scalebox{0.8}{%% Input file name: funcion_probabilidad_binomial.fig
%% FIG version: 3.2
%% Orientation: Landscape
%% Justification: Flush Left
%% Units: Inches
%% Paper size: A4
%% Magnification: 100.0
%% Resolution: 1200ppi
%% Include the following in the preamble:
%% \usepackage{textcomp}
%% End

\begin{pspicture}(5.45cm,3.48cm)(16.68cm,13.45cm)
\psset{unit=0.8cm}
%%
%% Depth: 2147483647
%%
\newrgbcolor{mycolor0}{1.00 0.50 0.31}\definecolor{mycolor0}{rgb}{1.00,0.50,0.31}
\newgray{mycolor1}{0.74}\definecolor{mycolor1}{gray}{0.74}
%%
%% Depth: 100
%%
\psset{linestyle=solid,linewidth=0.03175,linecolor=mycolor0}
\qdisk(10.61,6.83){0.1}
\qdisk(11.54,7.12){0.1}
\qdisk(12.47,8.23){0.1}
\qdisk(13.41,10.62){0.1}
\qdisk(14.34,13.49){0.1}
\qdisk(15.27,14.83){0.1}
\qdisk(16.21,13.49){0.1}
\qdisk(17.14,10.62){0.1}
\qdisk(18.07,8.23){0.1}
\qdisk(19.00,7.12){0.1}
\qdisk(19.94,6.83){0.1}
\psset{linecolor=black,fillstyle=none}
\psline(10.61,6.47)(19.94,6.47)
\psline(10.61,6.47)(10.61,6.26)
\psline(12.47,6.47)(12.47,6.26)
\psline(14.34,6.47)(14.34,6.26)
\psline(16.21,6.47)(16.21,6.26)
\psline(18.07,6.47)(18.07,6.26)
\psline(19.94,6.47)(19.94,6.26)
\rput(10.61,5.71){0}
\rput(12.47,5.71){2}
\rput(14.34,5.71){4}
\rput(16.21,5.71){6}
\rput(18.07,5.71){8}
\rput(19.94,5.71){10}
\psline(10.23,6.80)(10.23,14.95)
\psline(10.23,6.80)(10.02,6.80)
\psline(10.23,8.43)(10.02,8.43)
\psline(10.23,10.06)(10.02,10.06)
\psline(10.23,11.69)(10.02,11.69)
\psline(10.23,13.32)(10.02,13.32)
\psline(10.23,14.95)(10.02,14.95)
\rput{90}(9.73,6.80){0.00}
\rput{90}(9.73,8.43){0.05}
\rput{90}(9.73,10.06){0.10}
\rput{90}(9.73,11.69){0.15}
\rput{90}(9.73,13.32){0.20}
\rput{90}(9.73,14.95){0.25}
\psline(10.23,6.47)(20.31,6.47)(20.31,15.28)(10.23,15.28)(10.23,6.47)
\rput(15.27,15.99){Función de probabilidad de una binomial $B(10,0.5)$}
\rput(15.27,4.86){$X$}
\rput{90}(8.88,10.88){Probabilidad $f(x)$}
\psset{linecolor=mycolor1}
\psline(10.23,6.80)(20.31,6.80)
\end{pspicture}
%% End
} 
\caption{Función de probabilidad de una variable aleatoria binomial de 10 repeticiones y probabilidad de éxito 0.5}\label{g:binomial}
\end{figure}

A partir de la expresión anterior se puede demostrar que
\begin{align*}
\mu  &= n p\\
\sigma ^2  &= n p (1 - p)\\
\sigma  &=  + \sqrt {n p (1 - p)}
\end{align*}

En el caso particular de que el experimento se realice una sola vez, la variable aleatoria recibe el nombre de
\variable{Variable de Bernouilli}.
Una variable Binomial $X\sim B(n,\ p)$ se puede considerar como suma de $n$ variables de Bernouilli idénticas con
distribución $B(1,\ p)$.


\subsubsection{Variable de Poisson}
Las variables de Poisson surgen de la observación de un conjunto discreto de fenómenos puntuales en un soporte continuo
de tiempo, longitud o espacio.
Por ejemplo: nº de llamadas que llegan a una centralita telefónica en un tiempo establecido, nº de hematíes en un
volumen de sangre, etc.
Se supone además que en un soporte continuo suficientemente grande, el número medio de fenómenos ocurridos por unidad de
soporte considerado, es una constante que designaremos por $\lambda$.

A la v.a.d. $X$, que recoge el número de fenómenos puntuales que ocurren en un intervalo de amplitud establecida, se le
denomina \emph{Variable de Poisson} y se designa por $X\sim P(\lambda)$.

El recorrido de la variable $X$ es $\{0,1,2,...\}$, no existiendo un valor máximo que pueda alcanzar. Su función de
probabilidad viene dada por \[ f(x) = \frac{{\lambda ^x }}{{x!}}\  e^{ - \lambda } \] y su gráfica aparece en la
figura~\ref{g:poisson}

\begin{figure}[h!]
\centering
\scalebox{0.8}{%% Input file name: funcion_probabilidad_poisson.fig
%% FIG version: 3.2
%% Orientation: Landscape
%% Justification: Flush Left
%% Units: Inches
%% Paper size: A4
%% Magnification: 100.0
%% Resolution: 1200ppi
%% Include the following in the preamble:
%% \usepackage{textcomp}
%% End

\begin{pspicture}(5.45cm,3.48cm)(16.66cm,13.45cm)
\psset{unit=0.8cm}
%%
%% Depth: 2147483647
%%
\newrgbcolor{mycolor0}{1.00 0.50 0.31}\definecolor{mycolor0}{rgb}{1.00,0.50,0.31}
\newgray{mycolor1}{0.74}\definecolor{mycolor1}{gray}{0.74}
%%
%% Depth: 100
%%
\psset{linestyle=solid,linewidth=0.03175,linecolor=mycolor0}
\qdisk(10.61,7.55){0.1}
\qdisk(11.39,9.79){0.1}
\qdisk(12.16,12.77){0.1}
\qdisk(12.94,14.76){0.1}
\qdisk(13.72,14.76){0.1}
\qdisk(14.49,13.17){0.1}
\qdisk(15.27,11.05){0.1}
\qdisk(16.05,9.23){0.1}
\qdisk(16.83,8.01){0.1}
\qdisk(17.60,7.34){0.1}
\qdisk(18.38,7.01){0.1}
\qdisk(19.16,6.88){0.1}
\qdisk(19.94,6.83){0.1}
\psset{linecolor=black,fillstyle=none}
\psline(10.61,6.47)(19.94,6.47)
\psline(10.61,6.47)(10.61,6.26)
\psline(12.16,6.47)(12.16,6.26)
\psline(13.72,6.47)(13.72,6.26)
\psline(15.27,6.47)(15.27,6.26)
\psline(16.83,6.47)(16.83,6.26)
\psline(18.38,6.47)(18.38,6.26)
\psline(19.94,6.47)(19.94,6.26)
\rput(10.61,5.71){0}
\rput(12.16,5.71){2}
\rput(13.72,5.71){4}
\rput(15.27,5.71){6}
\rput(16.83,5.71){8}
\rput(18.38,5.71){10}
\rput(19.94,5.71){12}
\psline(10.23,6.80)(10.23,14.95)
\psline(10.23,6.80)(10.02,6.80)
\psline(10.23,8.84)(10.02,8.84)
\psline(10.23,10.88)(10.02,10.88)
\psline(10.23,12.91)(10.02,12.91)
\psline(10.23,14.95)(10.02,14.95)
\rput{90}(9.73,6.80){0.00}
\rput{90}(9.73,8.84){0.05}
\rput{90}(9.73,10.88){0.10}
\rput{90}(9.73,12.91){0.15}
\rput{90}(9.73,14.95){0.20}
\psline(10.23,6.47)(20.31,6.47)(20.31,15.28)(10.23,15.28)(10.23,6.47)
\rput(15.27,15.99){Función de probabilidad de una Poisson $P(4)$}
\rput(15.27,4.86){$X$}
\rput{90}(8.88,10.88){Probabilidad $f(x)$}
\psset{linecolor=mycolor1}
\psline(10.23,6.80)(20.31,6.80)
\end{pspicture}
%% End
} 
\caption{Función de probabilidad de una variable aleatoria Poisson de media $\lambda=4$}\label{g:poisson}
\end{figure}

Se puede demostrar que
\begin{align*}
\mu  &= \lambda\\
\sigma ^2  &= \lambda\\
\sigma  &=  + \sqrt {\lambda}
\end{align*}

La distribución de Poisson aparece como límite de la distribución Binomial cuando el número $n$ de repeticiones del
experimento es muy grande y la probabilidad $p$ de que ocurra el suceso $A$ considerado es muy pequeña.
Por ello, la distribución de Poisson se llama también \emph{Ley de los Casos Raros}.
En la práctica se considera aceptable realizar los cálculos de probabilidades correspondientes a una variable $B(n,\ p)$
mediante las fórmulas correspondientes a una variable $P(\lambda)$ con $\lambda=n p$, siempre que $n\geq 50$ y $p<0.1$.

\clearpage
\newpage

\section{Ejercicios resueltos}
\begin{enumerate}[leftmargin=*] \opt{largo}{ \item La ley de los grandes números establece que cuando un experimento
aleatorio se repite de manera indefinida, la frecuencia relativa de cada suceso tiende a estabilizarse en torno a un
valor que es la probabilidad del suceso. Para comprobar la ley se realiza un experimento que consiste en lanzar un dado
varias veces y anotar la frecuencia relativa de cada cara. Se pide:
\begin{enumerate}
\item Lanzar el dado 10 veces y calcular las frecuencias relativas de las caras obtenidas y el diagrama de barras
asociado. 
\begin{indicacion}{
Para generar los 10 lanzamientos del dado:
\begin{enumerate}
\item Seleccionar el menú \menu{Teaching>Simulaciones>Lanzamiento de dados}.
\item En el cuadro de diálogo que aparece, introducir 10 en el campo \campo{Número de lanzamientos}, introducir un
nombre para el conjunto de datos y hacer clic en el botón \boton{Enviar}.
\end{enumerate}
Para observar las frecuencias:
\begin{enumerate}
\item Seleccionar el menú \menu{Teaching>Distribución de frecuencias>Tabla de frecuencias}.
\item En el cuadro de diálogo que aparece, seleccionar como variable a tabular la variable \variable{dado1} y hacer clic en el botón
\boton{Enviar}.
\item Seleccionar el menú \menu{Teaching>Gráficos>Diagrama de barras}.
\item En el cuadro de diálogo que aparece seleccionar la variable \variable{dado1}.
\item En la solapa \menu{Opciones de las barras} seleccionar la opción \opcion{Frecuencias relativas} y hacer clic en
el botón \boton{Enviar}.
\item Observar las diferencias entre las frecuencias relativas y en la altura de las barras. 
\end{enumerate}}
\end{indicacion}

\item Repetir el apartado anterior para 100, 1000 y 1000000 lanzamientos. ¿Se cumple la ley de los grandes números?
¿En torno a qué valor se estabilizan las frecuencias relativas?
\end{enumerate}
}

\item Sea $X$ la variable que mide el número de caras obtenidas al lanzar 10 monedas. Para ver de manera experimental
la distribución de probabilidad de $X$ se realiza un experimento aleatorio que consiste en lanzar varias veces las 10
monedas y anotar el número de caras obtenido en cada lanzamiento. Se pide:
\begin{enumerate}
\item Lanzar las 10 monedas 1000 veces y calcular las frecuencias relativas de las caras obtenidas y el diagrama de
barras asociado.
\begin{indicacion}{Para generar los lanzamientos de monedas:
\begin{enumerate}
\item Seleccionar el menú \menu{Teaching>Simulaciones>Lanzamiento de monedas}.
\item En el cuadro de diálogo que aparece, introducir 10 en el campo \campo{Número de monedas}, 1000 en el campo
\campo{Número de lanzamientos}, introducir un nombre para el conjunto de datos y hacer clic en el
botón aceptar\boton{Enviar}.
\end{enumerate}
Para calcular las frecuencias relativas:
\begin{enumerate}
\item Seleccionar el menú \menu{Teaching>Distribución de frecuencias>Tabla de frecuencias}.
\item En el cuadro de diálogo que aparece, seleccionar como variable a tabular la variable \variable{sum} y hacer clic
en el botón \boton{Enviar}.
\end{enumerate}
Para dibujar el diagrama de barras:
\begin{enumerate}
\item Seleccionar el menú \menu{Teaching>Gráficos>Diagrama de barras}.
\item En el cuadro de diálogo que aparece seleccionar la variable \variable{sum}.
\item En la solapa \menu{Opciones de las barras} marcar la opción \opcion{Frecuencias relativas} y hacer clic en el
botón \boton{Enviar}.
\end{enumerate}}
\end{indicacion}

\item Generar la distribución de probabilidad de una variable Binomial $B(10\,,\,0.5)$ y compararla con la distribución de frecuencias relativas del apartado anterior.
\begin{indicacion}{
\begin{enumerate}
\item Seleccionar el menú \menu{teaching>Distribuciones>Discretas>Binomial\flecha
Probabilidades}.
\item En el cuadro de diálogo que aparece, introducir 0,1,2,3,4,5,6,7,8,9,10 en el campo \campo{Valores de la variable},
itroducir 10 en el campo \campo{Número de repeticiones}, $0.5$ en el campo \campo{Probabilidad de éxito}, y hacer clic
en el botón \boton{Enviar}.
\end{enumerate}}
\end{indicacion}

\item Dibujar la gráfica de la función de probabilidad de la Binomial $X\sim B(10\,,\,0.5)$ y compararla con el
diagrama de barras de frecuencias relativas del primer apartado.
\begin{indicacion}{
\begin{enumerate}
\item Seleccionar el menú \menu{Teaching>Distribuciones>Discretas>Binomial\flecha
Gráfico de probabilidad}.
\item En el cuadro de diálogo que aparece, introducir 10 en el campo \campo{Número de repeticiones},
$0.5$ en el campo \campo{Probabilidad de éxito} y hacer clic en el botón \boton{Enviar}.
\end{enumerate}}
\end{indicacion}

\item Dibujar la gráfica de la función de distribución.
\begin{indicacion}{
\begin{enumerate}
\item Seleccionar el menú \menu{Teaching>Distribuciones>Discretas>Binomial\flecha
Gráfico de probabilidad}.
\item En el cuadro de diálogo que aparece, introducir 10 en el campo \campo{Número de repeticiones}, $0.5$ en el campo
\campo{Probabilidad de éxito}, seleccionar la opción \opcion{Función de distribución} y hacer clic en el botón
\boton{Enviar}.
\end{enumerate}}
\end{indicacion}

\item Calcular $P(X=7)$.
\begin{indicacion}{
\begin{enumerate}
\item Seleccionar el menú \menu{teaching>Distribuciones>Discretas>Binomial\flecha
Probabilidades}.
\item En el cuadro de diálogo que aparece, introducir 7 en el campo \campo{Valores de la variable},
itroducir 10 en el campo \campo{Número de repeticiones}, $0.5$ en el campo \campo{Probabilidad de éxito}, y hacer clic
en el botón \boton{Enviar}.
\end{enumerate}}
\end{indicacion}

\item Calcular $P(X\leq 4)$.
\begin{indicacion}{
\begin{enumerate}
\item Seleccionar el menú \menu{Teaching>Distribuciones>Discretas>Binomial\flecha
Probabilidades}.
\item En el cuadro de diálogo que aparece, introducir 4 en el campo \campo{Valores de la variable}, 10 en el campo
\campo{Número de repeticiones}, $0.5$ en el campo \campo{Probabilidad de éxito}.
\item Seleccionar la opción \opcion{Probabilidades acumuladas} y hacer clic en el botón \boton{Enviar}.
\end{enumerate}}
\end{indicacion}

\item Calcular $P(X>5)$.
\begin{indicacion}{
\begin{enumerate}
\item Seleccionar el menú \menu{Teaching>Distribuciones>Discretas>Binomial\flecha
Probabilidades}.
\item En el cuadro de diálogo que aparece, introducir 5 en el campo \campo{Valores de la variable}, 10 en el campo
\campo{Número de repeticones}, $0.5$ en el campo \campo{Probabilidad de éxito}.
\item Seleccionar la opción \opcion{Probabilidades acumuladas}, seleccionar la opción \opcion{derecha}
en el campo \campo{cola de acumulación} y hacer clic en el botón \boton{Enviar}.
\end{enumerate}}
\end{indicacion}

\item Calcular $P(2\leq X < 9)$.
\begin{indicacion}{
\begin{enumerate}
\item Seleccionar el menú \menu{Teaching>Distribuciones>Discretas>Binomial\flecha
Probabilidades}.
\item En el cuadro de diálogo que aparece, introducir los valores $1$, $8$ en el campo \campo{Valores de la variable}, 10 en el campo
\campo{Número de repeticiones}, $0.5$ en el campo \campo{Probabilidad de éxito}.
\item Seleccionar la opción \opcion{Probabilidades acumuladas} y hacer clic en el botón \boton{Enviar}.
\end{enumerate}
La probabilidad del intervalo $P(2\leq X<9)$ es la resta de las probabilidades obtenidas $P(X<9)=P(X\leq 8)$ y $P(X<2)=P(X\leq 1)$.
}
\end{indicacion}
\end{enumerate}


\item El número de nacimientos diarios en una determinada población sigue una distribución de Poisson de media 6
nacimientos al día. Se pide:
\begin{enumerate}
\item Dibujar la gráfica de la función de probabilidad.
\begin{indicacion}{
\begin{enumerate}
\item Seleccionar el menú \menu{Teaching>Distribuciones>Discretas>Poisson\flecha
Gráfico de probabilidad}.
\item En el cuadro de diálogo que aparece, introducir el valor 6 en el campo \campo{Media} y hacer clic en el botón
\boton{Enviar}.
\end{enumerate}}
\end{indicacion}

\item Dibujar la gráfica de la función de distribución.
\begin{indicacion}{
\begin{enumerate}
\item Seleccionar el menú \menu{Teaching>Distribuciones>Discretas>Poisson\flecha
Gráfico de probabilidad}.
\item En el cuadro de diálogo que aparece, introducir el valor 6 en el campo \campo{Media}, marcar la opción
\opcion{Función de distribución} y hacer clic en el botón \boton{Enviar}.
\end{enumerate}}
\end{indicacion}

\item Calcular la probabilidad de un día haya 1 nacimiento.
\begin{indicacion}{
\begin{enumerate}
\item Seleccionar el menú \menu{Teaching>Distribuciones>Discretas>Poisson>Probabilidades}.
\item En el cuadro de diálogo que aparece, introducir el valor 1 en el campo \campo{Valores de la variable}, introducir
el valor 6 en el campo \campo{Media}, y hacer clic en el botón \boton{Enviar}.
\end{enumerate}}
\end{indicacion}

\item Calcular la probabilidad de que un día haya menos de 6 nacimientos.
\begin{indicacion}{
\begin{enumerate}
\item Seleccionar el menú \menu{Teaching>Distribuciones>Discretas>Poisson>Probabilidades}.
\item En el cuadro de diálogo que aparece, introducir 5 en el campo \campo{Valores de la variable} y 6 en el campo
\campo{Media}.
\item Seleccionar la opción \opcion{Probabilidades acumuladas} y hacer clic en el botón \boton{Enviar}.
\end{enumerate}}
\end{indicacion}

\item Calcular la probabilidad de que un día haya 4 o más nacimientos. 
\begin{indicacion}{
\begin{enumerate}
\item Seleccionar el menú \menu{Teaching>Distribuciones>Discretas>Poisson>Probabilidades}.
\item En el cuadro de diálogo que aparece, introducir 3 en el campo \campo{Valores de la variable} y 6 en el campo
\campo{Media}. 
\item Seleccionar la opción \opcion{Probabilidades acumuladas}, seleccionar la opción \opcion{derecha} en el campo
\campo{cola de acumulación} y hacer clic en el botón \boton{Enviar}.
\end{enumerate}}
\end{indicacion}

\item Calcular la probabilidad de que un día haya entre 4 y 8 nacimientos, inclusives.
\begin{indicacion}{
\begin{enumerate}
\item Seleccionar el menú \menu{Teaching>Distribuciones>Discretas>Poisson>Probabilidades}.
\item En el cuadro de diálogo que aparece, introducir los valores 3, 8 en el campo \campo{Valores de la variable} y
6 en el campo \campo{Media}.
\item Seleccionar la opción \opcion{Probabilidades acumuladas} y hacer clic en el botón \boton{Enviar}.
\end{enumerate}
La probabilidad del intervalo $P(4\leq X\leq 8)$ es la resta de las probabilidades obtenidas $P(X\leq 8)$ y $P(X<4)=P(X\leq 3)$.
}
\end{indicacion}
\end{enumerate}


\item La ley de los casos raros dice que en una distribución Binomial $B(n\,,\,p)$, cuando $n\geq 30$ y $p\leq
0.1$ la distribución se parece mucho a una distribución Poisson $P(np)$. 
Para comprobar hasta qué punto se parecen esta distribuciones, se pide:
\begin{enumerate}
\item Generar la distribución de probabilidad de una variable Binomial $B(30\,,0.1)$.
\begin{indicacion}{
\begin{enumerate}
\item Seleccionar el menú \menu{Teaching>Distribuciones>Discretas>Binomial>Probabilidades}.
\item En el cuadro de diálogo que aparece, introducir los valores 0,1,2,3,4,5,6,7,8,9,10 en el campo \campo{Valores de
la variable}, introducir el valor 30 en el campo \campo{Número de repeticiones}, $0.1$ en el campo \campo{Probabilidad
de éxito} y hacer clic en el botón \boton{Enviar}.
\end{enumerate}}
\end{indicacion}

\item Generar la distribución de probabilidad de una variable Poisson $P(3)$ y compararla con la de la binomial
$B(30\,,0.1)$.
\begin{indicacion}{
\begin{enumerate}
\item Seleccionar el menú \menu{Teaching>Distribuciones>Discretas>Poisson>Probabilidades}.
\item En el cuadro de diálogo que aparece, introducir los valores 0,1,2,3,4,5,6,7,8,9,10 en el campo \campo{Valores de
la variable}, introducir el valor 3 en el campo \campo{Media} y hacer clic en el botón \boton{Enviar}.
\end{enumerate}}
\end{indicacion}

\item Generar la distribución de probabilidad de una variable Binomial $B(100\,,0.03)$ y compararla con la de la
Poisson $P(3)$. ¿Se parecen más estas distribuciones que las anteriores? 
\begin{indicacion}{
\begin{enumerate}
\item Seleccionar el menú \menu{Teaching>Distribuciones>Discretas>Binomial>Probabilidades}.
\item En el cuadro de diálogo que aparece, introducir los valores 0,1,2,3,4,5,6,7,8,9,10 en el campo \campo{Valores de
la variable}, introducir el valor 100 en el campo \campo{Número de repeticiones}, $0.03$ en
el campo \campo{Probabilidad de éxito} y hacer clic en el botón \boton{Enviar}.
\end{enumerate}}
\end{indicacion}

\item Dibujar las gráficas de las distribuciones anteriores y ver cuáles se parecen más. 
¿Se cumple la ley de los casos raros?
\begin{indicacion}{
\begin{enumerate}
\item Seleccionar el menú \menu{Teaching>Simulaciones>Ley de los casos raros}.
\item En el cuadro de diálogo que aparece, desplazar el deslizador de \opcion{n} hasta 30 y el de \opcion{p} hasta 0.1.
\item Después desplazar el deslizador de \opcion{n} hasta 100 y el de \opcion{p} hasta 0.03.
\end{enumerate}}
\end{indicacion}
\end{enumerate}
\end{enumerate}


\section{Ejercicios propuestos}
\begin{enumerate}[leftmargin=*]
\item Al lanzar 100 veces una moneda, ¿cuál es la probabilidad de obtener entre 40 y 60 caras inclusive?

\item La probabilidad de curación de un paciente al ser sometido a un determinado tratamiento es $0.85$. Calcular la probabilidad de que en un grupo de 6 enfermos sometidos a tratamiento:
\begin{enumerate}
\item Se curen la mitad.
\item Se curen al menos 4.
\end{enumerate}

\item La probabilidad de que al administrar una vacuna dé una determinada reacción es $0.001$. 
Si se vacunan 2000 personas ¿cuál es la probabilidad de que aparezca alguna reacción adversa?

\item El número medio de llamadas por minuto que llegan a una centralita telefónica es igual a 120. Se pide:
\begin{enumerate}
\item Dar la distribución de probabilidad del número de llamadas en 2 segundos y dibujar su gráfica. 
\item Calcular al probabilidad de que durante 2 segundos lleguen a la centralita menos de 4 llamadas.
\item Calcular la probabilidad de que durante 3 segundos lleguen a la centralita 3 llamadas como mínimo.
\end{enumerate}

\item Se sabe que la probabilidad de que aparezca una bacteria en un mm$^3$ de cierta disolución es de $0.002$.
Si en cada mm$^3$ a los sumo puede aparecer una bacteria, determinar la probabilidad de que en un cm$^3$ haya como
máximo $5$ bacterias.
\end{enumerate}







