% Author: Alfredo Sánchez Alberca (asalber@ceu.es)

\chapter{Variables Aleatorias Continuas}

\section{Fundamentos teóricos}
\subsection{Variables Aleatorias}
Se define una \emph{variable aleatoria} asignando a cada resultado del experimento aleatorio un número.
Esta asignación puede realizarse de distintas maneras, obteniéndose de esta forma diferentes variables aleatorias.
Así, en el lanzamiento de dos monedas podemos considerar el número de caras o el número de cruces.
En general, si los resultados del experimento son numéricos, se tomarán dichos números como los valores de la variable,
y si los resultados son cualitativos, se hará corresponder a cada modalidad un número arbitrariamente.

Formalmente, una \emph{variable aleatoria} $X$ es una función real definida sobre los puntos del espacio muestral $E$ de
un experimento aleatorio. 
\[
X:E\rightarrow \mathbb{R}
\]

De esta manera, la distribución de probabilidad del espacio muestral $E$, se transforma en una distribución de
probabilidad para los valores de $X$.

El conjunto formado por todos los valores distintos que puede tomar la variable aleatoria se llama \emph{Rango} o
\emph{Recorrido} de la misma.

Las variables aleatorias pueden ser de dos tipos: discretas o continuas. Una variable es \emph{discreta} cuando sólo
puede tomar valores aislados, mientras que es \emph{continua} si puede tomar todos los valores posibles de un intervalo.

\subsection{Variables Aleatorias Continuas (v.a.c.)}
Se considera una v.a.c. $X$. En este tipo de variables, a diferencia de las discretas, la probabilidad de que la
variable tome un valor aislado cualquiera es nula, y sólo hablaremos de probabilidades asociadas a intervalos.

\subsubsection{Función de densidad}
La \emph{distribución de probabilidad} de $X$ se suele caracterizar mediante una función $f(x)$, conocida como
\emph{función de densidad}. Formalmente, una función de densidad es una función no negativa, integrable en $\mathbb{R}$,
que cumple 
\[
\int_{ - \infty }^\infty  {f(x)dx = 1} 
\]

A partir de esta función, se puede calcular la probabilidad de que el valor de la variable pertenezca a un intervalo
$[a,b]$, midiendo el área encerrada por dicha función y el eje de abscisas entre los límites del intervalo, como se
observa en la figura~\ref{g:probabilidadintervalo}, es decir
\[
P(a \le X \le b) = \int_a^b {f(x)dx}
\]

\begin{figure}[h!]
\centering
\scalebox{0.8}{%% Input file name: calculo_probabilidad_funcion_densidad.fig
%% FIG version: 3.2
%% Orientation: Landscape
%% Justification: Flush Left
%% Units: Inches
%% Paper size: A4
%% Magnification: 100.0
%% Resolution: 1200ppi

\begin{pspicture}(6.65cm,3.58cm)(17.03cm,13.49cm)
\psset{unit=0.8cm}
%%
%% Depth: 2147483647
%%
\newrgbcolor{mycolor0}{1.00 0.50 0.31}\definecolor{mycolor0}{rgb}{1.00,0.50,0.31}
%%
%% Depth: 100
%%
\psset{linestyle=solid,linewidth=0.03175,linecolor=black,fillstyle=solid,fillcolor=mycolor0}
\psline(12.47,6.97)(12.47,14.39)(12.56,14.80)(12.66,15.15)(12.76,15.44)(12.86,15.66)(12.96,15.83)(13.06,15.93)(13.16,15.98)(13.26,15.98)(13.36,15.92)(13.46,15.82)(13.55,15.68)(13.65,15.51)(13.75,15.30)(13.85,15.07)(13.95,14.81)(14.05,14.54)(14.15,14.25)(14.25,13.95)(14.35,13.64)(14.44,13.33)(14.54,13.02)(14.64,12.71)(14.74,12.40)(14.84,12.10)(14.94,11.80)(15.04,11.51)(15.14,11.23)(15.24,10.96)(15.34,10.70)(15.43,10.45)(15.53,10.21)(15.63,9.99)(15.73,9.77)(15.83,9.57)(15.93,9.38)(16.03,9.19)(16.13,9.03)(16.23,8.86)(16.33,8.72)(16.43,8.58)(16.43,6.97)
\psset{fillstyle=none}
\psline(10.59,6.97)(10.68,6.97)(10.78,6.98)(10.88,7.02)(10.98,7.11)(11.08,7.26)(11.18,7.50)(11.28,7.81)(11.38,8.20)(11.47,8.67)(11.57,9.19)(11.67,9.77)(11.77,10.38)(11.87,11.00)(11.97,11.63)(12.07,12.25)(12.17,12.85)(12.27,13.41)(12.37,13.93)(12.47,14.39)(12.56,14.80)(12.66,15.15)(12.76,15.44)(12.86,15.66)(12.96,15.83)(13.06,15.93)(13.16,15.98)(13.26,15.98)(13.36,15.92)(13.46,15.82)(13.55,15.68)(13.65,15.51)(13.75,15.30)(13.85,15.07)(13.95,14.81)(14.05,14.54)(14.15,14.25)(14.25,13.95)(14.35,13.64)(14.44,13.33)(14.54,13.02)(14.64,12.71)(14.74,12.40)(14.84,12.10)(14.94,11.80)(15.04,11.51)(15.14,11.23)(15.24,10.96)(15.34,10.70)(15.43,10.45)(15.53,10.21)(15.63,9.99)(15.73,9.77)(15.83,9.57)(15.93,9.38)(16.03,9.19)(16.13,9.03)(16.23,8.86)(16.33,8.72)(16.43,8.58)(16.52,8.45)(16.62,8.32)(16.72,8.21)(16.82,8.11)(16.92,8.01)(17.02,7.92)(17.12,7.84)(17.22,7.76)(17.32,7.69)(17.41,7.63)(17.51,7.57)(17.61,7.51)(17.71,7.46)(17.81,7.42)(17.91,7.38)(18.01,7.34)(18.11,7.30)(18.21,7.27)(18.30,7.24)(18.40,7.22)(18.50,7.19)(18.60,7.17)(18.70,7.15)(18.80,7.13)(18.90,7.11)(19.00,7.10)(19.10,7.09)(19.20,7.07)(19.30,7.06)(19.40,7.05)(19.49,7.04)(19.59,7.04)(19.69,7.03)(19.79,7.02)(19.89,7.02)(19.99,7.01)(20.09,7.01)(20.19,7.00)(20.28,7.00)(20.38,7.00)
\psline(10.19,6.60)(20.78,6.60)(20.78,16.34)(10.19,16.34)(10.19,6.60)
\rput(15.48,5.00){$X$}
\rput[c]{90}(8.83,11.35){Densidad de probabilidad $f(x)$}
\psline(12.47,6.60)(16.43,6.60)
\psline(12.47,6.60)(12.47,6.39)
\psline(16.43,6.60)(16.43,6.39)
\rput(12.47,5.84){$a$}
\rput(16.43,5.84){$b$}
\psline(10.19,6.97)(10.19,6.97)
\psline(10.19,6.97)(9.98,6.97)
\rput{90}(9.68,6.97){0}
\rput[c](14.4,8.70){$P(a\leq X\leq b)=$}
\rput[c](14.5,7.8){$=\dint_a^b f(x)\,dx$}
\psline(10.19,6.97)(20.78,6.97)
\end{pspicture}
%% End
} 
\caption{En una v.a.c. la probabilidad asociada a un intervalo $[a,b]$, es el área que queda encerrada por
la función de densidad y el eje de abscisas entre los límites del intervalo.}\label{g:probabilidadintervalo}
\end{figure}


\subsubsection{Función de distribución}
Otra forma equivalente de caracterizar la distribución de probabilidad de $X$ es mediante otra función $F(x)$, llamada
\emph{función de distribución}, que asigna a cada $x\in \mathbb{R}$ la probabilidad de que $X$ tome un valor menor o
igual que dicho número $x$. Así,
\[
F(x) = P(X \le x) = \int_{ - \infty }^x {f(t)dt}
\]

A partir de la definición anterior es claro que la probabilidad de que la variable tome un valor en el intervalo [a,b]
puede calcularse a partir de la función de distribución de la siguiente forma:
\[
P(a \le X \le b) = \int_a^b {f(x)dx = } \int_{ - \infty }^b {f(x)dx - } \int_{ - \infty }^a {f(x)dx = } F(b) - F(a)
\]


\subsubsection{Estadísticos poblacionales}
Los parámetros descriptivos más importantes de una v.a.c. $X$ son:
\begin{description}
\item [Media o Esperanza]
\[
E[X]=\mu  = \int_{ - \infty }^\infty  {xf(x)\,dx}
\]

\item [Varianza]
\[
V[X]=\sigma ^2  = \int_{ - \infty }^\infty {(x - \mu )^2 f(x)\,dx =
} \int_{ - \infty }^\infty  {x^2  f(x)\,dx - \mu ^2 }
\]

\item [Desviación típica]
\[
D[X]=\sigma  =  + \sqrt {\sigma ^2 }
\]
\end{description}

La media es una medida de tendencia central, mientras que la varianza y la desviación típica son medidas de dispersión.


\subsubsection{Distribución Uniforme Continua}
Una v.a.c. $X$ se dice que sigue una \emph{Distribución Uniforme Continua} de parámetros $a$ y $b$, y se designa por
$X\sim U(a,\ b)$, si su recorrido es el intervalo $[a,b]$ y su función de densidad es
\[ 
f(x) =
\begin{cases}
\frac{1}{b-a} & \mbox{si $a\leq x\leq b$},\\
0 & \mbox{en el resto}
\end{cases}
\]

Esta función es constante en el intervalo $[a,b]$ y nula fuera de él. 
Se cumple que
\[
\mu=\frac{a+b}{2} \qquad \sigma=+\frac{b-a}{\sqrt{12}}.
\]

\begin{figure}[h!]
\centering
\scalebox{0.8}{%% Input file name: funcion_densidad_uniforme.fig
%% FIG version: 3.2
%% Orientation: Landscape
%% Justification: Flush Left
%% Units: Inches
%% Paper size: A4
%% Magnification: 100.0
%% Resolution: 1200ppi
%% Include the following in the preamble:
%% \usepackage{textcomp}
%% End

\begin{pspicture}(6.68cm,3.48cm)(16.66cm,13.45cm)
\psset{unit=0.8cm}
%%
%% Depth: 2147483647
%%
\newgray{mycolor0}{0.74}\definecolor{mycolor0}{gray}{0.74}
%%
%% Depth: 100
%%
\psset{linestyle=solid,linewidth=0.03175,linecolor=black,fillstyle=none}
\psline(11.39,13.59)(12.25,13.59)(13.11,13.59)(13.98,13.59)(14.84,13.59)(15.70,13.59)(16.57,13.59)(17.43,13.59)(18.30,13.59)(19.16,13.59)
\psline(11.39,6.47)(19.16,6.47)
\psline(11.39,6.47)(11.39,6.26)
\psline(12.94,6.47)(12.94,6.26)
\psline(14.49,6.47)(14.49,6.26)
\psline(16.05,6.47)(16.05,6.26)
\psline(17.60,6.47)(17.60,6.26)
\psline(19.16,6.47)(19.16,6.26)
\rput(11.39,5.71){0.0}
\rput(12.94,5.71){0.2}
\rput(14.49,5.71){0.4}
\rput(16.05,5.71){0.6}
\rput(17.60,5.71){0.8}
\rput(19.16,5.71){1.0}
\psline(10.23,6.80)(10.23,14.95)
\psline(10.23,6.80)(10.02,6.80)
\psline(10.23,8.16)(10.02,8.16)
\psline(10.23,9.52)(10.02,9.52)
\psline(10.23,10.88)(10.02,10.88)
\psline(10.23,12.23)(10.02,12.23)
\psline(10.23,13.59)(10.02,13.59)
\psline(10.23,14.95)(10.02,14.95)
\rput{90}(9.73,6.80){0.0}
\rput{90}(9.73,8.16){0.2}
\rput{90}(9.73,9.52){0.4}
\rput{90}(9.73,10.88){0.6}
\rput{90}(9.73,12.23){0.8}
\rput{90}(9.73,13.59){1.0}
\rput{90}(9.73,14.95){1.2}
\psline(10.23,6.47)(20.31,6.47)(20.31,15.28)(10.23,15.28)(10.23,6.47)
\rput(15.27,15.99){Distribución Uniforme $U(0,1)$}
\rput(15.27,4.86){$X$}
\rput[l]{90}(8.86,9.54){Densidad $f(x)$}
\psset{linecolor=mycolor0}
\psline(10.23,6.80)(20.31,6.80)
\psset{linestyle=dashed}
\psline(11.39,6.80)(11.39,13.59)
\psline(19.16,6.80)(19.16,13.59)
\end{pspicture}
%% End
} 
\caption{Función de densidad de una variable aleatoria uniforme continua $U(0,\,1)$.}
\label{g:uniforme}
\end{figure}


\subsubsection{Distribución Normal}
Una v.a.c. $X$ se dice que sigue una \emph{Distribución Normal} o \emph{Gaussiana} de media $\mu$ y desviación típica
$\sigma$, y se designa por $X\sim N(\mu,\ \sigma)$, si su recorrido es todo $\mathbb{R}$ y su función de densidad es
\[
f(x) = \frac{1}{{\sigma \sqrt {2\pi } }}e^{ - \frac{{(x - \mu )^2 }}{{2\sigma ^2 }}}
\]

Esta función tiene forma acampanada y es simétrica con respeto a la media $\mu$.

La distribución Normal es la distribución continua más importante, ya que muchos de los fenómenos que aparecen en la
naturaleza presentan esta distribución.
Ello es debido a que, como establece el \emph{Teorema Central del Límite}, cuando los resultados de un experimento están
influidos por muchas causas independientes que actúan sumando sus efectos, se puede esperar que dichos resultados sigan
una distribución normal.

La v.a.c normal de media 0 y desviación típica 1, $Z\sim N(0,1)$, se conoce como \emph{variable normal estándar} o
\emph{tipificada} y se utiliza muy a menudo.
Su función de densidad aparece en la figura~\ref{g:funciondensidadnormal} y su función de distribución en la
figura~\ref{g:funciondistribucionnormal}.

\begin{figure}[h!]
\centering \subfigure[Función de
densidad.]{\label{g:funciondensidadnormal}
\scalebox{0.7}{%% Input file name: funcion_densidad_normal_estandar.fig
%% FIG version: 3.2
%% Orientation: Landscape
%% Justification: Flush Left
%% Units: Inches
%% Paper size: A4
%% Magnification: 100.0
%% Resolution: 1200ppi
%% Include the following in the preamble:
%% \usepackage{textcomp}
%% End

\begin{pspicture}(6.69cm,3.44cm)(16.66cm,13.45cm)
\psset{unit=0.8cm}
%%
%% Depth: 2147483647
%%
\newgray{mycolor0}{0.74}\definecolor{mycolor0}{gray}{0.74}
%%
%% Depth: 100
%%
\psset{linestyle=solid,linewidth=0.03175,linecolor=black,fillstyle=none}
\psline(10.61,6.75)(10.70,6.76)(10.80,6.77)(10.89,6.79)(10.99,6.80)(11.08,6.82)(11.17,6.84)(11.27,6.87)(11.36,6.90)(11.46,6.94)(11.55,6.98)(11.64,7.03)(11.74,7.08)(11.83,7.15)(11.93,7.22)(12.02,7.31)(12.12,7.40)(12.21,7.51)(12.30,7.63)(12.40,7.77)(12.49,7.92)(12.59,8.08)(12.68,8.26)(12.78,8.45)(12.87,8.67)(12.96,8.89)(13.06,9.14)(13.15,9.40)(13.25,9.67)(13.34,9.95)(13.43,10.26)(13.53,10.56)(13.62,10.88)(13.72,11.21)(13.81,11.54)(13.91,11.87)(14.00,12.20)(14.09,12.52)(14.19,12.84)(14.28,13.14)(14.38,13.43)(14.47,13.70)(14.56,13.96)(14.66,14.18)(14.76,14.38)(14.85,14.55)(14.94,14.69)(15.04,14.80)(15.13,14.87)(15.23,14.91)(15.32,14.91)(15.41,14.87)(15.51,14.80)(15.60,14.69)(15.70,14.55)(15.79,14.38)(15.89,14.18)(15.98,13.96)(16.07,13.70)(16.17,13.43)(16.26,13.14)(16.36,12.84)(16.45,12.52)(16.54,12.20)(16.64,11.87)(16.73,11.54)(16.83,11.21)(16.92,10.88)(17.02,10.56)(17.11,10.26)(17.20,9.95)(17.30,9.67)(17.39,9.40)(17.49,9.14)(17.58,8.89)(17.68,8.67)(17.77,8.45)(17.86,8.26)(17.96,8.08)(18.05,7.92)(18.15,7.77)(18.24,7.63)(18.33,7.51)(18.43,7.40)(18.52,7.31)(18.62,7.22)(18.71,7.15)(18.81,7.08)(18.90,7.03)(18.99,6.98)(19.09,6.94)(19.18,6.90)(19.28,6.87)(19.37,6.84)(19.47,6.82)(19.56,6.80)(19.66,6.79)(19.75,6.77)(19.84,6.76)(19.94,6.75)
\psline(11.02,6.43)(19.52,6.43)
\psline(11.02,6.43)(11.02,6.22)
\psline(12.44,6.43)(12.44,6.22)
\psline(13.86,6.43)(13.86,6.22)
\psline(15.27,6.43)(15.27,6.22)
\psline(16.69,6.43)(16.69,6.22)
\psline(18.11,6.43)(18.11,6.22)
\psline(19.52,6.43)(19.52,6.22)
\rput(11.02,5.67){-3}
\rput(12.44,5.67){-2}
\rput(13.86,5.67){-1}
\rput(15.27,5.67){0}
\rput(16.69,5.67){1}
\rput(18.11,5.67){2}
\rput(19.52,5.67){3}
\psline(10.23,6.72)(10.23,14.94)
\psline(10.23,6.72)(10.02,6.72)
\psline(10.23,8.77)(10.02,8.77)
\psline(10.23,10.83)(10.02,10.83)
\psline(10.23,12.88)(10.02,12.88)
\psline(10.23,14.94)(10.02,14.94)
\rput{90}(9.73,6.72){0.0}
\rput{90}(9.73,8.77){0.1}
\rput{90}(9.73,10.83){0.2}
\rput{90}(9.73,12.88){0.3}
\rput{90}(9.73,14.94){0.4}
\psline(10.23,6.43)(20.31,6.43)(20.31,15.23)(10.23,15.23)(10.23,6.43)
\rput[l](11.14,15.99){Distribución normal estándar $N(\mu=0,\sigma=1)$}
\rput(15.27,4.82){$X$}
\rput[l]{90}(8.88,9.71){Densidad de probabilidad $f(x)$}
\psset{linecolor=mycolor0}
\psline(10.23,6.72)(20.31,6.72)
\end{pspicture}
%% End
}}\qquad
\subfigure[Función de distribución.]{\label{g:funciondistribucionnormal}
\scalebox{0.7}{%% Input file name: funcion_distribucion_normal_estandar.fig
%% FIG version: 3.2
%% Orientation: Landscape
%% Justification: Flush Left
%% Units: Inches
%% Paper size: A4
%% Magnification: 100.0
%% Resolution: 1200ppi
%% Include the following in the preamble:
%% \usepackage{textcomp}
%% End

\begin{pspicture}(6.69cm,3.44cm)(16.66cm,13.45cm)
\psset{unit=0.8cm}
%%
%% Depth: 2147483647
%%
\newgray{mycolor0}{0.74}\definecolor{mycolor0}{gray}{0.74}
%%
%% Depth: 100
%%
\psset{linestyle=solid,linewidth=0.03175,linecolor=black,fillstyle=none}
\psline(10.61,6.75)(10.70,6.76)(10.80,6.76)(10.89,6.76)(10.99,6.76)(11.08,6.76)(11.17,6.77)(11.27,6.77)(11.36,6.78)(11.46,6.78)(11.55,6.79)(11.64,6.79)(11.74,6.80)(11.83,6.81)(11.93,6.83)(12.02,6.84)(12.12,6.86)(12.21,6.88)(12.30,6.90)(12.40,6.92)(12.49,6.95)(12.59,6.99)(12.68,7.03)(12.78,7.07)(12.87,7.12)(12.96,7.17)(13.06,7.23)(13.15,7.30)(13.25,7.37)(13.34,7.46)(13.43,7.55)(13.53,7.64)(13.62,7.75)(13.72,7.86)(13.81,7.99)(13.91,8.12)(14.00,8.26)(14.09,8.41)(14.19,8.56)(14.28,8.73)(14.38,8.90)(14.47,9.08)(14.56,9.27)(14.66,9.47)(14.76,9.67)(14.85,9.87)(14.94,10.08)(15.04,10.29)(15.13,10.51)(15.23,10.72)(15.32,10.94)(15.41,11.15)(15.51,11.37)(15.60,11.58)(15.70,11.79)(15.79,12.00)(15.89,12.20)(15.98,12.39)(16.07,12.58)(16.17,12.76)(16.26,12.93)(16.36,13.10)(16.45,13.26)(16.54,13.40)(16.64,13.54)(16.73,13.68)(16.83,13.80)(16.92,13.91)(17.02,14.02)(17.11,14.12)(17.20,14.21)(17.30,14.29)(17.39,14.36)(17.49,14.43)(17.58,14.49)(17.68,14.55)(17.77,14.59)(17.86,14.64)(17.96,14.67)(18.05,14.71)(18.15,14.74)(18.24,14.76)(18.33,14.79)(18.43,14.81)(18.52,14.82)(18.62,14.84)(18.71,14.85)(18.81,14.86)(18.90,14.87)(18.99,14.88)(19.09,14.88)(19.18,14.89)(19.28,14.89)(19.37,14.90)(19.47,14.90)(19.56,14.90)(19.66,14.90)(19.75,14.91)(19.84,14.91)(19.94,14.91)
\psline(11.02,6.43)(19.52,6.43)
\psline(11.02,6.43)(11.02,6.22)
\psline(12.44,6.43)(12.44,6.22)
\psline(13.86,6.43)(13.86,6.22)
\psline(15.27,6.43)(15.27,6.22)
\psline(16.69,6.43)(16.69,6.22)
\psline(18.11,6.43)(18.11,6.22)
\psline(19.52,6.43)(19.52,6.22)
\rput(11.02,5.67){-3}
\rput(12.44,5.67){-2}
\rput(13.86,5.67){-1}
\rput(15.27,5.67){0}
\rput(16.69,5.67){1}
\rput(18.11,5.67){2}
\rput(19.52,5.67){3}
\psline(10.23,6.75)(10.23,14.91)
\psline(10.23,6.75)(10.02,6.75)
\psline(10.23,8.38)(10.02,8.38)
\psline(10.23,10.02)(10.02,10.02)
\psline(10.23,11.65)(10.02,11.65)
\psline(10.23,13.28)(10.02,13.28)
\psline(10.23,14.91)(10.02,14.91)
\rput{90}(9.73,6.75){0.0}
\rput{90}(9.73,8.38){0.2}
\rput{90}(9.73,10.02){0.4}
\rput{90}(9.73,11.65){0.6}
\rput{90}(9.73,13.28){0.8}
\rput{90}(9.73,14.91){1.0}
\psline(10.23,6.43)(20.31,6.43)(20.31,15.23)(10.23,15.23)(10.23,6.43)
\rput[l](11.14,15.99){Distribución normal estándar $N(\mu=0,\sigma=1)$}
\rput(15.27,4.82){$X$}
\rput[l]{90}(8.88,8.51){Probabilidad acumulada $F(x)$}
\psset{linecolor=mycolor0}
\psline(10.23,6.75)(20.31,6.75)
\end{pspicture}
%% End
}}
\caption{Función de densidad y función de distribución de la
variable aleatoria continua $Z$ Normal de media 0 y desviación
típica 1 \,\, $Z\sim N(0,1)$} \label{g:graficasvac}
\end{figure}

\subsubsection{Distribución Chi-cuadrado}
Si $Z_1,\ldots,Z_n$ son $n$ v.a.c. normales estándar independientes, entonces la variable
\[ 
X=Z_1^2+\cdots+Z_n^2
\]
se dice que sigue una distribución \emph{Chi-cuadrado} con $n$ grados de libertad, y se nota $X\sim\chi^2(n)$.

\begin{figure}[h!]
\centering
\scalebox{0.8}{%% Input file name: funcion_densidad_chi_cuadrado.fig
%% FIG version: 3.2
%% Orientation: Landscape
%% Justification: Flush Left
%% Units: Inches
%% Paper size: A4
%% Magnification: 100.0
%% Resolution: 1200ppi

\begin{pspicture}(6.68cm,3.48cm)(16.80cm,13.45cm)
\psset{unit=0.8cm}
%%
%% Depth: 2147483647
%%
\newrgbcolor{mycolor0}{1.00 0.50 0.31}\definecolor{mycolor0}{rgb}{1.00,0.50,0.31}
\newrgbcolor{mycolor1}{0.28 0.46 1.00}\definecolor{mycolor1}{rgb}{0.28,0.46,1.00}
\newrgbcolor{mycolor2}{0.56 0.93 0.56}\definecolor{mycolor2}{rgb}{0.56,0.93,0.56}
\newgray{mycolor3}{0.74}\definecolor{mycolor3}{gray}{0.74}
%%
%% Depth: 100
%%
\psset{linestyle=solid,linewidth=0.03175,linecolor=mycolor0,fillstyle=none}
\psline(10.91,15.28)(10.99,13.55)(11.08,12.15)(11.17,11.12)(11.27,10.34)(11.36,9.73)(11.46,9.25)(11.55,8.86)(11.64,8.54)(11.74,8.27)(11.83,8.05)(11.93,7.87)(12.02,7.71)(12.12,7.58)(12.21,7.47)(12.30,7.38)(12.40,7.30)(12.49,7.23)(12.59,7.17)(12.68,7.12)(12.78,7.08)(12.87,7.04)(12.96,7.01)(13.06,6.98)(13.15,6.96)(13.25,6.94)(13.34,6.92)(13.43,6.90)(13.53,6.89)(13.62,6.88)(13.72,6.87)(13.81,6.86)(13.91,6.85)(14.00,6.85)(14.09,6.84)(14.19,6.83)(14.28,6.83)(14.38,6.83)(14.47,6.82)(14.56,6.82)(14.66,6.82)(14.76,6.82)(14.85,6.81)(14.94,6.81)(15.04,6.81)(15.13,6.81)(15.23,6.81)(15.32,6.81)(15.41,6.81)(15.51,6.81)(15.60,6.81)(15.70,6.80)(15.79,6.80)(15.89,6.80)(15.98,6.80)(16.07,6.80)(16.17,6.80)(16.26,6.80)(16.36,6.80)(16.45,6.80)(16.54,6.80)(16.64,6.80)(16.73,6.80)(16.83,6.80)(16.92,6.80)(17.02,6.80)(17.11,6.80)(17.20,6.80)(17.30,6.80)(17.39,6.80)(17.49,6.80)(17.58,6.80)(17.68,6.80)(17.77,6.80)(17.86,6.80)(17.96,6.80)(18.05,6.80)(18.15,6.80)(18.24,6.80)(18.33,6.80)(18.43,6.80)(18.52,6.80)(18.62,6.80)(18.71,6.80)(18.81,6.80)(18.90,6.80)(18.99,6.80)(19.09,6.80)(19.18,6.80)(19.28,6.80)(19.37,6.80)(19.47,6.80)(19.56,6.80)(19.66,6.80)(19.75,6.80)(19.84,6.80)(19.94,6.80)
\psset{linecolor=black}
\psline(10.61,6.47)(20.28,6.47)
\psline(10.61,6.47)(10.61,6.26)
\psline(12.54,6.47)(12.54,6.26)
\psline(14.48,6.47)(14.48,6.26)
\psline(16.41,6.47)(16.41,6.26)
\psline(18.35,6.47)(18.35,6.26)
\psline(20.28,6.47)(20.28,6.26)
\rput(10.61,5.71){0}
\rput(12.54,5.71){5}
\rput(14.48,5.71){10}
\rput(16.41,5.71){15}
\rput(18.35,5.71){20}
\rput(20.28,5.71){25}
\psline(10.23,6.80)(10.23,14.95)
\psline(10.23,6.80)(10.02,6.80)
\psline(10.23,8.16)(10.02,8.16)
\psline(10.23,9.52)(10.02,9.52)
\psline(10.23,10.88)(10.02,10.88)
\psline(10.23,12.23)(10.02,12.23)
\psline(10.23,13.59)(10.02,13.59)
\psline(10.23,14.95)(10.02,14.95)
\rput{90}(9.73,6.80){0.00}
\rput{90}(9.73,8.16){0.05}
\rput{90}(9.73,9.52){0.10}
\rput{90}(9.73,10.88){0.15}
\rput{90}(9.73,12.23){0.20}
\rput{90}(9.73,13.59){0.25}
\rput{90}(9.73,14.95){0.30}
\psline(10.23,6.47)(20.31,6.47)(20.31,15.28)(10.23,15.28)(10.23,6.47)
\rput(15.27,15.99){Distintas distribuciones chi-cuadrado}
\rput(15.27,4.86){$X$}
\rput[l]{90}(8.86,9.54){Densidad $f(x)$}
\psset{linecolor=mycolor1}
\psline(10.61,6.80)(10.70,11.54)(10.80,12.73)(10.89,13.23)(10.99,13.38)(11.08,13.31)(11.17,13.11)(11.27,12.84)(11.36,12.51)(11.46,12.16)(11.55,11.81)(11.64,11.45)(11.74,11.10)(11.83,10.76)(11.93,10.44)(12.02,10.14)(12.12,9.85)(12.21,9.58)(12.30,9.34)(12.40,9.11)(12.49,8.90)(12.59,8.70)(12.68,8.52)(12.78,8.36)(12.87,8.21)(12.96,8.08)(13.06,7.95)(13.15,7.84)(13.25,7.74)(13.34,7.64)(13.43,7.56)(13.53,7.48)(13.62,7.41)(13.72,7.35)(13.81,7.30)(13.91,7.25)(14.00,7.20)(14.09,7.16)(14.19,7.12)(14.28,7.09)(14.38,7.06)(14.47,7.03)(14.56,7.01)(14.66,6.99)(14.76,6.97)(14.85,6.95)(14.94,6.93)(15.04,6.92)(15.13,6.91)(15.23,6.90)(15.32,6.89)(15.41,6.88)(15.51,6.87)(15.60,6.86)(15.70,6.85)(15.79,6.85)(15.89,6.84)(15.98,6.84)(16.07,6.83)(16.17,6.83)(16.26,6.83)(16.36,6.82)(16.45,6.82)(16.54,6.82)(16.64,6.82)(16.73,6.82)(16.83,6.81)(16.92,6.81)(17.02,6.81)(17.11,6.81)(17.20,6.81)(17.30,6.81)(17.39,6.81)(17.49,6.81)(17.58,6.81)(17.68,6.81)(17.77,6.80)(17.86,6.80)(17.96,6.80)(18.05,6.80)(18.15,6.80)(18.24,6.80)(18.33,6.80)(18.43,6.80)(18.52,6.80)(18.62,6.80)(18.71,6.80)(18.81,6.80)(18.90,6.80)(18.99,6.80)(19.09,6.80)(19.18,6.80)(19.28,6.80)(19.37,6.80)(19.47,6.80)(19.56,6.80)(19.66,6.80)(19.75,6.80)(19.84,6.80)(19.94,6.80)
\psset{linecolor=mycolor2}
\psline(10.61,6.80)(10.70,6.80)(10.80,6.80)(10.89,6.81)(10.99,6.82)(11.08,6.84)(11.17,6.88)(11.27,6.93)(11.36,6.99)(11.46,7.07)(11.55,7.17)(11.64,7.28)(11.74,7.40)(11.83,7.53)(11.93,7.67)(12.02,7.81)(12.12,7.96)(12.21,8.11)(12.30,8.26)(12.40,8.40)(12.49,8.54)(12.59,8.68)(12.68,8.80)(12.78,8.92)(12.87,9.02)(12.96,9.11)(13.06,9.20)(13.15,9.27)(13.25,9.33)(13.34,9.37)(13.43,9.41)(13.53,9.44)(13.62,9.45)(13.72,9.45)(13.81,9.45)(13.91,9.43)(14.00,9.41)(14.09,9.38)(14.19,9.34)(14.28,9.29)(14.38,9.24)(14.47,9.19)(14.56,9.13)(14.66,9.07)(14.76,9.00)(14.85,8.93)(14.94,8.86)(15.04,8.79)(15.13,8.71)(15.23,8.64)(15.32,8.57)(15.41,8.49)(15.51,8.42)(15.60,8.35)(15.70,8.28)(15.79,8.21)(15.89,8.14)(15.98,8.07)(16.07,8.01)(16.17,7.94)(16.26,7.88)(16.36,7.83)(16.45,7.77)(16.54,7.71)(16.64,7.66)(16.73,7.61)(16.83,7.56)(16.92,7.52)(17.02,7.47)(17.11,7.43)(17.20,7.39)(17.30,7.36)(17.39,7.32)(17.49,7.29)(17.58,7.26)(17.68,7.23)(17.77,7.20)(17.86,7.17)(17.96,7.15)(18.05,7.12)(18.15,7.10)(18.24,7.08)(18.33,7.06)(18.43,7.04)(18.52,7.02)(18.62,7.01)(18.71,6.99)(18.81,6.98)(18.90,6.97)(18.99,6.95)(19.09,6.94)(19.18,6.93)(19.28,6.92)(19.37,6.91)(19.47,6.90)(19.56,6.90)(19.66,6.89)(19.75,6.88)(19.84,6.87)(19.94,6.87)
\psset{linecolor=mycolor3}
\psline(10.23,6.80)(20.31,6.80)
\psset{linecolor=mycolor0}
\psline(18.17,14.53)(18.80,14.53)
\psset{linecolor=mycolor1}
\psline(18.17,14.11)(18.80,14.11)
\psset{linecolor=mycolor2}
\psline(18.17,13.68)(18.80,13.68)
\rput[lb](19,14.33){$\chi^2(1)$}
\rput[lb](19,13.91){$\chi^2(3)$}
\rput[lb](19,13.49){$\chi^2(10)$}
\end{pspicture}
%% End
} 
\caption{Función de densidad de una variable aleatoria Chi cuadrado de 6 grados de libertad}
\label{g:chicuadrado}
\end{figure}

Se cumple que
\begin{align*}
\mu &= n\\
\sigma &=+\sqrt{2n}
\end{align*}

La distribución Chi-cuadrado se utiliza en inferencia estadística para cálculos de intervalos de confianza y contrastes
de hipótesis sobre la varianza de la población.

\subsubsection{Distribución $T$ de Student}
Si $Z$ es una v.a.c. normal estándar y $X$ es una v.a.c chi-cuadrado con $n$ grados de libertad, ambas variables
independientes, entonces la variable
\[
T=\frac{Z}{\sqrt{X/n}}
\]
se dice que sigue una distribución \emph{T de Student} con $n$ grados de libertad, y se nota $T\sim T(n)$.

\begin{figure}[h!]
\centering
\scalebox{0.8}{%% Input file name: funcion_densidad_t_student.fig
%% FIG version: 3.2
%% Orientation: Landscape
%% Justification: Flush Left
%% Units: Inches
%% Paper size: A4
%% Magnification: 100.0
%% Resolution: 1200ppi

\begin{pspicture}(6.68cm,3.48cm)(16.66cm,13.45cm)
\psset{unit=0.8cm}
%%
%% Depth: 2147483647
%%
\newrgbcolor{mycolor0}{1.00 0.50 0.31}\definecolor{mycolor0}{rgb}{1.00,0.50,0.31}
\newrgbcolor{mycolor1}{0.28 0.46 1.00}\definecolor{mycolor1}{rgb}{0.28,0.46,1.00}
\newrgbcolor{mycolor2}{0.56 0.93 0.56}\definecolor{mycolor2}{rgb}{0.56,0.93,0.56}
\newgray{mycolor3}{0.74}\definecolor{mycolor3}{gray}{0.74}
%%
%% Depth: 100
%%
\psset{linestyle=solid,linewidth=0.03175,linecolor=mycolor0,fillstyle=none}
\psline(10.61,7.09)(10.70,7.11)(10.80,7.12)(10.89,7.13)(10.99,7.15)(11.08,7.16)(11.17,7.18)(11.27,7.19)(11.36,7.21)(11.46,7.23)(11.55,7.25)(11.64,7.27)(11.74,7.30)(11.83,7.32)(11.93,7.35)(12.02,7.38)(12.12,7.41)(12.21,7.44)(12.30,7.48)(12.40,7.52)(12.49,7.56)(12.59,7.61)(12.68,7.66)(12.78,7.72)(12.87,7.79)(12.96,7.85)(13.06,7.93)(13.15,8.01)(13.25,8.10)(13.34,8.21)(13.43,8.32)(13.53,8.45)(13.62,8.59)(13.72,8.74)(13.81,8.92)(13.91,9.11)(14.00,9.33)(14.09,9.57)(14.19,9.84)(14.28,10.13)(14.38,10.45)(14.47,10.80)(14.56,11.17)(14.66,11.56)(14.76,11.95)(14.85,12.33)(14.94,12.67)(15.04,12.96)(15.13,13.16)(15.23,13.27)(15.32,13.27)(15.41,13.16)(15.51,12.96)(15.60,12.67)(15.70,12.33)(15.79,11.95)(15.89,11.56)(15.98,11.17)(16.07,10.80)(16.17,10.45)(16.26,10.13)(16.36,9.84)(16.45,9.57)(16.54,9.33)(16.64,9.11)(16.73,8.92)(16.83,8.74)(16.92,8.59)(17.02,8.45)(17.11,8.32)(17.20,8.21)(17.30,8.10)(17.39,8.01)(17.49,7.93)(17.58,7.85)(17.68,7.79)(17.77,7.72)(17.86,7.66)(17.96,7.61)(18.05,7.56)(18.15,7.52)(18.24,7.48)(18.33,7.44)(18.43,7.41)(18.52,7.38)(18.62,7.35)(18.71,7.32)(18.81,7.30)(18.90,7.27)(18.99,7.25)(19.09,7.23)(19.18,7.21)(19.28,7.19)(19.37,7.18)(19.47,7.16)(19.56,7.15)(19.66,7.13)(19.75,7.12)(19.84,7.11)(19.94,7.09)
\psset{linecolor=black}
\psline(11.21,6.47)(19.34,6.47)
\psline(11.21,6.47)(11.21,6.26)
\psline(13.24,6.47)(13.24,6.26)
\psline(15.27,6.47)(15.27,6.26)
\psline(17.31,6.47)(17.31,6.26)
\psline(19.34,6.47)(19.34,6.26)
\rput(11.21,5.71){-4}
\rput(13.24,5.71){-2}
\rput(15.27,5.71){0}
\rput(17.31,5.71){2}
\rput(19.34,5.71){4}
\psline(10.23,6.80)(10.23,14.95)
\psline(10.23,6.80)(10.02,6.80)
\psline(10.23,8.84)(10.02,8.84)
\psline(10.23,10.88)(10.02,10.88)
\psline(10.23,12.91)(10.02,12.91)
\psline(10.23,14.95)(10.02,14.95)
\rput{90}(9.73,6.80){0.0}
\rput{90}(9.73,8.84){0.1}
\rput{90}(9.73,10.88){0.2}
\rput{90}(9.73,12.91){0.3}
\rput{90}(9.73,14.95){0.4}
\psline(10.23,6.47)(20.31,6.47)(20.31,15.28)(10.23,15.28)(10.23,6.47)
\rput(15.27,15.99){Distintas distribuciones t de student}
\rput(15.27,4.86){$X$}
\rput[l]{90}(8.86,9.54){Densidad $f(x)$}
\psset{linecolor=mycolor1}
\psline(10.61,6.92)(10.70,6.92)(10.80,6.93)(10.89,6.94)(10.99,6.96)(11.08,6.97)(11.17,6.98)(11.27,7.00)(11.36,7.01)(11.46,7.03)(11.55,7.05)(11.64,7.07)(11.74,7.10)(11.83,7.12)(11.93,7.15)(12.02,7.19)(12.12,7.22)(12.21,7.26)(12.30,7.31)(12.40,7.36)(12.49,7.41)(12.59,7.48)(12.68,7.55)(12.78,7.63)(12.87,7.72)(12.96,7.81)(13.06,7.92)(13.15,8.05)(13.25,8.19)(13.34,8.34)(13.43,8.52)(13.53,8.71)(13.62,8.93)(13.72,9.17)(13.81,9.43)(13.91,9.72)(14.00,10.04)(14.09,10.38)(14.19,10.74)(14.28,11.13)(14.38,11.53)(14.47,11.94)(14.56,12.36)(14.66,12.76)(14.76,13.14)(14.85,13.49)(14.94,13.79)(15.04,14.03)(15.13,14.20)(15.23,14.28)(15.32,14.28)(15.41,14.20)(15.51,14.03)(15.60,13.79)(15.70,13.49)(15.79,13.14)(15.89,12.76)(15.98,12.36)(16.07,11.94)(16.17,11.53)(16.26,11.13)(16.36,10.74)(16.45,10.38)(16.54,10.04)(16.64,9.72)(16.73,9.43)(16.83,9.17)(16.92,8.93)(17.02,8.71)(17.11,8.52)(17.20,8.34)(17.30,8.19)(17.39,8.05)(17.49,7.92)(17.58,7.81)(17.68,7.72)(17.77,7.63)(17.86,7.55)(17.96,7.48)(18.05,7.41)(18.15,7.36)(18.24,7.31)(18.33,7.26)(18.43,7.22)(18.52,7.19)(18.62,7.15)(18.71,7.12)(18.81,7.10)(18.90,7.07)(18.99,7.05)(19.09,7.03)(19.18,7.01)(19.28,7.00)(19.37,6.98)(19.47,6.97)(19.56,6.96)(19.66,6.94)(19.75,6.93)(19.84,6.92)(19.94,6.92)
\psset{linecolor=mycolor2}
\psline(10.61,6.82)(10.70,6.82)(10.80,6.82)(10.89,6.82)(10.99,6.83)(11.08,6.83)(11.17,6.84)(11.27,6.85)(11.36,6.85)(11.46,6.86)(11.55,6.87)(11.64,6.89)(11.74,6.90)(11.83,6.92)(11.93,6.94)(12.02,6.96)(12.12,6.99)(12.21,7.03)(12.30,7.07)(12.40,7.11)(12.49,7.17)(12.59,7.23)(12.68,7.30)(12.78,7.39)(12.87,7.49)(12.96,7.61)(13.06,7.74)(13.15,7.89)(13.25,8.06)(13.34,8.26)(13.43,8.48)(13.53,8.72)(13.62,9.00)(13.72,9.30)(13.81,9.63)(13.91,9.98)(14.00,10.36)(14.09,10.77)(14.19,11.19)(14.28,11.62)(14.38,12.06)(14.47,12.49)(14.56,12.92)(14.66,13.32)(14.76,13.69)(14.85,14.01)(14.94,14.29)(15.04,14.50)(15.13,14.65)(15.23,14.72)(15.32,14.72)(15.41,14.65)(15.51,14.50)(15.60,14.29)(15.70,14.01)(15.79,13.69)(15.89,13.32)(15.98,12.92)(16.07,12.49)(16.17,12.06)(16.26,11.62)(16.36,11.19)(16.45,10.77)(16.54,10.36)(16.64,9.98)(16.73,9.63)(16.83,9.30)(16.92,9.00)(17.02,8.72)(17.11,8.48)(17.20,8.26)(17.30,8.06)(17.39,7.89)(17.49,7.74)(17.58,7.61)(17.68,7.49)(17.77,7.39)(17.86,7.30)(17.96,7.23)(18.05,7.17)(18.15,7.11)(18.24,7.07)(18.33,7.03)(18.43,6.99)(18.52,6.96)(18.62,6.94)(18.71,6.92)(18.81,6.90)(18.90,6.89)(18.99,6.87)(19.09,6.86)(19.18,6.85)(19.28,6.85)(19.37,6.84)(19.47,6.83)(19.56,6.83)(19.66,6.82)(19.75,6.82)(19.84,6.82)(19.94,6.82)
\psset{linecolor=mycolor3}
\psline(10.23,6.80)(20.31,6.80)
\psset{linecolor=mycolor0}
\psline(18.14,14.53)(18.78,14.53)
\psset{linecolor=mycolor1}
\psline(18.14,14.11)(18.78,14.11)
\psset{linecolor=mycolor2}
\psline(18.14,13.68)(18.78,13.68)
\rput[l](19,14.42){$T(1)$}
\rput[l](19,14.00){$T(3)$}
\rput[l](19,13.57){$T(10)$}
\end{pspicture}
%% End
} 
\caption{Función de densidad de una variable aleatoria t de student de 10 grados de libertad}
\label{g:tstudent}
\end{figure}

Esta variable es muy parecida a la normal estándar pero un poco menos apuntada, y se parece más a ésta a medida que
aumentan los grados de libertad, de manera que para $n\geq 30$ ambas distribuciones se consideran prácticamente iguales.
Se cumple que
\begin{align*}
\mu &= 0\\
\sigma &=+\sqrt{n/(n-2)}\quad \textrm{con } n>2
\end{align*}

La distribución $T$ de Student se utiliza en inferencia estadística para cálculos de intervalos de confianza y
contrastes de hipótesis sobre la media de la población.

\subsubsection{Distribución $F$ de Fisher-Snedecor}
Si $X$ e $Y$ son dos v.a.c chi-cuadrado con $m$ y $n$ grados de libertad respectivamente, ambas variables
independientes, entonces la variable
\[
F=\frac{X/m}{Y/n}
\]
se dice que sigue una distribución $F$ de Fisher-Snedecor con $m$ y $n$ grados de libertad, y se denota $F\sim F(m,n)$.

\begin{figure}[h!]
\centering
\scalebox{0.8}{%% Input file name: funcion_densidad_f_fisher.fig
%% FIG version: 3.2
%% Orientation: Landscape
%% Justification: Flush Left
%% Units: Inches
%% Paper size: A4
%% Magnification: 100.0
%% Resolution: 1200ppi

\begin{pspicture}(6.68cm,3.48cm)(16.66cm,13.45cm)
\psset{unit=0.8cm}
%%
%% Depth: 2147483647
%%
\newrgbcolor{mycolor0}{1.00 0.50 0.31}\definecolor{mycolor0}{rgb}{1.00,0.50,0.31}
\newrgbcolor{mycolor1}{0.28 0.46 1.00}\definecolor{mycolor1}{rgb}{0.28,0.46,1.00}
\newrgbcolor{mycolor2}{0.56 0.93 0.56}\definecolor{mycolor2}{rgb}{0.56,0.93,0.56}
\newgray{mycolor3}{0.74}\definecolor{mycolor3}{gray}{0.74}
%%
%% Depth: 100
%%
\psset{linestyle=solid,linewidth=0.03175,linecolor=mycolor0,fillstyle=none}
\psline(10.61,6.80)(10.66,10.19)(10.70,11.26)(10.75,11.89)(10.79,12.29)(10.84,12.53)(10.89,12.68)(10.94,12.75)(10.98,12.77)(11.03,12.75)(11.08,12.70)(11.12,12.63)(11.17,12.55)(11.22,12.45)(11.26,12.34)(11.31,12.22)(11.36,12.11)(11.40,11.99)(11.45,11.86)(11.50,11.74)(11.55,11.62)(11.59,11.50)(11.64,11.38)(11.69,11.26)(11.73,11.15)(11.78,11.04)(11.83,10.93)(11.87,10.82)(11.92,10.71)(11.97,10.61)(12.01,10.51)(12.06,10.42)(12.11,10.32)(12.16,10.23)(12.20,10.14)(12.25,10.06)(12.30,9.98)(12.34,9.90)(12.39,9.82)(12.44,9.74)(12.48,9.67)(12.53,9.60)(12.58,9.53)(12.62,9.46)(12.67,9.40)(12.72,9.33)(12.76,9.28)(12.81,9.22)(12.86,9.16)(12.91,9.10)(12.95,9.05)(13.00,9.00)(13.05,8.95)(13.09,8.90)(13.14,8.85)(13.19,8.81)(13.23,8.76)(13.28,8.72)(13.33,8.68)(13.37,8.64)(13.42,8.60)(13.47,8.56)(13.51,8.52)(13.56,8.48)(13.61,8.45)(13.65,8.41)(13.70,8.38)(13.75,8.35)(13.80,8.32)(13.84,8.29)(13.89,8.26)(13.94,8.23)(13.98,8.20)(14.03,8.17)(14.08,8.14)(14.12,8.12)(14.17,8.09)(14.22,8.07)(14.26,8.05)(14.31,8.02)(14.36,8.00)(14.41,7.98)(14.45,7.95)(14.50,7.93)(14.55,7.91)(14.59,7.89)(14.64,7.87)(14.69,7.85)(14.73,7.84)(14.78,7.82)(14.83,7.80)(14.87,7.78)(14.92,7.77)(14.97,7.75)(15.02,7.73)(15.06,7.72)(15.11,7.70)(15.16,7.69)(15.20,7.67)(15.25,7.66)(15.30,7.64)(15.34,7.63)(15.39,7.62)(15.44,7.60)(15.48,7.59)(15.53,7.58)(15.58,7.56)(15.63,7.55)(15.67,7.54)(15.72,7.53)(15.76,7.52)(15.81,7.51)(15.86,7.50)(15.90,7.49)(15.95,7.48)(16.00,7.47)(16.05,7.45)(16.09,7.45)(16.14,7.44)(16.19,7.43)(16.23,7.42)(16.28,7.41)(16.33,7.40)(16.37,7.39)(16.42,7.38)(16.47,7.37)(16.51,7.37)(16.56,7.36)(16.61,7.35)(16.66,7.34)(16.70,7.34)(16.75,7.33)(16.80,7.32)(16.84,7.32)(16.89,7.31)(16.94,7.30)(16.98,7.29)(17.03,7.29)(17.08,7.28)(17.12,7.27)(17.17,7.27)(17.22,7.26)(17.27,7.26)(17.31,7.25)(17.36,7.25)(17.41,7.24)(17.45,7.23)(17.50,7.23)(17.55,7.22)(17.59,7.22)(17.64,7.21)(17.69,7.21)(17.73,7.20)(17.78,7.20)(17.83,7.19)(17.88,7.19)(17.92,7.18)(17.97,7.18)(18.01,7.17)(18.06,7.17)(18.11,7.16)(18.16,7.16)(18.20,7.16)(18.25,7.15)(18.30,7.15)(18.34,7.14)(18.39,7.14)(18.44,7.14)(18.48,7.13)(18.53,7.13)(18.58,7.12)(18.62,7.12)(18.67,7.12)(18.72,7.11)(18.76,7.11)(18.81,7.11)(18.86,7.10)(18.91,7.10)(18.95,7.10)(19.00,7.09)(19.05,7.09)(19.09,7.09)(19.14,7.08)(19.19,7.08)(19.23,7.08)(19.28,7.08)(19.33,7.07)(19.37,7.07)(19.42,7.07)(19.47,7.06)(19.52,7.06)(19.56,7.06)(19.61,7.05)(19.66,7.05)(19.70,7.05)(19.75,7.05)(19.80,7.05)(19.84,7.04)(19.89,7.04)(19.94,7.04)
\psset{linecolor=black}
\psline(10.61,6.47)(19.94,6.47)
\psline(10.61,6.47)(10.61,6.26)
\psline(12.47,6.47)(12.47,6.26)
\psline(14.34,6.47)(14.34,6.26)
\psline(16.21,6.47)(16.21,6.26)
\psline(18.07,6.47)(18.07,6.26)
\psline(19.94,6.47)(19.94,6.26)
\rput(10.61,5.71){0}
\rput(12.47,5.71){1}
\rput(14.34,5.71){2}
\rput(16.21,5.71){3}
\rput(18.07,5.71){4}
\rput(19.94,5.71){5}
\psline(10.23,6.80)(10.23,14.05)
\psline(10.23,6.80)(10.02,6.80)
\psline(10.23,8.61)(10.02,8.61)
\psline(10.23,10.42)(10.02,10.42)
\psline(10.23,12.23)(10.02,12.23)
\psline(10.23,14.05)(10.02,14.05)
\rput{90}(9.73,6.80){0.0}
\rput{90}(9.73,8.61){0.2}
\rput{90}(9.73,10.42){0.4}
\rput{90}(9.73,12.23){0.6}
\rput{90}(9.73,14.05){0.8}
\psline(10.23,6.47)(20.31,6.47)(20.31,15.28)(10.23,15.28)(10.23,6.47)
\rput(15.27,15.99){Distintas distribuciones F de Fisher-Snedecor}
\rput(15.27,4.86){$X$}
\rput[l]{90}(8.86,9.54){Densidad $f(x)$}
\psset{linecolor=mycolor1}
\psline(10.61,6.80)(10.66,6.80)(10.70,6.85)(10.75,6.99)(10.79,7.24)(10.84,7.59)(10.89,8.02)(10.94,8.50)(10.98,9.00)(11.03,9.51)(11.08,10.00)(11.12,10.46)(11.17,10.88)(11.22,11.26)(11.26,11.59)(11.31,11.88)(11.36,12.12)(11.40,12.31)(11.45,12.47)(11.50,12.59)(11.55,12.68)(11.59,12.73)(11.64,12.76)(11.69,12.77)(11.73,12.75)(11.78,12.72)(11.83,12.67)(11.87,12.61)(11.92,12.53)(11.97,12.45)(12.01,12.36)(12.06,12.26)(12.11,12.16)(12.16,12.05)(12.20,11.94)(12.25,11.83)(12.30,11.72)(12.34,11.61)(12.39,11.49)(12.44,11.38)(12.48,11.26)(12.53,11.15)(12.58,11.04)(12.62,10.93)(12.67,10.83)(12.72,10.72)(12.76,10.62)(12.81,10.52)(12.86,10.42)(12.91,10.32)(12.95,10.22)(13.00,10.13)(13.05,10.04)(13.09,9.95)(13.14,9.87)(13.19,9.79)(13.23,9.70)(13.28,9.63)(13.33,9.55)(13.37,9.48)(13.42,9.40)(13.47,9.33)(13.51,9.27)(13.56,9.20)(13.61,9.14)(13.65,9.08)(13.70,9.01)(13.75,8.96)(13.80,8.90)(13.84,8.85)(13.89,8.79)(13.94,8.74)(13.98,8.69)(14.03,8.64)(14.08,8.60)(14.12,8.55)(14.17,8.50)(14.22,8.46)(14.26,8.42)(14.31,8.38)(14.36,8.34)(14.41,8.30)(14.45,8.27)(14.50,8.23)(14.55,8.20)(14.59,8.16)(14.64,8.13)(14.69,8.10)(14.73,8.07)(14.78,8.04)(14.83,8.01)(14.87,7.98)(14.92,7.95)(14.97,7.92)(15.02,7.90)(15.06,7.87)(15.11,7.85)(15.16,7.83)(15.20,7.80)(15.25,7.78)(15.30,7.76)(15.34,7.74)(15.39,7.72)(15.44,7.69)(15.48,7.68)(15.53,7.66)(15.58,7.64)(15.63,7.62)(15.67,7.60)(15.72,7.58)(15.76,7.57)(15.81,7.55)(15.86,7.54)(15.90,7.52)(15.95,7.51)(16.00,7.49)(16.05,7.48)(16.09,7.46)(16.14,7.45)(16.19,7.44)(16.23,7.42)(16.28,7.41)(16.33,7.40)(16.37,7.39)(16.42,7.38)(16.47,7.36)(16.51,7.35)(16.56,7.34)(16.61,7.33)(16.66,7.32)(16.70,7.31)(16.75,7.30)(16.80,7.29)(16.84,7.28)(16.89,7.27)(16.94,7.26)(16.98,7.26)(17.03,7.25)(17.08,7.24)(17.12,7.23)(17.17,7.22)(17.22,7.22)(17.27,7.21)(17.31,7.20)(17.36,7.19)(17.41,7.19)(17.45,7.18)(17.50,7.17)(17.55,7.17)(17.59,7.16)(17.64,7.15)(17.69,7.15)(17.73,7.14)(17.78,7.14)(17.83,7.13)(17.88,7.12)(17.92,7.12)(17.97,7.11)(18.01,7.11)(18.06,7.10)(18.11,7.10)(18.16,7.09)(18.20,7.09)(18.25,7.08)(18.30,7.08)(18.34,7.07)(18.39,7.07)(18.44,7.07)(18.48,7.06)(18.53,7.06)(18.58,7.05)(18.62,7.05)(18.67,7.05)(18.72,7.04)(18.76,7.04)(18.81,7.03)(18.86,7.03)(18.91,7.03)(18.95,7.02)(19.00,7.02)(19.05,7.02)(19.09,7.01)(19.14,7.01)(19.19,7.01)(19.23,7.00)(19.28,7.00)(19.33,7.00)(19.37,7.00)(19.42,6.99)(19.47,6.99)(19.52,6.99)(19.56,6.98)(19.61,6.98)(19.66,6.98)(19.70,6.98)(19.75,6.97)(19.80,6.97)(19.84,6.97)(19.89,6.97)(19.94,6.96)
\psset{linecolor=mycolor2}
\psline(10.61,6.80)(10.66,6.80)(10.70,6.81)(10.75,6.85)(10.79,6.94)(10.84,7.08)(10.89,7.29)(10.94,7.56)(10.98,7.90)(11.03,8.28)(11.08,8.71)(11.12,9.17)(11.17,9.65)(11.22,10.13)(11.26,10.62)(11.31,11.09)(11.36,11.54)(11.40,11.97)(11.45,12.37)(11.50,12.73)(11.55,13.06)(11.59,13.35)(11.64,13.60)(11.69,13.82)(11.73,13.99)(11.78,14.13)(11.83,14.24)(11.87,14.31)(11.92,14.35)(11.97,14.36)(12.01,14.35)(12.06,14.31)(12.11,14.25)(12.16,14.17)(12.20,14.07)(12.25,13.96)(12.30,13.84)(12.34,13.70)(12.39,13.55)(12.44,13.40)(12.48,13.24)(12.53,13.07)(12.58,12.90)(12.62,12.73)(12.67,12.55)(12.72,12.38)(12.76,12.20)(12.81,12.03)(12.86,11.86)(12.91,11.68)(12.95,11.51)(13.00,11.35)(13.05,11.18)(13.09,11.02)(13.14,10.86)(13.19,10.71)(13.23,10.56)(13.28,10.42)(13.33,10.27)(13.37,10.14)(13.42,10.01)(13.47,9.88)(13.51,9.75)(13.56,9.63)(13.61,9.51)(13.65,9.40)(13.70,9.30)(13.75,9.19)(13.80,9.09)(13.84,9.00)(13.89,8.90)(13.94,8.81)(13.98,8.73)(14.03,8.65)(14.08,8.57)(14.12,8.49)(14.17,8.42)(14.22,8.35)(14.26,8.28)(14.31,8.22)(14.36,8.16)(14.41,8.10)(14.45,8.04)(14.50,7.99)(14.55,7.94)(14.59,7.89)(14.64,7.84)(14.69,7.80)(14.73,7.76)(14.78,7.71)(14.83,7.68)(14.87,7.64)(14.92,7.60)(14.97,7.57)(15.02,7.53)(15.06,7.50)(15.11,7.47)(15.16,7.44)(15.20,7.41)(15.25,7.39)(15.30,7.36)(15.34,7.34)(15.39,7.32)(15.44,7.30)(15.48,7.27)(15.53,7.25)(15.58,7.23)(15.63,7.22)(15.67,7.20)(15.72,7.18)(15.76,7.17)(15.81,7.15)(15.86,7.14)(15.90,7.12)(15.95,7.11)(16.00,7.10)(16.05,7.08)(16.09,7.07)(16.14,7.06)(16.19,7.05)(16.23,7.04)(16.28,7.03)(16.33,7.02)(16.37,7.01)(16.42,7.00)(16.47,7.00)(16.51,6.99)(16.56,6.98)(16.61,6.97)(16.66,6.97)(16.70,6.96)(16.75,6.95)(16.80,6.95)(16.84,6.94)(16.89,6.93)(16.94,6.93)(16.98,6.92)(17.03,6.92)(17.08,6.92)(17.12,6.91)(17.17,6.90)(17.22,6.90)(17.27,6.90)(17.31,6.89)(17.36,6.89)(17.41,6.89)(17.45,6.88)(17.50,6.88)(17.55,6.88)(17.59,6.87)(17.64,6.87)(17.69,6.87)(17.73,6.87)(17.78,6.86)(17.83,6.86)(17.88,6.86)(17.92,6.86)(17.97,6.85)(18.01,6.85)(18.06,6.85)(18.11,6.85)(18.16,6.85)(18.20,6.85)(18.25,6.84)(18.30,6.84)(18.34,6.84)(18.39,6.84)(18.44,6.84)(18.48,6.83)(18.53,6.83)(18.58,6.83)(18.62,6.83)(18.67,6.83)(18.72,6.83)(18.76,6.83)(18.81,6.83)(18.86,6.83)(18.91,6.83)(18.95,6.82)(19.00,6.82)(19.05,6.82)(19.09,6.82)(19.14,6.82)(19.19,6.82)(19.23,6.82)(19.28,6.82)(19.33,6.82)(19.37,6.82)(19.42,6.82)(19.47,6.82)(19.52,6.82)(19.56,6.82)(19.61,6.82)(19.66,6.81)(19.70,6.81)(19.75,6.81)(19.80,6.81)(19.84,6.81)(19.89,6.81)(19.94,6.81)
\psset{linecolor=mycolor3}
\psline(10.23,6.80)(20.31,6.80)
\psset{linecolor=mycolor0}
\psline(17.51,14.53)(18.15,14.53)
\psset{linecolor=mycolor1}
\psline(17.51,14.11)(18.15,14.11)
\psset{linecolor=mycolor2}
\psline(17.51,13.68)(18.15,13.68)
\rput[l](18.3,14.42){$F(3,3)$}
\rput[l](18.3,14.00){$F(10,5)$}
\rput[l](18.3,13.57){$F(10,20)$}
\end{pspicture}
%% End
} 
\caption{Función de densidad de una variable aleatoria F de Fisher-Snedecor de 6 y 8 grados de libertad}
\label{g:ffisher}
\end{figure}

\begin{align*}
\mu &= \frac{n}{n-2}\\
\sigma &=+\sqrt{\frac{2n^2(m+n-2)}{m(n-2)^2(n-4)}}\quad \textrm{con
} n>4
\end{align*}

De la definición se deduce fácilmente que $F(m,n)=\dfrac{1}{F(n,m)}$, y si llamamos $F(m,n)_p$ al valor que cumple
que $P(F(m,n)\leq F(m,n)_p)=p$, entonces se verifica
\[
F(m,n)_p = \frac{1}{F(n,m)_{1-p}}
\]

La distribución $F$ de Fisher-Snedecor se utiliza en inferencia estadística para cálculos de intervalos
de confianza y contrastes de hipótesis sobre el cociente de varianzas de dos poblaciones, y en análisis de la varianza.

\clearpage
\newpage

\section{Ejercicios resueltos}
\begin{enumerate}[leftmargin=*]
\item Supongase que un autobús pasa por una parada cada 15 minutos y que una persona puede llegar a la parada en
cualquier instante, entonces la variable que mide el tiempo que la persona espera al autobús es una variable Uniforme
contnua $U(0,15)$, ya que cualquier valor entre 0 y 15 minutos es equiprobable.
Se pide:
\begin{enumerate}
\item Dibujar la gráfica de la función de densidad de la Uniforme $X\sim U(0,15)$. 
\begin{indicacion}{
\begin{enumerate}
\item Seleccionar el menú \menu{Teaching>Distribuciones>Continuas>Uniforme\flecha
Gráfico de probabilidad}.
\item En el cuadro de diálogo que aparece, introducir el valor 0 en el campo \campo{Mínimo}, 15 en el campo
\campo{Máximo} y hacer clic en el botón \boton{Enviar}.
\end{enumerate}}
\end{indicacion}

\item Dibujar la gráfica de la función de distribución. 
\begin{indicacion}{
\begin{enumerate}
\item Seleccionar el menú \menu{Teaching>Distribuciones>Continuas>Uniforme \flecha
Gráfico de probabilidad}.
\item En el cuadro de diálogo que aparece, introducir el valor 0 en el campo \campo{Mínimo}, 15 en el campo
\campo{Máximo}, marcar la opción \opcion{Función de distribución} y hacer clic en el botón \boton{Enviar}.
\end{enumerate}}\end{indicacion}

\item Calcular la probabilidad de esperar al autobús menos de $5$ minutos.
\begin{indicacion}{
\begin{enumerate}
\item Seleccionar el menú \menu{Teaching>Distribuciones>Continuas>Uniforme>Probabilidades}.
\item En el cuadro de diálogo que aparece, introducir el valor $5$ en el campo \campo{Valores de la variable}, 0 en
el campo \campo{Mínimo}, 15 en el campo \campo{Máximo} y hacer clic en el botón \boton{Enviar}.
\end{enumerate}}
\end{indicacion}

\item Calcular la probabilidad de esperar al autobús más de $12$ minutos.
\begin{indicacion}{
\begin{enumerate}
\item Seleccionar el menú \menu{Teaching>Distribuciones>Continuas>Uniforme>Probabilidades}.
\item En el cuadro de diálogo que aparece, introducir el valor $12$ en el campo \campo{Valores de la variable}, 0 en el
campo \campo{Mínimo}, 15 en el campo \campo{Máximo}, seleccionar la opción \opcion{derecha} en el campo \campo{cola de acumulación} y hacer clic en el
botón \boton{Enviar}.
\end{enumerate}}
\end{indicacion}

\item Calcular la probabilidad de esperar al autobús entre $5$ y $10$ minutos.
\begin{indicacion}{
\begin{enumerate}
\item Seleccionar el menú \menu{Teaching>Distribuciones>Continuas>Uniforme>Probabilidades}.
\item En el cuadro de diálogo que aparece, introducir los valores 10, 5 en el campo \campo{Valores de la variable}, 0 en el campo
\campo{Mínimo}, 15 en el campo \campo{Máximo} y hacer clic en el botón \boton{Enviar}.
\end{enumerate}
La probabilidad del intervalo $P(5\leq X\leq 10)$ es la resta de las probabilidades obtenidas $P(X\leq 10)-P(X\leq 5)$.
}
\end{indicacion}

\item ¿Por debajo de qué tiempo esperará al autobús la mitad de las veces?
\begin{indicacion}{
\begin{enumerate}
\item Seleccionar el menú \menu{Teaching>Distribuciones>Continuas>Uniforme\flecha
Cuantiles}.
\item En el cuadro de diálogo que aparece, introducir la probabilidad 0.5 en el campo \campo{Probabilidades acumuladas},
0 en el campo \campo{Mínimo}, 15 en el campo \campo{Máximo} y hacer clic en el botón \boton{Enviar}.
\end{enumerate}}
\end{indicacion}

\item ¿Por encima de qué tiempo esperará al autobús el 10\% de las veces?
\begin{indicacion}{
\begin{enumerate}
\item Seleccionar el menú \menu{Teaching>Distribuciones>Continuas>Uniforme>Cuantiles}.
\item En el cuadro de diálogo que aparece, introducir la probabilidad 0.1 en el campo \campo{Probabilidades acumuladas},
0 en el campo \campo{Mínimo}, 15 en el campo \campo{Máximo}, seleccionar la opción \opcion{derecha} en el campo
\campo{cola de acumulación} y hacer clic en el botón \boton{Enviar}.
\end{enumerate}}
\end{indicacion}
\end{enumerate}

\item La variable aleatoria normal de media 0 y desviación típica 1, $Z\sim N(0,1)$, se conoce como normal estándar y es
la variable normal más importante. 
Se pide:
\begin{enumerate}
\item Dibujar la gráfica de la función de densidad. 
\begin{indicacion}{
\begin{enumerate}
\item Seleccionar el menú \menu{Teaching>Distribuciones>Continuas>Normal\flecha
Gráfico de probabilidad}.
\item En el cuadro de diálogo que aparece, introducir el valor 0 el campo \campo{Media}, 1 en el campo \campo{Desviación
típica} y hacer clic en el botón \boton{Enviar}.
\end{enumerate}}
\end{indicacion}

\item ¿Cómo afectan los dos parámetros de la normal, su media y su desviación típica, a la forma de la campana de Gauss?
\begin{indicacion}{
\begin{enumerate}
\item Seleccionar el menú \menu{Teaching>Distribuciones>Continuas>Normal\flecha
Gráfico de probabilidad}.
\item En el cuadro de diálogo que aparece, seleccionar la opción \opcion{Previsualizar}.
\item Incrementar el valor de la media y ver cómo cambia la forma de la campana.
\item Después disminuir el valor de la desviación típica y ver cómo cambia la forma de la campana.
\end{enumerate}}
\end{indicacion}

\item Dibujar la gráfica de la función de distribución. 
\begin{indicacion}{
\begin{enumerate}
\item Seleccionar el menú \menu{Teaching>Distribuciones>Continuas>Normal\flecha
Gráfico de probabilidad}.
\item En el cuadro de diálogo que aparece, introducir el valor 0 en el campo \campo{Media}, 1 en el campo
\campo{Desviación típica}, marcar la opción \opcion{Función de distribución} y hacer clic en el botón \boton{Enviar}.
\end{enumerate}}
\end{indicacion}

\item Calcular la probabilidad de que la normal estándar tome un valor menor que $-1$. 
\begin{indicacion}{
\begin{enumerate}
\item Seleccionar el menú \menu{Teaching>Distribuciones>Continuas>Normal>Probabilidades}.
\item En el cuadro de diálogo que aparece, introducir el valor $-1$ en el campo \campo{Valores de la variable}, 0 en el campo  
\campo{Media}, 1 en el campo \campo{Desviación típica}, y hacer clic en el botón \boton{Enviar}.
\end{enumerate}}
\end{indicacion}

\item Calcular la probabilidad de que la normal estándar tome un valor mayor que $1$. 
\begin{indicacion}{
\begin{enumerate}
\item Seleccionar el menú \menu{Teaching>Distribuciones>Continuas>Normal>Probabilidades}.
\item En el cuadro de diálogo que aparece, introducir el valor $1$ en el campo \campo{Valores de la variable}, 0 en el campo  
\campo{Media}, 1 en el campo \campo{Desviación típica}, seleccionar la opción \opcion{derecha} en el campo
\campo{cola de acumulación} y hacer clic en el botón \boton{Enviar}.
\end{enumerate}}
\end{indicacion}

\item Calcular la probabilidad de que la normal estándar tome un valor entre $-1$ (la media menos la desviación típica)
y $1$ (la media más la desviación típica).
\begin{indicacion}{
\begin{enumerate}
\item Seleccionar el menú \menu{Teaching>Distribuciones>Continuas>Normal>Probabilidades}.
\item En el cuadro de diálogo que aparece, introducir los valores $1$, $-1$ en el campo \campo{Valores de la variable},
0 en el campo \campo{Media}, 1 en el campo \campo{Desviación típica} y hacer clic en el botón \boton{Enviar}.
\end{enumerate}
La probabilidad del intervalo $P(-1\leq Z\leq 1)$ es la resta de las probabilidades obtenidas $P(Z\leq 1)-P(Z\leq -1)$.
}
\end{indicacion}

\item Calcular la probabilidad de que la normal estándar tome un valor entre $-2$ (la media menos dos veces la desviación típica) y $2$ (la
media más dos veces la desviación típica). 
\begin{indicacion}{
\begin{enumerate}
\item Seleccionar el menú \menu{Teaching>Distribuciones>Continuas>Normal>Probabilidades}.
\item En el cuadro de diálogo que aparece, introducir los valores $2$, $-2$ en el campo \campo{Valores de la variable},
0 en el campo \campo{Media}, 1 en el campo \campo{Desviación típica} y hacer clic en el botón \boton{Enviar}.
\end{enumerate}
La probabilidad del intervalo $P(-2\leq Z\leq 2)$ es la resta de las probabilidades obtenidas $P(Z\leq 2)-P(Z\leq -2)$.
}
\end{indicacion}

\item Calcular la probabilidad de que la normal estándar tome un valor entre $-3$ (la media menos tres veces la desviación típica) y $3$ (la
media más tres veces la desviación típica). 
\begin{indicacion}{
\begin{enumerate}
\item Seleccionar el menú \menu{Teaching>Distribuciones>Continuas>Normal>Probabilidades}.
\item En el cuadro de diálogo que aparece, introducir los valores $3$, $-3$ en el campo \campo{Valores de la variable},
0 en el campo \campo{Media}, 1 en el campo \campo{Desviación típica} y hacer clic en el botón \boton{Enviar}.
\end{enumerate}
La probabilidad del intervalo $P(-3\leq Z\leq 3)$ es la resta de las probabilidades obtenidas $P(Z\leq 3)-P(Z\leq -3)$.
}
\end{indicacion}

\item Calcular los cuartiles.
\begin{indicacion}{
\begin{enumerate}
\item Seleccionar el menú \menu{Teaching>Distribuciones>Continuas>Normal>Cuantiles}.
\item En el cuadro de diálogo que aparece, introducir las probabilidades 0.25, 0.5, 0.75 en el campo \campo{Probabilidades acumuladas}, 0 en el
campo \campo{Media}, 1 en el campo \campo{Desviación típica} y hacer clic en el botón \boton{Enviar}.
\end{enumerate}}
\end{indicacion}

\item Calcular el valor que deja acumulada por debajo una probabilidad $0.95$.
\begin{indicacion}{
\begin{enumerate}
\item Seleccionar el menú \menu{Teaching>Distribuciones>Continuas>Normal>Cuantiles}.
\item En el cuadro de diálogo que aparece, introducir la probabilidad 0.95 en el campo \campo{Probabilidades acumuladas}, 0 en el
campo \campo{Media}, 1 en el campo \campo{Desviación típica} y hacer clic en el botón \boton{Enviar}.
\end{enumerate}}
\end{indicacion}

\item Calcular el valor que deja acumulada por encima una probabilidad $0.025$.
\begin{indicacion}{
\begin{enumerate}
\item Seleccionar el menú \menu{Teaching>Distribuciones>Continuas>Normal>Cuantiles}.
\item En el cuadro de diálogo que aparece, introducir la probabilidad 0.025 en el campo \campo{Probabilidades acumuladas}, 0
en el campo \campo{Media}, 1 en el campo \campo{Desviación típica}, seleccionar la opción \opcion{derecha} en el campo
\campo{cola de acumulación} y hacer clic en el botón \boton{Enviar}.
\end{enumerate}}
\end{indicacion}
\end{enumerate}


\item El teorema central del límite establece que la variable resultante de sumar 30 o más variables independientes
sigue una distribución normal de media la suma de las medias de cada una de las variables y de varianza la suma de sus
varianzas.
Esta es la explicación de que una gran parte de las variables continuas que aparecen en la naturaleza sean variables
normales.
Para observar de manera experimental el teorema central del límite se realiza un experimento que consiste en lanzar
varios dados muchas veces y sumar los valores obtenidos. 
Se pide:
\begin{enumerate}
\item Simular el lanzamiento de un dado 100000 veces y dibujar el diagrama de barras asociado. 
¿Tiene forma de campana de Gauss?
\begin{indicacion}{Para generar los lanzamientos del dado: 
\begin{enumerate}
\item Seleccionar el menú \menu{Teaching >Simulaciones>Lanzamiento de dados}.
\item En el cuadro de diálogo que aparece, introducir 1 en el campo \campo{Número de dados}, introducir 100000 en el
campo \campo{Número de lanzamientos}, seleccionar la opción \opcion{Incluir suma}, introducir un nombre para el conjunto
de datos y hacer clic en el botón \boton{Enviar}.
\end{enumerate}
Para dibujar el diagrama de barras:
\begin{enumerate}
\item Seleccionar el menú \menu{Teaching>Gráficos>Diagrama de barras}.
\item En el cuadro de diálogo que aparece seleccionar la variable \variable{sum}.
\item En la solapa \menu{Opciones de las barras}, seleccionar la opción \opcion{Frecuencias relativas} y hacer clic en
el botón \boton{Enviar}.
\end{enumerate}}
\end{indicacion}

\item Repetir el apartado anterior con 2 y 30 dados. 
¿Se cumple el teorema central del límite?
\end{enumerate}


\item La suma de $n$ variables normales estándar independientes elevadas al cuadrado es una variable con distribución Chi-cuadrado con $n$
grados de libertad $\chi^2(n)$. 
Sea $X$ una variable Chi-cuadrado con 6 grados de libertad $\chi^2(6)$. 
Se pide:
\begin{enumerate}
\item Dibujar la gráfica de la función de densidad. 
\begin{indicacion}{
\begin{enumerate}
\item Seleccionar el menú \menu{Teaching>Distribuciones>Continuas>Chi-cuadrado\flecha
Gráfico de probabilidad}.
\item En el cuadro de diálogo que aparece, introducir el valor 6 en el \campo{Grados de libertad} y hacer clic en el
botón \boton{Enviar}.
\end{enumerate}}
\end{indicacion}

\item Calcular la probabilidad de que la variable tome un valor menor que 6.
\begin{indicacion}{
\begin{enumerate}
\item Seleccionar el menú \menu{Teaching>Distribuciones>Continuas>Chi-cuadrado>Probabilidades}.
\item En el cuadro de diálogo que aparece, introducir el valor $6$ en el campo \campo{Valores de la variable}, 6 en
el campo \campo{Grados de libertad} y hacer clic en el botón \boton{Enviar}.
\end{enumerate}}
\end{indicacion}

\item Calcular el valor que deja acumulada por debajo una probabilidad $0.05$.
\begin{indicacion}{
\begin{enumerate}
\item Seleccionar el menú \menu{Teaching>Distribuciones>Continuas>Chi-cuadrado>Cuantiles}.
\item En el cuadro de diálogo que aparece, introducir la probabilidad 0.05 en el campo \campo{Probabilidades acumuladas}, 6 en
el campo \campo{Grados de libertad} y hacer clic en el botón \boton{Enviar}.
\end{enumerate}}
\end{indicacion}

\item Calcular el valor que deja acumulada por arriba una probabilidad $0.1$.
\begin{indicacion}{
\begin{enumerate}
\item Seleccionar el menú \menu{Teaching>Distribuciones>Continuas>Chi-cuadrado>Cuantiles}.
\item En el cuadro de diálogo que aparece, introducir la probabilidad 0.1 en el campo \campo{Probabilidades}, 6 en el
campo \campo{Grados de libertad}, seleccionar la opción \opcion{derecha} en el campo \campo{cola de acumulación} y hacer
clic en el botón \boton{Enviar}.
\end{enumerate}}
\end{indicacion}
\end{enumerate}


\item La variable que se obtiene al dividir una normal estándar entre la raíz de una variable Chi-cuadrado de $n$ grados
de libertad dividida por $n$, sigue una distribución $t$ de student de $n$ grados de libertad $T(n)$. 
Sea $X$ una variable $t$ de student de 8 grados de libertad $T(8)$. Se pide:
\begin{enumerate}
\item Dibujar la gráfica de la función de probabilidad y compararla con la de la normal estándar.
\begin{indicacion}{
\begin{enumerate}
\item Seleccionar el menú \menu{Teaching>Distribuciones>Continuas>T de student\flecha
Gráfico de probabilidad}.
\item En el cuadro de diálogo que aparece, introducir el valor 8 en el campo \campo{Grados de libertad} y hacer clic en
el botón \boton{Enviar}.
\end{enumerate}}
\end{indicacion}

\item Calcular el percentil octavo. 
\begin{indicacion}{
\begin{enumerate}
\item Seleccionar el menú \menu{Teaching>Distribuciones>Continuas>T de student>Cuantiles}.
\item En el cuadro de diálogo que aparece, introducir la probabilidad 0.08 en el campo \campo{Probabilidades
acumuladas}, 8 en el campo \campo{Grados de libertad} y hacer clic en el botón \boton{Enviar}.
\end{enumerate}}
\end{indicacion}

\item Calcular el valor por encima del cual está el 5\% de la población. 
\begin{indicacion}{
\begin{enumerate}
\item Seleccionar el menú \menu{Teaching>Distribuciones>Continuas>T de student>Cuantiles}.
\item En el cuadro de diálogo que aparece, introducir la probabilidad 0.05 en el campo \campo{Probabilidades}, 8 en el
campo \campo{Grados de libertad}, seleccionar la opción \opcion{derecha} en el campo \campo{cola de acumulación} y hacer
clic en el botón \boton{Enviar}.
\end{enumerate}}
\end{indicacion}
\end{enumerate}

\item La variable resultante de dividir una variable Chi-cuadrado de $n$ grados de libertad dividida por $n$, entre una
variable Chi-cuadrado de $m$ grados de libertad dividida por $m$, sigue un modelo de distribución $F$ de Fisher de $n$ y
$m$ grados de libertad $F(n,m)$. 
Sea $X$ una variable $F$ de Fisher de $10$ y $20$ grados de libertad $F(10,20)$. 
Se pide:
\begin{enumerate}
\item Dibujar la gráfica de la función de densidad
\begin{indicacion}{
\begin{enumerate}
\item Seleccionar el menú \menu{Teaching>Distribuciones>Continuas>F de Fisher\flecha
Gráfico de probabilidad}.
\item En el cuadro de diálogo que aparece, introducir 10 el campo \campo{Grados de libertad del
numerador}, introducir 20 en el campo \campo{Grados de libertad del denominador} y hacer clic en el botón \boton{Enviar}.
\end{enumerate}}
\end{indicacion}

\item Calcular la probabilidad acumulada por encima de 1. 
\begin{indicacion}{
\begin{enumerate}
\item Seleccionar el menú \menu{Teaching>Distribuciones>Continuas>F de Fisher>Probabilidades}.
\item En el cuadro de diálogo que aparece, introducir el valor $1$ en el campo \campo{Valores de la variable}, 10 en el
campo \campo{Grados de libertad del numerador}, 20 en el campo \campo{Grados de libertad del denominador}, seleccionar
la opción \opcion{derecha} en el campo \campo{cola de acumulación} y hacer clic en el botón \boton{Enviar}.
\end{enumerate}}
\end{indicacion}

\item Calcular el rango intercuartílico.
\begin{indicacion}{
\begin{enumerate}
\item Seleccionar el menú \menu{Teaching>Distribuciones>Continuas>F de Fisher>Cuantiles}.
\item En el cuadro de diálogo que aparece, introducir las probabilidades 0.75, 0.25 en el campo \campo{Probabilidades},
10 en el campo \campo{Grados de libertad del numerador}, 20  en el campo \campo{Grados de libertad del denominador} y
hacer clic en el botón \boton{Enviar}.
\end{enumerate}
El rango intercuartílico es la resta de los valores obtenidos correspondientes al tercer y primer cuartiles.
}
\end{indicacion}
\end{enumerate}

\end{enumerate}


\section{Ejercicios propuestos}
\begin{enumerate}[leftmargin=*]
\item Entre los diabéticos, el nivel de glucosa en la sangre en ayunas $X$, puede suponerse de distribución
aproximadamente normal, con media 106mg/100ml y desviación típica 8mg/100ml.
\begin{enumerate}
\item Hallar $P(X\leq120\textrm{mg}/100\textrm{ml})$
\item ¿Qué porcentaje de diabéticos tendrá niveles entre 90 y 120mg/100ml?
\item Encontrar un valor que tenga la propiedad de que el 25\% de los diabéticos tenga un nivel de glucosa por debajo de
dicho valor.
\end{enumerate}

\item Se sabe que el nivel de colesterol en varones de más de 30 años de una determinada población sigue una
distribución normal, de media 220mg/dl y desviación típica 30mg/dl.
Si la población tiene 20000 varones mayores de 30 años,
\begin{enumerate}
\item ¿Cuántos se espera que tengan su nivel de colesterol entre 210mg/dl y 240mg/dl?
\item ¿Cuántos se espera que tengan su nivel de colesterol por encima de 250mg/dl?
\item ¿Cuál será el nivel de colesterol por encima del cual se espera que esté el 20\% de la población?
\end{enumerate}

\item Calcular la probabilidad de obtener entre 40 y 60 caras, inclusive, al lanzar 100 veces una moneda. 
Utilizar la aproximación de la distribución binomial mediante una normal.
\end{enumerate}
