\chapter{Estadísticos Muestrales}

\section{Fundamentos teóricos}
Hemos visto cómo podemos presentar la información que obtenemos de la muestra, a través de tablas o bien a través de
gráficas.
La tabla de frecuencias contiene toda la información de la muestra pero resulta difícil sacar conclusiones sobre
determinados aspectos de la distribución con sólo mirarla.
Ahora veremos cómo a partir de esos mismos valores observados de la variable estadística, se calculan ciertos números
que resumen la información muestral.
Estos números, llamados \emph{Estadísticos}, se utilizan para poner de manifiesto ciertos aspectos de la distribución,
tales como la dispersión o concentración de los datos, la forma de su distribución, etc.
Según sea la característica que pretenden reflejar se pueden clasificar en medidas de posición, medidas de dispersión y medidas de forma.

\subsection{Medidas de posición}
Son valores que indican cómo se sitúan los datos. Los más importantes son la Media aritmética, la Mediana y la Moda.

\subsubsection{Media aritmética $ \overline{\mbox{\textit{x}}}$}
Se llama \emph{media aritmética} de una variable estadística $X$, y se representa por $\overline{x}$ , a la suma de
todos los resultados observados, dividida por el tamaño muestral. 
Es decir, la media de la variable estadística $X$, cuya distribución de frecuencias es $(x_i,n_i)$, viene dada por
\[
\overline{x}=\frac{x_1+\ldots+x_1+\ldots+x_k+\ldots+x_k}{n_1+\ldots+n_k}=\frac{x_1n_1+\ldots+x_kn_k}{n}=\frac{1}{n}\sum_{i=1}^{k}x_in_i
\]

La media aritmética sólo tiene sentido en variables cuantitativas.

\subsubsection{Mediana \textit{Me}}
Se llama \emph{mediana} y lo denotamos por $Me$, a aquel valor de la muestra que, una vez ordenados todos los valores de
la misma en orden creciente, tiene tantos términos inferiores a él como superiores.
En consecuencia, divide la distribución en dos partes iguales.

La mediana sólo tiene sentido en atributos ordinales y en variables cuantitativas.

\subsubsection{Moda \textit{Mo}}
La \emph{moda} es el valor de la variable que presenta una mayor frecuencia en la muestra.
Cuando haya más de un valor con frecuencia máxima diremos que hay más de una moda. En variables continuas o discretas
agrupadas llamaremos clase modal a la que tenga la máxima frecuencia.
Se puede calcular la moda tanto en variables cuantitativas como cualitativas.

\subsubsection{Cuantiles}
Si el conjunto total de valores observados se divide en $r$ partes que contengan cada una $\frac{n}{r}$ observaciones,
los puntos de separación de las mismas reciben el nombre genérico de \emph{cuantiles}.

Según esto la mediana también es un cuantil con $r=2$.
Algunos cuantiles reciben determinados nombres como:
\begin{description}
\item [Cuartiles.] Son los puntos que dividen la distribución en 4 partes iguales y se designan por $C_1,C_2,C_3$. Es
claro que $C_2=Me$.
\item[Deciles.] Son los puntos que dividen la distribución en 10 partes iguales y se designan por $D_1,D_2,\ldots,D_9$.
\item [Percentiles.] Son los puntos que dividen la distribución en 100 partes iguales y se designan por
$P_1,P_2,\ldots,P_{99}$.
\end{description}

\subsection{Medidas de dispersión}
Miden la separación existente entre los valores de la muestra.
Las más importantes son el Rango o Recorrido, el Rango Intercuartílico, la Varianza, la Desviación Típica y el
Coeficiente de Variación.

\subsubsection{Rango o Recorrido \textit{Re}}
La medida de dispersión más inmediata es el rango. Llamamos \emph{recorrido} o \emph{rango} y lo designaremos por
\textit{Re} a la diferencia entre los valores máximo y mínimo que toma la variable en la muestra, es decir
\[
Re = max\{x_i, i=1,2,\ldots,n\} - min\{x_i, i=1,2,\ldots,n\}.
\]

Este estadístico sirve para medir el campo de variación de la variable, aunque es la medida de dispersión que menos
información proporciona sobre la mayor o menor agrupación de los valores de la variable alrededor de las medidas de
tendencia central. 
Además tiene el inconveniente de que se ve muy afectado por los datos atípicos.

\subsubsection{Rango Intercuartílico \textit{RI}}
El \emph{rango intercuartílico} \textit{RI} es la diferencia entre el tercer y el primer cuartil, y mide, por tanto, el
campo de variación del 50\% de los datos centrales de la distribución. 
Por consiguiente 
\[
RI=C_3-C_1.
\]

La ventaja del rango intercuartílico frente al recorrido es que no se ve tan afectado por los datos atípicos.

\subsubsection{Varianza $\textit{s}_\textit{x}^\textrm{2}$}
Llamamos \emph{varianza} de una variable estadística $X$, y la designaremos por $\textit{s}_\textit{x}^\textrm{2}$, a la
media de los cuadrados de las desviaciones de los valores observados respecto de la media de la muestra, es decir,
\[
s_x^{2}=\frac{1}{n}\sum_{i=1}^{k}(x_i-\overline{x})^{2}n_i.
\]

\subsubsection{Desviación Típica $\textit{s}_\textit{x}$}
La raíz cuadrada positiva de la varianza se conoce como \emph{desviación típica} de la variable $X$, y se representa por
$s$, 
\[
s=+\sqrt{s_{x}^{2}}.
\]

\subsubsection{Coeficiente de Variación de Pearson $\textit{Cv}_\textit{x}$}
Al cociente entre la desviación típica y el valor absoluto de la media se le conoce como \emph{coeficiente de variación
de Pearson} o simplemente \emph{coeficiente de variación}:
\[
Cv_x=\frac{s_x}{|\overline{x}|}.
\]

El coeficiente de variación es adimensional, y por tanto permite hacer comparaciones entre variables expresadas en
distintas unidades.
Cuanto más próximo esté a 0, menor será la dispersión de la muestra en relación con la media, y más representativa será
ésta última del conjunto de observaciones.

\subsection{Medidas de forma}
Indican la forma que tiene la distribución de valores en la muestra.
Se pueden clasificar en dos grupos: Medidas de \emph{asimetría} y medidas de \emph{apuntamiento o curtosis}.

\subsubsection{Coeficiente de asimetría de Fisher $\textit{g}_\textit{1}$}
El \emph{coeficiente de asimetría de Fisher}, que se representa por $g_1$, se define
\[
g_1=\frac{\sum_{i=1}^{k}(x_i-\overline{x})^{3}f_i}{s_x^{3}}.
\]

Dependiendo del valor que tome tendremos:
\begin{itemize}
\item $g_1=0$. Distribución simétrica.
\item $g_1<0$. Distribución asimétrica hacia la izquierda.
\item $g_1>0$. Distribución asimétrica hacia la derecha.
\end{itemize}

\subsubsection{Coeficiente de apuntamiento o curtosis $\textit{g}_\textit{2}$}
El grado de apuntamiento de las observaciones de la muestra, se caracteriza por el \emph{coeficiente de apuntamiento o
curtosis}, que se representa por $g_2$, y se define
\[
g_2=\frac{\sum_{i=1}^{k}(x_i-\overline{x})^{4}f_i}{s_x^{4}}-3.
\]

Dependiendo del valor que tome tendremos:
\begin{itemize}
\item $g_2=0$. La distribución tiene un apuntamiento igual que el de la distribución normal de la misma media y
desviación típica. Se dice que es una distribución \emph{mesocúrtica}.
\item $g_2<0$. La distribución es menos apuntada que la distribución normal de la misma media y desviación típica. Se
dice que es una distribución \emph{platicúrtica}.
\item $g_2>0$. La distribución es más apuntada que la distribución normal de la misma media y desviación típica. Se dice
que es una distribución \emph{leptocúrtica}.
\end{itemize}

Tanto $g_1$ como $g_2$ suelen utilizarse para comprobar si los datos muestrales provienen de una población no normal.
Cuando $g_1$  está fuera del intervalo [-2,2] se dice que la distribución es demasiado asimétrica como para que los
datos provengan de una población normal.
Del mismo modo, cuando $g_2$ está fuera del intervalo [-2,2] se dice que la distribución es, o demasiado apuntada, o
demasiado plana, como para que los datos provengan de una población normal.

\subsection{Estadísticos de variables en las que se definen grupos}
Ya sabemos cómo resumir la información contenida en una muestra utilizando una serie de estadísticos.
Pero hasta ahora sólo hemos estudiado ejemplos con un único carácter objeto de estudio.

En la mayoría de las investigaciones no estudiaremos un único carácter, sino un conjunto de caracteres, y muchas veces
será conveniente obtener información de un determinado carácter, en función de los grupos creados por otro de los
caracteres estudiados en la investigación.
A estas variables que se utilizan para formar grupos se les conoce como \emph{variables clasificadoras} o
\emph{factores}.

Por ejemplo, si se realiza un estudio sobre un conjunto de niños recién nacidos, podemos estudiar su peso.
Pero si además sabemos si la madre de cada niño es fumadora o no, podremos hacer un estudio del peso de los niños de las
madres fumadoras por un lado y los de las no fumadoras por otro, para ver si existen diferencias entre ambos grupos.

\clearpage
\newpage

\section{Ejercicios resueltos}
\begin{enumerate}[leftmargin=*]
\item En una encuesta a 25 matrimonios sobre el número de hijos que tenían se obtuvieron los siguientes datos:
\begin{center}
1, 2, 4, 2, 2, 2, 3, 2, 1, 1, 0, 2, 2, 0, 2, 2, 1, 2, 2, 3, 1, 2, 2, 1, 2
\end{center}
Se pide:
\begin{enumerate}
\item Crear un conjunto de datos con la variable \variable{hijos} e introducir los datos. 
Si ya se tienen los datos, simplemente recuperarlos.

\item Calcular la media aritmética, varianza y desviación típica de dicha variable.
Interpretar los estadísticos. 
\begin{indicacion}
\begin{enumerate}
\item Seleccionar el menú \menu{Teaching > Estadística descriptiva > Estadísticos}.
\item En el cuadro de diálogo que aparece seleccionar la variable \variable{hijos} en el campo \campo{Variables}.
\item En la solapa \menu{Estadísticos básicos} seleccionar \opcion{Media} y \opcion{Desviación típica}, y hacer
click sobre el botón \boton{Enviar}.
\end{enumerate}
\end{indicacion}

\item Calcular los cuartiles, el recorrido, el rango intercuartílico, el tercer decil y el percentil 68. 
\begin{indicacion}
\begin{enumerate}
\item Seleccionar el menú \menu{Teaching > Estadística descriptiva > Estadísticos}.
\item En el cuadro de diálogo que aparece seleccionar la variable \variable{hijos} en el campo \campo{Variables}.
\item En la solapa \menu{Estadísticos básicos} seleccionar \opcion{Cuartiles}, \opcion{Rango}, \opcion{Rango
intercuartílico}, introducir los valores $0.3$ y $0.68$ en el campo $\campo{Percentiles}$, y hacer click sobre el botón
\boton{Enviar}.
\end{enumerate}
\end{indicacion}
\end{enumerate}

\item En un hospital se realizó un estudio sobre el número de personas que ingresaron en urgencias cada día del mes de
noviembre. 
Los datos observados fueron:
\begin{center}
15, 23, 12, 10, 28, 50, 12, 17, 20, 21, 18, 13, 11, 12, 26 \\
30, 6, 16, 19, 22, 14, 17, 21, 28, 9, 16, 13, 11, 16, 20
\end{center}
Se pide:

\begin{enumerate}
\item Crear un conjunto de datos con la variable \variable{urgencias} e introducir los datos.

\item Calcular la media aritmética, varianza, desviación típica y coeficiente de variación de dicha variable.
Interpretar los estadísticos. 
\begin{indicacion}
\begin{enumerate}
\item Seleccionar el menú \menu{Teaching > Estadística descriptiva > Estadísticos}.
\item En el cuadro de diálogo que aparece seleccionar la variable \variable{urgencias} en el campo \campo{Variables}.
\item En la solapa \menu{Estadísticos básicos} seleccionar \opcion{Media}, \opcion{Varianza}, 
\opcion{Desviación típica} y \opcion{Coeficiente de variación}, y hacer click sobre el botón \boton{Enviar}.
\end{enumerate}
\end{indicacion}

\item Calcular el coeficiente de asimetría y el de curtosis e interpretar los resultados
\begin{indicacion}
Seguir los mismos pasos del apartado anterior, seleccionando \opcion{Cofeficiente de asimetría} y \opcion{Coeficiente de
Curtosis} en la solapa \menu{Estadísticos básicos}.
\end{indicacion}
\end{enumerate}


\item En un grupo de 20 alumnos, las calificaciones obtenidas en Matemáticas fueron:
\begin{center}
SS, AP, SS, AP, AP, NT, NT, AP, SB, SS \\
SB, SS, AP, AP, NT, AP, SS, NT, SS, NT
\end{center}
Se pide:

\begin{enumerate}
\item  Crear un conjunto de datos \variable{curso} con la variable \variable{calificaciones} e introducir los datos.

\item  Recodificar esta variable, asignando $2.5$ al SS, $6$ al AP, $8$ al NT y $9.5$ al SB.
\begin{indicacion}
\begin{enumerate}
\item Selecionar el menú \menu{Teaching > Datos > Recodificar variable}.
\item En el cuadro de diálogo que aparece seleccionar como variable a recodificar la variable \variable{calificaciones}.
\item Introducir las reglas de recodificación en el campo \campo{Reglas de recodificación}:
\begin{quote}
\lstinline{"SS" = 2.5}\\
\lstinline{"AP" = 6}\\
\lstinline{"NT" = 8}\\
\lstinline{"SB" = 9.5}
\end{quote}
\item En el cuadro \campo{Guardar nueva variable} hacer click sobre el botón \boton{Cambiar}.
\item En el cuadro de diálogo que aparece seleccionar como objeto padre la el conjunto de datos \variable{curso} y hacer click sobre el botón \boton{Enviar}.
\item Introducir el nombre de la nueva variable \variable{nota}, desmarcar la casilla \opcion{Convertir en factor} y hacer click sobre el botón \boton{Enviar}.
\end{enumerate}
\end{indicacion}

\item La mediana y el rango intercuartílico.
\begin{indicacion}
\begin{enumerate}
\item Seleccionar el menú \menu{Teaching > Estadística descriptiva > Estadísticos}.
\item En el cuadro de diálogo que aparece seleccionar la variable \variable{nota} en el campo \campo{Variables}.
\item En la solapa \menu{Estadísticos básicos} seleccionar \opcion{Mediana} y \opcion{Rango intercuartílico}, 
y hacer click sobre el botón \boton{Enviar}.
\end{enumerate}
\end{indicacion}
\end{enumerate}

\item Para realizar un estudio sobre la estatura de los estudiantes universitarios se ha seleccionado mediante un
proceso de muestreo aleatorio, una muestra de 30 estudiantes, obteniendo los siguientes resultados (medidos en centímetros):
\begin{center}
\begin{tabular}{ll}
Mujeres: & 173, 158, 174, 166, 162, 177, 165, 154, 166, 182, 169, 172, 170, 168. \\
Hombres: & 179, 181, 172, 194, 185, 187, 198, 178, 188, 171, 175, 167, 186, 172, 176, 187. 
\end{tabular}
\end{center}
Se pide:

\begin{enumerate}
\item Crear un conjunto de datos con las variables \variable{estatura} y \variable{sexo} e introducir los datos.

\item Obtener un resumen de estadísticos en el que se muestren la media aritmética, mediana, varianza,
desviación típica y cuartiles según el sexo. Interpretar los estadísticos.
\begin{indicacion}
\begin{enumerate}
\item Seleccionar el menú \menu{Teaching > Estadística descriptiva > Estadísticos}.
\item En el cuadro de diálogo que aparece seleccionar la variable \variable{estatura} en el campo \campo{Variables},
marcar la casilla \opcion{Estadística por grupos} y seleccionar la variable \variable{sexo} en el campo \campo{Variables de agrupación}.
\item En la solapa \menu{Estadísticos básicos} seleccionar \opcion{Media}, \opcion{Mediana}, \opcion{Varianza},
\opcion{Desviación típica} y \opcion{Cuartiles}, y hacer click sobre el botón \boton{Enviar}.
\end{enumerate}
\end{indicacion}
\end{enumerate}

\end{enumerate}


\section{Ejercicios propuestos}
\begin{enumerate}[leftmargin=*]
\item  El número de lesiones padecidas durante una temporada por cada jugador de un equipo de fútbol fue el siguiente:
\begin{center}
0, 1, 2, 1, 3, 0, 1, 0, 1, 2, 0, 1, 1, 1, 2, 0, 1, 3, 2, 1, 2, 1, 0, 1
\end{center}

Se pide:
\begin{enumerate}
\item Calcular la media aritmética, mediana, varianza y desviación típica de las lesiones e interpretarlas.
\item Calcular los coeficientes de asimetría y curtosis e interpretarlos.
\item Calcular el cuarto y el octavo decil e interpretarlos.
\end{enumerate}

\item  En un estudio de población se tomó una muestra de 27 personas, y se les preguntó por su edad y estado civil,
obteniendo los siguientes resultados:
\begin{center}
\begin{tabular}{|l|rrrrrrrrr|}
\hline
Estado civil & \multicolumn{9}{c|}{Edad}\\
\hline
Soltero    & 31 & 45 & 35 & 65 & 21 & 38 & 62 & 22 & 31 \\
Casado     & 62 & 39 & 62 & 59 & 21 & 62 &    &    &    \\
Viudo      & 80 & 68 & 65 & 40 & 78 & 69 & 75 &    &    \\
Divorciado & 31 & 65 & 59 & 49 & 65 &    &    &    &    \\
\hline
\end{tabular}
\end{center}
Se pide:
\begin{enumerate}
\item Calcular la media y la desviación típica de la edad según el estado civil e interpretarlas.
\item ¿En qué grupo es más representativa la media?
\end{enumerate}

\item En un estudio se ha medido la tensión arterial de 25 individuos. Además se les ha preguntado si fuman y beben:
\begin{center}
\begin{tabular}{lccccccccccccc}
\hline
Fumador  & si & no & si & si & si & no & no & si & no & si & no & si & no \\
Bebedor & no & no & si & si & no & no & si & si & no & si & no & si & si \\
Tensión arterial & 80 & 92 & 75 & 56 & 89 & 93 & 101 & 67 & 89 & 63 & 98 & 58 & 91 \\
\hline
\\
\hline
Fumador  & si & no & no & si & no & no & no & si & no & si & no & si \\
Bebedor & si & no & si & si & no & no & si & si & si & no & si & no \\
Tensión arterial & 71 & 52 & 98 & 104 & 57 & 89 & 70 & 93 & 69 & 82 & 70 & 49 \\
\hline
\end{tabular}
\end{center}

Calcular la media aritmética, desviación típica, coeficiente de asimetría y curtosis de la tensión arterial por
grupos dependiendo de si beben o fuman e interpretarlos.

\item El conjunto de datos \variable{neonatos} del paquete \variable{rk.Teaching}, contiene información sobre una
muestra de 320 recién nacidos en un hospital durante un año que cumplieron el tiempo normal de gestación. 
Se pide:
\begin{enumerate}
\item Calcular la media y la mediana muestral del peso de los nacidos e interpretarlos. 
\item Calcular el peso medio de los recién nacidos de la muestra según si la madre ha fumado o no durante el embarazo.
Calcular también el peso medio de los recién nacidos de madres que no han fumado durante el embarazo, según si la madre
fumaba o no antes del embarazo. ¿Qué conclusiones se pueden sacar?
\item ¿Cuál es la puntuación Apgar al minuto de nacer más frecuente?
\item Calcular la media de la diferencia entre las puntuaciones Apgar a los 5 minutos y al minuto de nacer. ¿Cómo
evolucionan los recién nacidos?
\item Calcular los cuartiles muestrales del peso de los recién nacidos e interpretarlos.
\item Comparar los cuartiles muestrales del peso de los recién nacidos según el sexo. 
\item ¿Por encima de qué peso estarán el 10\% de los niños con mayor peso?
\item Si se considera que un niño es atípico por bajo peso si se encuentra entre el 5\% de los pesos más bajos, ¿por
debajo de qué peso tiene que estar?
\item Calcular el recorrido y el rango intercuartílico muestrales del peso de los recién nacidos e interpretarlos.
\item Calcular la varianza y la desviación típica del peso de los recién nacidos e interpretarlos.
\item ¿En qué grupo hay más variabilidad del peso de los recién nacidos, en las madres fumadoras o en las madres no
fumadoras durante el embarazo? ¿En qué grupo será más representativo el peso medio?
\item ¿Qué variable presenta más variabilidad relativa, el peso de los recién nacidos o el Apgar al minuto de nacer?
\item Calcular el coeficiente de asimetría y de apuntamiento muestrales del peso de los recién nacidos e interpretarlos.
\item ¿Qué distribución es más asimétrica, la de los pesos de recién nacidos en madres mayores de 20 años o en madres
menores de 20 años?
\item ¿Qué distribución es más apuntada, la del peso de los recién nacidos en hombres o en mujeres?
\item De acuerdo a la forma de la distribución, ¿puede considerarse la puntuación Apgar al minuto de nacer como una
variable normal? ¿Y el número de cigarros fumados al día durante el embarazo?
\end{enumerate}

\item Se quiere comparar la precisión de dos tensiómetros, uno de brazo y otro de muñeca, y para ello se han realizado 8
medidas repetidas de la tensión arterial de una misma persona con cada uno de ellos, obteniendo los siguientes valores
en mmHg:
\begin{itemize}
\item \variable{tens.brazo}: 111, 109, 112, 111, 113, 113, 114, 111.
\item \variable{tens.muñeca}: 115, 113, 117, 116, 112, 112, 117, 112.
\end{itemize}
¿Qué tensiómetro es más preciso?
\end{enumerate}
