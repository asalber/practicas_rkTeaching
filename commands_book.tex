%----------------------------------------------------------------------------------------
%	VARIOUS REQUIRED PACKAGES
%----------------------------------------------------------------------------------------

\usepackage{titlesec} % Allows customization of titles
\usepackage{tikz} % Required for drawing custom shapes
\usepackage{booktabs} % Required for nicer horizontal rules in tables
\usepackage{eso-pic} % Required for specifying an image background in the title page



%----------------------------------------------------------------------------------------
%	SECTIONS AND SUBSECTIONS
%----------------------------------------------------------------------------------------
%\titleformat*{\section}{\Huge}
%\titleformat*{\subsection}{\Large}


%----------------------------------------------------------------------------------------
%	MAIN TABLE OF CONTENTS
%----------------------------------------------------------------------------------------

\usepackage{titletoc} % Required for manipulating the table of contents

\contentsmargin{0cm} % Removes the default margin
% Chapter text styling
\titlecontents{chapter}[1.25cm] % Indentation
{\addvspace{15pt}\large\bfseries} % Spacing and font options for chapters
{\color{gray}\contentslabel[\Large\thecontentslabel.]{1.25cm}} % Chapter number
{}  
{\color{gray}\bfseries\hfill\thecontentspage} % Page number
% Section text styling
\titlecontents{section}[1.25cm] % Indentation
{\addvspace{5pt}\bfseries} % Spacing and font options for sections
{\contentslabel[\thecontentslabel]{1.25cm}} % Section number
{}
{\hfill\color{black}\thecontentspage} % Page number
[]
% Subsection text styling
\titlecontents{subsection}[1.25cm] % Indentation
{\addvspace{1pt}\small} % Spacing and font options for subsections
{\contentslabel[\thecontentslabel]{1.25cm}} % Subsection number
{}
{\;\titlerule*[.5pc]{.}\;\thecontentspage} % Page number
[] 

%----------------------------------------------------------------------------------------
%	MINI TABLE OF CONTENTS IN CHAPTER HEADS
%----------------------------------------------------------------------------------------

% Section text styling
\titlecontents{lsection}[0em] % Indendating
{\footnotesize\sffamily} % Font settings
{}
{}
{}

% Subsection text styling
\titlecontents{lsubsection}[.5em] % Indentation
{\normalfont\footnotesize\sffamily} % Font settings
{}
{}
{}
 
%----------------------------------------------------------------------------------------
%	PAGE HEADERS
%----------------------------------------------------------------------------------------

\usepackage{fancyhdr} % Required for header and footer configuration

\pagestyle{fancy}
\renewcommand{\chaptermark}[1]{\markboth{\color{grayceu}\sffamily\normalsize\textit{#1}}{}} % Chapter text font
% settings
%\renewcommand{\sectionmark}[1]{\markright{\color{grayceu}\sffamily\normalsize\textit{\thesection\hspace{5pt}#1}}{}} %
% Section text font settings
\fancyhf{} \fancyfoot[LE,RO]{\normalsize\thepage} % Font setting for the page number in the footer
\fancyhead[RE]{{\color{grayceu}\sffamily\normalsize\textit{Prácticas de Bioestadística con SPSS}}} % Print the nearest
% section name on the left side of odd pages
\fancyhead[LO]{\leftmark} % Print the current chapter name on the right side of even pages
\renewcommand{\headrulewidth}{0pt} % Removes the rule in the header
\addtolength{\headheight}{2.5pt} % Increase the spacing around the header slightly
\renewcommand{\footrulewidth}{0pt} % Removes the rule in the footer
\fancypagestyle{plain}{\fancyhead{}\renewcommand{\headrulewidth}{0pt}} % Style for when a plain pagestyle is specified

% Removes the header from odd empty pages at the end of chapters
\makeatletter
\renewcommand{\cleardoublepage}{
\clearpage\ifodd\c@page\else
\hbox{}
\vspace*{\fill}
\thispagestyle{empty}
\newpage
\fi}


%----------------------------------------------------------------------------------------
%	THEOREM STYLES
%----------------------------------------------------------------------------------------

\usepackage{amsthm} % For including math equations, theorems, symbols, etc

\newcommand{\intoo}[2]{\mathopen{]}#1\,;#2\mathclose{[}}
\newcommand{\ud}{\mathop{\mathrm{{}d}}\mathopen{}}
\newcommand{\intff}[2]{\mathopen{[}#1\,;#2\mathclose{]}}
\newtheorem{notation}{Notation}[chapter]

\newtheoremstyle{ocrenum} % Theorem style name
{7pt} % Space above
{7pt} % Space below
{\normalfont} % Body font
{} % Indent amount
{\small\bf\sffamily\color{ocre}} % Theorem head font
{\;\;} % Punctuation after theorem head
{0.25em} % Space after theorem head
{\small\sffamily\color{ocre}\thmname{#1}\thmnumber{\@ifnotempty{#1}{ }\@upn{#2}} % Theorem text (e.g. Theorem 2.1)
\thmnote{\ {\the\thm@notefont\sffamily\bfseries\color{black}--- #3.}}} % Optional theorem note
\renewcommand{\qedsymbol}{$\blacksquare$} % Optional qed square

\newtheoremstyle{blacknumex} % Theorem style name
{7pt} % Space above
{7pt} % Space below
{\normalfont} % Body font
{} % Indent amount
{\small\bf\sffamily} % Theorem head font
{\;\;} % Punctuation after theorem head
{0.25em} % Space after theorem head
{\small\sffamily{\tiny\ensuremath{\blacksquare}}\ \thmname{#1}\thmnumber{\@ifnotempty{#1}{ }\@upn{#2}} % Theorem text (e.g. Theorem 2.1)
\thmnote{\ {\the\thm@notefont\sffamily\bfseries--- #3.}}} % Optional theorem note

\newtheoremstyle{blacknum} % Theorem style name
{7pt} % Space above
{7pt} % Space below
{\normalfont} % Body font
{} % Indent amount
{\small\bf\sffamily} % Theorem head font
{\;\;} % Punctuation after theorem head
{0.25em} % Space after theorem head
{\small\sffamily\thmname{#1}\thmnumber{\@ifnotempty{#1}{ }\@upn{#2}} % Theorem text (e.g. Theorem 2.1)
\thmnote{\ {\the\thm@notefont\sffamily\bfseries--- #3.}}} % Optional theorem note


\newtheoremstyle{indication} % Theorem style name
{7pt} % Space above
{7pt} % Space below
{\normalfont} % Body font
{} % Indent amount
{\bf} % Theorem head font
{} % Punctuation after theorem head
{0pt} % Space after theorem head
{\begin{tikzpicture}[overlay]
\draw[anchor=west] (-26pt,10pt) node [fill=gray,opacity=1,inner
sep=1pt]{\Large\bfseries\textcolor{white}{\vphantom{plPQq}\makebox[17pt][c]{\emph{i}}}}; \end{tikzpicture}} % Theorem text (e.g. Theorem 2.1)

\makeatother

% Defines the theorem text style for each type of theorem to one of the three styles above
\theoremstyle{ocrenum}
\newtheorem{theoremeT}{Theorem}[chapter]
\newtheorem{proposition}{Proposition}[chapter]
\newtheorem{problem}{Problem}[chapter]
\newtheorem{exerciseT}{Exercise}[chapter]
\theoremstyle{blacknumex}
\newtheorem{exampleT}{Example}[chapter]
\theoremstyle{blacknum}
\newtheorem{vocabulary}{Vocabulary}[chapter]
\newtheorem{definitionT}{Definition}[chapter]
\newtheorem{corollaryT}{Corollary}[chapter]
\theoremstyle{indication}
\newtheorem{indicacionT}{Indicación}
\theoremstyle{nonumberplain}
\newtheorem{ejemplo}{Ejemplo}

%----------------------------------------------------------------------------------------
%	DEFINITION OF COLORED BOXES
%----------------------------------------------------------------------------------------

\RequirePackage[framemethod=default]{mdframed} % Required for creating the theorem, definition, exercise and corollary boxes

% Theorem box
\newmdenv[skipabove=7pt,
skipbelow=7pt,
backgroundcolor=black!5,
linecolor=ocre,
innerleftmargin=5pt,
innerrightmargin=5pt,
innertopmargin=5pt,
leftmargin=0cm,
rightmargin=0cm,
innerbottommargin=5pt]{tBox}

% Exercise box	  
\newmdenv[skipabove=7pt,
skipbelow=7pt,
rightline=false,
leftline=true,
topline=false,
bottomline=false,
backgroundcolor=ocre!10,
linecolor=ocre,
innerleftmargin=5pt,
innerrightmargin=5pt,
innertopmargin=5pt,
innerbottommargin=5pt,
leftmargin=0cm,
rightmargin=0cm,
linewidth=4pt]{eBox}	

% Definition box
\newmdenv[skipabove=10pt,
skipbelow=10pt,
rightline=false,
leftline=true,
topline=false,
bottomline=false,
linecolor=ocre,
innerleftmargin=5pt,
innerrightmargin=5pt,
innertopmargin=0pt,
leftmargin=0cm,
rightmargin=0cm,
linewidth=4pt,
innerbottommargin=0pt]{dBox}	

% Corollary box
\newmdenv[skipabove=7pt,
skipbelow=7pt,
rightline=false,
leftline=true,
topline=false,
bottomline=false,
linecolor=gray,
backgroundcolor=black!5,
innerleftmargin=5pt,
innerrightmargin=5pt,
innertopmargin=5pt,
leftmargin=0cm,
rightmargin=0cm,
linewidth=4pt,
innerbottommargin=5pt]{cBox}	

% Indication box
\newmdenv[skipabove=7pt,
skipbelow=7pt,
rightline=false,
leftline=true,
topline=false,
bottomline=false,
linecolor=gray,
backgroundcolor=black!5,
innerleftmargin=5pt,
innerrightmargin=5pt,
innertopmargin=5pt,
leftmargin=0pt,
rightmargin=0pt,
linewidth=4pt,
innerbottommargin=5pt]{iBox}				  
		  			  


% Creates an environment for each type of theorem and assigns it a theorem text style from the "Theorem Styles" section above and a colored box from above
\newenvironment{theorem}{\begin{tBox}\begin{theoremeT}}{\end{theoremeT}\end{tBox}}
\newenvironment{exercise}{\begin{eBox}\begin{exerciseT}}{\hfill{\color{ocre}\tiny\ensuremath{\blacksquare}}\end{exerciseT}\end{eBox}}				  
\newenvironment{definition}{\begin{dBox}\begin{definitionT}}{\end{definitionT}\end{dBox}}	
\newenvironment{example}{\begin{exampleT}}{\hfill{\tiny\ensuremath{\blacksquare}}\end{exampleT}}		
\newenvironment{corollary}{\begin{cBox}\begin{corollaryT}}{\end{corollaryT}\end{cBox}}	
\newenvironment{indicacion}{\begin{iBox}\begin{indicacionT}}{\end{indicacionT}\end{iBox}}	

%----------------------------------------------------------------------------------------
%	REMARK ENVIRONMENT
%----------------------------------------------------------------------------------------

%\newenvironment{remark}{\par\vskip10pt\small % Vertical white space above the remark and smaller font size
%\begin{list}{}{
%\leftmargin=35pt % Indentation on the left
%\rightmargin=25pt}\item\ignorespaces % Indentation on the right
%\makebox[-2.5pt]{\begin{tikzpicture}[overlay]
%\node[draw=ocre!60,line width=1pt,circle,fill=ocre!25,font=\sffamily\bfseries,inner sep=2pt,outer sep=0pt] at (-15pt,0pt){\textcolor{ocre}{R}};\end{tikzpicture}} % Orange R in a circle
%\advance\baselineskip -1pt}{\end{list}\vskip5pt} % Tighter line spacing and white space after remark

%----------------------------------------------------------------------------------------
%	SECTION NUMBERING IN THE MARGIN
%----------------------------------------------------------------------------------------

\makeatletter
%\renewcommand{\@seccntformat}[1]{\llap{\textcolor{grayceu}{\csname the#1\endcsname.}\hspace{1em}}}                    
\renewcommand{\section}{
	\@startsection{section}{1}{\z@}
	{-4ex \@plus -1ex \@minus -.4ex}{1ex \@plus.2ex }
	{\normalfont\LARGE}
}
\renewcommand{\subsection}{
	\@startsection{subsection}{2}{\z@}
	{-3ex \@plus -0.1ex \@minus -.4ex}{0.5ex \@plus.2ex }
	{\normalfont\Large}
}
\renewcommand{\subsubsection}{
	\@startsection{subsubsection}{3}{\z@}
	{-2ex \@plus -0.1ex \@minus -.2ex}{0.2ex \@plus.2ex }
	{\normalfont\large}
}                        
\renewcommand{\paragraph}{
	\@startsection{paragraph}{4}{\z@}
	{-2ex \@plus-.2ex \@minus .2ex}{0.1ex}
	{\normalfont\bfseries}
}

\def\thechapter{\arabic{chapter}}
\def\thesection{\thechapter.\arabic{section}.}
\def\thesubsection{\thesection\arabic{subsection}.}
\def\thesubsubsection{\thesubsection\arabic{section}.}

%----------------------------------------------------------------------------------------
%	CHAPTER HEADINGS
%----------------------------------------------------------------------------------------
%\newcommand{\thechapterimage}{}
%\newcommand{\chapterimage}[1]{\renewcommand{\thechapterimage}{#1}}
\def\@makechapterhead#1{
\thispagestyle{empty}
{
\ifnum \c@secnumdepth >\m@ne
\if@mainmatter
\startcontents
\begin{tikzpicture}[remember picture,overlay]
% \node at (current page.north west)
% {\begin{tikzpicture}[remember picture,overlay]
% 
% %\node[anchor=north west] at (-4pt,-27.5mm) {\includegraphics[width=\paperwidth]{\thechapterimage}};
% %Commenting the 3 lines below removes the small contents box in the chapter heading
% % \draw[fill=white,opacity=.6] (30mm,-30mm) rectangle (9cm,-9cm);
% % \node[anchor=north west] at (30mm,-30mm) {\parbox[t][9cm][t]{5.5cm}{\huge\bfseries\flushleft \printcontents{l}{1}{\setcounter{tocdepth}{2}}}};
% % \draw[anchor=west] (5cm,-11cm) node [rounded corners=25pt,fill=white,opacity=.7,inner sep=15.5pt]{\huge\sffamily\bfseries\textcolor{black}{\vphantom{plPQq}\makebox[20cm]{}}};
\draw[anchor=north west, fill=black] (0mm,50mm) rectangle (1pt,10mm);
\draw[anchor=north west] (0cm,20mm) node {\scalebox{1.5}{\Huge{\textcolor{grayceu}\thechapter}}}; 
%\draw[anchor=west] (0cm,0cm) node {\Huge{#1}}; 
\end{tikzpicture}
\par\vspace*{2cm}
{\Huge #1}
\par\vspace*{1cm}
% \end{tikzpicture}}\par\vspace*{230\p@}
}}
\def\@makeschapterhead#1{
\thispagestyle{empty}
{
\ifnum \c@secnumdepth >\m@ne
\if@mainmatter
\startcontents
\begin{tikzpicture}[remember picture,overlay]
% \node at (current page.north west)
% {\begin{tikzpicture}[remember picture,overlay]
% \node[anchor=north west] at (-4pt,-27.5mm) {\includegraphics[width=\paperwidth]{\thechapterimage}};
% \draw[anchor=west] (5cm,-11cm) node [rounded corners=25pt,fill=white,opacity=.7,inner sep=15.5pt]{\huge\sffamily\bfseries\textcolor{black}{\vphantom{plPQq}\makebox[20cm]{}}};
% \draw[anchor=west] (5cm,-11cm) node [rounded corners=25pt,inner sep=15.5pt]{\huge\sffamily\bfseries\textcolor{black}{#1\vphantom{plPQq}\makebox[20cm]{}}};
% \end{tikzpicture}};
% \end{tikzpicture}}\par\vspace*{230\p@}
%\draw[anchor=west] (0cm,0cm) node {\Huge{#1}}; 
\end{tikzpicture}
{\Huge #1}
\par\vspace*{1cm}
}}
\makeatother